\documentclass[12pt,paper=a4]{scrartcl}

% Author, Title, Subtitle etc.
\author{Carsten Becker}
\title{The Northwind and the Sun}
\subtitle{An Ayeri Translation, Revisited}
\date{\today} % Format: YYYY/MM/DD (rev. N, YYYY/MM/DD)

% Provide running author and date
\makeatletter
\let\runauthor\@author
\let\rundate\@date
\makeatother

% Handle language and quotation marks
\usepackage[american]{babel}
\usepackage{csquotes} % Put quotations in \enquote{}!
\SetBlockEnvironment{quote}
\renewcommand*{\mkccitation}[1]{ (#1)}

% Set all margins to 2.54 cm
\usepackage[margin=1in]{geometry}
\widowpenalty10000 % Avoid widows like the plague!
\clubpenalty10000 % Avoid orphans like the plage, too!

% Extended formatting of lists
\usepackage{enumitem}
\setlist[itemize]{noitemsep, leftmargin=3em, labelwidth=2.5em}%, labelsep=0em}

% Make multiple columns available in single-column document
\usepackage{multicol}

% Make text colors and color names available
\usepackage[xetex]{xcolor}

% Load font stuff for XeTeX
\usepackage{xltxtra}
\usepackage{fontspec}

% Set main fonts
%\usepackage[config=mt-Junicode]{microtype}

\newfontfamily{\Tagati}[
    Renderer=Graphite,
    Scale=1.1,
    BoldFont={* Italic},
    HyphenChar=·,
]{Tagati Book G}

\setmainfont[
    Ligatures=TeX,
    Numbers=Lowercase,
]{EB Garamond}

\setsansfont[
    Ligatures=TeX,
    Numbers=Lowercase,
    Scale=MatchUppercase,
    BoldFont={Open Sans Condensed Bold},
]{Open Sans Condensed Light}

% Load BibLaTeX (using Biber), configure citation styles
\usepackage[
    authordate-trad,
    backend=biber,
    safeinputenc,
]{biblatex-chicago}

% To make \textcite look like "Doe (2014: 213)"
\renewcommand*{\postnotedelim}{\addcolon\addspace}
\DeclareFieldFormat{postnote}{#1}
\DeclareFieldFormat{multipostnote}{#1}

% Enable generating files from this .tex file (for the bibliography) 
\usepackage{filecontents}

% Date etc.
\usepackage[
	datesep={/},
]{datetime2}

% Clickable links in footnotes, TOC, etc.
\usepackage[
    xetex,
    bookmarks=true,
    colorlinks=false,
    linktoc=section,
    hidelinks,
    pdfusetitle,
]{hyperref}

\usepackage{url}
\urlstyle{rm}

% Ability to include graphics and dealing with footnotes in descriptions
\usepackage{graphicx}
\usepackage[font={small,sf},labelfont={small,sf},format=plain]{caption}
\usepackage{subcaption}
\usepackage{wrapfig}
\setlength{\columnsep}{2\baselineskip}

% General headers and footers
\usepackage{fancyhdr}
\pagestyle{fancy}

\fancyhead[L]{} % empty
\fancyhead[C]{} % empty
\fancyhead[R]{\thepage}

\fancyfoot[L]{} % empty
\fancyfoot[C]{} % empty
\fancyfoot[R]{} % empty

\renewcommand{\headrulewidth}{0pt}
\renewcommand{\footrulewidth}{0pt}

% First page headers and footers are different
\fancypagestyle{firstpage}{
    \fancyhead[L]{\sffamily \footnotesize \textbf{Benung. The Ayeri Language Resource}}
    \fancyhead[C]{} % empty
    \fancyhead[R]{\sffamily \footnotesize \runauthor{} · \rundate{}}
    
    \fancyfoot[L]{} % empty
    \fancyfoot[C]{} % empty
    \fancyfoot[R]{\sffamily \footnotesize 
	\href{http://benung.nfshost.com}{http://benung.nfshost.com} · 
	\href{https://github.com/carbeck/benung-pdfs}{https://github.com/carbeck/benung-pdfs} · 
	\href{https://creativecommons.org/licenses/by-sa/4.0/}{CC~BY-SA~4.0}
    }
    
    \renewcommand{\headrulewidth}{0.5pt}
    \setlength\footskip{0.5in}
}

\usepackage{ifthen}
\ifthenelse{\value{page}=1}{\thispagestyle{firstpage}{\pagestyle{fancy}}}

% Line spacing
\usepackage{setspace}
\onehalfspacing

% Avoid pagebreaks right after sections and subsections
\usepackage{needspace}
\usepackage{etoolbox}
\preto\section{\needspace{6\baselineskip}}
\preto\subsection{\needspace{6\baselineskip}}

% Things for tables
% \usepackage{longtable}
% \usepackage{tabu}
% \usepackage{booktabs}
% \usepackage{rotating}

% Formatting of table of glossing abbreviations from Leipzig package manual
\usepackage[acronym,nomain,nonumberlist,nopostdot]{glossaries}
\usepackage{glossary-inline}%

\newglossarystyle{mysuper}{%
	\glossarystyle{super}% based on super
	\renewenvironment{theglossary}{%
		\begin{itemize}[align=left]
	}{%
		\end{itemize}%
	}%
	\renewcommand*{\glossaryheader}{}%
	\renewcommand*{\glsgroupheading}[1]{}%
	\renewcommand*{\glossaryentryfield}[5]{%
		\item[\glsentryitem{##1}\glstarget{##1}{##2}]
		\makefirstuc{##3}\glspostdescription{}
	}%
	\renewcommand*{\glsgroupskip}{}%
}%

% Formatting of glosses
\usepackage{expex}
\lingset{
	aboveexskip=0em,
	Everyex={\singlespacing},
}
\usepackage{leipzig}

\newleipzig{AgtT}{at}{agent topic}
\newleipzig{PatT}{pt}{patient topic}
\newleipzig{DatT}{datt}{dative topic}
\newleipzig{GenT}{gent}{genitive topic}
\newleipzig{LocT}{loct}{locative topic}
\newleipzig{InsT}{inst}{instrumental topic}
\newleipzig{CauT}{caut}{causative topic}
\newleipzig{An}{an}{animate}
\newleipzig{Inan}{inan}{inanimate}
\newleipzig{Hab}{hab}{habitative}
\newleipzig{Comp}{comp}{comparative}
\newleipzig{Ayr}{ayr}{Ayeri}

\makeglossaries

% Nicer footnotes
\usepackage[bottom,hang,norule]{footmisc}
\setlength{\footnotesep}{0.75\baselineskip}

% Smaller font in block quotes
\usepackage{relsize}
\AtBeginEnvironment{quote}{\noindent\smaller}
\AtBeginEnvironment{quotation}{\smaller}

% Macros
\newcommand{\fw}[1]{\textit{#1}} % Foreign Word
\newcommand{\tit}[1]{\textit{#1}} % Title of a work
\newcommand{\q}[1]{\enquote{#1}} % Context-aware quotation
\newcommand{\qq}[1]{\enquote*{#1}} % Explicit sublevel quotation
\newcommand{\tsup}[1]{\textsuperscript{#1}} % Superscript
\newcommand{\markyellow}[1]{\colorbox{yellow}{#1}} % Yellow highlighter
\newcommand{\ques}{\fakesuperscript{?}} % raised question mark

\newcommand{\ayr}[1]{{\Tagati #1}}
\newcommand{\xayr}[3]{{\Tagati #1} \emph{#2} \enquote*{#3}}

\newenvironment{ayeri}{
    %\hyphenpenalty=10000
    %\hbadness=10000
    \doublespacing
    \begin{multicols}{2}
    \Tagati
}{
    \end{multicols} \par
}

\newenvironment{mytitle}{
    \hfill
    \begin{minipage}{0.667\textwidth}
	\vspace{\baselineskip}
	\begin{center}
	    \Large
	    \sffamily\bfseries
	    \makeatletter
}{
	    \makeatother
	\end{center}
	\vspace{1em}
    \end{minipage}
    \hfill
}

% blah
\usepackage{lipsum}

%% BIBLIOGRAPHY DATABASE %%%%%%%%%%%%%%%%%%%%%%%%%%%%%%%%%%%%%%%%%%%%%%%%%%%%%%%

\begin{filecontents*}{\jobname.bib}
@inbook{aesop:northwind,
	crossref = {ipa2007},
	author = {Aesop},
	title = {{The North Wind and the Sun}},
	pages = {39},
}

@article{becker2004,
	journaltitle = {{Benung}},
	journalsubtitle = {{The Ayeri Language Resource}},
	author = {Carsten Becker},
	title = {{Temihin nay perin}},
	date = {2004-12-05},
	url = {http://benung.nfshost.com/wp-content/uploads/2011/06/xmp_northwind.pdf},
	urldate = {2016-04-26},
}

@article{becker2012,
	journaltitle = {{Benung}},
	journalsubtitle = {{The Ayeri Language Resource}},
	author = {Carsten Becker},
	title = {{Correlative Conjunctions}},
	date = {2012-12-10},
	url = {http://benung.nfshost.com/archives/2961},
	urldate = {2016-04-27},
}

@article{becker2015,
	journaltitle = {{Benung}},
	journalsubtitle = {{The Ayeri Language Resource}},
	author = {Carsten Becker},
	title = {{Fences and Gardens}},
	subtitle = {{An Ayeri Translation of a Medieval Neighborhood Dispute}},
	date = {2015-10-18},
	url = {http://benung.nfshost.com/wp-content/uploads/2015/10/2015-09-22-medieval-deed.pdf},
	urldate = {2016-04-26},
}

@book{corbett2006,
	author = {Greville G. Corbett},
	title = {Agreement},
	chapter = {8.3 and 8.4},
	pages = {243--253},
	series = {Cambridge Textbooks in Linguistics},
	publisher = {Cambridge University Press},
	location = {Cambridge},
	year = {2006},
}

@article{cysouwwälchli2007,
	author = {Michael Cysouw and Bernhard Wälchli},
	title = {{Parallel Texts}},
	subtitle = {{Using Translational Equivalents in Linguistic Typology}},
	journaltitle = {Sprachtypologie und Universalienforschung},
	volume = {60},
	number = {2},
	year = {2007},
	pages = {95--99},
%	doi = {10.1524/stuf.2007.60.2.95},
}

@book{ipa2007,
	title = {{Handbook of the International Phonetic Association}},
	subtitle = {{A Guide to the Use of the International Phonetic Alphabet}},
	edition = {9},
	author = {{International Phonetic Association}},
	location = {Cambridge},
	publisher = {Cambridge University Press},
	year = {2007},
}
\end{filecontents*}

\addbibresource{\jobname.bib}

%% END OF PREAMBLE %%%%%%%%%%%%%%%%%%%%%%%%%%%%%%%%%%%%%%%%%%%%%%%%%%%%%%%%%%%%%

\begin{document}

%% MAIN PART %%%%%%%%%%%%%%%%%%%%%%%%%%%%%%%%%%%%%%%%%%%%%%%%%%%%%%%%%%%%%%%%%%%

\begin{mytitle}
    \@title:\\
    \@subtitle
\end{mytitle}

\section{A Few Introductory Remarks}

The Aesopian fable, \enquote{\citetitle{aesop:northwind},} was one of the first texts -- if not \emph{the} first text -- I translated into Ayeri, back in \citeyear{becker2004}. Ayeri was still very young then and looked a little different from today. Just compare a sentence from the old translation (i) to its current reinterpretation (ii):

\ex[exno=i, belowexskip=0em]
\begingl
	\glpreamble \enquote{Viarilea ang macubriyàn Temihin nusatyo nay Perinin, sang luga samnoea ang engongiyànin numicyo, nay edauyiea loasanoin ang masahaiyè sasanoea, sang manaiconisaiyè cong metovaea eimato.} (\cite[1]{becker2004}) //
	\gla Viaril @ -ea ang ma- @ cubra @ -iy<à>n Temihin nu- @ satyo nay Perin @ -in, si @ -ang luga sam @ -no @ -ea ang eng @ -ong @ -iy<à>n @ -in nu- @ micyo, nay edauyi @ -ea lo- @ asano @ -in ang ma- @ saha @ -iy<è> sasano @ -ea, si @ -ang ma- @ naiconisa @ -iy<è> cong me- @ tova @ -ea ei- @ mato. //
	\glb sometime -\Loc{} \AgtT{} \Pst{}- quarrel -\Tpl{}<1> north.wind \Aarg{}- cold and sun -\Top{}, \Rel{} -\Aarg{} among two -\Nmlz{} -\Loc{} \AgtT{} be.more -\Irr{} -\Tpl{}<1> -\Top{} \Aarg{}- strong, and this.time -\Loc{} \Indf{}.\Aarg{}- traveler -\Top{} \AgtT{} \Pst{}- come -\Tsg{}<2> way -\Loc{}, \Rel{} -\Aarg{} \Pst{}- be.wrapped -\Tsg{}<2> inside \Indf{}.\Parg{}- cloak -\Loc{} \Obl{}- warm. //
	\glft `Once upon a time the cold North Wind and the Sun quarreled, who among the two would be stronger, and at this time a traveler came on the way, who was wrapped into a warm cloak.' //
\endgl\xe

\ex[exno=ii]
\begingl
	\glpreamble Ang manga ranyon adauyi Pintemis nay Perin, engyo mico sinyāng luga toya, lingya si lugaya asāyāng si sitang-naykonyāng kong tovaya mato. //
	\gla Ang manga ran @ -yon adauyi Ø= @ Pintemis nay Ø= @ Perin, eng @ -yo mico sinya @ -ang luga toya, ling @ -ya si luga @ -ya asāya @ -ang si sitang= @ naykon @ -yāng kong tova @ -ya mato. //
	\glb \AgtT{} \Prog{} argue -\Tpl{}.\N{} then \Top{}= {North Wind} and \Top{}= Sun, be.more -\Tsg{}.\N{} strong who -\Aarg{} among \Tpl{}.\N{}.\Loc{}, while -\Loc{} \Rel{} pass -\Tsg{}.\M{} traveler -\Aarg{} \Rel{} self= wrap -\Tsg{}.\M{}.\Aarg{} inside cloak -\Loc{} warm. //
	\glft `The North Wind and the Sun were then arguing which among them is stronger, all the while a traveler passed by who had wrapped himself in a warm cloak.' //
\endgl
\xe

The updated version of the Ayeri text which I am presenting here has been translated completely from scratch. Whereas I used a rendition of the text in German as the basis of my \citeyear{becker2004} version, I am now using the English text as provided by the \citetitle{ipa2007}, which has some notoriety as a parallel text in the linguistics community \autocite[see e.g.][97]{cysouwwälchli2007}. As in previous translation write-ups, I will progress through the text sentence by sentence. The translated sentences will be broken down by interlinear annotation, and I will comment on passages in terms of structural or lexical details that caught my attention while translating.

\section{The Text in English}

\blockcquote[39]{ipa2007}{The North Wind and the Sun were disputing which was the stronger, when a traveler came along wrapped in a warm cloak. They agreed that the one who first succeeded in making the traveller take his cloak off should be considered stronger than the other. Then the North Wind blew as hard as he could, but the more he blew the more closely did the traveller fold his cloak around him; and at last the North Wind gave up the attempt. Then the Sun shone out warmly, and immediately the traveller took off his cloak. And so the North Wind was obliged to confess that the Sun was the stronger of the two.}

\section{(Re-)Attempting an Ayeri Translation}

We have already seen the newly translated version of the first sentence of the text above, but I will repeat it here again for completeness:

\ex % The North Wind and the Sun were disputing ...
\begingl
	\glpreamble Ang manga ranyon adauyi Pintemis nay Perin, engyo mico sinyāng luga toya, lingya si lugaya asāyāng si sitang-naykonyāng kong tovaya mato. //
	\gla Ang manga ran @ -yon adauyi Ø= @ Pintemis nay Ø= @ Perin, eng @ -yo mico sinya @ -ang luga toya, ling @ -ya si luga @ -ya asāya @ -ang si sitang= @ naykon @ -yāng kong tova @ -ya mato. //
	\glb \AgtT{} \Prog{} argue -\Tpl{}.\N{} then \Top{}= {North Wind} and \Top{}= Sun, be.more -\Tsg{}.\N{} strong who -\Aarg{} among \Tpl{}.\N{}.\Loc{}, while -\Loc{} \Rel{} pass -\Tsg{}.\M{} traveler -\Aarg{} \Rel{} self= wrap -\Tsg{}.\M{}.\Aarg{} inside cloak -\Loc{} warm. //
	\glft `The North Wind and the Sun were then arguing which among them is stronger, all the while a traveler passed by who had wrapped himself in a warm cloak.' //
\endgl
\xe

The \citeyear{becker2004} version here uses a verb in the first clause, \xayr{kubFr/}{kubra-}{quarrel}, which I translated as \xayr{rnF/}{ran-}{argue} here. It may be noted that \ayr{kubFr/} \textit{kubra-} still exists, however, the dictionary gives it as `get into a conflict' these days, without further explanation. Like in my translation of a medieval deed \autocite[9]{becker2015}, I rendered the \enquote{when} clause not just with the plain preposition \xayr{liNF}{ling}{on (top of), while} used as a temporal adverb, but as a preposition proper with the complement given as a relative clause. Thus, this \ayr{liNFy si} \textit{lingya si} literally means `on top, where ...', though in context it might be better translated back into English as `all the while (that)'.

The words \xayr{pinF}{pin}{wind} and \xayr{perinF}{perin}{sun} are in the animate gender in Ayeri, but as they do not exhibit sex, they take neuter agreement. Since the North Wind and the Sun are anthropomorphized opponents here, it would make sense to assign masculine and feminine gender, respectively. For the time being, I can at least say that where gender resolution involving masculine and feminine constituents occurs, resolution to masculine forms is favored in Ayeri; I need to put more thought into this.\footnote{For example, \cite[Ch. 8.3 and 8.4]{corbett2006} would be very informative here.} Either way, though, I will give `it' instead of `he' or `she' for both North Wind and Sun in the English back-translations where it applies here, and no gender resolution will be necessary.

\ex % They agreed that the one ...
\begingl
	\glpreamble Sakantong, engongyo mico danyās palung menanang sirī ang pahongya asāya tovaley yana. //
	\gla Sakan @ -tong, eng @ -ong @ -yo mico danya @ -as palung menan @ -ang si @ -ri<i> ang pah @ -ong @ -ya asāya @ -Ø tova @ -ley yana. //
	\glb Agree -\Tpl{}.\N{}, be.more -\Irr{} -\Tsg{}.\N{} strong one -\Parg{} other first -\Aarg{} \Rel{} <-\Aarg{}>-\Ins{} \AgtT{} remove -\Irr{} -\Tsg{}.\M{} traveler -\Top{} cloak -\Parg{}.\Inan{} \Tsg{}.\M{}.\Gen{}. //
	\glft `They agreed that the first one due to whom the traveller would take off his cloak would be stronger than the other.' //
\endgl
\xe

I think that this sentence at least counters the claim that stories like this one would \textcquote[97]{cysouwwälchli2007}{deliberately evade complex linguistic constructions}. In the English version, this passage is pretty complex in containing a subject noun phrase that contains a relative clause that in turn contains a causative construction which gets nominalized, which is then followed by a verb phrase containing a verb whose complement is formed by a predicative noun phrase which consists of a comparative construction. This combines all of the things that Ayeri is ridiculously baroque about, which is why I could not just translate the English sentence in a very straightforward way, but had to rephrase things a little. The success brought about by making the traveler take off his cloak is thus implied by context in my Ayeri translation.

\pex % Then the North Wind blew ...
\a \begingl
	\glpreamble Gihayo ang Pintemis minganeri-hen yona. //
	\gla Giha @ -yo ang= @ Pintemis mingan @ -eri @ =hen yona. //
	\glb blow -\Tsg{}.\N{} A= {North Wind} ability -\Ins{} =all \Tsg{}.\N{}.\Gen{}. //
	\glft `The North Wind blew with all of his might.' //
\endgl
\a \begingl
	\glpreamble Nay gihayong mico nay mico-eng, nay ang da-naykonya rado nay rado-eng asāya tovaley yana. //
	\gla Nay giha @ -yong mico nay mico @ =eng, nay ang da= @ naykon @ -ya rado nay rado @ =eng asāya @ -Ø tova @ -ley yana. //
	\glb and blow -\Tsg.\N{}.\Aarg{} strong and strong =\Comp{}, and \AgtT{} so= wrap -\Tsg{}.\M{} tight and tight =\Comp{} traveler -\Top{} cloak -\Parg{}.\Inan{} \Tsg{}.\M{}.\Gen{}. //
	\glft `And it blew harder and harder, and the traveller so wrapped his cloak tighter and tighter.' //
\endgl

\a \begingl
	\glpreamble Subryo deramyam ang Pintemis. //
	\gla Subr @ -yo deramyam ang= @ Pintemis. //
	\glb give.up -\Tsg{}.\N{} {after all} \Aarg{}= {North Wind}. //
	\glft `The North Wind gave up after all.' //
\endgl
\xe

There is an article on my blog that deals with correlative conjunctions in Ayeri \autocite{becker2012}, but unfortunately, it leaves out a discussion of strategies to deal with `as ... as ...' and `the ... the ...'. I racked my brain for a while and came to the conclusion that I could as well paraphrase the respective passages.\footnote{A cursory search on the internet for \enquote{correlative conjunctions} or variations thereof proved pointless. Almost only exercise sheets for English classes and style-guide websites came up, but nothing that looked like the typology papers I had hoped for. I concede that more research on my part could have been done here, but I am going to again defer this to another time.} I suppose that it could be possible to shoehorn the `blew as hard as he could' part into a construction using \xayr{km/}{kama-}{be alike, be as … as …}, but I was feeling a little uneasy about something like:

\ex[exno=iii]
\begingl
	\glpreamble \judge\ques{}Gihayo kamayam mico mingyong ang Pintemis ... //
	\gla Giha @ -yo kama @ -yam mico ming @ -yong ang= @ Pintemis ... //
	\glb blow -\Tsg{}.\N{} be.alike -\Ptcp{} strong can -\Tsg{}.\N{}.\Aarg{} \Aarg{}= {North Wind} ... //
	\glft `The North Wind blew as hard as it could ...' //
\endgl
\xe

This seemed just too nested to be comfortable, and I am not sure if it should be permissible in Ayeri for the standard of comparison to be a finite clause -- \xayr{miNFyoNF}{mingyong}{it could} is a full clause by itself.

\ex % Then the Sun shined out warmly ...
\begingl
	\glpreamble Cunyo makayam mato epang ang Perin, nay ang pahya edauyikan asāya tovaley yana. //
	\gla Cun @ -yo maka @ -yam mato epang ang= @ Perin, nay ang pah @ -ya edauyikan asāya @ -Ø tova @ -ley yana. //
	\glb begin -\Tsg{}.\N{} shine -\Ptcp{} warm next \Aarg{}= Sun, and \AgtT{} remove -\Tsg{}.\M{} immediately traveler -\Top{} cloak -\Parg{}.\Inan{} \Tsg{}.\M.\Gen{} //
	\glft `Next, the Sun began to shine warmly, and the traveler immediately took off his cloak.' //
\endgl
\xe

The expression `shine out' is a peculiarity of English here, so I rephrased it as \xayr{kYunFyo mkymF}{cunyo makayam}{began to shine} according to narrative logic.

\ex % And so the North Wind was obliged ...
\begingl
	\glpreamble Kada rua bengyo ang Pintemis, ang engyo mico cuyam Perin luga toya sam. //
	\gla Kada rua beng @ -yo ang= @ Pintemis, ang eng @ -yo mico cuyam  Ø= @ Perin luga toya sam. //
	\glb thus must admit -\Tsg{} \Aarg{}= {North Wind}, \AgtT{} be.more -\Tsg{}.\N{} strong indeed \Top{}= Sun among \Tpl{}.\N{}.\Loc{} two //
	\glft `Thus the North Wind had to admit that the Sun was indeed the stronger among both of them.' //
\endgl
\xe

I simplified the wording \enquote{was obliged to confess} to an active construction here, \xayr{ru\_a beNFyo}{rua bengyo}{had to admit}. I interpreted this passive construction as a way to show that the North Wind has been humbled by using a more indirect turn of phrase. This being humbled is an important element in the story, so a way to express a concessive gesture on the North Wind's part had to somehow still be expressed. I tried to capture this by introducing the adverb \xayr{kYuymF}{cuyam}{actually, indeed, in fact} in the complement clause.

\section{The Text in Ayeri}

\begin{ayeri}
\noindent ANF mN rnFyonF Adauyi piMtemisF nj perinF, ENFyo mikYo sinYaaNF lug toy, liNFy si lugy AsaaːyNF si sitNF/njkonFyaaNF koNF tovy mto.
skMtoNF, ENoNFyo mikYo dnYaasF pluNF mennNF sirii ANF phoNFy Asaay tovlej yn.
gihyo ANF piM\-temisF miNneri/henF yon.
nj gihyoNF mikYo nj mikYo/ENF, nj ANF d/njkonFy rdo nj rdo/ENF Asaay tovlej yn.
subFrFyo dermFymF ANF piMtemisF.
kYunFyo mkymF mto EpNF ANF perinF, nj ANF phFy EdauyiknF Asaay tovlej yn.
kd ru\_a beNFyo ANF piMtemisF, ANF ENFyo mikYo kYuymF perinF lug toy smF.
\end{ayeri}

\noindent Ang manga ranyon adauyi Pintemis nay Perin, engyo mico sinyāng luga toya, lingya si lugaya asāyāng si sitang-naykonyāng kong tovaya mato.
Sakantong, engongyo mico danyās palung menanang sirī ang pahongya asāya tovaley yana.
Gihayo ang Pintemis minganeri-hen yona.
Nay gihayong mico nay mico-eng, nay ang da-naykonya rado nay rado-eng asāya tovaley yana.
Subryo deramyam ang Pintemis.
Cunyo makayam mato epang ang Perin, nay ang pahya edauyikan asāya tovaley yana.
Kada rua bengyo ang Pintemis, ang engyo mico cuyam Perin luga toya sam.\\

\begin{sloppypar}
\noindent /aŋ ˈmaŋa ˈranjɔn aˈdawi pɪnˈtemɪs naɪ ˈperɪn | ˈɛŋjo ˈmitʃo sɪnˈjaːŋ ˈluɡa ˈtoja | ˈlɪŋja si luˈɡaja aˈsaːjaːŋ si siˈtaŋ naɪkɔnˈjaːŋ kɔŋ toˈvaja ˈmato ||
sakanˈtɔŋ | ɛŋˈɔŋjo ˈmitʃo daˈnjaːs paˈlʊŋ mɛnaˈnaŋ siˈriː aŋ paˈhɔŋja aˈsaːja tovaˈlɛɪ ˈjana ||
ɡiˈhajo aŋ pɪnˈtemɪs mɪŋaˈneri hɛn ˈjona ||
naɪ ɡihaˈjɔŋ ˈmitʃo naɪ ˈmitʃoɛŋ | naɪ da naɪˈkɔnja ˈrado naɪ ˈradoɛŋ aˈsaːja tovaˈlɛɪ ˈjana ||
ˈsubrjo deˈramjam aŋ pɪnˈtemɪs ||
ˈtʃʊnjo maˈkajam ˈmato eˈpaŋ aŋ ˈperɪn | naɪ aŋ ˈpahja eˈdawikan aˈsaːja tovaˈlɛɪ ˈjana ||
ˈkada rwa ˈbɛŋjo aŋ pɪnˈtemɪs | aŋ ˈɛŋjo ˈmitʃo ˈtʃujam ˈperɪn ˈluɡa ˈtoja ˈsam/
\end{sloppypar}

%% BIBLIOGRAPHY %%%%%%%%%%%%%%%%%%%%%%%%%%%%%%%%%%%%%%%%%%%%%%%%%%%%%%%%%%%%%%%%

%\vfill
%\pagebreak

\addsec{Abbreviations}
\begin{multicols}{3}
\printglossary[style=mysuper,type=\leipzigtype]
\end{multicols}

%\nocite{*} % returns all entries from the bibliography database
\printbibliography[heading=bibintoc]

\end{document}
