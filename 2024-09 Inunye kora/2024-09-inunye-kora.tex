\documentclass[12pt,paper=a4]{scrartcl}

% Author, Title, Subtitle etc.
\author{Carsten Becker}
\title{Inunye kora}
\subtitle{Weirdly arpeggiated fishes}
\date{2024/09} % Format: YYYY/MM/DD (rev. N, YYYY/MM/DD)

% Provide running author and date
\makeatletter
\let\runauthor\@author
\let\rundate\@date
\makeatother

% Handle language and quotation marks
\usepackage{polyglossia}
\setdefaultlanguage{english}
\setotherlanguage{german}
\usepackage{csquotes} % Put quotations in \enquote{}!
\SetBlockEnvironment{quotation}
\renewcommand*{\mkccitation}[1]{ (#1)}
\let\q\textquote

% Quotation style for word definitions
\DeclareQuoteStyle{wdef}
	{\textquoteleft}{\textquoteright}
	{\textquotedblleft}{\textquotedblright}
\newcommand{\wdef}[1]{{\setquotestyle{wdef}\enquote{#1}}}

% Set all margins to 2.54 cm
\usepackage[margin=1in]{geometry}
\widowpenalty10000 % Avoid widows like the plague!
\clubpenalty10000 % Avoid orphans like the plage, too!

% Extended formatting of lists
\usepackage{enumitem}
\newlist{glossdefs}{itemize}{1}
\setlist[glossdefs]{nosep, leftmargin=3em, labelwidth=2.5em, align=left}
\setlist[itemize]{noitemsep}

% Make multiple columns available in single-column document
\usepackage{multicol}

% Make text colors and color names available
\usepackage{xcolor}

% Load font stuff for XeTeX
\usepackage{fontspec}
\usepackage{microtype}

% Set main fonts
\newfontfamily{\Tagati}[
	Renderer=Graphite,
	Scale=1.0,
	BoldFont={* Italic},
	HyphenChar=·,
]{Tagati Book G}

\usepackage{ebgaramond}

\setsansfont[
	Ligatures=TeX,
	Numbers=Lowercase,
	Scale=MatchUppercase,
	BoldFont={Open Sans Condensed Bold},
]{Open Sans Condensed Light}

% Load BibLaTeX (using Biber), configure citation styles
\usepackage[
	style=langsci-unified,
	backend=biber,
	safeinputenc,
	natbib,
]{biblatex}
\addbibresource{bibliography.bib}

% Date etc.
\usepackage[
	datesep={/},
]{datetime2}

% Verse
% \usepackage{verse}

% Tabbing
\usepackage{tabto}

% Ability to include graphics and dealing with footnotes in descriptions
\usepackage{graphicx}
\usepackage[font={small,sf},labelfont={small,sf},format=plain]{caption}
\usepackage{subcaption}
\usepackage{wrapfig}
\setlength{\columnsep}{2\baselineskip}

% General headers and footers
\usepackage{fancyhdr}
\pagestyle{fancy}

\fancyhead[L]{} % empty
\fancyhead[C]{} % empty
\fancyhead[R]{\thepage}

\fancyfoot[L]{} % empty
\fancyfoot[C]{} % empty
\fancyfoot[R]{} % empty

\renewcommand{\headrulewidth}{0pt}
\renewcommand{\footrulewidth}{0pt}

% First page headers and footers are different
\fancypagestyle{firstpage}{
	\fancyhead[L]{\sffamily \footnotesize \textbf{Benung. The Ayeri Language Resource}}
	\fancyhead[C]{} % empty
	\fancyhead[R]{\sffamily \footnotesize \runauthor{} · \rundate{}}
	
	\fancyfoot[L]{} % empty
	\fancyfoot[C]{} % empty
	\fancyfoot[R]{\sffamily \footnotesize 
	\href{https://ayeri.de}{https://ayeri.de} · 
	\href{https://github.com/carbeck/benung-pdfs}{https://github.com/carbeck/benung-pdfs} · 
	\href{https://creativecommons.org/licenses/by-nc-sa/4.0/}{CC~BY-NC-SA~4.0}
	}
	
	\renewcommand{\headrulewidth}{0.5pt}
	\setlength\footskip{0.5in}
}

\usepackage{ifthen}
\ifthenelse{\value{page}=1}{\thispagestyle{firstpage}{\pagestyle{fancy}}}

% Line spacing
\usepackage{setspace}
\onehalfspacing

% Avoid pagebreaks right after sections and subsections
\usepackage{needspace}
\usepackage{etoolbox}
\preto\section{\needspace{6\baselineskip}}
\preto\subsection{\needspace{6\baselineskip}}

% Things for tables
\usepackage{tabularx}
% \usepackage{booktabs}
% \usepackage{rotating}

% Formatting of table of glossing abbreviations from Leipzig package manual
\usepackage[
	acronym,
	nomain,
	nonumberlist,
	nopostdot,
	toc=true,
	xindy={codepage=utf8}, % language=english,
]{glossaries}
\usepackage{glossary-inline}

\newglossarystyle{threecolumn}{%
	\renewenvironment{theglossary}{%
		\begin{multicols}{3}
		\begin{glossdefs}%
	}{%
		\end{glossdefs}%
		\end{multicols}%
	}%
	\renewcommand*{\glossaryheader}{}%
	\renewcommand*{\glsgroupheading}[1]{}%
	\newcommand*{\glossaryentryfield}[5]{%
		\item[\glsentryitem{##1}\glstarget{##1}{##2}]
		% \makefirstuc{##3}\glspostdescription{}
		##3\glspostdescription{}
	}%
	\renewcommand*{\glsgroupskip}{}%
}%

% Formatting of glosses
\usepackage{langsci-gb4e}
\usepackage[glosses]{leipzig}
\renewcommand{\eachwordone}{\itshape}
\renewcommand{\eachwordtwo}{\rule[-.5\baselineskip]{0pt}{0pt}}

\newleipzig{AgtT}{at}{agent topic}
\newleipzig{PatT}{pt}{patient topic}
\newleipzig{DatT}{datt}{dative topic}
\newleipzig{GenT}{gent}{genitive topic}
\newleipzig{LocT}{loct}{locative topic}
\newleipzig{InsT}{inst}{instrumental topic}
\newleipzig{CauT}{caut}{causative topic}
\newleipzig{An}{an}{animate}
\newleipzig{Inan}{inan}{inanimate}
\newleipzig{Hab}{hab}{habitative}
\newleipzig{Sup}{sup}{superlative}
\newleipzig{St}{st}{strong}
\newleipzig{Wk}{wk}{weak}
\newleipzig{Ayr}{ayr}{Ayeri}
\newleipzig{Aff}{aff}{affectionate}

\makeglossaries

% Nicer footnotes
\usepackage[bottom,hang,norule]{footmisc}
\setlength{\footnotesep}{0.75\baselineskip}

% Smaller font in block quotes
\usepackage{relsize}
\AtBeginEnvironment{quote}{\noindent\smaller}
\AtBeginEnvironment{quotation}{\smaller}

% Clickable links in footnotes, TOC, etc.
\usepackage[
% 	xetex,
	bookmarks=true,
	colorlinks=false,
	linktoc=section,
	hidelinks,
	pdfusetitle,
]{hyperref}

% We want URLs to be italic and with regular uppercase numerals
\renewcommand{\UrlFont}{%
	\normalfont%
	\itshape%
	\addfontfeature{RawFeature=-onum}%
}

% In-text references
% cf. https://tex.stackexchange.com/a/139051
% Since German plural formation is not as regular as in English (-e, -en for
% Beispiel), we will define the label as empty
\usepackage[sort&compress,noabbrev]{cleveref}
% \newcommand{\crefrangeconjunction}{--}
\crefname{xnumi}{}{}
\creflabelformat{xnumi}{(#2#1#3)}
\crefname{xnumii}{}{}
\creflabelformat{xnumii}{(#2#1#3)}
\crefname{xnumiii}{}{}
\creflabelformat{xnumiii}{(#2#1#3)}
\crefname{xnumiv}{}{}
\creflabelformat{xnumiv}{(#2#1#3)}
\crefrangeformat{xnumi}{(#3#1#4) to~(#5#2#6)}
\crefrangeformat{xnumii}{(#3#1#4--#5\crefstripprefix{#1}{#2}#6)}
\crefrangemultiformat{xnumii}{(#3\arabic{xnumi}#1#4--#5#2#6)}
{ and~(#3\arabic{xnumi}#1#4--#5#2#6)}{, (#3\arabic{xnumi}#1#4--#5#2#6)}
{ and~(#3\arabic{xnumi}#1#4--#5#2#6)}


% Macros
\newcommand{\fw}[1]{\textit{#1}} % Foreign Word
\newcommand{\tit}[1]{\textit{#1}} % Title of a work
\newcommand{\tsup}[1]{\textsuperscript{#1}} % Superscript
\newcommand{\markyellow}[1]{\colorbox{yellow}{#1}} % Yellow highlighter
\newcommand{\ques}{\textsuperscript{?}} % raised question mark
\newcommand{\zwsp}{\mbox{​}} % Zero-width space (ZWSP)

\newcommand{\ayr}[1]{\zwsp\smash{{\Tagati #1}}} % Plain Ayeri orthography
\newcommand{\rayr}[2]{\zwsp\smash{{\Tagati #1}} \emph{#2}} % Ayeri orthography + *r*omanization
\newcommand{\tayr}[2]{#1 `#2'} % Romanization + *t*ranslation
\newcommand{\xayr}[3]{\zwsp\smash{\Tagati #1} \emph{#2} `#3'} % Ayeri orthography + romanization + translation

\usepackage{suffix}
\WithSuffix\newcommand{\ayr}*[1]{{\Tagati #1}} % Plain Ayeri orthography
\WithSuffix\newcommand{\rayr}*[2]{{\Tagati #1} \emph{#2}} % Ayeri orthography + *r*omanization
\WithSuffix\newcommand{\xayr}*[3]{{\Tagati #1} \emph{#2} `#3'} % Ayeri orthography + romanization + translation

\newenvironment{ayeri}{
	%\hyphenpenalty=10000
	%\hbadness=10000
	\doublespacing
	\begin{multicols}{2}
	\Tagati
}{
	\end{multicols} \par
}

\newenvironment{mytitle}{
	\hfill
	\begin{minipage}{0.667\textwidth}
	\vspace{\baselineskip}
	\begin{center}
		\Large
		\sffamily\bfseries
		\makeatletter
}{
		\makeatother
	\end{center}
	\vspace{1em}
	\end{minipage}
	\hfill
}

%% END OF PREAMBLE %%%%%%%%%%%%%%%%%%%%%%%%%%%%%%%%%%%%%%%%%%%%%%%%%%%%%%%%%%%%

\begin{document}

%% MAIN PART %%%%%%%%%%%%%%%%%%%%%%%%%%%%%%%%%%%%%%%%%%%%%%%%%%%%%%%%%%%%%%%%%%

\begin{mytitle}
	\@title: \@subtitle
\end{mytitle}

At least in my opinion, \citetitle{radiohead:weirdfishes} is the most
fascinating song on
\citeauthor{radiohead:weirdfishes}'s~(\citeyear{radiohead:weirdfishes}) album
\citefield[][]{radiohead:weirdfishes}[booktitle]{booktitle} with its
scintillating string sound. It recently occurred to me that the lyrics are very
straightforward and the spheric soundscape affords one not to be overly strict
with the meter when translating it. All things considered, it seemed like a
worthwhile to try making an Ayeri adaptation, because translating poetic texts
is always a fun challenge.%
%
	\footnote{This derivative fan work is neither endorsed nor licensed by
	Radiohead, their publisher, or the legal representatives of either of them.
	This work neither pursues nor implies any intent of financial gain.}

\begin{quote}
\begin{multicols}{2}
Karonya tasang-vā,\\
tasangya makārya---\\
Sā tila-\\
yang niva.\\
Sinyareng yamanley\\
perisānena nā?

Nasyoyya depāng-nama.\\
Māy sa kacvāng ay!\\
Sā tila-\\
yang niva\\
vana dimayjyam.			\tab \textit{(kunangreng)}\\
Yam nasyyang vās lito	\tab \textit{(kunangreng)}\\
na Mavay---				\tab \textit{(kunangreng)}\\
nay lesayang.\\
Ang sagonya enya māy,	\tab \textit{(kunangreng)}\\
ang sungya mimānley.	\tab \textit{(kunangreng)}\\
Nay eda-\dots			\tab \textit{(kunangreng)}\\
da-nā.

\columnbreak

Ang tavay kondanley\\
limajyam rina\\
nay kimbisanas\\
inunjyam kora,\\
inunjyam kora,\\
inunjyam kora.

Māy, ang---\\
Ang grenay avanya,\\
grenay avanya nay\\
nimpyang.\\
Ang---\\
Ang grenay avanya\\
grenay avanya nay\\
nimpyang.
\end{multicols}

% \begin{multicols}{2}
% \Tagati
% kronFy tsNF/vaa\\
% tsNFy mkaarFy\\
% saa til—\\
% —yNF niv.\\
% ymnFreNF sinFylej\\
% perisaanen naa?

% nsYojy depaaNF/nm.\\
% maaj s ktYFvaaNF Aj!\\
% saa til—\\
% —yNF niv\\
% vn dimjyeymF.			\tab \textit{kunNFreNF}\\
% ymF nsYyNF vaasF lito	\tab \textit{kunNFreNF}\\
% n mvj					\tab \textit{kunNFreNF}\\
% lesyNF.\\
% ANF sgonFy EnFy maaj	\tab \textit{kunNFreNF}\\
% ANF suNFy mimaanFlej.	\tab \textit{kunNFreNF}\\
% nj Ed/—				\tab \textit{kunNFreNF}\\
% d/naa.

% ANF tvj koMdnFlej\\
% limyeymF rin\\
% nj kiMbisnsF\\
% InunFyeymF kor\\
% InunFyeymF kor\\
% InunFyeymF kor

% maaj, ANF\\
% Ang gFrenj AvnFy\\
% gFrenj AvnFy nj\\
% niMpFyNF.\\
% ANF\\
% Ang gFrenj AvanFy\\
% gFrenj AvnFy nj\\
% niMpFyNF.
% \end{multicols}
\end{quote}

\section{Analysis of the translation}

The song is divided into four parts, of which the first two form a unit in both
text and melody as the guitars gradually build up. Then, after a break, the
lush arpeggi become more muted for the third part. A concluding and very sober
sounding coda follows after an instrumental. I will go through my translation
part by part, commenting on each sentence about what stands out to me as worth
mentioning regarding grammar, style, and the process of translation as such.

\subsection{First part}

My translation of the first stanza is given in examples number
\crefrange{ex:1a}{ex:1c}.%
%
	\footnote{Note that where animacy is concerned, I'm only explicitly marking
	inanimate referents in the glosses throughout.}
%
I think that in English, the text corresponding to \cref{ex:1a,ex:1b} can be
interpreted as a single sentence: \blockquote{In the deepest ocean, the bottom
of the sea, your eyes, they turn me.} However, due to the requirements of the
melody and syntax, as well as to keep up the parallelism with \cref{ex:2c}, it
wouldn't have made sense to pull \cref{ex:1b} all the way to the front and have
the location adverbials in \cref{ex:1a} act as modifiers to that. I thus
decided to keep the lines in \cref{ex:1a} for the beginning and use them to
establish context and mood, as in English, though at the expense of making the
second line an anacoluthon.

\begin{exe}
\ex \label{ex:1a}
	\gll Karonya tasang-vā, tasangya makārya~--- \\
		ocean-\Loc{} deep=\Sup{} abyss-\Loc{} black \\
	\trans `In the deepest ocean, in the black abyss ---'
\end{exe}

Consequentially, the text in \cref{ex:1a} doesn't form a complete sentence.
Since \rayr{kronF}{karon} can mean both `sea' and `ocean', I couldn't replicate
the way the English lyrics use these two words for large bodies of water as
basically synonyms. Instead, I played on the fact that \rayr{tsNF}{tasang} can
mean either `deep' or `abyss, chasm' depending on whether it's used as an
adjective or noun, and chose to use \xayr{mkaarFy}{makārya}{dark, black} as an
accompanying adjective accordingly for near-parallelism with \rayr{kronY
tsNF/vaa}{karonya tasang-vā}. The word used for `black' here,
\rayr{mkaarFy}{makārya}, literally means `unwhite'. I don't remember why I
chose to coin the word that way. There is also \xayr{gisu}{gisu}{dark} in the
dictionary, but this is a worse fit considering both the number of syllables of
the original line and assonance.

The next two lines posed something of a challenge as well, since the English
phrasing \blockquote{Your eyes, they turn me} with what basically amounts to a
topic--comment sentence structure doesn't work this way in Ayeri. Instead,
\cref{ex:1b} can only be translated as a regular statement with no syntactic
break in the middle. The topicalization of \textquote{eyes} can be adapted by
using a regular causative construction. Since my Ayeri translation of these two
lines only amounts to three words, another question is where to divide the
lines. I decided to make the cut according to the number of syllables, so right
inside the word \rayr{tilyNF}{tilayang} before the heavy syllable
\rayr{/yNF}{-yang}. This changes the distribution of words per line from 2~+~3
in English to 3~+~3 in Ayeri, though it works because Yorke is rather drawing
out \textquote{your eyes} in the recording, so another syllable can be sneaked
in easily.

\begin{exe}
\ex \label{ex:1b}
	\gll Sā tilayang niva.\normalfont\footnotemark{} \\
		\CauT{}= change=\Fsg.\Aarg{} gaze \\
	\trans `The eye/gaze makes me change.'
	\footnotetext{\xayr{niv}{niva}{eye}, extended here to mean `gaze'.}
\end{exe}

At least to my non-native ears, the phrase \fw{turn somebody} sounds strange,
and looking up the verb in a dictionary didn't yield any clarification either.
So for all intents and purposes, I suppose that the line can only be understood
in the context of its continuation in \cref{ex:2c}. Additionally, the owner of
the gaze (or eye?), \xayr{vn}{vana}{your}, falls victim to keeping the line
length similar to the English text. The Ayeri verb \rayr{til/}{tila-} can mean
`change' when used intransitively, so the sentence is still more or less
sensible on its own, but the context as to whose gaze and the how or why of
changing is similarly lost.

The next two lines in \cref{ex:1c} can be more easily expressed as a single
statement in Ayeri again, compared to the English \blockquote{Why should I stay
here? Why should I stay?} The 8~+~5 syllables of the English lyrics get
redistributed and shortened by one syllable overall to 6~+~6 in the Ayeri
translation, with the break right between the four words.

\begin{exe}
\ex \label{ex:1c}
	\gll Sinyareng yamanley perisānena nā? \\
		what-\Aarg.\Inan{} reason-\Parg.\Inan{} hesitation-\Gen{}
		\Fsg.\Gen{} \\
	\trans `What's the reason for my hesitation?'
\end{exe}

In consequence, the urgency of the question evoked by repetition in the English
text doesn't carry over well. I will grant that reading self-criticism
regarding hesitation into the English lines is my interpretation. Since the
song climaxes with thoughts of escape, though, an expression of ambivalence
towards feelings of infatuation at least doesn't seem far-fetched to me in the
context of the song (or Radiohead's oeuvre as such).

\subsection{Second part}

The next two lines, \blockquote{I'd be crazy not to follow, follow where you
lead} form a unit in English, with the repetition of \textquote{follow} binding
the two together. I made this less explicit in the corresponding Ayeri
translations in \cref{ex:2a,ex:2b}, also because anadiplosis doesn't work well
in Ayeri for syntactic reasons.%
%
	\footnote{I had to look up the term, admittedly.}
%
Literally using \xayr{sdyo}{sadayo}{crazy} in \cref{ex:2a} proved awkward as to
line length, so I rephrased the text as below, though I still had to reach into
my bag of tricks and contracted \rayr{depNNF}{depangang} to
\xayr{depaaNF}{depāng}{fool}. This also eschews the awkwardness of
\rayr{/ANNF}{-angang} with regard to the corresponding English line's rhythm.

\begin{exe}
\ex \label{ex:2a}
	\gll Nasyoyya depāng-nama.\normalfont\footnotemark{} \\
		follow-\Neg-\Tsg.\M{} fool-\Aarg=only \\
	\trans `Only a fool wouldn't follow.'
\end{exe}

For the follow-up line to \cref{ex:2a} I completely rephrased the English text
in \cref{ex:2b} as well to yield the same number of syllables. For rhetoric
effect I used a \q{pseudo-passive} construction with a patient topic and an
affective \rayr{maaj}{māy} at the start.

\begin{exe}
\ex \label{ex:2b}
	\gll Māy sa kacvāng ay! \\
		\Aff{} \PatT= draw=\Second.\Aarg{} \Fsg.\Top{} \\
	\trans `Oh how I'm drawn by you!' \\
\end{exe}

At first, I tried \xayr{y nsYFyNF ItiNF vn}{ya nasyyang iting vana}{your path,
I'm following it}, which is slightly closer to the English text, but is longer
by one syllable and thus doesn't fit the melody as well. The rephrased line now
lacks a mention of following, though this motif is at least alluded to by using
\xayr{ktYF/}{kac-}{draw}, and is still explicitly referred to in \cref{ex:2d}.

The single sentence in \cref{ex:2c} is actually spread over three lines and
picks up on the text in \cref{ex:1b}, which was noted before to lack context.
In English, context of \blockquote{your eyes, they turn me} is provided by
\blockquote{turn me into phantoms}. The Ayeri translation follows the English
text very closely here.

\begin{exe}
\ex \label{ex:2c}
	\gll Sā tilayang niva vana dimayjyam. \\
		\CauT= change=\Fsg.\Aarg{} gaze.\Top{} \Second.\Gen{}
		phantom-\Pl-\Dat{} \\
	\trans `Your gaze makes me turn into phantoms.'
\end{exe}

Still, the construction as such is maybe even trickier in relation to its
musical context. Relevant information about the origin of the gaze,
\xayr{vn}{vana}{your}, is stated almost like an afterthought by tearing the
noun phrase apart. This effect is reinforced by musical phrasing, since
\blockquote{turn me into phantoms} and \rayr{vn dimjyeymF}{vana dimayjam},
respectively, forms a unit with the next three lines subsumed in \cref{ex:2d}.
In Ayeri, the feeling of \fw{non sequitur} may thus be even stronger than in
English. Moreover, the haunting repetition of \textquote{way out} as part of
the background vocals starts here, foreshadowing the lines in
\cref{ex:2e,ex:2f}. I chose \xayr{kunNFreNF}{kunangreng}{the door} for the
Ayeri translation in order not to double the verb \xayr{sgonF/}{sagon-}{exit,
quit} with the corresponding noun \xayr{sgonnF}{sagonan}{exit} in \cref{ex:2e}.

The three lines subsumed by the sentence in \cref{ex:2d} again invoke the motif
of following, though the caution introduced in \cref{ex:1c} finds an echo here
in the consequence of falling off the edge of the world after being misled by
infatuation. The translation follows the English text closely, despite there
being no way to mark an action as involving a punctual change of state in the
same way English uses \fw{fall off} as opposed to just \fw{fall}. Moreover, I
decided to use \xayr{mvj}{mavay}{world} as a name to resemble the stress
pattern of the English line. The alternative would have been using
\rayr{mvyen}{mavayena} with penultimate stress and an additional syllable,
which seems worse.

\begin{exe}
\ex \label{ex:2d}
	\gll Yam nasyyang vās lito na Mavay~--- nay lesayang. \\
		\DatT= follow-\Fsg.\Aarg{} \Second.\Parg{} edge.\Top{} \Gen= World
		and fall=\Fsg.\Aarg{} \\
	\trans `To the edge of the Earth is where I'll follow you---and fall.'
\end{exe}

The sentence in \cref{ex:2e} corresponds to the English \blockquote{Yeah,
everybody leaves if they get the chance}, which divides into one clause per
line. The number of syllables in the English text is one less per clause
compared to the Ayeri one: 6~+~5 in English as opposed to 7~+~6 in Ayeri. The
overall stress pattern is similar enough that it works, though. The expression
\rayr{suNF– mimaanFlej}{find the opportunity} strikes me as rather
Germanic-influenced, but offhand, I couldn't think of a way to phrase it
differently and still make it fit the melody.

\begin{exe}
\ex \label{ex:2e}
	\gll Ang sagonya enya māy, ang sungya mimānley. \\
		\AgtT= quit-\Tsg.\M{} everyone.\Top{} \Aff{} \AgtT= find-\Tsg.\M{}
		opportunity-\Parg.\Inan{} \\
	\trans `Yeah, everybody quits, if they find the opportunity.'
\end{exe}

The last two lines in the second part are given in the sentence in
\cref{ex:2f}. The English lyrics are \blockquote{And this way out is my
chance.} Part of the sentence is provided by the background vocals, and I
replicated this peculiarity with \xayr{kunNFreNF}{kunangreng}{door}. I couldn't
find a succinct way to translate \textquote{way out} literally, for instance as
\xayr{ssaanFreNF AgonnFymF}{sasānreng agonanyam}{the way to the outside}, so I
chose metonymy as \emph{my} way out. I also used \xayr{d/naa}{da-nā}{mine}
instead of something more literal like \xayr{mimaanFlej naa}{mimānley nā}{my
chance} in order not to exceed the length of the English text here either.

\begin{exe}
\ex \label{ex:2f}
	\gll Nay eda-kunangreng da-nā. \\
		and this=door-\Aarg.\Inan{} one=\Fsg.\Gen{} \\
	\trans `And this door is mine.'
\end{exe}

From a morphosyntactic point of view, the construction
\xayr{d/naa}{da-nā}{mine} in \cref{ex:2f} is a little peculiar since Ayeri
normally marks predicative complements with the patient case. However, I
determined in \citet[310--311]{becker:ayrgram} that this is not the case with
predicative possessive pronouns. This rules out a form like
*\rayr{d/naalej}{*da-nāley}.

\subsection{Third part}

The third part of the song follows after a break and contains the lines that
provide the first part of the title, \textquote{Weird fishes}. As the sentence
\cref{ex:3} shows, extensive rephrasing was necessary here compared to the
English text, \blockquote{I get eaten by the worms and weird fishes, picked
over by the worms and weird fishes.} \xayr{InunFye\_amF kor}{inunjang
kora}{rare fishes} exceeds the number of the corresponding English text, but
the melody affords it here even though Ayeri's five syllables admittedly aren't
as smooth as English's three syllables.

\begin{exe}
\ex \label{ex:3}
	\gll Ang tavay kondanley limajyam rina\normalfont\footnotemark{}
		nay kimbisanas inunjyam kora. \\
		\AgtT= become=\Fsg.\Top{} food-\Parg.\Inan{} worm-\Pl-\Dat{} slithery
		and prey-\Parg{} fish-\Pl-\Dat{} rare \\
	\trans I become food for slithery worms and prey for rare fishes. \\
	\footnotetext{\xayr{rin}{rina}{slippery}, extended here to mean `slithery'.}
\end{exe}

The necessity to rephrase mainly arose from me avoiding to coin new words, so I
had to find a different way to express \textquote{picked over} while still
keeping the grammatical structure of the lines in parallel and the meaning
reasonably similar to the English text. The \xayr{limye}{limaye}{worms} thus
gained an adjective \xayr{rin}{rina}{slippery, slithery} in my translation for
parallelism with the adjective \xayr{kor}{kora}{rare} of
\xayr{InunFye}{inunye}{fishes}. In addition, the literal translation of the
first half, \xayr{s koMdFyonF Aj limye\_aNF nj InunFye\_aNF kepau}{Sa konjon ay
limajang nay inunjang kepau}{I get eaten by the worms and weird fishes}, proved
awkward in length and rhythm again. I also didn't like the sound of
\xayr{kepau}{kepau}{strange, weird, odd} with its word accent on the final
syllable and the rare \emph{au} diphthong. I replaced it with \rayr{kor}{kora},
referring here to the peculiarity of rarely observed things, compare Dutch
\fw{raar} `strange, odd, weird, unusual'.

\subsection{Fourth part}

\begin{exe}
\ex \label{ex:4b}
	\gll Māy, ang grenay avanya nay nimpyang. \\
		well \AgtT= reach=\Fsg.\Top{} ground-\Loc{} and run=\Fsg.\Aarg{} \\
	\trans `Well, I reach the bottom and run.'
\end{exe}

\section{Conclusion}


%% BIBLIOGRAPHY %%%%%%%%%%%%%%%%%%%%%%%%%%%%%%%%%%%%%%%%%%%%%%%%%%%%%%%%%%%%%%%

% \vfill
% \pagebreak

\begingroup\multicolsep=0pt
\printglossary[
	style=threecolumn,
	type=leipzig,
]
\endgroup

%\nocite{*} % returns all entries from the bibliography database
\printbibliography[heading=bibintoc]

\end{document}
