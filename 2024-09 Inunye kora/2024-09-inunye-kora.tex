\documentclass[12pt,paper=a4]{scrartcl}

% Author, Title, Subtitle etc.
\author{Carsten Becker}
\title{Inunye kora}
\subtitle{Arpeggiated fishes, weird}
\date{2024/09} % Format: YYYY/MM/DD (rev. N, YYYY/MM/DD)

% Provide running author and date
\makeatletter
\let\runauthor\@author
\let\rundate\@date
\makeatother

% Handle language and quotation marks
\usepackage{polyglossia}
\setdefaultlanguage{english}
\setotherlanguage{german}
\usepackage{csquotes} % Put quotations in \enquote{}!
\SetBlockEnvironment{quotation}
\renewcommand*{\mkccitation}[1]{ (#1)}
\let\q\textquote

% Quotation style for word definitions
\DeclareQuoteStyle{wdef}
	{\textquoteleft}{\textquoteright}
	{\textquotedblleft}{\textquotedblright}
\newcommand{\wdef}[1]{{\setquotestyle{wdef}\enquote{#1}}}

% Set all margins to 2.54 cm
\usepackage[margin=1in]{geometry}
\widowpenalty10000 % Avoid widows like the plague!
\clubpenalty10000 % Avoid orphans like the plage, too!

% Extended formatting of lists
\usepackage{enumitem}
\newlist{glossdefs}{itemize}{1}
\setlist[glossdefs]{nosep, leftmargin=3em, labelwidth=2.5em, align=left}
\setlist[itemize]{noitemsep}

% Make multiple columns available in single-column document
\usepackage{multicol}

% Make text colors and color names available
\usepackage{xcolor}

% Load font stuff for XeTeX
\usepackage{fontspec}

% Set main fonts
\newfontfamily{\Tagati}[
	Renderer=Graphite,
	Scale=1.0,
	BoldFont={* Italic},
	HyphenChar=·,
]{Tagati Book G}

\usepackage{ebgaramond}

\setsansfont[
	Ligatures=TeX,
	Numbers=Lowercase,
	Scale=MatchUppercase,
	BoldFont={Open Sans Condensed Bold},
]{Open Sans Condensed Light}

% % Load BibLaTeX (using Biber), configure citation styles
% \usepackage[
% 	authordate-trad,
% 	backend=biber,
% 	safeinputenc,
% ]{biblatex-chicago}

% % To make \textcite look like "Doe (2014: 213)"
% \renewcommand*{\postnotedelim}{\addcolon\addspace}
% \DeclareFieldFormat{postnote}{#1}
% \DeclareFieldFormat{multipostnote}{#1}

% % Enable generating files from this .tex file (for the bibliography) 
% \usepackage{filecontents}

% Date etc.
\usepackage[
	datesep={/},
]{datetime2}

% Verse
\usepackage{verse}

% Ability to include graphics and dealing with footnotes in descriptions
\usepackage{graphicx}
\usepackage[font={small,sf},labelfont={small,sf},format=plain]{caption}
\usepackage{subcaption}
\usepackage{wrapfig}
\setlength{\columnsep}{2\baselineskip}

% General headers and footers
\usepackage{fancyhdr}
\pagestyle{fancy}

\fancyhead[L]{} % empty
\fancyhead[C]{} % empty
\fancyhead[R]{\thepage}

\fancyfoot[L]{} % empty
\fancyfoot[C]{} % empty
\fancyfoot[R]{} % empty

\renewcommand{\headrulewidth}{0pt}
\renewcommand{\footrulewidth}{0pt}

% First page headers and footers are different
\fancypagestyle{firstpage}{
	\fancyhead[L]{\sffamily \footnotesize \textbf{Benung. The Ayeri Language Resource}}
	\fancyhead[C]{} % empty
	\fancyhead[R]{\sffamily \footnotesize \runauthor{} · \rundate{}}
	
	\fancyfoot[L]{} % empty
	\fancyfoot[C]{} % empty
	\fancyfoot[R]{\sffamily \footnotesize 
	\href{https://ayeri.de}{https://ayeri.de} · 
	\href{https://github.com/carbeck/benung-pdfs}{https://github.com/carbeck/benung-pdfs} · 
	\href{https://creativecommons.org/licenses/by-sa/4.0/}{CC~BY-SA~4.0}
	}
	
	\renewcommand{\headrulewidth}{0.5pt}
	\setlength\footskip{0.5in}
}

\usepackage{ifthen}
\ifthenelse{\value{page}=1}{\thispagestyle{firstpage}{\pagestyle{fancy}}}

% Line spacing
\usepackage{setspace}
\onehalfspacing

% Avoid pagebreaks right after sections and subsections
\usepackage{needspace}
\usepackage{etoolbox}
\preto\section{\needspace{6\baselineskip}}
\preto\subsection{\needspace{6\baselineskip}}

% Things for tables
\usepackage{tabularx}
% \usepackage{booktabs}
% \usepackage{rotating}

% Formatting of table of glossing abbreviations from Leipzig package manual
\usepackage[
	acronym,
	nomain,
	nonumberlist,
	nopostdot,
	toc=true,
	xindy={codepage=utf8}, % language=english,
]{glossaries}
\usepackage{glossary-inline}

\newglossarystyle{threecolumn}{%
	\renewenvironment{theglossary}{%
		\begin{multicols}{3}
		\begin{glossdefs}%
	}{%
		\end{glossdefs}%
		\end{multicols}%
	}%
	\renewcommand*{\glossaryheader}{}%
	\renewcommand*{\glsgroupheading}[1]{}%
	\newcommand*{\glossaryentryfield}[5]{%
		\item[\glsentryitem{##1}\glstarget{##1}{##2}]
		% \makefirstuc{##3}\glspostdescription{}
		##3\glspostdescription{}
	}%
	\renewcommand*{\glsgroupskip}{}%
}%

% Formatting of glosses
\usepackage{langsci-gb4e}
\usepackage[glosses]{leipzig}
\renewcommand{\eachwordone}{\itshape}
\renewcommand{\eachwordtwo}{\rule[-.5\baselineskip]{0pt}{0pt}}

\newleipzig{AgtT}{at}{agent topic}
\newleipzig{PatT}{pt}{patient topic}
\newleipzig{DatT}{datt}{dative topic}
\newleipzig{GenT}{gent}{genitive topic}
\newleipzig{LocT}{loct}{locative topic}
\newleipzig{InsT}{inst}{instrumental topic}
\newleipzig{CauT}{caut}{causative topic}
\newleipzig{An}{an}{animate}
\newleipzig{Inan}{inan}{inanimate}
\newleipzig{Hab}{hab}{habitative}
\newleipzig{Sup}{sup}{superlative}
\newleipzig{St}{st}{strong}
\newleipzig{Wk}{wk}{weak}
\newleipzig{Ayr}{ayr}{Ayeri}
\newleipzig{Ints}{ints}{Intensifier}

\makeglossaries

% Nicer footnotes
\usepackage[bottom,hang,norule]{footmisc}
\setlength{\footnotesep}{0.75\baselineskip}

% Smaller font in block quotes
\usepackage{relsize}
\AtBeginEnvironment{quote}{\noindent\smaller}
\AtBeginEnvironment{quotation}{\smaller}

% Clickable links in footnotes, TOC, etc.
\usepackage[
% 	xetex,
	bookmarks=true,
	colorlinks=false,
	linktoc=section,
	hidelinks,
	pdfusetitle,
]{hyperref}

% We want URLs to be italic and with regular uppercase numerals
\renewcommand{\UrlFont}{%
	\normalfont%
	\itshape%
	\addfontfeature{RawFeature=-onum}%
}

% In-text references
% cf. https://tex.stackexchange.com/a/139051
% Since German plural formation is not as regular as in English (-e, -en for
% Beispiel), we will define the label as empty
\usepackage[sort&compress,noabbrev]{cleveref}
\newcommand{\crefrangeconjunction}{--}
\crefname{xnumi}{}{}
\creflabelformat{xnumi}{(#2#1#3)}
\crefname{xnumii}{}{}
\creflabelformat{xnumii}{(#2#1#3)}
\crefname{xnumiii}{}{}
\creflabelformat{xnumiii}{(#2#1#3)}
\crefname{xnumiv}{}{}
\creflabelformat{xnumiv}{(#2#1#3)}
\crefrangeformat{xnumi}{(#3#1#4)--(#5#2#6)}
\crefrangeformat{xnumii}{(#3#1#4--#5\crefstripprefix{#1}{#2}#6)}
\crefrangemultiformat{xnumii}{(#3\arabic{xnumi}#1#4--#5#2#6)}
{ and~(#3\arabic{xnumi}#1#4--#5#2#6)}{, (#3\arabic{xnumi}#1#4--#5#2#6)}
{ and~(#3\arabic{xnumi}#1#4--#5#2#6)}


% Macros
\newcommand{\fw}[1]{\textit{#1}} % Foreign Word
\newcommand{\tit}[1]{\textit{#1}} % Title of a work
\newcommand{\tsup}[1]{\textsuperscript{#1}} % Superscript
\newcommand{\markyellow}[1]{\colorbox{yellow}{#1}} % Yellow highlighter
\newcommand{\ques}{\textsuperscript{?}} % raised question mark
\newcommand{\zwsp}{\mbox{​}} % Zero-width space (ZWSP)

\newcommand{\ayr}[1]{\zwsp\smash{{\Tagati #1}}} % Plain Ayeri orthography
\newcommand{\rayr}[2]{\zwsp\smash{{\Tagati #1}} \emph{#2}} % Ayeri orthography + *r*omanization
\newcommand{\tayr}[2]{#1 `#2'} % Romanization + *t*ranslation
\newcommand{\xayr}[3]{\zwsp\smash{\Tagati #1} \emph{#2} `#3'} % Ayeri orthography + romanization + translation

\usepackage{suffix}
\WithSuffix\newcommand{\ayr}*[1]{{\Tagati #1}} % Plain Ayeri orthography
\WithSuffix\newcommand{\rayr}*[2]{{\Tagati #1} \emph{#2}} % Ayeri orthography + *r*omanization
\WithSuffix\newcommand{\xayr}*[3]{{\Tagati #1} \emph{#2} `#3'} % Ayeri orthography + romanization + translation

\newenvironment{ayeri}{
	%\hyphenpenalty=10000
	%\hbadness=10000
	\doublespacing
	\begin{multicols}{2}
	\Tagati
}{
	\end{multicols} \par
}

\newenvironment{mytitle}{
	\hfill
	\begin{minipage}{0.667\textwidth}
	\vspace{\baselineskip}
	\begin{center}
		\Large
		\sffamily\bfseries
		\makeatletter
}{
		\makeatother
	\end{center}
	\vspace{1em}
	\end{minipage}
	\hfill
}

%% BIBLIOGRAPHY DATABASE %%%%%%%%%%%%%%%%%%%%%%%%%%%%%%%%%%%%%%%%%%%%%%%%%%%%%%%

% \begin{filecontents*}{\jobname.bib}
% @book{blah,
%     title = {A Grammar of {Blah}},
%     author = {Alfred E. Neuman and John X. Doe},
%     publisher = {Maximegalon UP},
%     location = {Maximegalon},
%     date = {1972},
%     series = {Reference Grammars of Inexistent Languages},
%     number = {4},
%     pages = {123--145},
% }
% \end{filecontents*}

% \addbibresource{\jobname.bib}

%% END OF PREAMBLE %%%%%%%%%%%%%%%%%%%%%%%%%%%%%%%%%%%%%%%%%%%%%%%%%%%%%%%%%%%%

\begin{document}

%% MAIN PART %%%%%%%%%%%%%%%%%%%%%%%%%%%%%%%%%%%%%%%%%%%%%%%%%%%%%%%%%%%%%%%%%%

\begin{mytitle}
	\@title: \@subtitle
\end{mytitle}

\begin{quote}
\begin{minipage}[t]{.5\linewidth}
\begin{verse}
\renewcommand*{\vrightskip}{-2em}
Karonya tasang-vā,\\
tasangya makārya ---\\
Sā tila-\\
yang niva.\\
Yamanreng sinyaley\\
perisānena nā?\\!

Nasyoyya depāng-nama.\\
Ya nasyyang iting vana.\\
Sā tila-\\
yang niva\\
vana dimayjyam. (tinkānreng)\\
Ya nasyyang vās lito (tinkānreng)\\
lito na Mavay, (tinkānreng)\\
lesayang.\\
Ang sagonya enya māy, (tinkānreng)\\
ang sungya mimānley. (tinkānreng)\\
Nay eda-(tinkānreng)\\
da-nā.\\!

Ang tavay kondanley\\
limayena rina\\
nay kimbisanas\\
inunyena kora,\\
inunyena kora,\\
inunyena kora.\\!

Māy, yang ---\\
Ang grenay avanya,\\
grenay avanya nay\\
nimpyang.\\
Yang ---\\
Ang grenay avanya\\
grenay avanya nay\\
nimpyang.\\!
\end{verse}
\end{minipage}
~
\begin{minipage}[t]{.5\linewidth}
\Tagati % \smaller
\begin{verse}
kronFy tsNF/vaa\\
tsNFy mkaarFy\\
saa til—\\
—yNF niv.\\
ymnFreNF sinFylej\\
perisaanen naa?\\!

nsYojy depaaNF/nm.\\
y nsYFyNF ItiNF vn.\\
saa til—\\
—yNF niv\\
vn dimjyeymF. — \textit{tiMkaanFreNF}\\
y nsYyNF vaasF lito — \textit{tiMkaanFreNF}\\
lito n mvj — \textit{tiMkaanFreNF}\\
lesyNF.\\
ANF sgonFy EnFy maaj — \textit{tiMkaanFreNF}\\
ANF suNFy mimaanFlej. — \textit{tiMkaanFreNF}\\
nj Ed/ — \textit{tiMkaanFreNF}\\
d/naa.\\!

ANF tvj koMdnFlej\\
limyen rin\\
nj kiMbisnsF\\
InunFyen kor\\
InunFyen kor\\
InunFyen kor\\!

maaj, yNF\\
Ang gFrenj AvnFy\\
gFrenj AvnFy nj\\
niMpFyNF.\\
yNF\\
Ang gFrenj AvanFy\\
gFrenj AvnFy nj\\
niMpFyNF.\\!
\end{verse}
\end{minipage}
\end{quote}

\section{Analysis of the translation}

\begin{exe}
\ex \begin{xlist}
	\ex \label{ex:1a}
		\gll Karonya tasang-vā, tasangya makārya --- \\
			ocean-\Loc{} deep=\Sup{} abyss-\Loc{} black \\
		\trans `In the deepest ocean, in the black abyss ---'

	\ex \label{ex:1b}
		\gll Sā tilayang niva.\normalfont\footnotemark{} \\
			\CauT{}= change=\Fsg.\Aarg{} gaze \\
		\trans `Your gaze makes me change.'
		\footnotetext{\xayr{niv}{niva}{eye}, extended here to mean `gaze'.}

	\ex \label{ex:1c}
		\gll Yamanreng sinyaley perisānena nā? \\
			reason-\Aarg.\Inan{} which-\Parg.\Inan{} hesitation-\Gen{}
			\Fsg.\Gen{} \\
		\trans `What's the reason for my hesitation?'
\end{xlist}
\end{exe}

\begin{exe}
\ex \begin{xlist}
	\ex \label{ex:2a}
		\gll Nasyoyya depāng-nama.\normalfont\footnotemark{} \\
			follow-\Neg-\Tsg.\M{} fool-\Aarg=only \\
		\trans `Only a fool doesn't follow.'
		\footnotetext{\xayr{depaaNF}{depāng}{fool}, contraction of
			\rayr{depNNF}{depangang} to make the line fit and because
			\emph{-angang} is awkward.}

	\ex \label{ex:2b}
		\gll Ya nasyyang iting vana. \\
			\LocT= follow=\Fsg.\Aarg{} path-\Top{} \Second.\Gen{} \\
		\trans `Your path, I'm following it.'

	\ex \label{ex:2c}
		\gll Sā tilayang niva vana dimayjyam. \\
			\CauT= change=\Fsg.\Aarg{} gaze-\Top{} \Second.\Gen{}
			phantom-\Pl-\Dat{} \\
		\trans `Your gaze makes me turn into phantoms.'

	\ex \label{ex:2d}
		\gll Ya nasyyang vās lito na Mavay, lesayang. \\
			\LocT= follow-\Fsg.\Aarg{} \Second.\Parg{} edge-\Top{} \Gen= World
			fall=\Fsg.\Aarg{} \\
		\trans `To the edge of the Earth is where I'll follow you, and fall.'

	\ex \label{ex:2e}
		\gll Ang sagonya enya māy, ang sungya mimānley. \\
			\AgtT= quit-\Tsg.\M{} everyone-\Top{} \Ints{} \AgtT= find-\Tsg.\M{}
			opportunity-\Parg.\Inan{} \\
		\trans `Yeah, everybody quits, if they find the opportunity.'

	\ex \label{ex:2f}
		\gll Nay eda-tinkānreng da-nā. \\
			and this=opening-\Aarg.\Inan{} one=\Fsg.\Gen{} \\
		\trans `And this opening is mine.'
\end{xlist}
\end{exe}

\begin{exe}
\ex \label{ex:3}
	\gll Ang tavay kondanley limayena rina\normalfont\footnotemark{}
		nay kimbisanas inunyena kora. \\
		\AgtT= become=\Fsg.\Top{} food-\Parg.\Inan{} worm-\Pl-\Gen{} slithery
		and prey-\Parg{} fish-\Pl-\Gen{} rare \\
	\trans I become the food of slithery worms and the prey of rare fishes. \\
	\footnotetext{\xayr{rin}{rina}{slippery}, extended here to mean `slithery'.}
\end{exe}

\begin{exe}
\ex \begin{xlist}
	\ex \label{ex:4a}
		\gll Māy, yang --- \\
			well \Fsg.\Aarg{} \\
		\trans `Well, I ---'

	\ex \label{ex:4b}
		\gll Ang grenay avanya nay nimpyang. \\
			\AgtT= reach=\Fsg.\Top{} ground-\Loc{} and run=\Fsg.\Aarg{} \\
		\trans `I reach the bottom and run.'
\end{xlist}
\end{exe}

\section{Conclusion}


%% BIBLIOGRAPHY %%%%%%%%%%%%%%%%%%%%%%%%%%%%%%%%%%%%%%%%%%%%%%%%%%%%%%%%%%%%%%%

% \vfill
% \pagebreak

\begingroup\multicolsep=0pt
\printglossary[
	style=threecolumn,
	type=leipzig,
]
\endgroup

%\nocite{*} % returns all entries from the bibliography database
% \printbibliography[heading=bibintoc]

\end{document}
