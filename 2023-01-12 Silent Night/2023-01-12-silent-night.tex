\documentclass[12pt,paper=a4]{scrartcl}

% Author, Title, Subtitle etc.
\author{Carsten Becker}
\title{Silent Night}
\subtitle{A Christmas Carol in Ayeri}
\date{\today} % Format: YYYY/MM/DD (rev. N, YYYY/MM/DD)

% Provide running author and date
\makeatletter
\let\runauthor\@author
\let\rundate\@date
\makeatother

% Handle language and quotation marks
\usepackage{polyglossia}
\setdefaultlanguage{english}
\setotherlanguage{german}
\usepackage{csquotes} % Put quotations in \enquote{}!
\SetBlockEnvironment{quotation}
\renewcommand*{\mkccitation}[1]{ (#1)}
\let\q\textquote

% Quotation style for word definitions
\DeclareQuoteStyle{wdef}
	{\textquoteleft}{\textquoteright}
	{\textquotedblleft}{\textquotedblright}
\newcommand{\wdef}[1]{{\setquotestyle{wdef}\enquote{#1}}}

% Set all margins to 2.54 cm
\usepackage[margin=1in]{geometry}
\widowpenalty10000 % Avoid widows like the plague!
\clubpenalty10000 % Avoid orphans like the plage, too!

% Extended formatting of lists
\usepackage{enumitem}
\newlist{glossdefs}{itemize}{1}
\setlist[glossdefs]{nosep, leftmargin=3em, labelwidth=2.5em, align=left}
\setlist[itemize]{noitemsep}

% Make multiple columns available in single-column document
\usepackage{multicol}

% Make text colors and color names available
\usepackage{xcolor}

% Load font stuff for XeTeX
\usepackage{fontspec}

% Set main fonts
\usepackage[config=mt-Junicode]{microtype}

\newfontfamily{\Tagati}[
	Renderer=Graphite,
	Scale=1.0,
	BoldFont={* Italic},
	HyphenChar=·,
]{Tagati Book G}

% \setmainfont[
% 	Ligatures=TeX,
% 	Numbers=Lowercase,
% ]{Junicode}

\setmainfont[
	Ligatures=TeX,
	Numbers={OldStyle,Proportional},
	BoldFont=*-Bold,
	ItalicFont=*-Italic,
	BoldItalicFont=*-BoldItalic,
]{Junicode Two Beta}

\setsansfont[
	Ligatures=TeX,
	Numbers=Lowercase,
	Scale=MatchUppercase,
	BoldFont={Open Sans Condensed Bold},
]{Open Sans Condensed Light}

% % Load BibLaTeX (using Biber), configure citation styles
% \usepackage[
% 	authordate-trad,
% 	backend=biber,
% 	safeinputenc,
% ]{biblatex-chicago}

% % To make \textcite look like "Doe (2014: 213)"
% \renewcommand*{\postnotedelim}{\addcolon\addspace}
% \DeclareFieldFormat{postnote}{#1}
% \DeclareFieldFormat{multipostnote}{#1}

% % Enable generating files from this .tex file (for the bibliography) 
% \usepackage{filecontents}

% Date etc.
\usepackage[
	datesep={/},
]{datetime2}

% Verse
\usepackage{verse}

% Ability to include graphics and dealing with footnotes in descriptions
\usepackage{graphicx}
\usepackage[font={small,sf},labelfont={small,sf},format=plain]{caption}
\usepackage{subcaption}
\usepackage{wrapfig}
\setlength{\columnsep}{2\baselineskip}

% General headers and footers
\usepackage{fancyhdr}
\pagestyle{fancy}

\fancyhead[L]{} % empty
\fancyhead[C]{} % empty
\fancyhead[R]{\thepage}

\fancyfoot[L]{} % empty
\fancyfoot[C]{} % empty
\fancyfoot[R]{} % empty

\renewcommand{\headrulewidth}{0pt}
\renewcommand{\footrulewidth}{0pt}

% First page headers and footers are different
\fancypagestyle{firstpage}{
	\fancyhead[L]{\sffamily \footnotesize \textbf{Benung. The Ayeri Language Resource}}
	\fancyhead[C]{} % empty
	\fancyhead[R]{\sffamily \footnotesize \runauthor{} · \rundate{}}
	
	\fancyfoot[L]{} % empty
	\fancyfoot[C]{} % empty
	\fancyfoot[R]{\sffamily \footnotesize 
	\href{https://ayeri.de}{https://ayeri.de} · 
	\href{https://github.com/carbeck/benung-pdfs}{https://github.com/carbeck/benung-pdfs} · 
	\href{https://creativecommons.org/licenses/by-sa/4.0/}{CC~BY-SA~4.0}
	}
	
	\renewcommand{\headrulewidth}{0.5pt}
	\setlength\footskip{0.5in}
}

\usepackage{ifthen}
\ifthenelse{\value{page}=1}{\thispagestyle{firstpage}{\pagestyle{fancy}}}

% Line spacing
\usepackage{setspace}
\onehalfspacing

% Avoid pagebreaks right after sections and subsections
\usepackage{needspace}
\usepackage{etoolbox}
\preto\section{\needspace{6\baselineskip}}
\preto\subsection{\needspace{6\baselineskip}}

% Things for tables
\usepackage{tabularx}
% \usepackage{booktabs}
% \usepackage{rotating}

% Formatting of table of glossing abbreviations from Leipzig package manual
\usepackage[
	acronym,
	nomain,
	nonumberlist,
	nopostdot,
	toc=true,
	xindy={codepage=utf8}, % language=english,
]{glossaries}
\usepackage{glossary-inline}

\newglossarystyle{threecolumn}{%
	\renewenvironment{theglossary}{%
		\begin{multicols}{3}
		\begin{glossdefs}%
	}{%
		\end{glossdefs}%
		\end{multicols}%
	}%
	\renewcommand*{\glossaryheader}{}%
	\renewcommand*{\glsgroupheading}[1]{}%
	\newcommand*{\glossaryentryfield}[5]{%
		\item[\glsentryitem{##1}\glstarget{##1}{##2}]
		% \makefirstuc{##3}\glspostdescription{}
		##3\glspostdescription{}
	}%
	\renewcommand*{\glsgroupskip}{}%
}%

% Formatting of glosses
\usepackage{langsci-gb4e}
\usepackage[glosses]{leipzig}
\renewcommand{\eachwordone}{\itshape}
\renewcommand{\eachwordtwo}{\rule[-.5\baselineskip]{0pt}{0pt}}

\newleipzig{AgtT}{at}{agent topic}
\newleipzig{PatT}{pt}{patient topic}
\newleipzig{DatT}{datt}{dative topic}
\newleipzig{GenT}{gent}{genitive topic}
\newleipzig{LocT}{loct}{locative topic}
\newleipzig{InsT}{inst}{instrumental topic}
\newleipzig{CauT}{caut}{causative topic}
\newleipzig{An}{an}{animate}
\newleipzig{Inan}{inan}{inanimate}
\newleipzig{Hab}{hab}{habitative}
\newleipzig{Sup}{sup}{superlative}
\newleipzig{St}{st}{strong}
\newleipzig{Ayr}{ayr}{Ayeri}

\makeglossaries

% Nicer footnotes
\usepackage[bottom,hang,norule]{footmisc}
\setlength{\footnotesep}{0.75\baselineskip}

% Smaller font in block quotes
\usepackage{relsize}
\AtBeginEnvironment{quote}{\noindent\smaller}
\AtBeginEnvironment{quotation}{\smaller}

% Clickable links in footnotes, TOC, etc.
\usepackage[
% 	xetex,
	bookmarks=true,
	colorlinks=false,
	linktoc=section,
	hidelinks,
	pdfusetitle,
]{hyperref}

% We want URLs to be italic and with regular uppercase numerals
\renewcommand{\UrlFont}{%
	\normalfont%
	\itshape%
	\addfontfeature{RawFeature=-onum}%
}

% In-text references
% cf. https://tex.stackexchange.com/a/139051
% Since German plural formation is not as regular as in English (-e, -en for
% Beispiel), we will define the label as empty
\usepackage[sort&compress,noabbrev]{cleveref}
\newcommand{\crefrangeconjunction}{--}
\crefname{xnumi}{}{}
\creflabelformat{xnumi}{(#2#1#3)}
\crefname{xnumii}{}{}
\creflabelformat{xnumii}{(#2#1#3)}
\crefname{xnumiii}{}{}
\creflabelformat{xnumiii}{(#2#1#3)}
\crefname{xnumiv}{}{}
\creflabelformat{xnumiv}{(#2#1#3)}
\crefrangeformat{xnumi}{(#3#1#4)--(#5#2#6)}
\crefrangeformat{xnumii}{(#3#1#4--#5\crefstripprefix{#1}{#2}#6)}
\crefrangemultiformat{xnumii}{(#3\arabic{xnumi}#1#4--#5#2#6)}
{ and~(#3\arabic{xnumi}#1#4--#5#2#6)}{, (#3\arabic{xnumi}#1#4--#5#2#6)}
{ and~(#3\arabic{xnumi}#1#4--#5#2#6)}


% Macros
\newcommand{\fw}[1]{\textit{#1}} % Foreign Word
\newcommand{\tit}[1]{\textit{#1}} % Title of a work
\newcommand{\tsup}[1]{\textsuperscript{#1}} % Superscript
\newcommand{\markyellow}[1]{\colorbox{yellow}{#1}} % Yellow highlighter
\newcommand{\ques}{\fakesuperscript{?}} % raised question mark
\newcommand{\zwsp}{\mbox{​}} % Zero-width space (ZWSP)

\newcommand{\ayr}[1]{\zwsp\smash{{\Tagati #1}}} % Plain Ayeri orthography
\newcommand{\rayr}[2]{\zwsp\smash{{\Tagati #1}} \emph{#2}} % Ayeri orthography + *r*omanization
\newcommand{\tayr}[2]{#1 `#2'} % Romanization + *t*ranslation
\newcommand{\xayr}[3]{\zwsp\smash{\Tagati #1} \emph{#2} `#3'} % Ayeri orthography + romanization + translation

\usepackage{suffix}
\WithSuffix\newcommand{\ayr}*[1]{{\Tagati #1}} % Plain Ayeri orthography
\WithSuffix\newcommand{\rayr}*[2]{{\Tagati #1} \emph{#2}} % Ayeri orthography + *r*omanization
\WithSuffix\newcommand{\xayr}*[3]{{\Tagati #1} \emph{#2} `#3'} % Ayeri orthography + romanization + translation

\newenvironment{ayeri}{
	%\hyphenpenalty=10000
	%\hbadness=10000
	\doublespacing
	\begin{multicols}{2}
	\Tagati
}{
	\end{multicols} \par
}

\newenvironment{mytitle}{
	\hfill
	\begin{minipage}{0.667\textwidth}
	\vspace{\baselineskip}
	\begin{center}
		\Large
		\sffamily\bfseries
		\makeatletter
}{
		\makeatother
	\end{center}
	\vspace{1em}
	\end{minipage}
	\hfill
}

%% BIBLIOGRAPHY DATABASE %%%%%%%%%%%%%%%%%%%%%%%%%%%%%%%%%%%%%%%%%%%%%%%%%%%%%%%

% \begin{filecontents*}{\jobname.bib}
% @book{blah,
%     title = {A Grammar of {Blah}},
%     author = {Alfred E. Neuman and John X. Doe},
%     publisher = {Maximegalon UP},
%     location = {Maximegalon},
%     date = {1972},
%     series = {Reference Grammars of Inexistent Languages},
%     number = {4},
%     pages = {123--145},
% }
% \end{filecontents*}

% \addbibresource{\jobname.bib}

%% END OF PREAMBLE %%%%%%%%%%%%%%%%%%%%%%%%%%%%%%%%%%%%%%%%%%%%%%%%%%%%%%%%%%%%%

\begin{document}

%% MAIN PART %%%%%%%%%%%%%%%%%%%%%%%%%%%%%%%%%%%%%%%%%%%%%%%%%%%%%%%%%%%%%%%%%%%

\begin{mytitle}
	\@title: \@subtitle
\end{mytitle}

In December 2022 I posted on my Mastodon account a photo from the Berlin State
Library's Unter den Linden branch featuring a pinboard on which were posted
festive tags with Christmas greetings in a slew of languages spoken by library
patrons.%
%
	\footnote{See \url%
		{https://mastodon.online/@chrpistorius/109522620399297747}%
	.}
%
User Scott Hühnerkrisp
% %
% 	\footnote{He goes by \href%
% 		{https://chaos.social/@ScottHuehnerkrisp}%
% 		{\itshape @ScottHuehnerkrisp@chaos.social}%
% 	.}
% %
replied that he missed Ayeri among the pinned tags,%
%
	\footnote{Sadly, there were none left when I returned a few days later.}
%
and also wondered whether there already exists a translation of \tit{Stille
Nacht} into Ayeri. Even though it's past Christmas now and this year's is still
a ways off, I wanted to make good on this challenge. This is \rayr{sirutj
terFnu kluj}{Sirutay ternu kaluy}, my attempt to translate the Austrian
Christmas carol \tit{Stille Nacht, heilige Nacht}---to English speakers known
as \tit{Silent Night}---into Ayeri.

\section{The German Text}

The text of the carol in German as it is commonly sung today---along with a
more or less literal English translation---goes as follows. This serves as the
base for the Ayeri version since I speak German natively and my inner
philologist is skeptical of second-hand translations.

\begin{quote}
\begin{minipage}{.5\linewidth}
\begin{verse}
\renewcommand*{\vrightskip}{-2em}
\verselinenumbersleft
\poemlines{5}
\indentpattern{000011}
\begin{patverse}
Stille Nacht, heilige Nacht!\\
Alles schläft, einsam wacht\\
nur das traute, hochheilige Paar.\\
Holder Knabe im lockigen Haar,\\
schlaf in seliger Ruh,\\
schlaf in seliger Ruh.\\!
\end{patverse}

\begin{patverse}
Stille Nacht, heilige Nacht!\\
Gottes Sohn, o wie lacht\\
Lieb aus deinem göttlichen Mund,\\
da uns schlägt die rettende Stund,\\
Christ, in deiner Geburt,\\
Christ, in deiner Geburt.\\!
\end{patverse}

\begin{patverse}
Stille Nacht, heilige Nacht!\\
Hirten erst kundgemacht,\\
durch der Engel Halleluja\\
tönt es laut von fern und nah:\\
Christ, der Retter, ist da,\\
Christ, der Retter, ist da!\\!
\end{patverse}
\end{verse}
\end{minipage}
~
\begin{minipage}{.5\linewidth}
\itshape
\begin{verse}
\poemlines{0}
\indentpattern{000011}
\begin{patverse}
Silent night, holy night!\\
All is asleep, lonely wakes\\
only the intimate, most holy couple.\\
Lovely boy with curly hair\\
sleep in blissful rest,\\
sleep in blissful rest.\\!
\end{patverse}

\begin{patverse}
Silent night, holy night!\\
Son of God, oh how is laughing\\
love from your divine mouth\\
as the hour of salvation tolls for us,\\
Christ, with your birth,\\
Christ, with your birth.\\!
\end{patverse}

\begin{patverse}
Silent night, holy night!\\
First announced to shepherds,\\
per the angels' hallelujah\\
loudly it sounds from near and far:\\
Christ, the Savior, is here,\\
Christ, the Savior, is here!\\!
\end{patverse}
\end{verse}
\end{minipage}
\end{quote}

\section{The Ayeri translation}
The text below is my translation of the above into Ayeri. Since this is in
verse, the translation ought to fit the original's meter as closely as
possible, which is always a challenge. However, with songs the melody
fortunately creates some leeway. The German text sometimes stretches single
syllables over two notes (\fw{Sti-il-le Nacht}), so the number of syllables per
line slightly varies across stanzas. In translation one can make use of such
built-in tolerance as well. A greater challenge is that Ayeri tends to have
disyllabic roots to which are attached grammatical suffixes creating even more
syllables. Not exceeding the overall syllable count while staying true to
the content of the text requires a little creativity at times. Bonus points if
you can fit the foreign words' natural stress pattern to what the melody
dictates.

\begin{quote}
\begin{minipage}[t]{.5\linewidth}
\begin{verse}
\renewcommand*{\vrightskip}{-2em}
\verselinenumbersleft
\poemlines{5}
\indentpattern{000011}
\begin{patverse}
Ah sirutay ternu kaluy!\\
Torya enyāng, nārya-nama\\
sānang sitang-setim ternu-vā.\\
Yanang val' mitrangeri gura,\\
toru nibānya aray,\\
toru nibānya aray.\\!
\end{patverse}

\begin{patverse}
Ah sirutay ternu kaluy!\\
Yampangal, sā d'-apayo\\
cān bantāng van'. Eng yomara\\
pidim madānena nana,\\
Yesu, vesangeri vana,\\
Yesu, vesangeri van'.\\!
\end{patverse}

\begin{patverse}
Ah sirutay ternu kaluy!\\
Ang tangyan nantongye\\
aleluyās kelangyena.\\
Edauyi tangnang baho naynay:\\
Yomaya ang Yesu Madaya,\\
yomaya ang Yesu Maday'!\\!
\end{patverse}
\end{verse}
\end{minipage}
~
\begin{minipage}[t]{.5\linewidth}
\Tagati % \smaller
\begin{verse}
\poemlines{0}
\indentpattern{000011}
\begin{patverse}
AH sirutj terFnu kluj!\\
torFy EnFyaaNF, naarFy/nm\\
saanNF sitNF/setimF terFnu/vaa.\\
ynNF vlF/ mitFrNeri gur,\\
toru nibaanFy Arj,\\
toru nibaanFy Arj.\\!
\end{patverse}

\begin{patverse}
AH sirutj terFnu kluj!\\
yMpNlF, saa dF/Apyo\\
tYaanF bMtaaNF vnF/. ENF yomr\\
pidimF mdaanen nn,\\
yesu, vesNeri vn,\\
yesu, vesNeri vnF/.\\!
\end{patverse}

\begin{patverse}
AH sirutj terFnu kluj!\\
ANF tNFynF nMtoNFye\\
AleluyaasF kelNFyen.\\
Edauyi tNFnNF bho njnj –\\
yomy ANF yesu mdy,\\
yomy ANF yesu mdj/!\\!
\end{patverse}
\end{verse}
\end{minipage}
\end{quote}

\section{Analysis of the translation}

% - FIRST STANZA --------------------------------------------------------------
\subsection{First stanza}

Let us now analyze the translation in tried-and-tested fashion, sentence by
sentence. I will start with the verse that heads every stanza to evoke a
peaceful and solemn atmosphere.

\begin{exe}
\ex \label{ex:ln01}
	\gll Ah sirutay ternu kaluy!\\
	oh night holy silent\\
	\trans \wdef{Oh silent, holy night!}\\
\end{exe}

Here already, concessions had to be made for a more natural stress pattern and
to fit the words. The melody prescribes the stress pattern     |    .
\fw{Nacht} \wdef{night} is conveniently a single-syllable word while the
adjectives \fw{stille} \wdef{silent-\Nom.\Sg.\F.\St{}} and \fw{heilige}
\wdef{holy-\Nom.\Sg.\F.\St{}} each occupy three beats. It is apparent from
\cref{ex:ln01} that \xayr{sirutj}{sirutay}{night} alone already contains three
syllables while the adjectives are each disyllabic. Repetition of
\rayr{sirutj}{sirutay} is thus out of the question. Moreover,
\rayr{sirutj}{sirutay} and \rayr{kluj}{kaluy} both have final-syllable stress
due to diphthongs being heavy in terms of syllable weight. To keep the
translation in line with the original pattern, \rayr{sirutj}{sirutay} and
\rayr{kluj}{kaluy} need to end on the fourth and eighth beat. Thus,
\xayr{terFnu}{ternu}{silent} needs to precede \rayr{kluj}{kaluy} and there is
an unoccupied stressed beat before \rayr{sirutj}{sirutay}. I decided not to
stretch \rayr{sirutj}{sirutay} over all four beats in the same way the German
text stretches the two syllables of \fw{stille} over the first three beats.
Instead, I opted to introduce the vocative particle \xayr{AH}{ah}{oh}, which
fits the function of this line and is appropriately solemn.

Since \rayr{sirutj}{sirutay} is addressed here, I decided not to give it a case
marker, as when addressing people. Besides, this would've increased the
syllable count. The order of the adjectives, even though reversed from the
original, is natural in Ayeri due to its very consistent head-first nature.

The second and third verse in \cref{ex:ln02} present the nativity scene with
Mary and Joseph in intimate togetherness with their newborn child, Jesus
Christ.

\begin{exe}
\ex \label{ex:ln02}
	\gll Torya enyāng, nārya-nama\\
		sleep-\Tsg.\M{} everyone-\Aarg{} but=only\\
\sn \gll sānang sitang-setim ternu-vā.\\
		couple-\Aarg{} self=familiar holy=\Sup{}\\
	\trans \wdef{Everyone is sleeping, except for the intimate, most holy
		couple.}
\end{exe}

These two verses contain the newly coined words \rayr{naarY/nm}{nārya-nama} and
\xayr{sitNF/setimF}{sitang-setim}{intimate}. The former adds the adverbial
quantifier \xayr{/nm}{-nama}{only, just} to the adversative conjunction
\xayr{naarY}{nārya}{but, though, yet, etc.} to express something along the
lines of \wdef{except for}. In German, \fw{außer} \wdef{except for, apart from}
would require a complement in the dative because it's a preposition. In
contrast, \rayr{naary}{nārya} is a conjunction, so I went with the agent case
in parallel to \rayr{torFy EnFyaaNF}{torya enyāng}. Context would then imply
that everyone is sleeping, except the couple isn't asleep. The adjective
\fw{traut} \wdef{intimately familiar with each other} from the German text is
not in current use anymore, however, adding the reflexive particle
\rayr{sitNF/}{sitang-} to \xayr{setimF}{setim}{familiar} seemed a good way to
convey the same meaning, at the expense of conflating reflexivity with
reciprocity.%
%
	\footnote{This also happens with \fw{sich} \wdef{oneself, him/her/itself,
		themselves; each other} in colloquial German, so it doesn't seem too
		far-fetched to me. At the point of writing this, I haven't done
		research on how common it is cross-linguistically for languages to
		conflate the two categories.}

Technically, the order of adjectives in Ayeri is reversed here by using the
head-final order of the German text, but this as well flows better with the
melody. \rayr{/nm}{-nama} untypically receives stress on the final syllable
here because German \fw{wacht} \wdef{wakes} is stressed. However, stress in
Ayeri is not commonly phonemic, so to speakers is would simply sound slightly
off.

Verses 4 to 6 now also introduce the main figure of the nativity scene, the
baby in the manger, Jesus Christ. The boy having curly hair is poetic license,
I suppose.

\begin{exe}
\ex \label{ex:ln04}
	\gll Yanang val' mitrangeri gura,\\
		boy-\Aarg{} lovely hair-\Ins{} curly \\

\sn \gll toru nibānya aray.\\
		sleep-\Imp{} rest-\Loc{} blissful \\
	\trans \wdef{Lovely boy with curly hair, sleep in blissful rest.}
\end{exe}

Another two words had to be coined here, namely \xayr{gur}{gura}{curly} and
\xayr{Arj}{aray}{blissful}. The former is derived from the verb
\xayr{gurF/}{gur-}{turn around, wind} and the latter from the noun
\xayr{Arj}{aray}{bliss}. \rayr{gur}{gura} is also untypically stressed on the
second syllable here. I am not quite sure if the locative is the best choice in
the repeated verse 5 and 6. Alternatively, one might prefer the
instrumental form, \xayr{nibaanF/ri}{nibān'ri}{with rest}.

% - SECOND STANZA -------------------------------------------------------------

\subsection{Second stanza}

Verse 7, the first line of the second stanza, is the same as the first verse,
compare \cref{ex:ln01}. I have omitted it here as well as in the next section
for redundance.

\begin{exe}
\ex \label{ex:ln08}
	\gll Yampangal, sā d'-apayo\\
		Son.God \CauT= thus=laugh-\Tsg.\N{} \\

\sn \gll cān bantāng van'.\\
		love.\Top{} mouth-\Aarg{} \Second.\Gen{} \\
	\trans \wdef{God-son, how love makes your mouth laugh.}
\end{exe}

The German text corresponding to the couplet in \cref{ex:ln08} describes
baby Jesus as literally \wdef{laughing} love from his divine mouth in
connection with the next two verses about the prospect of salvation. I suppose,
the overall image is one of looking at a baby beaming with love and happiness.
Due to limitations in the available number of syllables,
\xayr{pNFr}{pangra}{divine} didn't fit anywhere. Making a compound
\xayr{yMpNlF}{Yampangal}{God-son} instead of literally adapting \fw{Gottes
Sohn} \wdef{son of God} as \xayr{ynF pNlen}{yan pangalena}{son God-\Gen} saves
two syllables.

Verses 10 to 12 in German give the theological reason for baby Jesus' joy: the
birth of Christ heralds human salvation---which is consummated on Easter,
except according to the carol, Jesus' birth is already the key moment. In
ignorance of further theological details and ramifications I shrugged and
translated this more or less literally in \cref{ex:ln09}.

\begin{exe}
\ex \label{ex:ln09}
	\gll Eng yomara pidim madānena nana,\\
		\AgtT.\Inan= be.there-\Tsg.\Inan{} hour.\Top{} rescue-\Gen{}
		\Fpl.\Gen \\
\sn \gll Yesu, vesangeri vana.\\
		Jesus birth-\Ins{} \Second.\Gen{} \\
	\trans \wdef{The hour of our salvation is here, Jesus, by means of your
		birth.}
\end{exe}

The first line above actually already begins in verse 9, creating an enjambment
because it was convenient given the syllable count. What is unfortunately lost
is the logical connection to the previous two verses implied by \fw{da}
\wdef{as} in German, likewise due to meeting the syllable count. Furthermore, I
chose to use \xayr{yesu}{Yesu}{Jesus} here instead of trying to adapt
\fw{Christ} in some way---whether borrowing it as \rayr{kFrisFto}{Kristo} or
attempting to loan-translate Greek \fw{Khrīstós} \wdef{the anointed one}. The
stress pattern dictated by the melody (       ) goes completely counter
to the natural pattern of \rayr{pidimF mdaanen nn}{pidim madānena nana} (  
    ).

% - THIRD STANZA --------------------------------------------------------------

\subsection{Third stanza}

The third stanza is the most challenging one yet because in German, \fw{durch
der Engel Halleluja} \wdef{per the angles' hallelujah} (verse 15) can be read
as referring to both the previous and the next line. Moreover, a rather literal
translation is too long. Instead of the shepherds being announced to as in the
German \fw{Hirten erst kundgemacht} \wdef{first announced to shepherds}, in
\cref{ex:ln14}, I rephrased the text rather neutrally as the shepherds hearing
the angels' jubilation.

\begin{exe}
\ex \label{ex:ln14}
	\gll Ang tangyan nantongye\\
		\AgtT= hear-\Tpl.\M{} shepherd-\Pl.\Top{} \\

\sn \gll aleluyās kelangyena.\\
		halelujah-\Parg{} angel-\Pl-\Gen{} \\
	\trans \wdef{The shepherds heard the angels' hallelujah.}
\end{exe}

Anyone familiar with the biblical story will know the context anyway, so I only
slightly regretted also dropping \fw{first} for exceeding the syllable count. I
suppose \xayr{mennFy}{menanya}{(at) first} (one-\Nmlz-\Loc) could be a viable
translation due to the peculiarity of how ordinals are formed in Ayeri as
nouns. The corresponding dative form,
\rayr{mennFymF}{menanyam}, means \wdef{once} as a multiplicative.
Alternatively, one might use \xayr{ku/mennFyeANF}{ku-menanjang}{as the first
ones} (like=one-\Nmlz-\Pl-\Aarg) in reference to the shepherds, but of course,
this is far too long as well.

Translating Greek \fw{ángelos} \wdef{messenger}, from which English gets
\fw{angel}, literally as \rayr{niny}{ninaya} would've yielded the rather
tongue-twisty form \rayr{ninyyen}{ninayayena} in Ayeri's iotaphile fashion. It
would again have been too long. Thus, I decided to extend the meaning of
\xayr{kelNF}{kelang}{chain, garland} with \wdef{angel}. The common denominator
is the corresponding verb \xayr{kelNF/}{kelang-}{connect}, which otherwise
gives \xayr{kelNnF}{kelangan}{connection}, and possibly
\xayr{kelNy}{kelangaya}{connector}---likewise too long with \rayr{/yen}{-yena}
added for genitive plural.

The remaining verses 16 to 18 in \cref{ex:ln16} differ from the German text as
well. I tried to compensate for the missing temporal aspect in the previous two
lines by extending German's \fw{tönt es laut von fern und nah} \wdef{loudly it
sounds from near and far} with \q{us now, too} as recipients of the message.

\begin{exe}
\ex \label{ex:ln16}
	\gll Edauyi tangnang baho naynay:\\
		now hear=\Fpl.\Aarg{} loudly too \\

\sn \gll yomaya ang Yesu Madaya!\\
		be.there-\Tsg.\M{} \Aarg= Jesus Savior \\
	\trans \wdef{Now we hear loudly as well: Jesus the Savior is here.}
\end{exe}

The difference in translation omits the \fw{fern und nah} \wdef{near and far}
aspect completely, again because there was no space to fit anything along those
lines. Replacing \xayr{njnj}{naynay}{as well, too} with a wrongly stressed
\xayr{ynenF}{yanen}{everywhere} seemed comparatively more awkward to
me.\bigskip

Now off to grab the chords or sheet music to the song and practice playing and
singing it, and maybe have an amateurish recording ready by this year's
Christmas.

%% BIBLIOGRAPHY %%%%%%%%%%%%%%%%%%%%%%%%%%%%%%%%%%%%%%%%%%%%%%%%%%%%%%%%%%%%%%%

% \vfill
% \pagebreak

\begingroup\multicolsep=0pt
\printglossary[
	style=threecolumn,
	type=leipzig,
]
\endgroup

%\nocite{*} % returns all entries from the bibliography database
% \printbibliography[heading=bibintoc]

\end{document}
