\documentclass[12pt,paper=a4]{scrartcl}

% Author, Title, Subtitle etc.
\author{Carsten Becker}
\title{Silent Night}
\subtitle{A Christmas Carol in Ayeri}
\date{\today} % Format: YYYY/MM/DD (rev. N, YYYY/MM/DD)

% Provide running author and date
\makeatletter
\let\runauthor\@author
\let\rundate\@date
\makeatother

% Handle language and quotation marks
\usepackage{polyglossia}
\setdefaultlanguage{english}
\setotherlanguage{german}
\usepackage{csquotes} % Put quotations in \enquote{}!
\SetBlockEnvironment{quotation}
\renewcommand*{\mkccitation}[1]{ (#1)}

% Set all margins to 2.54 cm
\usepackage[margin=1in]{geometry}
\widowpenalty10000 % Avoid widows like the plague!
\clubpenalty10000 % Avoid orphans like the plage, too!

% Extended formatting of lists
\usepackage{enumitem}
\newlist{glossdefs}{itemize}{1}
\setlist[glossdefs]{nosep, leftmargin=3em, labelwidth=2.5em, align=left}
\setlist[itemize]{noitemsep}

% Make multiple columns available in single-column document
\usepackage{multicol}

% Make text colors and color names available
\usepackage{xcolor}

% Load font stuff for XeTeX
\usepackage{fontspec}

% Set main fonts
\usepackage[config=mt-Junicode]{microtype}

\newfontfamily{\Tagati}[
	Renderer=Graphite,
	Scale=1.1,
	BoldFont={* Italic},
	HyphenChar=·,
]{Tagati Book G}

\setmainfont[
	Ligatures=TeX,
	Numbers=Lowercase,
]{Junicode}

\setsansfont[
	Ligatures=TeX,
	Numbers=Lowercase,
	Scale=MatchUppercase,
	BoldFont={Open Sans Condensed Bold},
]{Open Sans Condensed Light}

% Load BibLaTeX (using Biber), configure citation styles
\usepackage[
	authordate-trad,
	backend=biber,
	safeinputenc,
]{biblatex-chicago}

% To make \textcite look like "Doe (2014: 213)"
\renewcommand*{\postnotedelim}{\addcolon\addspace}
\DeclareFieldFormat{postnote}{#1}
\DeclareFieldFormat{multipostnote}{#1}

% Enable generating files from this .tex file (for the bibliography) 
\usepackage{filecontents}

% Date etc.
\usepackage[
	datesep={/},
]{datetime2}

% Verse
\usepackage{verse}

% Clickable links in footnotes, TOC, etc.
\usepackage[
% 	xetex,
	bookmarks=true,
	colorlinks=false,
	linktoc=section,
	hidelinks,
	pdfusetitle,
]{hyperref}

% We want URLs to be italic and with regular uppercase numerals
\renewcommand{\UrlFont}{%
	\normalfont%
	\itshape%
	\addfontfeature{RawFeature=-onum}%
}

% Ability to include graphics and dealing with footnotes in descriptions
\usepackage{graphicx}
\usepackage[font={small,sf},labelfont={small,sf},format=plain]{caption}
\usepackage{subcaption}
\usepackage{wrapfig}
\setlength{\columnsep}{2\baselineskip}

% General headers and footers
\usepackage{fancyhdr}
\pagestyle{fancy}

\fancyhead[L]{} % empty
\fancyhead[C]{} % empty
\fancyhead[R]{\thepage}

\fancyfoot[L]{} % empty
\fancyfoot[C]{} % empty
\fancyfoot[R]{} % empty

\renewcommand{\headrulewidth}{0pt}
\renewcommand{\footrulewidth}{0pt}

% First page headers and footers are different
\fancypagestyle{firstpage}{
	\fancyhead[L]{\sffamily \footnotesize \textbf{Benung. The Ayeri Language Resource}}
	\fancyhead[C]{} % empty
	\fancyhead[R]{\sffamily \footnotesize \runauthor{} · \rundate{}}
	
	\fancyfoot[L]{} % empty
	\fancyfoot[C]{} % empty
	\fancyfoot[R]{\sffamily \footnotesize 
	\href{https://ayeri.de}{https://ayeri.de} · 
	\href{https://github.com/carbeck/benung-pdfs}{https://github.com/carbeck/benung-pdfs} · 
	\href{https://creativecommons.org/licenses/by-sa/4.0/}{CC~BY-SA~4.0}
	}
	
	\renewcommand{\headrulewidth}{0.5pt}
	\setlength\footskip{0.5in}
}

\usepackage{ifthen}
\ifthenelse{\value{page}=1}{\thispagestyle{firstpage}{\pagestyle{fancy}}}

% Line spacing
\usepackage{setspace}
\onehalfspacing

% Avoid pagebreaks right after sections and subsections
\usepackage{needspace}
\usepackage{etoolbox}
\preto\section{\needspace{6\baselineskip}}
\preto\subsection{\needspace{6\baselineskip}}

% Things for tables
\usepackage{tabularx}
% \usepackage{booktabs}
% \usepackage{rotating}

% Formatting of table of glossing abbreviations from Leipzig package manual
\usepackage[acronym,nomain,nonumberlist,nopostdot]{glossaries}
\usepackage{glossary-inline}%

\newglossarystyle{mysuper}{%
	\glossarystyle{super}% based on super
	\renewenvironment{theglossary}{%
		\begin{glossdefs}%
	}{%
		\end{glossdefs}%
	}%
	\renewcommand*{\glossaryheader}{}%
	\renewcommand*{\glsgroupheading}[1]{}%
	\renewcommand*{\glossaryentryfield}[5]{%
		\item[\glsentryitem{##1}\glstarget{##1}{##2}]
		\makefirstuc{##3}\glspostdescription{}
	}%
	\renewcommand*{\glsgroupskip}{}%
}%

% Formatting of glosses
\usepackage{langsci-gb4e}
\usepackage{leipzig}

\newleipzig{AgtT}{at}{agent topic}
\newleipzig{PatT}{pt}{patient topic}
\newleipzig{DatT}{datt}{dative topic}
\newleipzig{GenT}{gent}{genitive topic}
\newleipzig{LocT}{loct}{locative topic}
\newleipzig{InsT}{inst}{instrumental topic}
\newleipzig{CauT}{caut}{causative topic}
\newleipzig{An}{an}{animate}
\newleipzig{Inan}{inan}{inanimate}
\newleipzig{Hab}{hab}{habitative}
\newleipzig{Ayr}{ayr}{Ayeri}

\makeglossaries

% Nicer footnotes
\usepackage[bottom,hang,norule]{footmisc}
\setlength{\footnotesep}{0.75\baselineskip}

% Smaller font in block quotes
\usepackage{relsize}
\AtBeginEnvironment{quote}{\noindent\smaller}
\AtBeginEnvironment{quotation}{\smaller}

% Macros
\newcommand{\fw}[1]{\textit{#1}} % Foreign Word
\newcommand{\tit}[1]{\textit{#1}} % Title of a work
\newcommand{\q}[1]{\enquote{#1}} % Context-aware quotation
\newcommand{\qq}[1]{\enquote*{#1}} % Explicit sublevel quotation
\newcommand{\tsup}[1]{\textsuperscript{#1}} % Superscript
\newcommand{\markyellow}[1]{\colorbox{yellow}{#1}} % Yellow highlighter
\newcommand{\ques}{\fakesuperscript{?}} % raised question mark
\newcommand{\zwsp}{\mbox{​}} % Zero-width space (ZWSP)

\newcommand{\ayr}[1]{\zwsp\smash{{\Tagati #1}}} % Plain Ayeri orthography
\newcommand{\rayr}[2]{\zwsp\smash{{\Tagati #1}} \emph{#2}} % Ayeri orthography + *r*omanization
\newcommand{\tayr}[2]{#1 `#2'} % Romanization + *t*ranslation
\newcommand{\xayr}[3]{\zwsp\smash{\Tagati #1} \emph{#2} `#3'} % Ayeri orthography + romanization + translation

\usepackage{suffix}
\WithSuffix\newcommand{\ayr}*[1]{{\Tagati #1}} % Plain Ayeri orthography
\WithSuffix\newcommand{\rayr}*[2]{{\Tagati #1} \emph{#2}} % Ayeri orthography + *r*omanization
\WithSuffix\newcommand{\xayr}*[3]{{\Tagati #1} \emph{#2} `#3'} % Ayeri orthography + romanization + translation

\newenvironment{ayeri}{
	%\hyphenpenalty=10000
	%\hbadness=10000
	\doublespacing
	\begin{multicols}{2}
	\Tagati
}{
	\end{multicols} \par
}

\newenvironment{mytitle}{
	\hfill
	\begin{minipage}{0.667\textwidth}
	\vspace{\baselineskip}
	\begin{center}
		\Large
		\sffamily\bfseries
		\makeatletter
}{
		\makeatother
	\end{center}
	\vspace{1em}
	\end{minipage}
	\hfill
}

%% BIBLIOGRAPHY DATABASE %%%%%%%%%%%%%%%%%%%%%%%%%%%%%%%%%%%%%%%%%%%%%%%%%%%%%%%

% \begin{filecontents*}{\jobname.bib}
% @book{blah,
%     title = {A Grammar of {Blah}},
%     author = {Alfred E. Neuman and John X. Doe},
%     publisher = {Maximegalon UP},
%     location = {Maximegalon},
%     date = {1972},
%     series = {Reference Grammars of Inexistent Languages},
%     number = {4},
%     pages = {123--145},
% }
% \end{filecontents*}

% \addbibresource{\jobname.bib}

%% END OF PREAMBLE %%%%%%%%%%%%%%%%%%%%%%%%%%%%%%%%%%%%%%%%%%%%%%%%%%%%%%%%%%%%%

\begin{document}

%% MAIN PART %%%%%%%%%%%%%%%%%%%%%%%%%%%%%%%%%%%%%%%%%%%%%%%%%%%%%%%%%%%%%%%%%%%

\begin{mytitle}
	\@title: \@subtitle
\end{mytitle}

This is my attempt to translate the Austrian Christmas carol \tit{Stille Nacht,
heilige Nacht}---to English speakers simply known as \tit{Silent Night}---into
Ayeri. In December 2022, I posted on my Mastodon account a photo from the
atrium of the Berlin State Library's Unter den Linden branch featuring a
pinboard on which were posted festive tags with Christmas greetings in a slew
of languages spoken by library patrons.%
%
	\footnote{See \url%
		{https://mastodon.online/@chrpistorius/109522620399297747}%
	.}
%
User Scott Hühnerkrisp%
%
	\footnote{He goes by \href%
		{https://chaos.social/@ScottHuehnerkrisp}%
		{\itshape @ScottHuehnerkrisp@chaos.social}%
	.}
%
replied that he missed Ayeri from the pinned tags,%
%
	\footnote{Sadly, there were none left when I returned the next day.}
%
and also wondered whether there already exists a translation of \tit{Stille
Nacht} into Ayeri. Even though it's past Christmas now and this year's is still
a ways off, I wanted to make good on this challenge. This is \rayr{sirutj kluj
nj terFnu}{Sirutay kaluy nay ternu}.

\section{The German Text}

The text of the carol in German as it is commonly sung today---along with a
more or less literal English translation---goes as follows. This serves as the
base for the Ayeri version rather than the English translation proper.

\begin{quote}
\begin{minipage}{.5\linewidth}
\begin{verse}
\renewcommand*{\vrightskip}{-2em}
\verselinenumbersleft
\poemlines{5}
\indentpattern{000011}
\begin{patverse}
Stille Nacht, heilige Nacht!\\
Alles schläft, einsam wacht\\
nur das traute, hochheilige Paar.\\
Holder Knabe im lockigen Haar,\\
schlaf in seliger Ruh,\\
schlaf in seliger Ruh.\\!
\end{patverse}

\begin{patverse}
Stille Nacht, heilige Nacht!\\
Gottes Sohn, o wie lacht\\
Lieb aus deinem göttlichen Mund,\\
da uns schlägt die rettende Stund,\\
Christ, in deiner Geburt,\\
Christ, in deiner Geburt.\\!
\end{patverse}

\begin{patverse}
Stille Nacht, heilige Nacht!\\
Hirten erst kundgemacht,\\
durch der Engel Halleluja\\
tönt es laut von fern und nah:\\
Christ, der Retter, ist da,\\
Christ, der Retter, ist da!\\!
\end{patverse}
\end{verse}
\end{minipage}
~
\begin{minipage}{.5\linewidth}
\itshape
\begin{verse}
\poemlines{0}
\indentpattern{000011}
\begin{patverse}
Silent night, holy night!\\
All is asleep, lonely wakes\\
only the intimate, most holy couple.\\
Lovely boy with curly hair\\
sleep in blissful calm,\\
sleep in blissful calm.\\!
\end{patverse}

\begin{patverse}
Silent night, holy night!\\
Son of God, oh your divine mouth\\
is laughing with love\\
as the hour of salvation tolls for us,\\
Christ, with your birth,\\
Christ, with your birth.\\!
\end{patverse}

\begin{patverse}
Silent night, holy night!\\
First announced to shepherds,\\
per the angels' hallelujah\\
loudly is sounding from near and far:\\
Christ, the Savior, is here,\\
Christ, the Savior, is here!\\!
\end{patverse}
\end{verse}
\end{minipage}
\end{quote}

\section{Ayeri translation}

\begin{quote}
\begin{minipage}[t]{.5\linewidth}
\begin{verse}
\renewcommand*{\vrightskip}{-2em}
\verselinenumbersleft
\poemlines{5}
\indentpattern{000011}
\begin{patverse}
Āh sirutay kaluy nay ternu!\\
Torya enyāng, nārya-nama\\
sānang sitang-setim ternu-vā.\\
Yanang val' mitrangeri gura,\\
toru tarānya aray,\\
toru tarānya aray.\\!
\end{patverse}

\begin{patverse}
Āh sirutay kaluy nay ternu!\\
Yampangal, sā d'-apayo\\
cān bantāng van'. Eng yomara\\
pidim madānena nana,\\
Yesu, vesang'ri vana,\\
Yesu, vesang'ri vana.\\!
\end{patverse}

\begin{patverse}
Āh sirutay kaluy nay ternu!\\
Ang tangyan nantongye\\
aleluyās kelangyena.\\
Edauyi tangnang baho naynay:\\
Vesaya sa Yesu Maday',\\
vesaya sa Yesu Maday'!\\!
\end{patverse}
\end{verse}
\end{minipage}
~
\begin{minipage}[t]{.5\linewidth}
\Tagati % \smaller
\begin{verse}
\poemlines{0}
\indentpattern{000011}
\begin{patverse}
AH sirutj kluj nj terFnu!\\
torFy EnFyaaNF, naarFy/nm\\
saanNF sitNF/setimF terFnu/vaa.\\
ynNF vlF/ mitFrNeri gur,\\
toru traanY Arj,\\
toru traanY Arj.\\!
\end{patverse}

\begin{patverse}
AH sirutj kluj nj terFnu!\\
yMpNlF, saa dF/Apyo\\
tYaanF bMtaaNF vnF/. ENF yomr\\
pidimF mdaanen nn,\\
yesu, vesNF/ri vn,\\
yesu, vesNF/ri vn.\\!
\end{patverse}

\begin{patverse}
AH sirutj kluj nj terFnu!\\
ANF tNFynF nMtoNFye\\
AleluyaasF kelNFyen.\\
Edauyi tNFnNF bho njnj --\\
vesy s yesu mdjF,\\
vesy s yesu mdjF\\!
\end{patverse}
\end{verse}
\end{minipage}
\end{quote}

%% BIBLIOGRAPHY %%%%%%%%%%%%%%%%%%%%%%%%%%%%%%%%%%%%%%%%%%%%%%%%%%%%%%%%%%%%%%%%

%\vfill
\pagebreak

% \addsec{Abbreviations}
% \begin{multicols}{3}
% \printglossary[style=mysuper,type=\leipzigtype]
% \end{multicols}

%\nocite{*} % returns all entries from the bibliography database
\printbibliography[heading=bibintoc]

\end{document}
