\documentclass[12pt,paper=a4]{scrartcl}

% Author, Title, Subtitle etc.
\author{Carsten Becker}
\title{Silent Night}
\subtitle{A Christmas Carol in Ayeri}
\date{\today} % Format: YYYY/MM/DD (rev. N, YYYY/MM/DD)

% Provide running author and date
\makeatletter
\let\runauthor\@author
\let\rundate\@date
\makeatother

% Handle language and quotation marks
\usepackage{polyglossia}
\setdefaultlanguage{english}
\setotherlanguage{german}
\usepackage{csquotes} % Put quotations in \enquote{}!
\SetBlockEnvironment{quotation}
\renewcommand*{\mkccitation}[1]{ (#1)}

% Quotation style for word definitions
\DeclareQuoteStyle{wdef}
	{\textquoteleft}{\textquoteright}
	{\textquotedblleft}{\textquotedblright}
\newcommand{\wdef}[1]{{\setquotestyle{wdef}\enquote{#1}}}

% Set all margins to 2.54 cm
\usepackage[margin=1in]{geometry}
\widowpenalty10000 % Avoid widows like the plague!
\clubpenalty10000 % Avoid orphans like the plage, too!

% Extended formatting of lists
\usepackage{enumitem}
\newlist{glossdefs}{itemize}{1}
\setlist[glossdefs]{nosep, leftmargin=3em, labelwidth=2.5em, align=left}
\setlist[itemize]{noitemsep}

% Make multiple columns available in single-column document
\usepackage{multicol}

% Make text colors and color names available
\usepackage{xcolor}

% Load font stuff for XeTeX
\usepackage{fontspec}

% Set main fonts
\usepackage[config=mt-Junicode]{microtype}

\newfontfamily{\Tagati}[
	Renderer=Graphite,
	Scale=1.0,
	BoldFont={* Italic},
	HyphenChar=·,
]{Tagati Book G}

% \setmainfont[
% 	Ligatures=TeX,
% 	Numbers=Lowercase,
% ]{Junicode}

\setmainfont[
	Ligatures=TeX,
	Numbers={OldStyle,Proportional},
	BoldFont=*-Bold,
	ItalicFont=*-Italic,
	BoldItalicFont=*-BoldItalic,
]{Junicode Two Beta}

\setsansfont[
	Ligatures=TeX,
	Numbers=Lowercase,
	Scale=MatchUppercase,
	BoldFont={Open Sans Condensed Bold},
]{Open Sans Condensed Light}

% Load BibLaTeX (using Biber), configure citation styles
\usepackage[
	authordate-trad,
	backend=biber,
	safeinputenc,
]{biblatex-chicago}

% To make \textcite look like "Doe (2014: 213)"
\renewcommand*{\postnotedelim}{\addcolon\addspace}
\DeclareFieldFormat{postnote}{#1}
\DeclareFieldFormat{multipostnote}{#1}

% Enable generating files from this .tex file (for the bibliography) 
\usepackage{filecontents}

% Date etc.
\usepackage[
	datesep={/},
]{datetime2}

% Verse
\usepackage{verse}

% Clickable links in footnotes, TOC, etc.
\usepackage[
% 	xetex,
	bookmarks=true,
	colorlinks=false,
	linktoc=section,
	hidelinks,
	pdfusetitle,
]{hyperref}

% We want URLs to be italic and with regular uppercase numerals
\renewcommand{\UrlFont}{%
	\normalfont%
	\itshape%
	\addfontfeature{RawFeature=-onum}%
}

% Ability to include graphics and dealing with footnotes in descriptions
\usepackage{graphicx}
\usepackage[font={small,sf},labelfont={small,sf},format=plain]{caption}
\usepackage{subcaption}
\usepackage{wrapfig}
\setlength{\columnsep}{2\baselineskip}

% General headers and footers
\usepackage{fancyhdr}
\pagestyle{fancy}

\fancyhead[L]{} % empty
\fancyhead[C]{} % empty
\fancyhead[R]{\thepage}

\fancyfoot[L]{} % empty
\fancyfoot[C]{} % empty
\fancyfoot[R]{} % empty

\renewcommand{\headrulewidth}{0pt}
\renewcommand{\footrulewidth}{0pt}

% First page headers and footers are different
\fancypagestyle{firstpage}{
	\fancyhead[L]{\sffamily \footnotesize \textbf{Benung. The Ayeri Language Resource}}
	\fancyhead[C]{} % empty
	\fancyhead[R]{\sffamily \footnotesize \runauthor{} · \rundate{}}
	
	\fancyfoot[L]{} % empty
	\fancyfoot[C]{} % empty
	\fancyfoot[R]{\sffamily \footnotesize 
	\href{https://ayeri.de}{https://ayeri.de} · 
	\href{https://github.com/carbeck/benung-pdfs}{https://github.com/carbeck/benung-pdfs} · 
	\href{https://creativecommons.org/licenses/by-sa/4.0/}{CC~BY-SA~4.0}
	}
	
	\renewcommand{\headrulewidth}{0.5pt}
	\setlength\footskip{0.5in}
}

\usepackage{ifthen}
\ifthenelse{\value{page}=1}{\thispagestyle{firstpage}{\pagestyle{fancy}}}

% Line spacing
\usepackage{setspace}
\onehalfspacing

% Avoid pagebreaks right after sections and subsections
\usepackage{needspace}
\usepackage{etoolbox}
\preto\section{\needspace{6\baselineskip}}
\preto\subsection{\needspace{6\baselineskip}}

% Things for tables
\usepackage{tabularx}
% \usepackage{booktabs}
% \usepackage{rotating}

% Formatting of table of glossing abbreviations from Leipzig package manual
\usepackage[
	acronym,
	nomain,
	nonumberlist,
	nopostdot,
	toc=true,
	xindy={codepage=utf8}, % language=english,
]{glossaries}
\usepackage{glossary-inline}

\newglossarystyle{threecolumn}{%
	\renewenvironment{theglossary}{%
		\begin{multicols}{3}
		\begin{glossdefs}%
	}{%
		\end{glossdefs}%
		\end{multicols}%
	}%
	\renewcommand*{\glossaryheader}{}%
	\renewcommand*{\glsgroupheading}[1]{}%
	\newcommand*{\glossaryentryfield}[5]{%
		\item[\glsentryitem{##1}\glstarget{##1}{##2}]
		% \makefirstuc{##3}\glspostdescription{}
		##3\glspostdescription{}
	}%
	\renewcommand*{\glsgroupskip}{}%
}%

% Formatting of glosses
\usepackage{langsci-gb4e}
\usepackage[glosses]{leipzig}
\renewcommand{\eachwordone}{\itshape}
\renewcommand{\eachwordtwo}{\rule[-.5\baselineskip]{0pt}{0pt}}

\newleipzig{AgtT}{at}{agent topic}
\newleipzig{PatT}{pt}{patient topic}
\newleipzig{DatT}{datt}{dative topic}
\newleipzig{GenT}{gent}{genitive topic}
\newleipzig{LocT}{loct}{locative topic}
\newleipzig{InsT}{inst}{instrumental topic}
\newleipzig{CauT}{caut}{causative topic}
\newleipzig{An}{an}{animate}
\newleipzig{Inan}{inan}{inanimate}
\newleipzig{Hab}{hab}{habitative}
\newleipzig{Sup}{sup}{superlative}
\newleipzig{St}{st}{strong}
\newleipzig{Ayr}{ayr}{Ayeri}

\makeglossaries

% In-text references
% cf. https://tex.stackexchange.com/a/139051
% Since German plural formation is not as regular as in English (-e, -en for
% Beispiel), we will define the label as empty
\usepackage[sort&compress,noabbrev]{cleveref}
\newcommand{\crefrangeconjunction}{--}
\crefname{xnumi}{}{}
\creflabelformat{xnumi}{(#2#1#3)}
\crefname{xnumii}{}{}
\creflabelformat{xnumii}{(#2#1#3)}
\crefname{xnumiii}{}{}
\creflabelformat{xnumiii}{(#2#1#3)}
\crefname{xnumiv}{}{}
\creflabelformat{xnumiv}{(#2#1#3)}
\crefrangeformat{xnumi}{(#3#1#4)--(#5#2#6)}
\crefrangeformat{xnumii}{(#3#1#4--#5\crefstripprefix{#1}{#2}#6)}
\crefrangemultiformat{xnumii}{(#3\arabic{xnumi}#1#4--#5#2#6)}
{ and~(#3\arabic{xnumi}#1#4--#5#2#6)}{, (#3\arabic{xnumi}#1#4--#5#2#6)}
{ and~(#3\arabic{xnumi}#1#4--#5#2#6)}

% Nicer footnotes
\usepackage[bottom,hang,norule]{footmisc}
\setlength{\footnotesep}{0.75\baselineskip}

% Smaller font in block quotes
\usepackage{relsize}
\AtBeginEnvironment{quote}{\noindent\smaller}
\AtBeginEnvironment{quotation}{\smaller}

% Macros
\newcommand{\fw}[1]{\textit{#1}} % Foreign Word
\newcommand{\tit}[1]{\textit{#1}} % Title of a work
\newcommand{\q}[1]{\enquote{#1}} % Context-aware quotation
\newcommand{\qq}[1]{\enquote*{#1}} % Explicit sublevel quotation
\newcommand{\tsup}[1]{\textsuperscript{#1}} % Superscript
\newcommand{\markyellow}[1]{\colorbox{yellow}{#1}} % Yellow highlighter
\newcommand{\ques}{\fakesuperscript{?}} % raised question mark
\newcommand{\zwsp}{\mbox{​}} % Zero-width space (ZWSP)

\newcommand{\ayr}[1]{\zwsp\smash{{\Tagati #1}}} % Plain Ayeri orthography
\newcommand{\rayr}[2]{\zwsp\smash{{\Tagati #1}} \emph{#2}} % Ayeri orthography + *r*omanization
\newcommand{\tayr}[2]{#1 `#2'} % Romanization + *t*ranslation
\newcommand{\xayr}[3]{\zwsp\smash{\Tagati #1} \emph{#2} `#3'} % Ayeri orthography + romanization + translation

\usepackage{suffix}
\WithSuffix\newcommand{\ayr}*[1]{{\Tagati #1}} % Plain Ayeri orthography
\WithSuffix\newcommand{\rayr}*[2]{{\Tagati #1} \emph{#2}} % Ayeri orthography + *r*omanization
\WithSuffix\newcommand{\xayr}*[3]{{\Tagati #1} \emph{#2} `#3'} % Ayeri orthography + romanization + translation

\newenvironment{ayeri}{
	%\hyphenpenalty=10000
	%\hbadness=10000
	\doublespacing
	\begin{multicols}{2}
	\Tagati
}{
	\end{multicols} \par
}

\newenvironment{mytitle}{
	\hfill
	\begin{minipage}{0.667\textwidth}
	\vspace{\baselineskip}
	\begin{center}
		\Large
		\sffamily\bfseries
		\makeatletter
}{
		\makeatother
	\end{center}
	\vspace{1em}
	\end{minipage}
	\hfill
}

%% BIBLIOGRAPHY DATABASE %%%%%%%%%%%%%%%%%%%%%%%%%%%%%%%%%%%%%%%%%%%%%%%%%%%%%%%

% \begin{filecontents*}{\jobname.bib}
% @book{blah,
%     title = {A Grammar of {Blah}},
%     author = {Alfred E. Neuman and John X. Doe},
%     publisher = {Maximegalon UP},
%     location = {Maximegalon},
%     date = {1972},
%     series = {Reference Grammars of Inexistent Languages},
%     number = {4},
%     pages = {123--145},
% }
% \end{filecontents*}

% \addbibresource{\jobname.bib}

%% END OF PREAMBLE %%%%%%%%%%%%%%%%%%%%%%%%%%%%%%%%%%%%%%%%%%%%%%%%%%%%%%%%%%%%%

\begin{document}

%% MAIN PART %%%%%%%%%%%%%%%%%%%%%%%%%%%%%%%%%%%%%%%%%%%%%%%%%%%%%%%%%%%%%%%%%%%

\begin{mytitle}
	\@title: \@subtitle
\end{mytitle}

This is my attempt to translate the Austrian Christmas carol \tit{Stille Nacht,
heilige Nacht}---to English speakers known as \tit{Silent Night}---into Ayeri.
In December 2022 I posted on my Mastodon account a photo from the Berlin State
Library's Unter den Linden branch featuring a pinboard on which were posted
festive tags with Christmas greetings in a slew of languages spoken by library
patrons.%
%
	\footnote{See \url%
		{https://mastodon.online/@chrpistorius/109522620399297747}%
	.}
%
User Scott Hühnerkrisp
% %
% 	\footnote{He goes by \href%
% 		{https://chaos.social/@ScottHuehnerkrisp}%
% 		{\itshape @ScottHuehnerkrisp@chaos.social}%
% 	.}
% %
replied that he missed Ayeri among the pinned tags,%
%
	\footnote{Sadly, there were none left when I returned a few days later.}
%
and also wondered whether there already exists a translation of \tit{Stille
Nacht} into Ayeri. Even though it's past Christmas now and this year's is still
a ways off, I wanted to make good on this challenge. This is \rayr{sirutj
terFnu kluj}{Sirutay ternu kaluy}.

\section{The German Text}

The text of the carol in German as it is commonly sung today---along with a
more or less literal English translation---goes as follows. This serves as the
base for the Ayeri version rather than the English translation proper.

\begin{quote}
\begin{minipage}{.5\linewidth}
\begin{verse}
\renewcommand*{\vrightskip}{-2em}
\verselinenumbersleft
\poemlines{5}
\indentpattern{000011}
\begin{patverse}
Stille Nacht, heilige Nacht!\\
Alles schläft, einsam wacht\\
nur das traute, hochheilige Paar.\\
Holder Knabe im lockigen Haar,\\
schlaf in seliger Ruh,\\
schlaf in seliger Ruh.\\!
\end{patverse}

\begin{patverse}
Stille Nacht, heilige Nacht!\\
Gottes Sohn, o wie lacht\\
Lieb aus deinem göttlichen Mund,\\
da uns schlägt die rettende Stund,\\
Christ, in deiner Geburt,\\
Christ, in deiner Geburt.\\!
\end{patverse}

\begin{patverse}
Stille Nacht, heilige Nacht!\\
Hirten erst kundgemacht,\\
durch der Engel Halleluja\\
tönt es laut von fern und nah:\\
Christ, der Retter, ist da,\\
Christ, der Retter, ist da!\\!
\end{patverse}
\end{verse}
\end{minipage}
~
\begin{minipage}{.5\linewidth}
\itshape
\begin{verse}
\poemlines{0}
\indentpattern{000011}
\begin{patverse}
Silent night, holy night!\\
All is asleep, lonely wakes\\
only the intimate, most holy couple.\\
Lovely boy with curly hair\\
sleep in blissful calm,\\
sleep in blissful calm.\\!
\end{patverse}

\begin{patverse}
Silent night, holy night!\\
Son of God, oh how your divine mouth\\
is laughing with love\\
as the hour of salvation tolls for us,\\
Christ, with your birth,\\
Christ, with your birth.\\!
\end{patverse}

\begin{patverse}
Silent night, holy night!\\
First announced to shepherds,\\
per the angels' hallelujah\\
loudly is sounding from near and far:\\
Christ, the Savior, is here,\\
Christ, the Savior, is here!\\!
\end{patverse}
\end{verse}
\end{minipage}
\end{quote}

\section{The Ayeri translation}
The text below is my translation of the above into Ayeri. Since this is a
verse, the translation ought to fit the original's meter as closely as
possible, which is always a challenge. However, with songs the melody
fortunately creates some leeway. The German text sometimes stretches single
syllables over two notes (\fw{Sti-il-le Nacht}). The number of syllables per
line thus slightly varies across stanzas. In translation one can make use of
such tolerance as well. A greater challenge is that Ayeri tends to have
disyllabic roots to which are attached suffixes creating even more syllables,
so not exceeding the overall syllable count while staying true to the content
of the text requires a little creativity at times. Bonus points if you can fit
the foreign words' natural stress pattern to what the melody dictates.

\begin{quote}
\begin{minipage}[t]{.5\linewidth}
\begin{verse}
\renewcommand*{\vrightskip}{-2em}
\verselinenumbersleft
\poemlines{5}
\indentpattern{000011}
\begin{patverse}
Ah sirutay ternu kaluy!\\
Torya enyāng, nārya-nama\\
sānang sitang-setim ternu-vā.\\
Yanang val' mitrangeri gura,\\
toru tarānya aray,\\
toru tarānya aray.\\!
\end{patverse}

\begin{patverse}
Ah sirutay ternu kaluy!\\
Yampangal, sā d'-apayo\\
cān bantāng van'. Eng yomara\\
pidim madānena nana,\\
Yesu, vesang'ri vana,\\
Yesu, vesang'ri vana.\\!
\end{patverse}

\begin{patverse}
Ah sirutay ternu kaluy!\\
Ang tangyan nantongye\\
aleluyās kelangyena.\\
Edauyi tangnang baho naynay:\\
Vesaya sa Yesu Maday',\\
vesaya sa Yesu Maday'!\\!
\end{patverse}
\end{verse}
\end{minipage}
~
\begin{minipage}[t]{.5\linewidth}
\Tagati % \smaller
\begin{verse}
\poemlines{0}
\indentpattern{000011}
\begin{patverse}
AH sirutj terFnu kluj!\\
torFy EnFyaaNF, naarFy/nm\\
saanNF sitNF/setimF terFnu/vaa.\\
ynNF vlF/ mitFrNeri gur,\\
toru traanY Arj,\\
toru traanY Arj.\\!
\end{patverse}

\begin{patverse}
AH sirutj terFnu kluj!\\
yMpNlF, saa dF/Apyo\\
tYaanF bMtaaNF vnF/. ENF yomr\\
pidimF mdaanen nn,\\
yesu, vesNF/ri vn,\\
yesu, vesNF/ri vn.\\!
\end{patverse}

\begin{patverse}
AH sirutj terFnu kluj!\\
ANF tNFynF nMtoNFye\\
AleluyaasF kelNFyen.\\
Edauyi tNFnNF bho njnj –\\
vesy s yesu mdjF,\\
vesy s yesu mdjF\\!
\end{patverse}
\end{verse}
\end{minipage}
\end{quote}

\section{Analysis of the translation}

Let us now analyze the translation in tried-and-tested fashion, sentence by
sentence. I will start with the verse that heads every stanza to evoke a
peaceful and solemn atmosphere.

% - FIRST STANZA --------------------------------------------------------------

\begin{exe}
\ex \label{ex:ln01}
	\gll Ah sirutay ternu kaluy!\\
	oh night holy silent\\
	\trans \wdef{Oh silent, holy night!}\\
\end{exe}

Here already, concessions had to be made for a more natural stress pattern and
to fit the words. The German text as well as the melody prescribes the stress
pattern     |    . \fw{Nacht} \wdef{night} is conveniently a
single-syllable word while the adjectives \fw{stille}
\wdef{silent-\Nom.\Sg.\F.\St{}} and \fw{heilige} \wdef{holy-\Nom.\Sg.\F.\St{}}
each occupy three beats. It is apparent from \cref{ex:ln01} that
\xayr{sirutj}{sirutay}{night} alone already contains three syllables while the
adjectives are each disyllabic. Repetition of \rayr{sirutj}{sirutay} is thus
out of the question. Moreover, \rayr{sirutj}{sirutay} and \rayr{kluj}{kaluy}
both have final-syllable stress due to diphthongs being heavy in terms of
syllable weight. To keep the translation in line with the original pattern,
\rayr{sirutj}{sirutay} and \rayr{kluj}{kaluy} need to end on the fourth and
eighth beat. Thus, \xayr{terFnu}{ternu}{silent} needs to precede
\rayr{kluj}{kaluy} and there is an unoccupied stressed beat before
\rayr{sirutj}{sirutay}. I decided not to stretch \rayr{sirutj}{sirutay} over
all four beats in the same way the German text stretches the two syllables of
\fw{stille} over the first three beats. Instead, I opted to introduce the
vocative particle \xayr{AH}{ah}{oh}, which fits the function of this line and
is appropriately solemn.

Since \rayr{sirutj}{sirutay} is addressed here, I decided not to give it a case
marker as with names. Besides, this would have increased the syllable count by
at least one syllable. The order of the adjectives, even though reversed from
the original, is natural in Ayeri due to its rather consistent head-first
nature.

\begin{exe}
\ex \label{ex:ln02}
	\gll Torya enyāng, nārya-nama\\
		sleep-\Tsg.\M{} everyone-\Aarg{} but=only\\
\sn \gll sānang sitang-setim ternu-vā.\\
		couple-\Aarg{} self=familiar holy=\Sup{}\\
	\trans \wdef{Everyone is sleeping, except for the intimate, most holy
		couple.}
\end{exe}

\begin{exe}
\ex \label{ex:ln04}
	\gll Yanang val' mitrangeri gura,\\
		boy-\Aarg{} lovely hair-\Ins{} curly \\

\sn \gll toru tarānya aray.\\
		sleep-\Imp{} calm-\Loc{} blissful \\
	\trans \wdef{Lovely boy with curly hair, sleep in blissful calm.}
\end{exe}

% - SECOND STANZA -------------------------------------------------------------

\begin{exe}
\ex \label{ex:ln08}
	\gll Yampangal, sā d'-apayo\\
		Son.God \CauT= thus=laugh-\Tsg.\M{} \\

\sn \gll cān bantāng van'.\\
		love.\Top{} mouth-\Aarg{} \Second.\Gen{} \\
	\trans \wdef{Son of God, how love makes your mouth laugh.}
\end{exe}

\begin{exe}
\ex \label{ex:ln09}
	\gll Eng yomara pidim madānena nana,\\
		\AgtT.\Inan= be.there-\Tsg.\Inan{} hour.\Top{} rescue-\Gen{}
		\Fpl.\Gen \\
\sn \gll Yesu, vesang'ri vana\\
		Jesus birth-\Ins{} \Second.\Gen{} \\
	\trans \wdef{The hour of our salvation is here, Jesus, by means of your
		birth.}
\end{exe}

% - THIRD STANZA --------------------------------------------------------------

\begin{exe}
\ex \label{ex:ln14}
	\gll Ang tangyan nantongye\\
		\AgtT= hear-\Tpl.\M{} shepherd-\Pl.\Top{} \\

\sn \gll aleluyās kelangyena.\\
		halelujah-\Parg{} angel-\Pl-\Gen{} \\
	\trans \wdef{The shepherds heard the angels' hallelujah.}
\end{exe}

\begin{exe}
\ex \label{ex:ln16}
	\gll Edauyi tangnang baho naynay:\\
		now hear=\Fpl.\Aarg{} loud as\_well \\

\sn \gll vesaya sa Yesu Maday'!\\
		birth-\Tsg.\M{} \Parg= Jesus Savior \\
	\trans \wdef{Jesus the Savior is born.}
\end{exe}

%% BIBLIOGRAPHY %%%%%%%%%%%%%%%%%%%%%%%%%%%%%%%%%%%%%%%%%%%%%%%%%%%%%%%%%%%%%%%%

%\vfill
\pagebreak

\begingroup\multicolsep=0pt
\printglossary[
	style=threecolumn,
	type=leipzig,
]
\endgroup

%\nocite{*} % returns all entries from the bibliography database
\printbibliography[heading=bibintoc]

\end{document}
