\documentclass[12pt,paper=a4]{scrartcl}

% Author, Title, Subtitle etc.
\author{Carsten Becker}
\title{SE Kunst- und Plansprachen -- von Esperanto bis Dothraki}
\subtitle{Antworten zum 16.06.2016}
\date{\today} % Format: YYYY/MM/DD (rev. N, YYYY/MM/DD)

% Provide running author and date
\makeatletter
\let\runauthor\@author
\let\rundate\@date
\makeatother

% Handle language and quotation marks
\usepackage{polyglossia}
\setdefaultlanguage{german}
\setotherlanguage{english}
\usepackage{csquotes} % Put quotations in \enquote{}!
\SetBlockEnvironment{quotation}
\renewcommand*{\mkccitation}[1]{ (#1)}

% Set all margins to 2.54 cm
\usepackage[margin=1in]{geometry}
\widowpenalty10000 % Avoid widows like the plague!
\clubpenalty10000 % Avoid orphans like the plage, too!

% Extended formatting of lists
\usepackage{enumitem}
% \newlist{glossdefs}{itemize}{1}
% \setlist[glossdefs]{nosep, leftmargin=3em, labelwidth=2.5em, align=left}
\setlist[itemize]{noitemsep}

% Make multiple columns available in single-column document
\usepackage{multicol}

% Make text colors and color names available
\usepackage[xetex]{xcolor}

% Load font stuff for XeTeX
\usepackage{xltxtra}
\usepackage{fontspec}

\newfontfamily{\Tagati}[
    Renderer=Graphite,
    Scale=1.1,
    BoldFont={* Italic},
    HyphenChar=·,
]{Tagati Book G}

\setmainfont[
    Ligatures=TeX,
    Numbers=Lowercase,
]{EB Garamond}

\setsansfont[
    Ligatures=TeX,
    Numbers=Lowercase,
    Scale=MatchUppercase,
    BoldFont={Open Sans Condensed Bold},
]{Open Sans Condensed Light}

% Load BibLaTeX (using Biber), configure citation styles
\usepackage[
    authordate-trad,
    backend=biber,
    safeinputenc,
]{biblatex-chicago}
\addbibresource{bibliografie.bib}

% To make \textcite look like "Doe (2014: 213)"
\renewcommand*{\postnotedelim}{\addcolon\addspace}
\DeclareFieldFormat{postnote}{#1}
\DeclareFieldFormat{multipostnote}{#1}

% Enable generating files from this .tex file (for the bibliography) 
\usepackage{filecontents}

% Clickable links in footnotes, TOC, etc.
\usepackage[
    xetex,
    bookmarks=true,
    colorlinks=false,
    linktoc=section,
    hidelinks,
    pdfusetitle,
]{hyperref}

\usepackage{url}
\urlstyle{rm}

% Ability to include graphics and dealing with footnotes in descriptions
% \usepackage{graphicx}
% \usepackage[font={small,sf},labelfont={small,sf},format=plain]{caption}
% \usepackage{subcaption}
% \usepackage{wrapfig}
% \setlength{\columnsep}{2\baselineskip}

% General headers and footers
\usepackage{fancyhdr}
\pagestyle{fancy}

\fancyhead[L]{} % empty
\fancyhead[C]{} % empty
\fancyhead[R]{\thepage}

\fancyfoot[L]{} % empty
\fancyfoot[C]{} % empty
\fancyfoot[R]{} % empty

\renewcommand{\headrulewidth}{0pt}
\renewcommand{\footrulewidth}{0pt}

% First page headers and footers are different
\fancypagestyle{firstpage}{
    \fancyhead[L]{\sffamily \footnotesize \textbf{Benung. The Ayeri Language Resource}}
    \fancyhead[C]{} % empty
    \fancyhead[R]{\sffamily \footnotesize \runauthor{} · \rundate{}}
    
    \fancyfoot[L]{} % empty
    \fancyfoot[C]{} % empty
    \fancyfoot[R]{\sffamily \footnotesize 
	\href{http://benung.nfshost.com}{http://benung.nfshost.com} · 
	\href{https://github.com/carbeck/benung-pdfs}{https://github.com/carbeck/benung-pdfs} · 
	\href{https://creativecommons.org/licenses/by-sa/4.0/}{CC~BY-SA~4.0}
    }
    
    \renewcommand{\headrulewidth}{0.5pt}
    \setlength\footskip{0.5in}
}

\usepackage{ifthen}
\ifthenelse{\value{page}=1}{\thispagestyle{firstpage}{\pagestyle{fancy}}}

% Line spacing
\usepackage{setspace}
\onehalfspacing

% Avoid pagebreaks right after sections and subsections
\usepackage{needspace}
\usepackage{etoolbox}
\preto\section{\needspace{6\baselineskip}}
\preto\subsection{\needspace{6\baselineskip}}

% % Things for tables
\usepackage{longtable}
\usepackage{tabu}
\usepackage{booktabs}
\usepackage{rotating}

% Column types
\newcolumntype{S}{>{\scshape}X}

% Formatting of table of glossing abbreviations from Leipzig package manual
% \usepackage[acronym,nomain,nonumberlist,nopostdot]{glossaries}
% \usepackage{glossary-inline}%
% 
% \newglossarystyle{mysuper}{%
% 	\glossarystyle{super}% based on super
% 	\renewenvironment{theglossary}{%
% 		\begin{glossdefs}%
% 	}{%
% 		\end{glossdefs}%
% 	}%
% 	\renewcommand*{\glossaryheader}{}%
% 	\renewcommand*{\glsgroupheading}[1]{}%
% 	\renewcommand*{\glossaryentryfield}[5]{%
% 		\item[\glsentryitem{##1}\glstarget{##1}{##2}]%
% 		\makefirstuc{##3}\glspostdescription{}%
% 	}%
% 	\renewcommand*{\glsgroupskip}{}%
% }%

% Tree diagrams
% \usepackage[linguistics]{forest}

% Formatting of glosses
% \usepackage{expex}
% \usepackage{leipzig}
% 
% \newleipzig{AgtT}{at}{agent topic}
% \newleipzig{PatT}{pt}{patient topic}
% \newleipzig{DatT}{datt}{dative topic}
% \newleipzig{GenT}{gent}{genitive topic}
% \newleipzig{LocT}{loct}{locative topic}
% \newleipzig{InsT}{inst}{instrumental topic}
% \newleipzig{CauT}{caut}{causative topic}
% \newleipzig{An}{an}{animate}
% \newleipzig{Inan}{inan}{inanimate}
% \newleipzig{Hab}{hab}{habitative}
% \newleipzig{Ayr}{ayr}{Ayeri}
% 
% \makeglossaries

% Nicer footnotes
\usepackage[bottom,hang,norule]{footmisc}
\setlength{\footnotesep}{0.75\baselineskip}

% Smaller font in block quotes
\usepackage{relsize}
\AtBeginEnvironment{quote}{\noindent\smaller}
\AtBeginEnvironment{quotation}{\smaller}

% Date and time
\usepackage[
	datesep=.,
	style=ddmmyyyy,
]{datetime2}

% Macros
\newcommand{\fw}[1]{\textit{#1}} % Foreign Word
\newcommand{\tit}[1]{\textit{#1}} % Title of a work
\newcommand{\q}[1]{\enquote{#1}} % Context-aware quotation
\newcommand{\qq}[1]{\enquote*{#1}} % Explicit sublevel quotation
\newcommand{\tsup}[1]{\textsuperscript{#1}} % Superscript
\newcommand{\tsub}[1]{\textsubscript{#1}} % Superscript
\newcommand{\markyellow}[1]{\colorbox{yellow}{#1}} % Yellow highlighter
\newcommand{\ques}{\fakesuperscript{?}} % raised question mark

% \newcommand{\ayr}[1]{{\Tagati #1}}
% \newcommand{\xayr}[3]{{\Tagati #1} \emph{#2} `#3'}
% 
% \newenvironment{ayeri}{
%     %\hyphenpenalty=10000
%     %\hbadness=10000
%     \doublespacing
%     \begin{multicols}{2}
%     \Tagati
% }{
%     \end{multicols} \par
% }

\newenvironment{mytitle}{
    \hfill
    \begin{minipage}{0.667\textwidth}
	\vspace{\baselineskip}
	\begin{center}
	    \Large
	    \sffamily\bfseries
	    \makeatletter
}{
	    \makeatother
	\end{center}
	\vspace{1em}
    \end{minipage}
    \hfill
}

% blah
\usepackage{lipsum}

%% END OF PREAMBLE %%%%%%%%%%%%%%%%%%%%%%%%%%%%%%%%%%%%%%%%%%%%%%%%%%%%%%%%%%%%%

\begin{document}

%% MAIN PART %%%%%%%%%%%%%%%%%%%%%%%%%%%%%%%%%%%%%%%%%%%%%%%%%%%%%%%%%%%%%%%%%%%

\begin{mytitle}
    \@title: \@subtitle\footnotemark
\end{mytitle}
\footnotetext{\cite[Vgl.][]{buch2016ss}.}

\section{Quellen zur Satzkonstruktion}

\noindent Informationen zur Konstruktion der Sätze und Wörter aus den letzten 
beiden Hausaufgaben können zu den jeweiligen Punkten den aufgelisteten Quellen 
entnommen werden:

\begin{itemize}
\item Wortstämme: Einträge zu `boat', `float', `my', `full', `fill' 
	\parencite[Dictionary]{benung};
\item Silbenstruktur von Wortstämmen: \cite{becker2010}; \cite[5]{becker2011}, 
	dazu ausführlicher \cite{benung:syllablestress};
\item Satzstellung: \cites[27]{becker2011}[§~2.1]{benung:flickingswitches};
\item Wortstellung: \cite[20--21, 28--29]{becker2011};
\item Kasusmarkierungen: \cite[36--37, 39]{becker2011};
\item Konjugation von Verben: \cites[17--20]{becker2011}{benung:verbagreement};
\item Prädikative NPs und Existenzialsätze: 
	\cite[43--44]{becker2011};
\item Topikalisierung: \cite[27--28]{becker2011} (hier noch als \qq{Fokus}), 
	dazu aktualisierend \cite[§§~1 und 2.1]{benung:flickingswitches};
\item Schrift: \cites[Alphabet]{benung}{tagatibookg:readme}.
\end{itemize}

\section{Bewertung}

Die folgende Bewertung richtet sich nach \cite{sai:conlangeval}. Obwohl 
\citeauthor{sai:conlangeval} um objektive Kriterien bemüht ist, ist die 
Einschätzung des Ausprägungsgrades des jeweiligen Kriteriums trotz allem relativ 
subjektiv.\footnote{Daher würde mich die Einschätzung der Kursteilnehmer, die 
Ayeri bearbeitet haben, umso mehr interessieren!}

{\singlespacing
\begin{longtabu} to \textwidth {X[7] S[3] X[12]}
\toprule
\rowfont{\normalfont\sffamily\bfseries}
Kriterium
	& Code
	& Begründung
\\* \toprule
\endhead

Naturalness
	& nat+
	& Im Großen und Ganzen relativ normal, dürfte als natürliche Sprache 
	  durchgehen, wenn Regel- und Unregelmäßigkeiten etwas natürlicher 
	  verteilt wären.\phantom{p} Das Topik-System und die 
	  Adjektiv\-steigerung sind etwas eigen.\phantom{p}
\\* \midrule

Completeness
	& cpl+(++)
	& Zahlreiche auch komplexe Beispieltexte, zumindest Versuche zur 
	  Versform (\q{Ozymandias}, \q{LCC4 Relay})
\\* \midrule

Complexity
	& cpx+(++)
	& Im Großen und Ganzen \fw{copy-and-paste}-ag\-glu\-tinierend, aber 
	  Pronomen und Zahlwörter haben es in sich.\phantom{p}
\\* \midrule

Personal Innovation
	& pin++
	& Relationale Typologie (Anleihen an \fw{Austronesian align\-ment}) 
	  findet sich so in Europa nicht.\phantom{p}
\\* \midrule

Global Innovation
	& gin-
	& Relativ wiedererkennbar, was Kunstsprachen angeht, versucht aber 
	  vielleicht etwas zu sehr, stereotyp südostasiatisch 
	  auszusehen.\phantom{p}
\\* \midrule

Coherence
	& chr+(++)
	& Die Morphosyntax ist relativ aus einem Guss, nur hat der Kontrast 
	  zwischen belebten und unbelebten Substantiven keine prägenden 
	  Aus\-wirkungen.\phantom{p}
\\* \midrule

Cultural Expressiveness
	& clt
	& Spezifische kulturelle Bezüge wurden größtenteils vermieden, aber 
	  Entwicklungsmöglichkeiten angedeutet.\phantom{p}
\\* \midrule

Liberalness
	& lib-{}-
	& Sexus-basierte Trennung in der 3.~Person belebt bei Pronomen und 
	Verbkongruenz; maskulin als Resolutionspräferenz.\phantom{p}
\\* \midrule

Mellifluousness
	& mlf++
	& Schimpf- und sonstige \qq{dreckigen} Wörter hören sich aufgrund der 
	  Silben- und Lautstruktur der Sprache harmlos an.\phantom{p}
\\* \midrule

Sapir-Whorf
	& !spw
	& Erfordert keine besondere Weltsicht.\phantom{p}
\\* \midrule

Ease of Learning
	& eas++
	& Relativ hohe Regelmäßigkeit dürfte das Lernen leicht machen.\phantom{p}
\\* \midrule

Documentation
	& doc++
	& Detailliert beschreibende Materialien sind verfügbar, wenn auch etwas 
	  verstreut (Blog, Kommentar zu Beispieltexten). Eine Grammatik ist 
	  vorhanden, aber seit Jahren unvollständig.\phantom{p}
\\* \midrule

Corpus
	& crp++
	& Zahlreiche kürzere Beispieltexte.\phantom{p}
\\* \midrule

Finishedness
	& fin++
	& Ziemlich stabil, hin und wieder kleine Än\-de\-rungen oder 
	  Variationsmöglichkeiten.\phantom{p}
\\* \midrule

Fidelity
	& fid
	& Keine besonderen Strategien, die die Dinge entweder sehr leicht 
	  oder sehr schwer verständlich machen.\phantom{p}
\\* \midrule

Effort
	& eff++
	& Durch die im Vergleich mit den meisten europäischen Sprachen 
	  ungewöhnliche Satzstellung braucht es etwas Überlegung.\phantom{p}
\\* \midrule

Density
	& dns
	& Information ist etwa genauso dicht wie Englisch oder Deutsch;
	  Übersetzungen sind nicht signifikant länger oder kürzer.\phantom{p}
\\* \midrule

Clarity
	& clr
	& Nicht mehr oder weniger ambig als natürliche Sprachen.\phantom{p}
\\* \midrule

Noise Resistance
	& nse
	& Vielleicht etwas schwierig zu verstehen bei größerem Lärm, aber 
	  nicht außergewöhnlich.\phantom{p}
\\* \midrule

Form/Concept Complexity
	& fcc
	& \qq{Einfache} Wörter haben meist 1--2 Silben; Komposita (≥~3 Silben) 
	  drücken komplexere Dinge aus.\phantom{p}
\\* \midrule

Family
	& fam-{}-{}-{}-
	& Bisher alleinstehend.\phantom{p} Sprachfamilie angedacht, aber nie in 
	  Angriff genommen.\phantom{p}
\\* \midrule

Modalities
	& mod
	& Eine eigene Schrift ist vorhanden, diese spiegelt jedoch die 
	  gesprochene Sprache wider.\phantom{p}
\\* \midrule

Directness
	& !dct
	& Es gibt nicht viele idiomatische Ausdrücke, da der kulturelle Bezug 
	  nicht ausgearbeitet ist.\phantom{p}
\\* \midrule

Overall rating
	& tlt+
	& Ayeri ist vermutlich eine der komplexeren, detailreicheren 
	  persönlichen Kunstsprachen im Internet. Lobend erwähnt in 
	  \cite[15, 249]{peterson2015}.\phantom{p}
\\* \midrule

Ambition
	& amb++
	& Man ist im Grunde nie fertig damit, eine naturalistische Kunstsprache 
	  zu schaffen.\phantom{p}
\\* \midrule

Success
	& suc++
	& Ziel/Wertung: \textsc{clt}~2/0; \textsc{cpl}~2/1,5; \textsc{doc}~3/2; 
	  \textsc{fam}~2/–4 = 2,4.\footnotemark
\\
\bottomrule
\end{longtabu}
}
\footnotetext{Hier sind nur die Kriterien in die Berechnung eingeflossen, bei 
denen meiner Einschätzung nach meine Ziele und die hier angegebene Bewertung zur 
realen Ausführung auseinanderfallen. Ansonsten dürften sich Ziel und Bewertung 
relativ decken. Sich selbst realistisch zu bewerten ist allerdings schwierig.}

%% BIBLIOGRAPHY %%%%%%%%%%%%%%%%%%%%%%%%%%%%%%%%%%%%%%%%%%%%%%%%%%%%%%%%%%%%%%%%

\vfill

\addsec{Literaturverzeichnis}
\printbibliography[heading=none]

\end{document}
