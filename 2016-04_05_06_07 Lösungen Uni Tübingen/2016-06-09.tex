\documentclass[12pt,paper=a4]{scrartcl}

% Author, Title, Subtitle etc.
\author{Carsten Becker}
\title{SE Kunst- und Plansprachen -- von Esperanto bis Dothraki}
\subtitle{Antworten zum 09.06.2016}
\date{\today} % Format: YYYY/MM/DD (rev. N, YYYY/MM/DD)

% Provide running author and date
\makeatletter
\let\runauthor\@author
\let\rundate\@date
\makeatother

% Handle language and quotation marks
\usepackage{polyglossia}
\setdefaultlanguage{german}
\usepackage{csquotes} % Put quotations in \enquote{}!
\SetBlockEnvironment{quotation}
\renewcommand*{\mkccitation}[1]{ (#1)}

% Set all margins to 2.54 cm
\usepackage[margin=1in]{geometry}
\widowpenalty10000 % Avoid widows like the plague!
\clubpenalty10000 % Avoid orphans like the plage, too!

% Extended formatting of lists
\usepackage{enumitem}
\newlist{glossdefs}{itemize}{1}
\setlist[glossdefs]{nosep, leftmargin=3em, labelwidth=2.5em, align=left}
\setlist[itemize]{noitemsep}

% Make multiple columns available in single-column document
\usepackage{multicol}

% Make text colors and color names available
\usepackage[xetex]{xcolor}

% Load font stuff for XeTeX
\usepackage{xltxtra}
\usepackage{fontspec}

\newfontfamily{\Tagati}[
    Renderer=Graphite,
    Scale=1.1,
    BoldFont={* Italic},
    HyphenChar=·,
]{Tagati Book G}

\setmainfont[
    Ligatures=TeX,
    Numbers=Lowercase,
]{EB Garamond}

\setsansfont[
    Ligatures=TeX,
    Numbers=Lowercase,
    Scale=MatchUppercase,
    BoldFont={Open Sans Condensed Bold},
]{Open Sans Condensed Light}

% Load BibLaTeX (using Biber), configure citation styles
\usepackage[
    authordate-trad,
    backend=biber,
    safeinputenc,
]{biblatex-chicago}
\addbibresource{bibliografie.bib}

% To make \textcite look like "Doe (2014: 213)"
\renewcommand*{\postnotedelim}{\addcolon\addspace}
\DeclareFieldFormat{postnote}{#1}
\DeclareFieldFormat{multipostnote}{#1}

% Enable generating files from this .tex file (for the bibliography) 
\usepackage{filecontents}

% Clickable links in footnotes, TOC, etc.
\usepackage[
    xetex,
    bookmarks=true,
    colorlinks=false,
    linktoc=section,
    hidelinks,
    pdfusetitle,
]{hyperref}

\usepackage{url}
\urlstyle{rm}

% Ability to include graphics and dealing with footnotes in descriptions
\usepackage{graphicx}
\usepackage[font={small,sf},labelfont={small,sf},format=plain]{caption}
\usepackage{subcaption}
\usepackage{wrapfig}
\setlength{\columnsep}{2\baselineskip}

% General headers and footers
\usepackage{fancyhdr}
\pagestyle{fancy}

\fancyhead[L]{} % empty
\fancyhead[C]{} % empty
\fancyhead[R]{\thepage}

\fancyfoot[L]{} % empty
\fancyfoot[C]{} % empty
\fancyfoot[R]{} % empty

\renewcommand{\headrulewidth}{0pt}
\renewcommand{\footrulewidth}{0pt}

% First page headers and footers are different
\fancypagestyle{firstpage}{
    \fancyhead[L]{\sffamily \footnotesize \textbf{Benung. The Ayeri Language Resource}}
    \fancyhead[C]{} % empty
    \fancyhead[R]{\sffamily \footnotesize \runauthor{} · \rundate{}}
    
    \fancyfoot[L]{} % empty
    \fancyfoot[C]{} % empty
    \fancyfoot[R]{\sffamily \footnotesize 
	\href{http://benung.nfshost.com}{http://benung.nfshost.com} · 
	\href{https://github.com/carbeck/benung-pdfs}{https://github.com/carbeck/benung-pdfs} · 
	\href{https://creativecommons.org/licenses/by-sa/4.0/}{CC~BY-SA~4.0}
    }
    
    \renewcommand{\headrulewidth}{0.5pt}
    \setlength\footskip{0.5in}
}

\usepackage{ifthen}
\ifthenelse{\value{page}=1}{\thispagestyle{firstpage}{\pagestyle{fancy}}}

% Line spacing
\usepackage{setspace}
\onehalfspacing

% Avoid pagebreaks right after sections and subsections
\usepackage{needspace}
\usepackage{etoolbox}
\preto\section{\needspace{6\baselineskip}}
\preto\subsection{\needspace{6\baselineskip}}

% Things for tables
\usepackage{longtable}
\usepackage{tabu}
\usepackage{booktabs}
\usepackage{rotating}

% Formatting of table of glossing abbreviations from Leipzig package manual
\usepackage[acronym,nomain,nonumberlist,nopostdot]{glossaries}
\usepackage{glossary-inline}%

\newglossarystyle{mysuper}{%
	\glossarystyle{super}% based on super
	\renewenvironment{theglossary}{%
		\begin{glossdefs}%
	}{%
		\end{glossdefs}%
	}%
	\renewcommand*{\glossaryheader}{}%
	\renewcommand*{\glsgroupheading}[1]{}%
	\renewcommand*{\glossaryentryfield}[5]{%
		\item[\glsentryitem{##1}\glstarget{##1}{##2}]%
		\makefirstuc{##3}\glspostdescription{}%
	}%
	\renewcommand*{\glsgroupskip}{}%
}%

% Tree diagrams
\usepackage[linguistics]{forest}

% Formatting of glosses
\usepackage{expex}
\usepackage{leipzig}

\newleipzig{AgtT}{at}{agent topic}
\newleipzig{PatT}{pt}{patient topic}
\newleipzig{DatT}{datt}{dative topic}
\newleipzig{GenT}{gent}{genitive topic}
\newleipzig{LocT}{loct}{locative topic}
\newleipzig{InsT}{inst}{instrumental topic}
\newleipzig{CauT}{caut}{causative topic}
\newleipzig{An}{an}{animate}
\newleipzig{Inan}{inan}{inanimate}
\newleipzig{Hab}{hab}{habitative}
\newleipzig{Ayr}{ayr}{Ayeri}

\makeglossaries

% Section formatting
% cf. http://tex.stackexchange.com/a/80114/60686
\makeatletter
\renewcommand\thesection{}
\renewcommand\thesubsection{\@arabic\c@section.\@arabic\c@subsection}
\makeatother

% Nicer footnotes
\usepackage[bottom,hang,norule]{footmisc}
\setlength{\footnotesep}{0.75\baselineskip}

% Smaller font in block quotes
\usepackage{relsize}
\AtBeginEnvironment{quote}{\noindent\smaller}
\AtBeginEnvironment{quotation}{\smaller}

% Date and time
\usepackage[
	datesep=.,
	style=ddmmyyyy,
]{datetime2}

% Macros
\newcommand{\fw}[1]{\textit{#1}} % Foreign Word
\newcommand{\tit}[1]{\textit{#1}} % Title of a work
\newcommand{\q}[1]{\enquote{#1}} % Context-aware quotation
\newcommand{\qq}[1]{\enquote*{#1}} % Explicit sublevel quotation
\newcommand{\tsup}[1]{\textsuperscript{#1}} % Superscript
\newcommand{\tsub}[1]{\textsubscript{#1}} % Superscript
\newcommand{\markyellow}[1]{\colorbox{yellow}{#1}} % Yellow highlighter
\newcommand{\ques}{\fakesuperscript{?}} % raised question mark

\newcommand{\ayr}[1]{{\Tagati #1}}
\newcommand{\xayr}[3]{{\Tagati #1} \emph{#2} `#3'}

\newenvironment{ayeri}{
    %\hyphenpenalty=10000
    %\hbadness=10000
    \doublespacing
    \begin{multicols}{2}
    \Tagati
}{
    \end{multicols} \par
}

\newenvironment{mytitle}{
    \hfill
    \begin{minipage}{0.667\textwidth}
	\vspace{\baselineskip}
	\begin{center}
	    \Large
	    \sffamily\bfseries
	    \makeatletter
}{
	    \makeatother
	\end{center}
	\vspace{1em}
    \end{minipage}
    \hfill
}

% blah
\usepackage{lipsum}

%% END OF PREAMBLE %%%%%%%%%%%%%%%%%%%%%%%%%%%%%%%%%%%%%%%%%%%%%%%%%%%%%%%%%%%%%

\begin{document}

%% MAIN PART %%%%%%%%%%%%%%%%%%%%%%%%%%%%%%%%%%%%%%%%%%%%%%%%%%%%%%%%%%%%%%%%%%%

\begin{mytitle}
    \@title: \@subtitle\footnotemark
\end{mytitle}
\footnotetext{\cite[Vgl.][]{buch2016ss}.}

\noindent Die direkteste Möglichkeit, den Satz \q{Mein Luftkissenboot ist 
voller Aale} zu übersetzen, ist diese:

\ex\label{ex:predicative} \begingl
	\glpreamble \ayr{tFrpsFreNF svitnF naa deNF bmNFyeri.} 
		\textit{Trapasreng savitan nā deng bamangyeri.} //
	\gla Trapas @ -reng savit @ -an nā deng bamang @ 
		-ye @ -eri //
	\glb Boot -\Aarg{}.\Inan{} schweben -\Nmlz{} \Fsg{}.\Gen{} 
		voll Aal -\Pl{} -\Ins{} //
	\glft `Mein Luftkissenboot ist voller Aale.' //
\endgl \xe

\noindent Alternativ ist jedoch auch diese Möglichkeit denkbar:

\ex\label{ex:regular} \begingl
	\glpreamble \ayr{le suytonF bmNFye\_aNF tFrpsFvitnF naa.} 
		\textit{Le suyaton bamangjang trapasavitan nā.} //
	\gla Le suya @ -ton bamang @ -j @ -ang trapasavitan @ -Ø nā //
	\glb \PatT{} füllen -\Tpl{}.\N{} Aal -\Pl{} -\Aarg{} Luftkissenboot 
		-\Top{} \Fsg{}.\Gen{} //
	\glft `Mein Luftkissenboot, Aale füllen es.' //
\endgl \xe

\noindent Informationen zur Konstruktion der Sätze können entnommen werden aus:

\begin{itemize}
	\item Wortstämme: Wörterbucheinträge zu `boat', `float', `my', `full',
		`fill' \parencite{benung};
	\item Satzstellung: \cites[27]{becker2011}[§~2.1]{benung:flickingswitches};
	\item Nachstellung von Modifikatoren: \cite[20--21, 28--29]{becker2011};
	\item Kasusmarkierungen: \cite[36, 39]{becker2011};
	\item Prädikative NPs und Existenzialsätze: \cite[43--44]{becker2011};
	\item Konjugation von Verben: \cites[17--20]{becker2011}[]{benung:verbagreement};
	\item Topikalisierung: \cite[27--28]{becker2011} (hier noch als 
		\qq{Fokus}), \cite[§§~1 und 2.1]{benung:flickingswitches}.
\end{itemize}

\begin{figure}[t]
\label{fig:ex1analysis}
\ex\label{ex1analysis}\scriptsize
\vspace{-1.75\baselineskip}
\begin{multicols}{2}
\begin{forest}
where n children=0{tier=word,font=\itshape}{}
[S
	[NP
		[N [{Trapasreng savitan},roof]]
		[Det [nā]]
	]
	[VP
		[V [Ø]]
		[AP
			[Adj [deng]]
			[NP
				[N [bamangyeri]]
			]
		]
	]
]
\end{forest}

\scriptsize
\begin{forest}
where n children=0{tier=word,edge=dotted,font=\itshape}{}
[Ø, calign=child, calign child=2
	[trapasreng, for tree={calign=first}
		[Trapasreng]
		[savitan
			[savitan]
		]
		[nā
			[nā]
		]
	]
	[Ø]
	[deng, for tree={calign=first}
		[deng]
		[bamangyeri
			[bamangyeri]
		]
	]
]
\end{forest}
\end{multicols}
\xe
\end{figure}

Ein Phrasenstrukturbaum für (\ref{ex:predicative}) könnte vielleicht wie in 
(\ref{ex1analysis}) aussehen, ohne dabei allzu sehr ins Detail (und in favorisierte 
Syntaxtheorien) zu gehen; im selben Beispiel findet sich daneben außerdem ein 
Dependenzbaum. Beide Analysen gehen davon aus, dass Ayeri bei Sätzen mit 
prädikativen NPs die Konstituentenfolge SV statt -- wie sonst üblich -- 
VS benutzt. Für (\ref{ex:regular}) sieht der Dependenzbaum wie in (\ref{ex2dep})
gezeigt aus.

\ex\label{ex2dep}\scriptsize
\begin{forest}
where n children=0{tier=word,edge=dotted,font=\itshape}{}
[{le suyaton}, for tree={calign=first}
	[{Le suyaton}]
	[bamangjang
		[bamangjang]
	]
	[trapasavitan
		[trapasavitan]
		[nā
			[nā]
		]
	]
]
\end{forest}
\xe

Beim Phrasenstrukturbaum zu (\ref{ex:regular}) stellt sich die Frage, wie mit 
der VP verfahren wird, da diese diskontinuierlich ist, wenn man wie gewöhnlich 
annimmt, dass die Objekt-NP ein Komplement der VP ist. Aufschluss könnte 
eventuell \cite{benung:verbagreement} geben. Unter der Annahme, dass VSO eine 
von SVO abgeleitete Satzstellung ist, wäre vielleicht folgende Analyse möglich:

\ex\label{ex:movement}\scriptsize
\begin{forest}
where n children=0{tier=word,font=\itshape}{}
[S
	[V [{Le suyaton}, roof, name=verb]]
	[NP [N [bamangjang]]]
	[VP
		[V [\textup{t}, name=verbtrace]]
		[NP
			[N [trapasavitan]]
			[Det [nā]]
		]
	]
]
\draw[->, dotted] (verbtrace) to[out=south west,in=south east] (verb);
\end{forest}
\xe

%% BIBLIOGRAPHY %%%%%%%%%%%%%%%%%%%%%%%%%%%%%%%%%%%%%%%%%%%%%%%%%%%%%%%%%%%%%%%%

\vfill

\addsec{Abkürzungen}
\begin{multicols}{3}%
\printglossary[style=mysuper,type=\leipzigtype]
\end{multicols}

\addsec{Literaturverzeichnis}
\printbibliography[heading=none]

\end{document}
