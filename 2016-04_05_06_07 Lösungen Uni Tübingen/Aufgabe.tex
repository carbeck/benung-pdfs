\documentclass[12pt,paper=a4]{scrartcl}

% Author, Title, Subtitle etc.
\author{Carsten Becker}
\title{SE Kunst- und Plansprachen -- von Esperanto bis Dothraki}
\subtitle{Übersetzungsaufgabe}
\date{\today} % Format: YYYY/MM/DD (rev. N, YYYY/MM/DD)

% Provide running author and date
\makeatletter
\let\runauthor\@author
\let\rundate\@date
\makeatother

% Handle language and quotation marks
\usepackage{polyglossia}
\setdefaultlanguage{german}
\setotherlanguage{english}
\usepackage{csquotes} % Put quotations in \enquote{}!
\SetBlockEnvironment{quotation}
\renewcommand*{\mkccitation}[1]{ (#1)}

% Set all margins to 2.54 cm
\usepackage[margin=1in]{geometry}
\widowpenalty10000 % Avoid widows like the plague!
\clubpenalty10000 % Avoid orphans like the plage, too!

% Extended formatting of lists
\usepackage{enumitem}
\newlist{glossdefs}{itemize}{1}
\setlist[glossdefs]{nosep, leftmargin=3em, labelwidth=2.5em, align=left}
\setlist[itemize]{noitemsep}

% Make multiple columns available in single-column document
\usepackage{multicol}

% Make text colors and color names available
\usepackage[xetex]{xcolor}

% Load font stuff for XeTeX
\usepackage{xltxtra}
\usepackage{fontspec}

\newfontfamily{\Tagati}[
    Renderer=Graphite,
    Scale=1.1,
    BoldFont={* Italic},
    HyphenChar=·,
]{Tagati Book G}

\setmainfont[
    Ligatures=TeX,
    Numbers=Lowercase,
]{Junicode}

\setsansfont[
    Ligatures=TeX,
    Numbers=Lowercase,
    Scale=MatchUppercase,
    BoldFont={Open Sans Condensed Bold},
]{Open Sans Condensed Light}

% Load BibLaTeX (using Biber), configure citation styles
\usepackage[
    authordate-trad,
    backend=biber,
    safeinputenc,
]{biblatex-chicago}
\addbibresource{bibliografie.bib}

% To make \textcite look like "Doe (2014: 213)"
\renewcommand*{\postnotedelim}{\addcolon\addspace}
\DeclareFieldFormat{postnote}{#1}
\DeclareFieldFormat{multipostnote}{#1}

% Enable generating files from this .tex file (for the bibliography) 
\usepackage{filecontents}

% Clickable links in footnotes, TOC, etc.
\usepackage[
    xetex,
    bookmarks=true,
    colorlinks=false,
    linktoc=section,
    hidelinks,
    pdfusetitle,
]{hyperref}

\usepackage{url}
\urlstyle{rm}

% Ability to include graphics and dealing with footnotes in descriptions
% \usepackage{graphicx}
% \usepackage[font={small,sf},labelfont={small,sf},format=plain]{caption}
% \usepackage{subcaption}
% \usepackage{wrapfig}
% \setlength{\columnsep}{2\baselineskip}

% General headers and footers
\usepackage{fancyhdr}
\pagestyle{fancy}

\fancyhead[L]{} % empty
\fancyhead[C]{} % empty
\fancyhead[R]{\thepage}

\fancyfoot[L]{} % empty
\fancyfoot[C]{} % empty
\fancyfoot[R]{} % empty

\renewcommand{\headrulewidth}{0pt}
\renewcommand{\footrulewidth}{0pt}

% First page headers and footers are different
\fancypagestyle{firstpage}{
    \fancyhead[L]{\sffamily \footnotesize \textbf{Benung. The Ayeri Language Resource}}
    \fancyhead[C]{} % empty
    \fancyhead[R]{\sffamily \footnotesize \runauthor{} · \rundate{}}
    
    \fancyfoot[L]{} % empty
    \fancyfoot[C]{} % empty
    \fancyfoot[R]{\sffamily \footnotesize 
	\href{https://benung.nfshost.com}{https://benung.nfshost.com} · 
	\href{https://github.com/carbeck/benung-pdfs}{https://github.com/carbeck/benung-pdfs} · 
	\href{https://creativecommons.org/licenses/by-sa/4.0/}{CC~BY-SA~4.0}
    }
    
    \renewcommand{\headrulewidth}{0.5pt}
    \setlength\footskip{0.5in}
}

\usepackage{ifthen}
\ifthenelse{\value{page}=1}{\thispagestyle{firstpage}{\pagestyle{fancy}}}

% Line spacing
\usepackage{setspace}
\onehalfspacing

% Avoid pagebreaks right after sections and subsections
\usepackage{needspace}
\usepackage{etoolbox}
\preto\section{\needspace{6\baselineskip}}
\preto\subsection{\needspace{6\baselineskip}}

% Smaller font in block quotes
\usepackage{relsize}
\AtBeginEnvironment{quote}{\noindent\smaller}
\AtBeginEnvironment{quotation}{\smaller}

% % Things for tables
\usepackage{longtable}
\usepackage{tabu}
% \usepackage{booktabs}
% \usepackage{rotating}

% Column types
% \newcolumntype{S}{>{\scshape}X}

% Formatting of table of glossing abbreviations from Leipzig package manual
\usepackage[acronym,nomain,nonumberlist,nopostdot]{glossaries}
\usepackage{glossary-inline}%

\newglossarystyle{mysuper}{%
	\glossarystyle{super}% based on super
	\renewenvironment{theglossary}{%
		\begin{glossdefs}%
	}{%
		\end{glossdefs}%
	}%
	\renewcommand*{\glossaryheader}{}%
	\renewcommand*{\glsgroupheading}[1]{}%
	\renewcommand*{\glossaryentryfield}[5]{%
		\item[\glsentryitem{##1}\glstarget{##1}{##2}]%
		%\makefirstuc{##3}\glspostdescription{}%
		##3\glspostdescription{}%
	}%
	\renewcommand*{\glsgroupskip}{}%
}%

% Tree diagrams
\usepackage[linguistics]{forest}

% Formatting of glosses
\usepackage{expex}
\lingset{Everyex=\smaller}
\usepackage{leipzig}

\newleipzig{Aarg}{a}{Agens}
\newleipzig{Parg}{p}{Patiens}
\newleipzig{Dat}{dat}{Dativ}
\newleipzig{Gen}{gen}{Genitiv}
\newleipzig{Loc}{loc}{Lokativ}
\newleipzig{Caus}{caus}{Kausativ}
\newleipzig{Top}{top}{Topik}

\newleipzig{AgtT}{at}{Agens-Topik}
\newleipzig{PatT}{pt}{Patiens-Topik}
\newleipzig{DatT}{datt}{Dativ-Topik}
\newleipzig{GenT}{gent}{Genitiv-Topik}
\newleipzig{LocT}{loct}{Lokativ-Topik}
\newleipzig{InsT}{inst}{Instrumental-Topik}
\newleipzig{CauT}{caut}{Kausativ-Topik}

\newleipzig{First}{1}{erste Person}
\newleipzig{Second}{2}{zweite Person}
\newleipzig{Third}{3}{dritte Person}
\newleipzig{M}{m}{Maskulinum}
\newleipzig{F}{f}{Femininum}
\newleipzig{N}{n}{Neutrum}
\newleipzig{Sg}{sg}{Singular}
\newleipzig{Pl}{pl}{Plural}
\newleipzig{Inan}{inan}{unbelebt}

\newleipzig{Hab}{hab}{Habitativ}
\newleipzig{Neg}{neg}{Negativ}
\newleipzig{Irr}{irr}{Irrealis}
\newleipzig{Imp}{imp}{Imperativ}
\newleipzig{Prog}{prog}{Progressiv}
\newleipzig{Dyn}{dyn}{dynamisch}
\newleipzig{Refl}{refl}{reflexiv}
\newleipzig{NPst}{npst}{nahe Vergangenheit}

\newleipzig{Rel}{rel}{Relativ}

\makeglossaries

\newcommand{\AargI}{{\Aarg}.{\Inan}}
\newcommand{\PargI}{{\Parg}.{\Inan}}
\newcommand{\AgtTI}{{\AgtT}.{\Inan}}
\newcommand{\PatTI}{{\PatT}.{\Inan}}

\newcommand{\TsgM}{{\Tsg}.{\M}}
\newcommand{\TsgF}{{\Tsg}.{\F}}
\newcommand{\TsgN}{{\Tsg}.{\N}}
\newcommand{\TsgI}{{\Tsg}.{\Inan}}
\newcommand{\TplM}{{\Tpl}.{\M}}
\newcommand{\TplF}{{\Tpl}.{\F}}
\newcommand{\TplN}{{\Tpl}.{\N}}
\newcommand{\TplI}{{\Tpl}.{\Inan}}

% Nicer footnotes
\usepackage[bottom,hang,norule]{footmisc}
\setlength{\footnotesep}{0.75\baselineskip}

% Date and time
\usepackage[
	datesep=.,
	style=ddmmyyyy,
]{datetime2}

% Macros
\newcommand{\fw}[1]{\textit{#1}} % Foreign Word
\newcommand{\tit}[1]{\textit{#1}} % Title of a work
\newcommand{\q}[1]{\enquote{#1}} % Context-aware quotation
\newcommand{\qq}[1]{\enquote*{#1}} % Explicit sublevel quotation
\newcommand{\tsup}[1]{\textsuperscript{#1}} % Superscript
\newcommand{\tsub}[1]{\textsubscript{#1}} % Superscript
\newcommand{\markyellow}[1]{\colorbox{yellow}{#1}} % Yellow highlighter
\newcommand{\ques}{\fakesuperscript{?}} % raised question mark
\newcommand{\zwsp}{\mbox{​}} % Zero-width space (ZWSP)

\newcommand{\ayr}[1]{\zwsp\smash{{\Tagati #1}}} % Plain Ayeri orthography
\newcommand{\rayr}[2]{\zwsp\smash{{\Tagati #1}} \emph{#2}} % Ayeri orthography + *r*omanization
\newcommand{\tayr}[2]{\emph{#1} `#2'} % Romanization + *t*ranslation
\newcommand{\xayr}[3]{\zwsp\smash{\Tagati #1} \emph{#2} `#3'} % Ayeri orthography + romanization + translation

% \newenvironment{ayeri}{
%     %\hyphenpenalty=10000
%     %\hbadness=10000
%     \doublespacing
%     \begin{multicols}{2}
%     \Tagati
% }{
%     \end{multicols} \par
% }

\newenvironment{mytitle}{
    \hfill
    \begin{minipage}{0.667\textwidth}
	\vspace{\baselineskip}
	\begin{center}
	    \Large
	    \sffamily\bfseries
	    \makeatletter
}{
	    \makeatother
	\end{center}
	\vspace{1em}
    \end{minipage}
    \hfill
}

% blah
\usepackage{lipsum}

%% END OF PREAMBLE %%%%%%%%%%%%%%%%%%%%%%%%%%%%%%%%%%%%%%%%%%%%%%%%%%%%%%%%%%%%%

\begin{document}

%% MAIN PART %%%%%%%%%%%%%%%%%%%%%%%%%%%%%%%%%%%%%%%%%%%%%%%%%%%%%%%%%%%%%%%%%%%

\begin{mytitle}
    \@title: \@subtitle\footnotemark
\end{mytitle}
\footnotetext{\cite[Vgl.][]{buch2016ss}.}

\section{Text der Aufgabe}

\begin{enumerate}[noitemsep]
\item Ein hungriger Fuchs kam einst in ein Dorf. Er sprach zu einem Hahn: \enquote{Lass mich Deine schöne Stimme hören!}
\item Der stolze Hahn schloss seine Augen und krähte laut. Da schnappte der Fuchs ihn und trug ihn in den Wald.
\item Als die Bauern das merkten, liefen sie dem Fuchs nach und riefen: \enquote{Der Fuchs trägt unseren Hahn fort!}
\item Da sprach der Hahn zum Fuchs: \enquote{Sag ihnen: \enquote{Ich trage meinen Hahn und nicht euren!}}
\item Der Fuchs ließ den Hahn aus dem Maul und rief: \enquote{Ich trage meinen Hahn und nicht euren!}
\item Der Hahn aber flog schnell auf einen Baum. Der Fuchs schalt sich selbst einen Narren und trottete davon.
\end{enumerate}

(Nach Äsop)

\section{Übersetzung}

\pex % 1
\a\begingl
	\gla Mə-bahisya, ang sahaya runay mabo minkayya. //
	\glb Mə=bahis-ya ang saha-ya runay-Ø mabo minkay-ya //
	\glc irgend=Tag-\Loc{} \AgtT{} kommen-\TsgM{} Fuchs-\Top{} hungrig Dorf-\Loc{} //
	\glft \enquote{Eines Tages kam ein hungriger Fuchs an ein Dorf.} //
\endgl

\a\label{ex:umformuliert}\begingl[glspace=0.25em]
	\gla Ang naraya aguyanya: Garu, sa ming tangyang kadāre sekay veno vana! //
	\glb Ang nara=ya.Ø aguyan-ya Gara-u sa ming tang=yang kadāre sekay-Ø veno vana //
	\glc \AgtT{} sprechen=\TsgM{}.\Top{} Hahn-\Loc{} rufen-\Imp{} \PatT{} können hören=\Fsg{}.\Aarg{} damit Stimme-\Top{} schön \Ssg{}.\Gen{} //
	\glft \enquote{Er sprach zu einem Hahn: \enquote{Rufe, damit ich deine schöne Stimme hören kann!}} //
\endgl

\xe

In dieser Fabel wird der Fuchs als erstes in den Diskurs eingeführt und er behält auch zunächst die Hauptrolle, deswegen bildet er die Topik. Das Wort \xayr{runj}{runay}{Fuchs} wurde dabei neu gebildet, in unregelmäßiger Ableitung von \xayr{Aruno}{aruno}{braun}.\footnote{Man könnte hier genauso gut auch \xayr{venej}{veney}{Hund} verwenden, um eine Neubildung zu vermeiden. Da es in Ayeri an kulturellem Kontext mangelt, habe ich mich entschieden, die Tiere wie im Original zu belassen. Der Fairness halber habe ich es weitestgehend vermieden, neue Wörter einzuführen.} Die Bewegungsrichtung ist durch das Verb \xayr{sh/}{saha-}{kommen} mehr oder weniger eindeutig angegeben, daher kann das Dorf, \rayr{miMkj}{minkay}, im Lokativ stehenbleiben; wenn man das \fw{zu} oder \fw{in} genauer bestimmen möchte, kämen auch der Dativ \rayr{miMkjymF}{minkayyam} oder der präpositionale Ausdruck \xayr{mN koNF miMkjy}{manga kong minkayya}{in ein Dorf hinein} (\Dyn{} in Dorf-\Loc{}) in Frage. Ayeri unterscheidet außerdem nicht zwischen Präsens und epischem Präteritum, weswegen alle Verben ohne Tempusmarkierung erscheinen. Bei den Rückübersetzungen habe ich der Konvention halber trotzdem das Präteritum gewählt.

Der Aufforderungssatz ist im Original kausativ formuliert (\enquote{Lass mich [...] hören}), doch kann Ayeri keine morphologischen Imperative im Kausativ bilden, da das Imperativsuffix \rayr{/U}{-u} nicht zur Verfügung steht -- \rayr{tNu}{tangu} würde nicht `lasse hören' bedeuten, sondern `höre'; ein kausatives Hilfsverb ist nicht vorgesehen. Wenn man die Kausativstruktur beibehalten möchte, muss man den Imperativ also umschreiben. Eine wörtlichere Übersetzung des Satzes oben zeigt das folgende Beispiel; \fw{lasse hören} ist hier im Prinzip umformuliert zu \fw{ich soll hören}:

\exdisplay[everygl=\hspace*{1em}]\noexno
\begingl
	\gla Rī mya tangyang sekayas veno vana va! //
	\glb Rī mya tang=yang sekay-as veno vana va.Ø //
	\glc \CauT{} sollen hören=\Fsg{}.\Aarg{} Stimme-\Parg{} schön \Ssg{}.\Gen{} \Ssg{}.\Top{} //
	\glft \enquote{Dass du mich deine schöne Stimme hören lassen mögest!},\\
		oder wörtlich: \enquote{Deinetwegen soll ich deine schöne Stimme hören} //
\endgl
\xe

Bei der syntaktischen Kaustivkonstruktion handelt es sich im Grunde eine Grammatikalisierung der Kausativ-Topik, sodass die anstiftende Agens als Topik markiert ist, die anderen Kasusrelationen aber nicht um eine Stufe zurückgesetzt werden \autocite[vgl.][Kap. 8.2]{comrie1989}. Die untergeordnete, ausführende Agens bleibt als Agens markiert, Patiens bleibt Patiens etc. Während hier die Aufforderung indirekt an den Hahn gerichtet wird, ist sie in dem in (\ref{ex:umformuliert}) präsentierten Satz dagegen direkt: Der Fuchs sagt nicht \enquote{Lass mich [...] hören} sondern \enquote{Rufe}. Der Zweck der Handlung kann in einem Nebensatz ausgedrückt werden. Diese Formulierung scheint mir etwas natürlicher, da sie weniger kompliziert ist. Im umformulierten Satz schien mir außerdem die \enquote{schöne Stimme} als die markanteste Information des Satzes, sodass ich diesen Satzteil topikalisiert habe, wenn auch eine erste Person \enquote*{belebter} ist als eine dritte.\footnote{\textcite[197--199]{comrie1989} diskutiert \fw{topic-worthiness} im Gegensatz zur Belebtheit.}

Im folgenden Satz wechselt der Blickwinkel zum Hahn, der aufgrund des Erzählflusses auch im zweiten Teil die Topik bildet. Entsprechend habe ich den zweiten Teil mit passiven Verbformen zurückübersetzt. Durch die Markierung der Patiens als Topik und eine an semantischen Kategorien orientierte Kasusmarkierung kann die Grundstruktur des Satzes beibehalten werden. Eine Konversion des Akkusativobjekts (Patiens) ist in Ayeri also nicht nötig. Verglichen mit dem Deutschen übernimmt die Topikmarkierung sozusagen die Rolle der Markierung der Patiens als Subjekt; eine Markierung der ursprünglichen Patiens als Agens wäre logischerweise nicht grammatikalisch korrekt.\footnote{Topik und semantisch-syntaktische Kernrollen sind in Ayeri einigermaßen unabhängig voneinander. Die Agens ist nicht strikt mit einem Subjekt im Nominativ gleichzusetzen.} Die Agens wird dabei auch nicht als instrumentale Adverbiale ausgedrückt, sondern verbleibt als Agens im Satz.\footnote{Ayeri kennt darüber hinaus allerdings eine \enquote*{echtere} Art Passiv, insofern als die Agens-NP weggelassen werden kann. Diese wird dann aber auch nicht als Instrumental wieder in den Satz eingeführt. In diesen Fällen kongruiert das Verb mit der Patiens-NP oder das Patiens-Pronomen tritt anstelle des Agens-Pronomens als Enklitikum an den Verbstamm heran.}

\pex % 2
\a\label{ex:itrvb}\begingl
	\gla Ang rimaya aguyan viyu nivajas yana nay garayāng baho. //
	\glb Ang rima-ya aguyan-Ø viyu niva-ye-as yana nay gara=yāng baho //
	\glc \AgtT{} schließen-\TsgM{} Hahn-\Top{} stolz Auge-\Pl{}-\Parg{} \TsgM{}.\Gen{} und rufen=\TsgM{}.\Aarg{} laut //
	\glft \enquote{Der stolze Hahn schloss seine Augen und rief laut.} //
\endgl

\a\label{ex:coordvb}\begingl
	\gla Sa da-kacisaya runayang ya nay sa ninyāng ya manga kong vinimya. //
	\glb Sa da=kacisa-ya runay-ang ya.Ø nay sa nin=yāng ya.Ø manga kong vinim-ya //
	\glc \PatT{} so=packen-\TsgM{} Fuchs-\Aarg{} \TsgM{}.\Top{} und \PatT{} tragen=\TsgM{}.\Aarg{} \TsgM{}.\Top{} \Dyn{} in Wald-\Loc{} //
	\glft \enquote{Da wurde er vom Fuchs gepackt und er wurde von ihm in den Wald getragen.} //
\endgl

\xe

Bisher gab es keine expliziten Regeln zur Kongruenz bei Koordination, aber durch Gebrauch hat sich ergeben, dass es bei koordinierten Verb\emph{phrasen} nicht möglich ist, die Topikmarkierung und ein sonst klitisches Agenspronomen wegzulassen und letzteres durch einfache Kongruenzmarkierung zu ersetzen \autocite[vgl.][288]{zwicky1985}, daher muss das Verb in der zweiten Hälfte von (\ref{ex:coordvb}) \rayr{s ninFyaaNF}{sa ninyāng} lauten, nicht einfach *\rayr{ninY}{*ninya}. In der zweiten Hälfte von (\ref{ex:itrvb}) fällt bei \rayr{gryaaNF}{garayāng} die Topikmarkierung weg, da das Verb intransitiv gebraucht wird. Auch im dritten Teil ist Koordination von Verbphrasen anzutreffen:

\pex % 3
\a\begingl
	\gla Tadayya si ang kengyan bedangye adaley, ang nimpyan manga pang runayya nay bahatang: //
	\glb Taday-ya si ang keng-yan bedang-ye-Ø ada-ley ang nimp=yan.Ø manga pang runay-ya nay nay baha=tang //
	\glc Zeit-\Loc{} \Rel{} \AgtT{} bemerken-\TplM{} Bauer-\Pl{}-\Top{} jenes-\PargI{} \AgtT{} rennen=\TplM{}.\Top{} \Dyn{} hinter Fuchs-\Loc{} und schreien=\TplM{}.\Aarg{} //
	\glft \enquote{Als die Bauern das bemerkten, rannten sie dem Fuchs hinterher und sie schrien:} //
\endgl

\a\begingl
	\gla Ang manga pahya runay aguyanas nana! //
	\glb Ang manga pah-ya runay-Ø aguyan-as nana //
	\glc \AgtT{} \Prog{} wegnehmen-\TsgM{} Fuchs-\Top{} Hahn-\Parg{} \Fsg{}.\Gen{} //
	\glft \enquote{Der Fuchs nimmt gerade unseren Hahn fort!} //
\endgl

\xe

Bei diesem Satz ist des Weiteren anzumerken, dass Ayeri, anders als zum Beispiel das Deutsche, nur sehr wenige um Präpositionen erweiterte Verben kennt. Dies äußert sich zum einen darin, dass `wegnehmen' und `nehmen' verschiedene (allerdings wahrscheinlich verwandte) Verben sind: \rayr{phF/}{pah-} und \rayr{p/}{pa-}. Zum anderen ist die Verbindung zwischen Verb und Präpositionalphrase bei dem Ausdruck \rayr{niMpF/— mN pNF ArilinFy}{nimp-... manga pang arilinya} (wörtlich läuft man `zum Rücken von') tendenziell weniger fest gefügt als im Deutschen.

\rayr{mN}{manga} bei Verben und bei Präpositionen sind miteinander verwandt und fügen beiden eine dynamische Bedeutung zu, die beim Verb als Progressiv grammatikalisiert ist. Das Progressiv ist aber nicht generell obligatorisch, sondern dient mehr der Betonung einer andauernden Handlung zum gegenwärtigen Zeitpunkt, ähnlich der Verlaufsform mit \fw{am} im Deutschen.

Der folgende, vierte Part ist morphosyntaktisch potentiell interssant, da er ineinander geschachtelte wörtliche Rede enthält. Dies wird in Ayeri allerdings sehr unkompliziert gehandhabt, insofern es keine Morpheme gibt, die Quotative, Evidentialität oder gar Egophorizität markieren. Syntaktische Effekte ergeben sich auch keine besonderen, da die wörtliche Rede, wie im Deutschen auch, einfach angehängt wird.

\pex % 4
\a\begingl
	\gla Nay ang naraya aguyan runayya: Ningu cam: //
	\glb Nay ang nara-ya aguyan-Ø runay-ya Ning-u cam //
	\glc und \AgtT{} sprechen-\TsgM{} Hahn-\Top{} Fuchs-\Loc{} sagen-\Imp{} \TplM{}.\Dat{} //
	\glft \enquote{Und der Hahn sprach zum Fuchs: \enquote{Sage ihnen:}} //
\endgl

\a\label{ex:negativbindung}\begingl
	\gla Sa ninyang aguyan nā; ninoyyang da-vana. //
	\glb Sa nin=yang aguyan-Ø nā nin-oy=yang da=vana //
	\glc \PatT{} tragen=\Fsg{}.\Aarg{} Hahn-\Top{} \Fsg{}.\Gen{} tragen-\Neg{}=\Fsg{}.\Aarg{} so=\Spl{}.\Gen{} //
	\glft \enquote{Ich trage meinen Hahn; ich trage nicht den euren.} //
\endgl

\xe

In (\ref{ex:negativbindung}) muss in der zweiten Satzhälfte wieder die volle Verbform stehen, da Negation in Ayeri durch ein gebundenes Suffix \rayr{/Oj}{-oy} und nicht durch ein freies Adverb geschieht (\xayr{voj}{voy}{nein} tritt nur in prädikativen NPs in der Bedeutung `nicht' auf). Possessivpronomen werden zwar wie Adjektive behandelt, doch kann das Negativsuffix nicht an den Stamm eines Personalpronomens herantreten (Adjektive sind negierbar). Interessant dürfte außerdem sein, dass das indefinite Demonstrativpräfix \rayr{d/}{da-} proklitisch an das Possessivpronomen tritt, im Grunde, um das Pronomen zu nominalisieren. Das Possessivpronomen erhält allerdings in diesem Fall -- wie ein Adjektiv auch -- keine zusätzliche Kasusmarkierung als Patiens (Adjektive sind indeklinabel).

Das fünfte Satzpaar wiederholt einen Teil des vierten. Satz (\ref{ex:genabl}) zeigt den Genitiv in seiner erweiterten Funktion als Ablativ. Eine Präposition wird nicht unbedingt benötigt, um `aus dem Maul' auszudrücken, die Markierung der NP mit dem Genitiv reicht aus.

\pex % 5
\a\label{ex:genabl}\begingl
	\gla Ang bomya runay aguyanas bantana yana nay garayāng: //
	\glb Ang bom-ya runay-Ø aguyan-as banta-na yana nay gara=yāng //
	\glc \AgtT{} freilassen-\TsgM{} Fuchs-\Top{} Hahn-\Parg{} Maul-\Gen{} \TsgM{}.\Gen{} und rufen=\TsgM{}.\Aarg{} //
	\glft \enquote{Der Fuchs ließ den Hahn aus seinem Maul frei und er rief:} //
\endgl

\a\begingl
	\gla Sa ninyang aguyan nā; ninoyyang da-vana. //
	\glb Sa nin=yang aguyan-Ø nā nin-oy=yang da=vana //
	\glc \PatT{} tragen=\Fsg{}.\Aarg{} Hahn-\Top{} \Fsg{}.\Gen{} tragen-\Neg{}=\Fsg{}.\Aarg{} so=\Spl{}.\Gen{} //
	\glft \enquote{Ich trage meinen Hahn; ich trage nicht den euren.} //
\endgl

\xe

Neben der Genitivform ist es möglich, eine explizitere Formulierung mit einer PP zu verwenden. In diesem Fall bietet sich die Präposition \xayr{AgonnF}{agonan}{außerhalb von} an, die um die Partikel \rayr{mN}{manga} erweitert wird, um eine Bewegung in diese Richtung anzuzeigen. Das Präpositionalobjekt steht dabei konventionell im Lokativ: \xayr{mN AgonnF bMty}{manga agonan bantaya}{nach außerhalb des Mundes} (\Dyn{} außen Mund-\Loc{}). Das Verb des Satzes sollte in diesem Fall allerdings eher \xayr{rmYF/}{ramy-}{lassen} lauten.

Im sechsten und letzten Teil ist der erste Satz nicht weiter bemerkenswert. Der zweite Satz ist allerdings aufgrund seiner Objektsprädikativ-Konstruktion interessant.

\pex % 6
\a\begingl
	\gla Ang nunaya para nārya aguyan manga ling mehirya. //
	\glb Ang nuna-ya para nārya aguyan-Ø manga ling mehir-ya //
	\glc \AgtT{} fliegen-\TsgM{} schnell jedoch Hahn-\Top{} \Dyn{} auf Baum-\Loc{} //
	\glft \enquote{Der Hahn flog aber schnell auf einen Baum.} //
\endgl

\a\label{ex:objpred}\begingl
	\gla Sitang-gasiya runayang, yāng depangas, nay lampyāng mangasara. //
	\glb Sitang=gasi-ya runay-ang yāng depang-as nay lamp=yāng mangasara //
	\glc \Refl{}=schelten-\TsgM{} Fuchs-\Aarg{} \TsgM{}.\Aarg{} Narr-\Parg{} und laufen=\TsgM{}.\Aarg{} weg //
	\glft \enquote{Der Fuchs schalt\footnotemark{} sich, er sei ein Narr, und er lief davon.} //
\endgl

\xe

\footnotetext{Bei dem Verb \xayr{gsi/}{gasi-}{schelten, schimpfen} handelt es sich um eine Neubildung.}

Bei adjektivischen Objektsprädikativen besteht in Ayeri tendenziell die Schwierigkeit, ein prädikatives Adjektiv von einem deskriptiven zu unterscheiden: (i) \fw{Er malt die Wand weiß} ist nicht dasselbe wie (ii) \fw{Er malt die weiße Wand}. Da in Ayeri das Adjektiv dem Substantiv nachgestellt wird, sind bei strikt logischer Abfolge der Konstituenten (i) und (ii) nicht unterscheidbar; beide würden lauten:

\exdisplay[everygl=\hspace*{1em}]\noexno
\begingl
	\gla Ang vitaya merengley maka. //
	\glb Ang vita=ya.Ø mereng-ley maka //
	\glc \AgtT{} anmalen=\TsgM{}.\Top{} Wand-\PargI{} weiß //
\endgl
\xe

Das prädikative Adjektiv steht daher in diesen Fällen im Unterschied zum deskriptiven zwischen Verb und Substantiv:

\exdisplay[everygl=\hspace*{1em}]\noexno
\begingl
	\gla Ang vitaya maka merengley. //
	\glb Ang vita=ya.Ø maka mereng-ley //
	\glc \AgtT{} anmalen=\TsgM{}.\Top{} weiß Wand-\PargI{} //
	\glft `Er malt die Wand weiß.' //
\endgl
\xe

Bei nominalen Objektprädikativen besteht die Schwierigkeit, dass die objektsprädikative NP logischerweise als Patiens markiert werden müsste, Ayeri Agens und Patiens aber nur einmal pro Satz vergibt. Wenn der Fuchs sich also einen Narren schilt, dann sind Agens und Patiens schon durch \fw{Fuchs} und \fw{sich} besetzt. Die Lösung besteht darin, die prädikative NP als Nebensatz auszudrücken, wie in (\ref{ex:objpred}) demonstriert: \fw{Der Fuchs schalt sich, dass er ein Narr sei}. Zu (\ref{ex:objpred}) ist darüber hinaus anzumerken, dass Ayeri bei prädikativen Nominalen keine Kopula verwendet und die prädikative NP -- wenn deklinabel -- als Patiens markiert, daher steht hier \rayr{depNsF}{depangas} (Narr-\Parg{}) anstelle von *\rayr{depNNF}{*depangang} (Narr-\Aarg{}).

Außerdem werden reflexive Personalpronomen normalerweise gebildet, indem das reflexive Präfix \rayr{sitNF/}{sitang-} an das Pronomen tritt, zum Beispiel: \xayr{sitNF/yaamF}{sitang-yām}{für mich/(zu) mir selbst} (\Refl{}=\Fsg{}.\Dat{}). Bei einer reflexiven Patiens ist es jedoch möglich, das reflexive Präfix ans Verb zu hängen, sodass der erste Satzteil wie in (\ref{ex:objpred}) lauten könnte, aber auch die folgende Formulierung möglich ist:

\exdisplay[everygl=\hspace*{1em}]\noexno
\begingl
	\gla Ang gasiya runay sitang-yās, ... //
	\glb Ang gasi-ya runay-Ø sitang=yās ... //
	\glc \AgtT{} schelten-\TsgM{} Fuchs-\Top{} \Refl{}=\TsgM{}.\Parg{} ... //
\endgl
\xe

\section{Text in Ayeri}

\tabulinesep=0.5em
\begin{longtabu} to \linewidth {X[1r] X[12] X[12l]}

1.
	& Mə-bahisya, ang sahaya runay mabo minkayya. Ang naraya aguyanya: \enquote{Garu, sa ming tangyang kadāre sekay veno vana!}
	& {\Tagati me/bhisFy, ANF shy runj mbo miMkjy. ANF nry AguynFy — gru, s miNF tNFyNF kdaare sekj veno vn!}
	\\
	
2.
	& Ang rimaya aguyan viyu nivajas yana nay garayāng baho. Sa da-kacisaya runayang ya nay sa ninyāng ya manga kong vi\-nim\-ya.
	& {\Tagati ANF rimy AguynF viyu nivye\_asF yn nj gryaaNF bho. s d/ktYisy runyNF y nj s ninFyaaNF y mN koNF vinimFy.}
	\\
	
3.
	& Tadayya si ang kengyan bedangye adaley, ang nimpyan manga pang runayya nay bahatang: \enquote{Ang manga pahya runay aguyanas nana!}
	& {\Tagati tdjy si ANF keNFynF bedNFye Adlej, ANF niMpFynF mN pNF runjy nj bhtNF — ANF mN phFy runj AguynsF nn!}
	\\
	
4.
	& Nay ang naraya aguyan runayya: \enquote{Ningu cam: \enquote{Sa ninyang aguyan nā; ninoyyang da-vana.}}
	& {\Tagati nj ANF nry AguynF runjy – niNu tYmF – s ninFyNF AguynF naa – ninojyNF d/vn.}
	\\
	
5.
	& Ang bomya runay aguyanas bantana yana nay garayāng: \enquote{Sa ninyang aguyan nā; ninoyyang da-vana.}
	& {\Tagati ANF bomFy runj AguynsF bMtn yn nj gryaaNF – s ninFyNF AguynF naa – ninojyNF d/vn.}
	\\
	
6.
	& Ang nunaya para nārya aguyan manga ling mehirya. Sitang-gasiya runayang, yāng depangas, nay lampyāng mangasara.
	& {\Tagati ANF nuny pr naarY AguynF mN liNF mehirFy. sitNF/gsiy runyNF, yaaNF depNsF, nj lMpFyaaNF mNsr.}
	\\

\end{longtabu}

%% BIBLIOGRAPHY %%%%%%%%%%%%%%%%%%%%%%%%%%%%%%%%%%%%%%%%%%%%%%%%%%%%%%%%%%%%%%%%

%\vfill
\clearpage

\addsec{Abkürzungen}
\begin{multicols}{3}%
\printglossary[style=mysuper,type=\leipzigtype]
\end{multicols}

\addsec{Literaturverzeichnis}
\printbibliography[heading=none]

\end{document}
