\documentclass[12pt,paper=a4]{scrartcl}

% Author, Title, Subtitle etc.
\author{Carsten Becker}
\title{SE Kunst- und Plansprachen -- von Esperanto bis Dothraki}
\subtitle{Übersetzungsaufgabe}
\date{\today} % Format: YYYY/MM/DD (rev. N, YYYY/MM/DD)

% Provide running author and date
\makeatletter
\let\runauthor\@author
\let\rundate\@date
\makeatother

% Handle language and quotation marks
\usepackage{polyglossia}
\setdefaultlanguage{german}
\setotherlanguage{english}
\usepackage{csquotes} % Put quotations in \enquote{}!
\SetBlockEnvironment{quotation}
\renewcommand*{\mkccitation}[1]{ (#1)}

% Set all margins to 2.54 cm
\usepackage[margin=1in]{geometry}
\widowpenalty10000 % Avoid widows like the plague!
\clubpenalty10000 % Avoid orphans like the plage, too!

% Extended formatting of lists
\usepackage{enumitem}
\newlist{glossdefs}{itemize}{1}
\setlist[glossdefs]{nosep, leftmargin=3em, labelwidth=2.5em, align=left}
\setlist[itemize]{noitemsep}

% Make multiple columns available in single-column document
\usepackage{multicol}

% Make text colors and color names available
\usepackage[xetex]{xcolor}

% Load font stuff for XeTeX
\usepackage{xltxtra}
\usepackage{fontspec}

\newfontfamily{\Tagati}[
    Renderer=Graphite,
    Scale=1.1,
    BoldFont={* Italic},
    HyphenChar=·,
]{Tagati Book G}

\setmainfont[
    Ligatures=TeX,
    Numbers=Lowercase,
]{Junicode}

\setsansfont[
    Ligatures=TeX,
    Numbers=Lowercase,
    Scale=MatchUppercase,
    BoldFont={Open Sans Condensed Bold},
]{Open Sans Condensed Light}

% Load BibLaTeX (using Biber), configure citation styles
\usepackage[
    authordate-trad,
    backend=biber,
    safeinputenc,
]{biblatex-chicago}
\addbibresource{bibliografie.bib}

% To make \textcite look like "Doe (2014: 213)"
\renewcommand*{\postnotedelim}{\addcolon\addspace}
\DeclareFieldFormat{postnote}{#1}
\DeclareFieldFormat{multipostnote}{#1}

% Enable generating files from this .tex file (for the bibliography) 
\usepackage{filecontents}

% Clickable links in footnotes, TOC, etc.
\usepackage[
    xetex,
    bookmarks=true,
    colorlinks=false,
    linktoc=section,
    hidelinks,
    pdfusetitle,
]{hyperref}

\usepackage{url}
\urlstyle{rm}

% Ability to include graphics and dealing with footnotes in descriptions
% \usepackage{graphicx}
% \usepackage[font={small,sf},labelfont={small,sf},format=plain]{caption}
% \usepackage{subcaption}
% \usepackage{wrapfig}
% \setlength{\columnsep}{2\baselineskip}

% General headers and footers
\usepackage{fancyhdr}
\pagestyle{fancy}

\fancyhead[L]{} % empty
\fancyhead[C]{} % empty
\fancyhead[R]{\thepage}

\fancyfoot[L]{} % empty
\fancyfoot[C]{} % empty
\fancyfoot[R]{} % empty

\renewcommand{\headrulewidth}{0pt}
\renewcommand{\footrulewidth}{0pt}

% First page headers and footers are different
\fancypagestyle{firstpage}{
    \fancyhead[L]{\sffamily \footnotesize \textbf{Benung. The Ayeri Language Resource}}
    \fancyhead[C]{} % empty
    \fancyhead[R]{\sffamily \footnotesize \runauthor{} · \rundate{}}
    
    \fancyfoot[L]{} % empty
    \fancyfoot[C]{} % empty
    \fancyfoot[R]{\sffamily \footnotesize 
	\href{https://benung.nfshost.com}{https://benung.nfshost.com} · 
	\href{https://github.com/carbeck/benung-pdfs}{https://github.com/carbeck/benung-pdfs} · 
	\href{https://creativecommons.org/licenses/by-sa/4.0/}{CC~BY-SA~4.0}
    }
    
    \renewcommand{\headrulewidth}{0.5pt}
    \setlength\footskip{0.5in}
}

\usepackage{ifthen}
\ifthenelse{\value{page}=1}{\thispagestyle{firstpage}{\pagestyle{fancy}}}

% Line spacing
\usepackage{setspace}
\onehalfspacing

% Avoid pagebreaks right after sections and subsections
\usepackage{needspace}
\usepackage{etoolbox}
\preto\section{\needspace{6\baselineskip}}
\preto\subsection{\needspace{6\baselineskip}}

% Smaller font in block quotes
\usepackage{relsize}
\AtBeginEnvironment{quote}{\noindent\smaller}
\AtBeginEnvironment{quotation}{\smaller}

% % Things for tables
% \usepackage{longtable}
% \usepackage{tabu}
% \usepackage{booktabs}
% \usepackage{rotating}

% Column types
% \newcolumntype{S}{>{\scshape}X}

% Formatting of table of glossing abbreviations from Leipzig package manual
\usepackage[acronym,nomain,nonumberlist,nopostdot]{glossaries}
\usepackage{glossary-inline}%

\newglossarystyle{mysuper}{%
	\glossarystyle{super}% based on super
	\renewenvironment{theglossary}{%
		\begin{glossdefs}%
	}{%
		\end{glossdefs}%
	}%
	\renewcommand*{\glossaryheader}{}%
	\renewcommand*{\glsgroupheading}[1]{}%
	\renewcommand*{\glossaryentryfield}[5]{%
		\item[\glsentryitem{##1}\glstarget{##1}{##2}]%
		\makefirstuc{##3}\glspostdescription{}%
	}%
	\renewcommand*{\glsgroupskip}{}%
}%

% Tree diagrams
\usepackage[linguistics]{forest}

% Formatting of glosses
\usepackage{expex}
\lingset{Everyex=\smaller}
\usepackage{leipzig}

\newleipzig{Aarg}{a}{Agens}
\newleipzig{Parg}{p}{Patiens}
\newleipzig{Dat}{dat}{Dativ}
\newleipzig{Gen}{gen}{Genitiv}
\newleipzig{Loc}{loc}{Lokativ}
\newleipzig{Caus}{caus}{Kausativ}
\newleipzig{Top}{top}{Topik}

\newleipzig{AgtT}{at}{Agens-Topik}
\newleipzig{PatT}{pt}{Patiens-Topik}
\newleipzig{DatT}{datt}{Dativ-Topik}
\newleipzig{GenT}{gent}{Genitiv-Topik}
\newleipzig{LocT}{loct}{Lokativ-Topik}
\newleipzig{InsT}{inst}{Instrumental-Topik}
\newleipzig{CauT}{caut}{Kausativ-Topik}

\newleipzig{First}{1}{erste Person}
\newleipzig{Second}{2}{zweite Person}
\newleipzig{Third}{3}{dritte Person}
\newleipzig{M}{m}{maskulin}
\newleipzig{F}{f}{feminin}
\newleipzig{N}{n}{neutrum}
\newleipzig{An}{an}{belebt}
\newleipzig{Inan}{inan}{unbelebt}

\newleipzig{Hab}{hab}{Habitativ}
\newleipzig{Neg}{neg}{Negativ}
\newleipzig{Irr}{irr}{Irrealis}
\newleipzig{Imp}{imp}{Imperativ}
\newleipzig{Dyn}{dyn}{dynamisch}

\newleipzig{NPst}{npst}{nahe Vergangenheit}

\makeglossaries

\newcommand{\AargI}{{\Aarg}.{\Inan}}
\newcommand{\PargI}{{\Parg}.{\Inan}}
\newcommand{\AgtTI}{{\AgtT}.{\Inan}}
\newcommand{\PatTI}{{\PatT}.{\Inan}}

\newcommand{\TsgM}{{\Tsg}.{\M}}
\newcommand{\TsgF}{{\Tsg}.{\F}}
\newcommand{\TsgN}{{\Tsg}.{\N}}
\newcommand{\TsgI}{{\Tsg}.{\Inan}}
\newcommand{\TplM}{{\Tpl}.{\M}}
\newcommand{\TplF}{{\Tpl}.{\F}}
\newcommand{\TplN}{{\Tpl}.{\N}}
\newcommand{\TplI}{{\Tpl}.{\Inan}}

% Nicer footnotes
\usepackage[bottom,hang,norule]{footmisc}
\setlength{\footnotesep}{0.75\baselineskip}

% Date and time
\usepackage[
	datesep=.,
	style=ddmmyyyy,
]{datetime2}

% Macros
\newcommand{\fw}[1]{\textit{#1}} % Foreign Word
\newcommand{\tit}[1]{\textit{#1}} % Title of a work
\newcommand{\q}[1]{\enquote{#1}} % Context-aware quotation
\newcommand{\qq}[1]{\enquote*{#1}} % Explicit sublevel quotation
\newcommand{\tsup}[1]{\textsuperscript{#1}} % Superscript
\newcommand{\tsub}[1]{\textsubscript{#1}} % Superscript
\newcommand{\markyellow}[1]{\colorbox{yellow}{#1}} % Yellow highlighter
\newcommand{\ques}{\fakesuperscript{?}} % raised question mark
\newcommand{\zwsp}{\mbox{​}} % Zero-width space (ZWSP)

\newcommand{\ayr}[1]{\zwsp\smash{{\Tagati #1}}} % Plain Ayeri orthography
\newcommand{\rayr}[2]{\zwsp\smash{{\Tagati #1}} \emph{#2}} % Ayeri orthography + *r*omanization
\newcommand{\tayr}[2]{\emph{#1} `#2'} % Romanization + *t*ranslation
\newcommand{\xayr}[3]{\zwsp\smash{\Tagati #1} \emph{#2} `#3'} % Ayeri orthography + romanization + translation

% \newenvironment{ayeri}{
%     %\hyphenpenalty=10000
%     %\hbadness=10000
%     \doublespacing
%     \begin{multicols}{2}
%     \Tagati
% }{
%     \end{multicols} \par
% }

\newenvironment{mytitle}{
    \hfill
    \begin{minipage}{0.667\textwidth}
	\vspace{\baselineskip}
	\begin{center}
	    \Large
	    \sffamily\bfseries
	    \makeatletter
}{
	    \makeatother
	\end{center}
	\vspace{1em}
    \end{minipage}
    \hfill
}

% blah
\usepackage{lipsum}

%% END OF PREAMBLE %%%%%%%%%%%%%%%%%%%%%%%%%%%%%%%%%%%%%%%%%%%%%%%%%%%%%%%%%%%%%

\begin{document}

%% MAIN PART %%%%%%%%%%%%%%%%%%%%%%%%%%%%%%%%%%%%%%%%%%%%%%%%%%%%%%%%%%%%%%%%%%%

\begin{mytitle}
    \@title: \@subtitle\footnotemark
\end{mytitle}
\footnotetext{\cite[Vgl.][]{buch2016ss}.}

\section{Text der Aufgabe}
\begin{enumerate}[noitemsep]
\item Ein hungriger Fuchs kam einst in ein Dorf. Er sprach zu einem Hahn: \enquote{Lass mich Deine schöne Stimme hören!}
\item Der stolze Hahn schloss seine Augen und krähte laut. Da schnappte der Fuchs ihn und trug ihn in den Wald.
\item Als die Bauern das merkten, liefen sie dem Fuchs nach und riefen: \enquote{Der Fuchs trägt unseren Hahn fort!}
\item Da sprach der Hahn zum Fuchs: \enquote{Sag ihnen: \enquote{Ich trage meinen Hahn und nicht euren!}}
\item Der Fuchs ließ den Hahn aus dem Maul und rief: \enquote{Ich trage meinen Hahn und nicht euren!}
\item Der Hahn aber flog schnell auf einen Baum. Der Fuchs schalt sich selbst einen Narren und trottete davon.
\end{enumerate}

\section{Übersetzung}

\pex % 1
\a\begingl
	\gla Məbahisya, ang sahaya runay mabo minkayya. //
	\glb Mə=bahis-ya, ang saha-ya runay-Ø mabo minkay-ya //
	\glc irgend=Tag-\Loc{}, \AgtT{} kommen-\TsgM{} Fuchs-\Top{} hungrig Dorf-\Loc{} //
	\glft \enquote{Eines Tages kam ein hungriger Fuchs an ein Dorf.} //
\endgl

\a\begingl
	\gla Ang naraya aguyanya: Rī mya tangyang sekayas veno vana va! //
	\glb Ang nara=ya.Ø aguyan-ya: Rī mya tang=yang sekay-as veno vana va.Ø //
	\glc \AgtT{} sprechen=\TsgM{}.\Top{} Hahn-\Loc{}: \CauT{} sollen hören=\Fsg{}.\Aarg{} Stimme-\Parg{} schön \Ssg{}.\Gen{} \Ssg{}.\Top{} //
	\glft \enquote{Er sprach zu einem Hahn: \enquote{Dass du mich deine schöne Stimme hören lassen sollst!}} //
\endgl

\xe

In dieser Fabel wird der Fuchs als erstes in den Diskurs eingeführt und er behält auch zunächst die Hauptrolle, deswegen bildet er die Topik. Das Wort \xayr{runj}{runay}{Fuchs} wurde dabei neu gebildet, in unregelmäßiger Ableitung von \xayr{Aruno}{aruno}{braun}.\footnote{Man könnte hier genauso gut auch \xayr{venej}{veney}{Hund} verwenden, um eine Neubildung zu vermeiden. Da es in Ayeri an kulturellem Kontext mangelt, habe ich mich entschieden, die Tiere wie im Original zu belassen.} Die Bewegungsrichtung ist durch das Verb \xayr{sh/}{saha-}{kommen} mehr oder weniger eindeutig angegeben, daher kann das Dorf, \rayr{miMkj}{minkay}, im Lokativ stehenbleiben; wenn man das \fw{zu} oder \fw{in} genauer bestimmen möchte, kämen auch der Dativ \rayr{miMkjymF}{minkayyam} oder der präpositionale Ausdruck \xayr{mN koNF miMkjy}{manga kong minkayya}{in ein Dorf} (\Dyn{} in Dorf-\Loc{}) in Frage. Ayeri unterscheidet außerdem nicht zwischen Präsens und epischem Präteritum, weswegen alle Verben unmarkiert bezüglich des Tempus erscheinen. Bei den Rückübersetzungen habe ich der Konvention halber trotzdem das Präteritum gewählt.

Der Aufforderungssatz ist im Original kausativ formuliert (\enquote{Lass mich [...] hören}), doch kann Ayeri keine morphologischen Imperative im Kausativ bilden, da das Imperativsuffix \rayr{/U}{-u} nicht zur Verfügung steht -- \rayr{tNu}{tangu} würde nicht `lasse hören' bedeuten, sondern `höre'. Wenn man die Kausativstruktur beibehalten möchte, muss man den Imperativ also umschreiben. In der Übersetzung oben habe ich dies durch Hinzufügen des Hilfsverbs \xayr{mY}{mya}{sollen} gelöst; wörtlich heißt der Satz \enquote{Deinetwegen, ich soll deine schöne Stimme hören!} Andernfalls ist es natürlich auch möglich, den Satz ohne Kausativ umzuformulieren, zum Beispiel als:

\exdisplay[everygl=\hspace*{1em}]\noexno
\begingl
	\gla Garu, kadāre sa ming tangyang sekay veno vana! //
	\glb Gara-u, kadāre sa ming tang=yang sekay-Ø veno vana //
	\glc rufen-\Imp{}, damit \PatT{} können hören=\Fsg{}.\Aarg{} Stimme-\Top{} schön \Ssg{}.\Gen{} //
	\glft \enquote{Rufe, damit ich deine schöne Stimme hören kann!} //
\endgl
\xe

Hier geht die Aufforderung direkt an den Hahn: Die Aufforderung lautet nicht \enquote{Lass mich [...] hören} sondern \enquote{Rufe}. Der Zweck der Handlung kann in einem Nebensatz ausgedrückt werden. Diese Formulierung ist vielleicht auch etwas natürlicher als die zwar relativ textnähere, doch wesentlich kompliziertere Übersetzung oben. Die \enquote{schöne Stimme} erschien mir als die markanteste Information des Satzes, sodass ich diesen Satzteil topikalisiert habe, wenn auch eine erste Person \enquote*{belebter} ist als eine dritte.\footnote{\textcite[197--199]{comrie1989} diskutiert \fw{topic-worthiness} im Kontrast zu Belebtheit.}

Im folgenden Satz wechselt der Blickwinkel zum Hahn, der aufgrund des Erzählflusses auch im zweiten Teil die Topik bildet. Entsprechend habe ich den zweiten Teil mit passiven Verbformen zurückübersetzt. 

\pex % 2
\a\begingl
	\gla Ang rimaya aguyan viyu nivajas yana nay garayāng baho. //
	\glb Ang rima-ya aguyan-Ø viyu niva-ye-as yana nay gara=yāng baho //
	\glc \AgtT{} schließen-\TsgM{} Hahn-\Top{} stolz Auge-\Pl{}-\Parg{} \TsgM{}.\Gen{} und rufen=\TsgM{}.\Aarg{} laut //
	\glft \enquote{Der stolze Hahn schloss seine Augen und rief laut.} //
\endgl

\a\begingl
	\gla Sa da-kacisaya runayang ya nay sa ninyāng ya manga kong vinimya. //
	\glb Sa da=kacisa-ya runay-ang ya.Ø nay sa nin=yāng ya manga kong vinim-ya //
	\glc \PatT{} so=packen-\TsgM{} Fuchs-\Aarg{} \TsgM{}.\Top{} und \PatT{} tragen=\TsgM{}.\Aarg{} \TsgM{}.\Top{} \Dyn{} in Wald-\Loc{} //
	\glft \enquote{Da wurde er vom Fuchs gepackt und von ihm in den Wald getragen.} //
\endgl

\xe

Bisher gab es keine expliziten Regeln zur Kongruenz bei Koordination, aber sagen wir einfach, dass es bei koordinierten Verb\emph{phrasen} nicht möglich ist, die Topikmarkierung und ein sonst klitisches Agenspronomen wegzulassen und letzteres durch einfache Kongruenzmarkierung zu ersetzen, daher muss das Verb in der zweiten Hälfte des zweiten Satzes \rayr{s ninFyaaNF}{sa ninyāng} lauten, nicht einfach *\rayr{ninY}{ninya}. In der zweiten Hälfte des ersten Satzes fällt die Topikmarkierung weg, da das Verb intransitiv gebraucht wird. Auch im dritten Teil ist Koordination von Verbphrasen anzutreffen:

\pex % 3
\a\begingl
	\gla Tadayya si ang kengyan bedangye adanyaley, ang nimpyan manga pang runayya nay bahatang: //
	\glb Taday-ya si ang keng-yan bedang-ye-Ø adanya-ley, ang nimp=yan.Ø manga pang runay-ya nay nay baha=tang //
	\glc time-\Loc{} \Rel{} \AgtT{} bemerken-\TplM{} Bauer-\Pl{}-\Top{} jenes-\PargI{}, \AgtT{} rennen=\TplM{}.\Top{} \Dyn{} hinter Fuchs-\Loc{} und schreien=\TplM{}.\Aarg{} //
	\glft \enquote{Als die Bauern das bemerkten, rannten sie dem Fuchs hinterher und schrien:} //
\endgl

\a\begingl
	\gla Ang kəpahya runay aguyanas nana! //
	\glb Ang kə-pah-ya runay-Ø aguyan-as nana //
	\glc \AgtT{} \NPst{}-wegnehmen-\TsgM{} Fuchs-\Top{} Hahn-\Parg{} \Fsg{}.\Gen{} //
	\glft \enquote{Der Fuchs hat gerade unseren Hahn fortgenommen!} //
\endgl

\xe

Bei diesem Satz ist des Weiteren anzumerken, dass Ayeri, anders als zum Beispiel das Deutsche, nur sehr wenige um Präpositionen erweiterte Verben kennt. Dies äußert sich zum einen darin, dass `wegnehmen' und `nehmen' verschiedene (allerdings wahrscheinlich verwandte) Verben sind: \rayr{phF/}{pah-} und \rayr{p/}{pa-}. Zum anderen mag der Ausdruck für `jemandem hinterherlaufen' zwar gebräuchlich sein als \rayr{niMpF/— mN pNF ArilinFy}{nimp-... manga pang arilinya} (wörtlich läuft man `im Rücken von'), Verb und Präpositionalphrase sind dabei aber weniger fest gefügt als im Deutschen.

\pex % 4 Da sprach der Hahn zum Fuchs: \enquote{Sag ihnen: \enquote{Ich trage meinen Hahn und nicht euren!}}
\a\begingl
	\gla //
	\glb //
	\glc //
	\glft \enquote{} //
\endgl

\a\begingl
	\gla //
	\glb //
	\glc //
	\glft \enquote{} //
\endgl

\xe

\pex % 5 Der Fuchs ließ den Hahn aus dem Maul und rief: \enquote{Ich trage meinen Hahn und nicht euren!}
\a\begingl
	\gla //
	\glb //
	\glc //
	\glft \enquote{} //
\endgl

\a\begingl
	\gla //
	\glb //
	\glc //
	\glft \enquote{} //
\endgl

\xe

\pex % 6 Der Hahn aber flog schnell auf einen Baum. Der Fuchs schalt sich selbst einen Narren und trottete davon.
\a\begingl
	\gla //
	\glb //
	\glc //
	\glft \enquote{} //
\endgl

\a\begingl
	\gla //
	\glb //
	\glc //
	\glft \enquote{} //
\endgl

\xe

%% BIBLIOGRAPHY %%%%%%%%%%%%%%%%%%%%%%%%%%%%%%%%%%%%%%%%%%%%%%%%%%%%%%%%%%%%%%%%

\vfill

\addsec{Abkürzungen}
\begin{multicols}{3}%
\printglossary[style=mysuper,type=\leipzigtype]
\end{multicols}

\addsec{Literaturverzeichnis}
\printbibliography[heading=none]

\end{document}
