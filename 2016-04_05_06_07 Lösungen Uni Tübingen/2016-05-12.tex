\documentclass[12pt,paper=a4]{scrartcl}

% Author, Title, Subtitle etc.
\author{Carsten Becker}
\title{SE Kunst- und Plansprachen -- von Esperanto bis Dothraki}
\subtitle{Antworten zum 12.05.2016}
\date{05.06.2016} % Format: YYYY/MM/DD (rev. N, YYYY/MM/DD)

% Provide running author and date
\makeatletter
\let\runauthor\@author
\let\rundate\@date
\makeatother

% Handle language and quotation marks
\usepackage[english, ngerman]{babel}
\usepackage{csquotes} % Put quotations in \enquote{}!
\SetBlockEnvironment{quotation}
\renewcommand*{\mkccitation}[1]{ (#1)}

% Set all margins to 2.54 cm
\usepackage[margin=1in]{geometry}
\widowpenalty10000 % Avoid widows like the plague!
\clubpenalty10000 % Avoid orphans like the plage, too!

% Extended formatting of lists
\usepackage{enumitem}
\setlist[itemize]{noitemsep}

% Make multiple columns available in single-column document
\usepackage{multicol}

% Make text colors and color names available
\usepackage[xetex]{xcolor}

% Load font stuff for XeTeX
\usepackage{xltxtra}
\usepackage{fontspec}

\newfontfamily{\Tagati}[
    Renderer=Graphite,
    Scale=1.1,
    BoldFont={* Italic},
    HyphenChar=·,
]{Tagati Book G}

\setmainfont[
    Ligatures=TeX,
    Numbers=Lowercase,
]{EB Garamond}

\setsansfont[
    Ligatures=TeX,
    Numbers=Lowercase,
    Scale=MatchUppercase,
    BoldFont={Open Sans Condensed Bold},
]{Open Sans Condensed Light}

% Load BibLaTeX (using Biber), configure citation styles
\usepackage[
    authordate-trad,
    backend=biber,
    safeinputenc,
]{biblatex-chicago}
\addbibresource{bibliografie.bib}

% To make \textcite look like "Doe (2014: 213)"
\renewcommand*{\postnotedelim}{\addcolon\addspace}
\DeclareFieldFormat{postnote}{#1}
\DeclareFieldFormat{multipostnote}{#1}

% Enable generating files from this .tex file (for the bibliography) 
\usepackage{filecontents}

% Clickable links in footnotes, TOC, etc.
\usepackage[
    xetex,
    bookmarks=true,
    colorlinks=false,
    linktoc=section,
    hidelinks,
    pdfusetitle,
]{hyperref}

\usepackage{url}
\urlstyle{rm}

% Ability to include graphics and dealing with footnotes in descriptions
\usepackage{graphicx}
\usepackage[font={small,sf},labelfont={small,sf},format=plain]{caption}
\usepackage{subcaption}
\usepackage{wrapfig}
\setlength{\columnsep}{2\baselineskip}
\usepackage{tikz}

% General headers and footers
\usepackage{fancyhdr}
\pagestyle{fancy}

\fancyhead[L]{} % empty
\fancyhead[C]{} % empty
\fancyhead[R]{\thepage}

\fancyfoot[L]{} % empty
\fancyfoot[C]{} % empty
\fancyfoot[R]{} % empty

\renewcommand{\headrulewidth}{0pt}
\renewcommand{\footrulewidth}{0pt}

% First page headers and footers are different
\fancypagestyle{firstpage}{
    \fancyhead[L]{\sffamily \footnotesize \textbf{Benung. The Ayeri Language Resource}}
    \fancyhead[C]{} % empty
    \fancyhead[R]{\sffamily \footnotesize \runauthor{} · \rundate{}}
    
    \fancyfoot[L]{} % empty
    \fancyfoot[C]{} % empty
    \fancyfoot[R]{\sffamily \footnotesize 
	\href{http://benung.nfshost.com}{http://benung.nfshost.com} · 
	\href{https://github.com/carbeck/benung-pdfs}{https://github.com/carbeck/benung-pdfs} · 
	\href{https://creativecommons.org/licenses/by-sa/4.0/}{CC~BY-SA~4.0}
    }
    
    \renewcommand{\headrulewidth}{0.5pt}
    \setlength\footskip{0.5in}
}

\usepackage{ifthen}
\ifthenelse{\value{page}=1}{\thispagestyle{firstpage}{\pagestyle{fancy}}}

% Line spacing
\usepackage{setspace}
\onehalfspacing

% Avoid pagebreaks right after sections and subsections
\usepackage{needspace}
\usepackage{etoolbox}
\preto\section{\needspace{6\baselineskip}}
\preto\subsection{\needspace{6\baselineskip}}

% Things for tables
\usepackage{longtable}
\usepackage{tabu}
\usepackage{booktabs}
\usepackage{rotating}

% Column types
\newcolumntype{H}{>{\bfseries\footnotesize}X}

% Formatting of table of glossing abbreviations from Leipzig package manual
\usepackage[acronym,nomain,nonumberlist,nopostdot]{glossaries}
\usepackage{glossary-inline}%

\newglossarystyle{mysuper}{%
	\glossarystyle{super}% based on super
	\renewenvironment{theglossary}{%
		\begin{itemize}[align=left]
	}{%
		\end{itemize}%
	}%
	\renewcommand*{\glossaryheader}{}%
	\renewcommand*{\glsgroupheading}[1]{}%
	\renewcommand*{\glossaryentryfield}[5]{%
		\item[\glsentryitem{##1}\glstarget{##1}{##2}]
		\makefirstuc{##3}\glspostdescription{}
	}%
	\renewcommand*{\glsgroupskip}{}%
}%

% Formatting of glosses
\usepackage{expex}
\usepackage{leipzig}

\newleipzig{AgtT}{at}{agent topic}
\newleipzig{PatT}{pt}{patient topic}
\newleipzig{DatT}{datt}{dative topic}
\newleipzig{GenT}{gent}{genitive topic}
\newleipzig{LocT}{loct}{locative topic}
\newleipzig{InsT}{inst}{instrumental topic}
\newleipzig{CauT}{caut}{causative topic}
\newleipzig{An}{an}{animate}
\newleipzig{Inan}{inan}{inanimate}
\newleipzig{Hab}{hab}{habitative}
\newleipzig{Ayr}{ayr}{Ayeri}

\makeglossaries

% Section formatting
\usepackage{titlesec}
\usepackage{needspace}

\titleformat{\section}{\needspace{6\baselineskip}\sffamily\bfseries}{\noindent\thesection}{0.5\baselineskip}{}

% Nicer footnotes
\usepackage[bottom,hang,norule]{footmisc}
\setlength{\footnotesep}{0.75\baselineskip}

% Smaller font in block quotes
\usepackage{relsize}
\AtBeginEnvironment{quote}{\noindent\smaller}
\AtBeginEnvironment{quotation}{\smaller}

% Macros
\newcommand{\fw}[1]{\textit{#1}} % Foreign Word
\newcommand{\tit}[1]{\textit{#1}} % Title of a work
\newcommand{\q}[1]{\enquote{#1}} % Context-aware quotation
\newcommand{\qq}[1]{\enquote*{#1}} % Explicit sublevel quotation
\newcommand{\tsup}[1]{\textsuperscript{#1}} % Superscript
\newcommand{\markyellow}[1]{\colorbox{yellow}{#1}} % Yellow highlighter
\newcommand{\ques}{\fakesuperscript{?}} % raised question mark

\newcommand{\ayr}[1]{{\Tagati #1}}
\newcommand{\xayr}[3]{{\Tagati #1} \emph{#2} `#3'}

\newenvironment{ayeri}{
    %\hyphenpenalty=10000
    %\hbadness=10000
    \doublespacing
    \begin{multicols}{2}
    \Tagati
}{
    \end{multicols} \par
}

\newenvironment{mytitle}{
    \hfill
    \begin{minipage}{0.667\textwidth}
	\vspace{\baselineskip}
	\begin{center}
	    \Large
	    \sffamily\bfseries
	    \makeatletter
}{
	    \makeatother
	\end{center}
	\vspace{1em}
    \end{minipage}
    \hfill
}

% blah
\usepackage{lipsum}

%% END OF PREAMBLE %%%%%%%%%%%%%%%%%%%%%%%%%%%%%%%%%%%%%%%%%%%%%%%%%%%%%%%%%%%%%

\begin{document}

%% MAIN PART %%%%%%%%%%%%%%%%%%%%%%%%%%%%%%%%%%%%%%%%%%%%%%%%%%%%%%%%%%%%%%%%%%%

\begin{mytitle}
    \@title: \@subtitle\footnotemark
\end{mytitle}
\footnotetext{\fullcite[Vgl.][]{buch2016ss}.}

\section{Konsonanten}
Lateinische und native Graphien für die jeweiligen Phoneme:

\begin{longtabu} to \textwidth {H[2l] X[c] X[c] X[c] X[c] X[c] X[c] X[c] X[c] X[c] X[c] X[c] X[c]}
\toprule
\rowfont{\bfseries\footnotesize}
	%
	& \multicolumn2{c}{Bilabiale}
	& \multicolumn2{c}{Labiodentale}
	& \multicolumn2{c}{Alveolare}
	& \multicolumn2{c}{Palatale}
	& \multicolumn2{c}{Velare}
	& \multicolumn2{c}{Glottale}
	\\

\midrule

Plosive
	& p \ayr{p} & b \ayr{b}	% Bilabiale
	&           &          	% Labiodentale
	& t \ayr{t} & d \ayr{d}	% Alveolare
	&           &          	% Palatale
	& k \ayr{k} & g \ayr{g}	% Velare
	&           &          	% Glottale
	\\

\midrule

Affrikaten
	&                             &                            	% Bilabiale
	&                             &                            	% Labiodentale
	& c\newline\ayr{tY}, \ayr{kY} & j\newline\ayr{dY}, \ayr{gY}	% Alveolare
	&                             &                            	% Palatale
	&                             &                            	% Glottale
	\\

\midrule

Nasale
	&           & m \ayr{m} % Bilabiale
	&           &          	% Labiodentale
	&           & n \ayr{n}	% Alveolare
	&           &          	% Palatale
	&           & ŋ \ayr{N}	% Velare
	&           &          	% Glottale
	\\

\midrule

Frikative
	&           &          	% Bilabiale
	&           & v \ayr{v}	% Labiodentale
	& s \ayr{s} &          	% Alveolare
	&           &          	% Palatale
	&           &          	% Velare
	& h \ayr{h} &          	% Glottale
	\\

\midrule

Taps/Flaps
	&           &          	% Bilabiale
	&           &          	% Labiodentale
	&           & r \ayr{r}	% Alveolare
	&           &          	% Palatale
	&           &          	% Velare
	&           &          	% Glottale
	\\

\midrule

Approximanten
	&           &          	% Bilabiale
	&           &          	% Labiodentale
	&           & l \ayr{l}	% Alveolare
	&           & j \ayr{y}	% Palatale
	&           &          	% Velare
	&           &          	% Glottale
	\\

\bottomrule
\end{longtabu}

\section{Vokale}

Lateinische und native Graphien für die jeweiligen Phoneme (Nullkonsonant \ayr{ʔ} als Träger):

\begin{center}
\begin{tikzpicture}[scale=2]
% cf. http://tex.stackexchange.com/a/171541/60686
\large
\tikzset{
    vowel/.style={fill=white, anchor=mid, text depth=0ex, text height=1ex,}, %font=\small},
    dot/.style={circle,fill=black,minimum size=0.4ex,inner sep=0pt,outer sep=-1pt},
}
\coordinate (hf) at (0,2); % high front
\coordinate (hb) at (2,2); % high back
\coordinate (lf) at (1,0); % low front
\coordinate (lb) at (2,0); % low back
\def\V(#1,#2){barycentric cs:hf={(3-#1)*(2-#2)},hb={(3-#1)*#2},lf={#1*(2-#2)},lb={#1*#2}}

% Draw the horizontal lines first.
\draw (\V(0,0)) -- (\V(0,2));
\draw (\V(1,0)) -- (\V(1,2));
\draw (\V(2,0)) -- (\V(2,2));
\draw (\V(3,0)) -- (\V(3,2));

% Place all the unrounded-rounded pairs next, on top of the horizontal lines.
\path (\V(0,0))     node[vowel, left]  {i \ayr{I}} node[dot] {};
\path (\V(0,2))     node[vowel, right] {u \ayr{U}} node[dot] {};
\path (\V(1,0))     node[vowel, left]  {e \ayr{E}} node[dot] {};
\path (\V(1,2))     node[vowel, right] {o \ayr{O}} node[dot] {};

% Draw the vertical lines.
\draw (\V(0,0)) -- (\V(3,0));
\draw (\V(0,1)) -- (\V(3,1));
\draw (\V(0,2)) -- (\V(3,2));

% Place the unpaired symbols last, on top of the vertical lines.
\path (\V(3  ,1))   node[vowel]        {a \ayr{A}}             ;
\end{tikzpicture}
\end{center}

Daneben gibt es noch die folgenden Diphthonge: au \ayr{AU}, ey \ayr{Ej}, ay 
\ayr{Aj}, oy \ayr{Oj}, uy \ayr{Uj}. Langvokale existieren mit Ausnahme von /aː/ 
lexikalisch nur in wenigen Wörtern, z.\,B. \xayr{niis}{nīsa}{gesucht}, 
\xayr{psiis}{pasīsa}{interessant}; \xayr{AreenF}{arēn}{wie auch immer}, 
\xayr{leer}{lēra}{Hure}; \xayr{noonF}{nōn}{Wunsch}. /uː/ kommt gar nicht vor. 
In der Regel entstehen sie durch das Zusammentreffen zweier gleicher Vokale; ob 
sie tatsächlich phonemisch sind, müsste untersucht werden.

\end{document}
