\documentclass[12pt,paper=a4]{scrartcl}

% Author, Title, Subtitle etc.
\author{Carsten Becker}
\title{SE Kunst- und Plansprachen -- von Esperanto bis Dothraki}
\subtitle{Antworten zum 02.06.2016}
\date{08.06.2016} % Format: YYYY/MM/DD (rev. N, YYYY/MM/DD)

% Provide running author and date
\makeatletter
\let\runauthor\@author
\let\rundate\@date
\makeatother

% Handle language and quotation marks
\usepackage{polyglossia}
\setdefaultlanguage{german}
\usepackage{csquotes} % Put quotations in \enquote{}!
\SetBlockEnvironment{quotation}
\renewcommand*{\mkccitation}[1]{ (#1)}

% Set all margins to 2.54 cm
\usepackage[margin=1in]{geometry}
\widowpenalty10000 % Avoid widows like the plague!
\clubpenalty10000 % Avoid orphans like the plage, too!

% Extended formatting of lists
\usepackage{enumitem}
\setlist[itemize]{noitemsep}

% Make multiple columns available in single-column document
\usepackage{multicol}

% Make text colors and color names available
\usepackage[xetex]{xcolor}

% Load font stuff for XeTeX
\usepackage{xltxtra}
\usepackage{fontspec}

% Set main fonts
\usepackage[config=mt-Junicode]{microtype}

\newfontfamily{\Tagati}[
    Renderer=Graphite,
    Scale=1.1,
    BoldFont={* Italic},
    HyphenChar=·,
]{Tagati Book G}

\setmainfont[
    Ligatures=TeX,
    Numbers=Lowercase,
]{Junicode}

\setsansfont[
    Ligatures=TeX,
    Numbers=Lowercase,
    Scale=MatchUppercase,
    BoldFont={Open Sans Condensed Bold},
]{Open Sans Condensed Light}

% Load BibLaTeX (using Biber), configure citation styles
\usepackage[
    authordate-trad,
    backend=biber,
    safeinputenc,
]{biblatex-chicago}
\addbibresource{bibliografie.bib}

% To make \textcite look like "Doe (2014: 213)"
\renewcommand*{\postnotedelim}{\addcolon\addspace}
\DeclareFieldFormat{postnote}{#1}
\DeclareFieldFormat{multipostnote}{#1}

% Enable generating files from this .tex file (for the bibliography) 
\usepackage{filecontents}

% Clickable links in footnotes, TOC, etc.
\usepackage[
    xetex,
    bookmarks=true,
    colorlinks=false,
    linktoc=section,
    hidelinks,
    pdfusetitle,
]{hyperref}

\usepackage{url}
\urlstyle{rm}

% Ability to include graphics and dealing with footnotes in descriptions
\usepackage{graphicx}
\usepackage[font={small,sf},labelfont={small,sf},format=plain]{caption}
\usepackage{subcaption}
\usepackage{wrapfig}
\setlength{\columnsep}{2\baselineskip}

% General headers and footers
\usepackage{fancyhdr}
\pagestyle{fancy}

\fancyhead[L]{} % empty
\fancyhead[C]{} % empty
\fancyhead[R]{\thepage}

\fancyfoot[L]{} % empty
\fancyfoot[C]{} % empty
\fancyfoot[R]{} % empty

\renewcommand{\headrulewidth}{0pt}
\renewcommand{\footrulewidth}{0pt}

% First page headers and footers are different
\fancypagestyle{firstpage}{
    \fancyhead[L]{\sffamily \footnotesize \textbf{Benung. The Ayeri Language Resource}}
    \fancyhead[C]{} % empty
    \fancyhead[R]{\sffamily \footnotesize \runauthor{} · \rundate{}}
    
    \fancyfoot[L]{} % empty
    \fancyfoot[C]{} % empty
    \fancyfoot[R]{\sffamily \footnotesize 
	\href{http://benung.nfshost.com}{http://benung.nfshost.com} · 
	\href{https://github.com/carbeck/benung-pdfs}{https://github.com/carbeck/benung-pdfs} · 
	\href{https://creativecommons.org/licenses/by-sa/4.0/}{CC~BY-SA~4.0}
    }
    
    \renewcommand{\headrulewidth}{0.5pt}
    \setlength\footskip{0.5in}
}

\usepackage{ifthen}
\ifthenelse{\value{page}=1}{\thispagestyle{firstpage}{\pagestyle{fancy}}}

% Line spacing
\usepackage{setspace}
\onehalfspacing

% Avoid pagebreaks right after sections and subsections
\usepackage{needspace}
\usepackage{etoolbox}
\preto\section{\needspace{6\baselineskip}}
\preto\subsection{\needspace{6\baselineskip}}

% Things for tables
% \usepackage{longtable}
% \usepackage{tabu}
% \usepackage{booktabs}
% \usepackage{rotating}

% Formatting of table of glossing abbreviations from Leipzig package manual
\usepackage[acronym,nomain,nonumberlist,nopostdot]{glossaries}
\usepackage{glossary-inline}%

\newglossarystyle{mysuper}{%
	\glossarystyle{super}% based on super
	\renewenvironment{theglossary}{%
		\begin{itemize}[align=left]
	}{%
		\end{itemize}%
	}%
	\renewcommand*{\glossaryheader}{}%
	\renewcommand*{\glsgroupheading}[1]{}%
	\renewcommand*{\glossaryentryfield}[5]{%
		\item[\glsentryitem{##1}\glstarget{##1}{##2}]
		\makefirstuc{##3}\glspostdescription{}
	}%
	\renewcommand*{\glsgroupskip}{}%
}%

% Formatting of glosses
\usepackage{expex}
\usepackage{leipzig}

\newleipzig{AgtT}{at}{agent topic}
\newleipzig{PatT}{pt}{patient topic}
\newleipzig{DatT}{datt}{dative topic}
\newleipzig{GenT}{gent}{genitive topic}
\newleipzig{LocT}{loct}{locative topic}
\newleipzig{InsT}{inst}{instrumental topic}
\newleipzig{CauT}{caut}{causative topic}
\newleipzig{An}{an}{animate}
\newleipzig{Inan}{inan}{inanimate}
\newleipzig{Hab}{hab}{habitative}
\newleipzig{Ayr}{ayr}{Ayeri}

\makeglossaries

% Section formatting
\usepackage{titlesec}
\usepackage{needspace}

\titleformat{\section}{
	\needspace{4\baselineskip}\sffamily\bfseries
}{
	\noindent\thesection
}{
	0.5\baselineskip
}{
}

% Nicer footnotes
\usepackage[bottom,hang,norule]{footmisc}
\setlength{\footnotesep}{0.75\baselineskip}

% Smaller font in block quotes
\usepackage{relsize}
\AtBeginEnvironment{quote}{\noindent\smaller}
\AtBeginEnvironment{quotation}{\smaller}

% Date and time
\usepackage[
	datesep=.,
	style=ddmmyyyy,
]{datetime2}

% Macros
\newcommand{\fw}[1]{\textit{#1}} % Foreign Word
\newcommand{\tit}[1]{\textit{#1}} % Title of a work
\newcommand{\q}[1]{\enquote{#1}} % Context-aware quotation
\newcommand{\qq}[1]{\enquote*{#1}} % Explicit sublevel quotation
\newcommand{\tsup}[1]{\textsuperscript{#1}} % Superscript
\newcommand{\markyellow}[1]{\colorbox{yellow}{#1}} % Yellow highlighter
\newcommand{\ques}{\fakesuperscript{?}} % raised question mark

\newcommand{\ayr}[1]{{\Tagati #1}}
\newcommand{\xayr}[3]{{\Tagati #1} \emph{#2} `#3'}

\newenvironment{ayeri}{
    %\hyphenpenalty=10000
    %\hbadness=10000
    \doublespacing
    \begin{multicols}{2}
    \Tagati
}{
    \end{multicols} \par
}

\newenvironment{mytitle}{
    \hfill
    \begin{minipage}{0.667\textwidth}
	\vspace{\baselineskip}
	\begin{center}
	    \Large
	    \sffamily\bfseries
	    \makeatletter
}{
	    \makeatother
	\end{center}
	\vspace{1em}
    \end{minipage}
    \hfill
}

% blah
\usepackage{lipsum}

%% END OF PREAMBLE %%%%%%%%%%%%%%%%%%%%%%%%%%%%%%%%%%%%%%%%%%%%%%%%%%%%%%%%%%%%%

\begin{document}

%% MAIN PART %%%%%%%%%%%%%%%%%%%%%%%%%%%%%%%%%%%%%%%%%%%%%%%%%%%%%%%%%%%%%%%%%%%

\begin{mytitle}
    \@title: \@subtitle\footnotemark
\end{mytitle}
\footnotetext{\cite[Vgl.][]{buch2016ss}.}

\section{Aufgabe 1}
`voll sein (von)' --- \ayr{deNF} \textit{deng} (Adj.)

\section{Aufgabe 2}
`Boot' --- \ayr{tFrpsF} \textit{trapas} (N., unbelebt)

\section{Aufgabe 3}
`Aal':

\begin{enumerate}[label={(\alph*)}]
	\item z.\,B. Kompositum aus bestehenden Lexemen:
	
		\ayr{InunFsegsF} \textit{inunsegas} (\ayr{InunF/ segsF} 
		\textit{inun- segas}) `Schlangenfisch' (N., belebt) ← 
		\xayr{InunF}{inun}{Fisch} + \xayr{segsF}{segas}{Schlange}

	\item z.\,B. neues Lexem:
	
		\ayr{bmNF} \textit{bamang} (N., belebt)
		
\end{enumerate}

\section{Aufgabe 4}
`Luftkissenboot':

\begin{enumerate}[label={(\alph*)}]
	
	\item z.\,B. Entlehnung aus Engl. \textit{hovercraft}:
	
		\ayr{hvkFraafF} \textit{havakrāf} (N., unbelebt)
		
		Der Laut /f/ kommt in Ayeri nativ nicht vor, sollte aber kein 
		Problem für Sprecher sein, da /v/ vorhanden ist. Das /t/ am Ende 
		muss entfallen, da \textsc{-cc} nicht zulässig ist.
		
		\item Native Neuschöpfung:
		
		z.\,B. \ayr{tFrpsvitnF} \textit{trapasavitan} 
		(\ayr{tFrpsF/ svitnF} \textit{trapas- savitan})
		`Schwebeboot' (N., unbelebt) ← \xayr{tFrpsF}{trapas}{Boot} + 
		\xayr{svitF/}{savit-}{schweben} + \xayr{/AnF}{-an}{\Nmlz{}}
		
	\end{enumerate}

\section{Aufgabe 5}
Ayeri hat weitgehend agglutinierende Morphologie; durch Krasis kommt es 
unter Umständen zu Allomorphie. Beispiele für fusionale Morphologie sind die 
Pronomen bzw. ans Verb angehängten Pronominalendungen.\footnote{\xayr{yaaNF}{yāng}{er}
ist sehr transparent \xayr{y}{ya}{\Tsg{}.\M{}} + \xayr{/ANF}{-ang}{-\Aarg{}}; 
für \xayr{naa}{nā}{mein} funktioniert das allerdings nicht: Die Form 
wäre sonst *\ayr{Ayen} *\textit{ayena} < \xayr{Aj}{ay}{\Fsg{}} + 
\xayr{/En}{-ena}{\Gen{}}.}
	
\ex \begingl
	\glpreamble \ayr{le koMdsojyaaNF nikye bennFyaanFy.}\\
		\textit{Le kondasoyyāng nikaye benanyānya.} //
	\gla Le \textbf{kond} @ \textbf{-asa} @ \textbf{-oy} @ \textbf{-yāng} 
		nika @ -ye @ -Ø benan @ -ya @ -an @ -ya //
	\glb \PatT{}.\Inan{} \textbf{essen} \textbf{-\Hab}{} \textbf{-\Neg}{} 
		\textbf{-\Tsg{}.\M{}.\Aarg{}} Kartoffel -\Pl{} -\Top{} 
		Morgen -\Loc{} -\Nmlz{} -\Loc{} //
	\glft `Er isst normalerweise keine Kartoffeln zum 
		Frühstück.' //
\endgl \xe

\ex \begingl
	\glpreamble \ayr{tFrpsFreNF svitnF naa deNF bmNFyeri.} \\
		\textit{Trapasreng savitan nā deng bamangyeri.} //
	\gla Trapas @ -reng savit @ -an \textbf{nā} deng \textbf{bamang} @ 
		\textbf{-ye} @ \textbf{-eri} //
	\glb Boot -\Aarg{}.\Inan{} schweben -\Nmlz{} \textbf{\Fsg{}.\Gen{}} 
		voll \textbf{Aal} \textbf{-\Pl{}} \textbf{-\Ins{}} //
	\glft `Mein Luftkissenboot ist voller Aale.' //
\endgl \xe

%% BIBLIOGRAPHY %%%%%%%%%%%%%%%%%%%%%%%%%%%%%%%%%%%%%%%%%%%%%%%%%%%%%%%%%%%%%%%%

\vfill

\addsec{Abkürzungen}
\begin{multicols}{3}
\printglossary[style=mysuper,type=\leipzigtype]
\end{multicols}

\printbibliography[heading=bibintoc]

\end{document}
