\documentclass[12pt,paper=a4]{scrartcl}

% Author, Title, Subtitle etc.
\author{Carsten Becker}
\title{The ‘Lord's Prayer’ in Ayeri}
\subtitle{An Updated Translation}

% Handle language and quotation marks
\usepackage[ngerman,english]{babel}
\usepackage{csquotes} % Put quotations in \enquote{}!
\SetBlockEnvironment{quotation}
\renewcommand*{\mkccitation}[1]{ (#1)}

% Set all margins to 2.54 cm
\usepackage[margin=1in]{geometry}
\widowpenalty10000 % Avoid widows like the plague!
\clubpenalty10000 % Avoid orphans like the plage, too!

% Make multiple columns available in single-column document
\usepackage{multicol}

% Make text colors and color names available
\usepackage[xetex]{xcolor}

% Load font stuff for XeTeX
\usepackage{xltxtra}
\usepackage{fontspec}

% Set main fonts
\usepackage[config=mt-Junicode]{microtype}

\newfontfamily{\Tagati}[
    Renderer=Graphite,
    Scale=1.1,
    BoldFont={* Italic},
    HyphenChar=·,
]{Tagati Book G}

\setmainfont[
    Ligatures=TeX,
    Numbers=Lowercase,
]{Junicode}

\setsansfont[
    Ligatures=TeX,
    Numbers=Lowercase,
    Scale=MatchUppercase,
    BoldFont={Open Sans Condensed Bold},
]{Open Sans Condensed Light}

\newfontfamily{\OSCLucno}[
    Numbers=Uppercase,
    Scale=MatchUppercase,
]{Open Sans Condensed Light}

% Load BibLaTeX (using Biber), configure citation styles
% Uses <https://github.com/semprag/biblatex-sp-unified>
\usepackage[
    style=biblatex-sp-unified,
    citestyle=sp-authoryear-comp,
    maxnames=2,
    backend=biber,
    safeinputenc,
]{biblatex}

% To make \textcite look like "Doe (2014: 213)"
\renewcommand*{\postnotedelim}{\addcolon\addspace}
\DeclareFieldFormat{postnote}{#1}
\DeclareFieldFormat{multipostnote}{#1}

% Enable generating files from this .tex file (for the bibliography) 
\usepackage{filecontents}

% Date etc.
\usepackage{datetime}
\renewcommand{\dateseparator}{/}

% Clickable links in footnotes, TOC, etc.
\usepackage[
    xetex,
    bookmarks=true,
    colorlinks=false,
    linktoc=section,
    hidelinks,
    pdfusetitle,
]{hyperref}

\usepackage{url}
\urlstyle{rm}

% Ability to include graphics and dealing with footnotes in descriptions
\usepackage{graphicx}
\usepackage[font={small,sf},labelfont={small,sf},format=plain]{caption}
\usepackage{subcaption}
\usepackage{wrapfig}
\setlength{\columnsep}{2\baselineskip}

% General headers and footers
\usepackage{fancyhdr}
\pagestyle{fancy}

\fancyhead[L]{} % empty
\fancyhead[C]{} % empty
\fancyhead[R]{\thepage}

\fancyfoot[L]{} % empty
\fancyfoot[C]{} % empty
\fancyfoot[R]{} % empty

\renewcommand{\headrulewidth}{0pt}
\renewcommand{\footrulewidth}{0pt}

% First page headers and footers are different
\fancypagestyle{firstpage}{
    \fancyhead[L]{\sffamily \footnotesize \textbf{Benung. The Ayeri Language Resource}}
    \fancyhead[C]{} % empty
    \fancyhead[R]{\sffamily \footnotesize Carsten Becker · \yyyymmdddate\today}
    
    \fancyfoot[L]{} % empty
    \fancyfoot[C]{} % empty
    \fancyfoot[R]{\sffamily \footnotesize 
	\href{http://benung.nfshost.com}{http://benung.nfshost.com} · 
	\href{https://github.com/carbeck/benung-pdfs}{https://github.com/carbeck/benung-pdfs} · 
	\href{https://creativecommons.org/licenses/by-sa/4.0/}{CC~BY-SA~4.0}
    }
    
    \renewcommand{\headrulewidth}{0.5pt}
    \setlength\footskip{0.5in}
}

\usepackage{ifthen}
\ifthenelse{\value{page}=1}{\thispagestyle{firstpage}{\pagestyle{fancy}}}

% Line spacing
\usepackage{setspace}
\onehalfspacing

% Line numbering
\usepackage[left, modulo]{lineno}
\renewcommand\linenumberfont{\normalfont\tiny\OSCLucno}

% Things for tables
% \usepackage{longtable}
% \usepackage{tabu}
% \usepackage{rotating}

% Formatting of glosses
\usepackage{expex}
\usepackage{leipzig}

\newleipzig{AgtT}{at}{agent topic}
\newleipzig{PatT}{pt}{patient topic}
\newleipzig{DatT}{datt}{dative topic}
\newleipzig{GenT}{gent}{genitive topic}
\newleipzig{LocT}{loct}{locative topic}
\newleipzig{InsT}{inst}{instrumentative topic}
\newleipzig{CauT}{caut}{causative topic}
\newleipzig{An}{an}{animate}
\newleipzig{Inan}{inan}{inanimate}

% Nicer footnotes
\usepackage[bottom]{footmisc}
\deffootnote{0em}{1.5em}{\textsuperscript\thefootnotemark\enskip}
\renewcommand{\footnoterule}{\rule{0pt}{0pt}{\vspace*{-0pt}}}
\setlength{\footnotesep}{1.5em}

% Smaller font in block quotes
\usepackage{relsize,etoolbox}
\AtBeginEnvironment{quote}{\noindent\smaller}
\AtBeginEnvironment{quotation}{\smaller}

% Macros
\newcommand{\fw}[1]{\textit{#1}} % Foreign Word
\newcommand{\tit}[1]{\textit{#1}} % Title of a work
\newcommand{\q}[1]{\enquote{#1}} % Context-aware quotation
\newcommand{\qq}[1]{\enquote*{#1}} % Explicit sublevel quotation
\newcommand{\tsup}[1]{\textsuperscript{#1}} % Superscript
\newcommand{\markyellow}[1]{\colorbox{yellow}{#1}} % Yellow highlighter
\newcommand{\ques}{\fakesuperscript{?}} % raised question mark

\newcommand{\ayr}[1]{{\Tagati #1}}
\newcommand{\xayr}[3]{{\Tagati #1} \emph{#2} \enquote*{#3}}

\newenvironment{ayeri}{
    %\hyphenpenalty=10000
    %\hbadness=10000
    %\doublespacing
    %\begin{multicols}{2}
    \Tagati
}{
    %\end{multicols} \par
}

\newenvironment{mytitle}{
    \hfill
    \begin{minipage}{0.667\textwidth}
	\vspace{\baselineskip}
	\begin{center}
	    \Large
	    \sffamily\bfseries
	    \makeatletter
}{
	    \makeatother
	\end{center}
	\vspace{1em}
    \end{minipage}
    \hfill
}

% blah
%\usepackage{lipsum}

%% BIBLIOGRAPHY DATABASE %%%%%%%%%%%%%%%%%%%%%%%%%%%%%%%%%%%%%%%%%%%%%%%%%%%%%%%

\begin{filecontents*}{\jobname.bib}
@article{becker2008,
	author = {Carsten Becker},
	title = {{The Lord's Prayer}},
	pagination = {verse},
	journaltitle = {Benung. The Ayeri Language Resource},
	date = {2008-07-05},
	url = {http://benung.nfshost.com/wp-content/uploads/2011/06/xmp_lordsprayer.pdf},
	urldate = {2015-06-29},
}
\end{filecontents*}

\addbibresource{\jobname.bib}

%% END OF PREAMBLE %%%%%%%%%%%%%%%%%%%%%%%%%%%%%%%%%%%%%%%%%%%%%%%%%%%%%%%%%%%%%

\begin{document}

%% MAIN PART %%%%%%%%%%%%%%%%%%%%%%%%%%%%%%%%%%%%%%%%%%%%%%%%%%%%%%%%%%%%%%%%%%%

\begin{mytitle}
    \@title: \@subtitle
\end{mytitle}

\section{The Text}
\begin{raggedright}
\begin{multicols}{2}
\textbf{Natran na Nahang} \\ [0.5\baselineskip]

\begin{linenumbers*}

Badan nana lenoya \\
Ang mya hangyo ternu garan vana \\
Ang mya sahayo lagasim vana \\
Ang mya amangyo nōn vana \\
Avanya ku-lenoya. \\ [0.5\baselineskip]

Le ilu dabas vadisān diyan nana nyam \\
Nay tatamu semānjas nana \\
Ku-tatamnang adayam \\
Si ang semayan nya ran \\
Nay lantoyu nas nembangya \\
Nārya na madu nas nerau. \\ [0.5\baselineskip]

Yanoyam vana lagasimas \\
Nay litās nay vomayas asayam. \\ [0.5\baselineskip]

Amen.

\end{linenumbers*}

%\end{multicols}
\columnbreak
%\begin{multicols}{2}

\begin{ayeri}
\textbf{ntFrnF n nhNF} \\ [0.5\baselineskip]

\begin{linenumbers*}

bdnF nn lenoy \\
ANF mY hNYo terFnu grnF vn \\
ANF mY shyo lgsimF vn \\
ANF mY AmNYo noonF vn \\
AvnY ku/lenoy~– \\ [0.5\baselineskip]

le Ilu dbsF vdiːsnF diynF nn nYmF \\
nj ttmu semaanYe\_asF nn \\
ku/ ttmFnNF AdymF \\
si ANF semynF nY rnF \\
nj lMtoyu nsF nemFbNY \\
naarY n mdu nsF nerau~— \\ [0.5\baselineskip]

ynoymF vn lgsimsF \\
nj litaasF nj vomysF AsymF~— \\ [0.5\baselineskip]

AmenF

\end{linenumbers*}

\end{ayeri}
\end{multicols}
\end{raggedright}

\section{Some Notes}
The first time I translated the \tit{Lord's Prayer} was in 2008 
\parencite[cf.][]{becker2008}, and the version then contained a few errors in 
hindsight: for example, ll.~3--4 were translated with patient topics when there 
are no patients there, according to Ayeri's grammar. I also updated the 
vocabulary in places, as well as the spelling, e.g. \ayr{/ye\_asF} \fw{-yeas} 
\qq{-\Pl{}-\Parg{}} was changed to \fw{-jas} to better reflect actual 
pronunciation (cf. l.~7). Phrasings have also been simplified in a few places, 
e.g. \ayr{AvnY ku/lenoy} \fw{avanya ku-lenoya} `on earth as in heaven' instead 
of the wordier \ayr{ku/lenoy, Adaare y Arek njnj} \fw{ku-lenoya, adāre 
ya Areka naynay} `as in Heaven, in the same way as well on Earth' (cf. 
l.~5).\footnote{The word for `earth' has also since changed from \ayr{Arek} 
\fw{Areka} to \ayr{AvnF} \fw{avan}, which simply means `bottom; ground, soil', 
cf. German \fw{Erde} or French \fw{terre}.} Curiously, I treated \ayr{rnF} 
\fw{ran} `against' like a preposition with a patient NP dependent in my first 
translation; custom has changed it into a postposition since then, with regular 
locative marking on the postpositional object (cf. l.~9): it used to be 
\ayr{rnF nsF} \fw{ran nas} `against us.\Parg{}', but it is \ayr{nY rnF} 
\fw{nya ran} `us.\Loc{} against' now.

Moreover, I have made efforts to digitalize Tahano Hikamu as well since my 
original translation of the \tit{Lord's Prayer} seven years ago, so the text 
in Ayeri can be presented here in its native script.

\section{Interlinear Glossing}

\ex % 1
\begingl
    \glpreamble Badan nana lenoya //
    \gla badan nana leno -ya //
    \glb father \Fpl{}.\Gen{} heaven -\Loc{} //
    \glft `Our father in heaven' //
\endgl
\xe

\ex % 2
\begingl
    \glpreamble Ang mya hangyo ternu garan vana //
    \gla ang mya hang -yo ternu garan -Ø vana //
    \glb \AgtT{} shall stay -\Tsg{}.\N{} holy name -\Top{} \Ssg{}.\Gen{} //
    \glft `Thy name stay holy' //
\endgl
\xe

\ex % 3
\begingl
    \glpreamble Ang mya sahayo lagasim vana //
    \gla ang mya saha -yo lagasim -Ø vana //
    \glb \AgtT{} shall come -\Tsg{}.\N{} kingdom -\Top{} \Ssg{}.\Gen{} //
    \glft `Thy kingdom come' //
\endgl
\xe

\ex % 4
\begingl
    \glpreamble Ang mya amangyo nōn vana //
    \gla ang mya amang -yo nōn -Ø vana //
    \glb \AgtT{} shall happen -\Tsg{}.\N{} will -\Top{} \Ssg{}.\Gen{} //
    \glft `Thy will happen' //
\endgl
\xe

\ex % 5
\begingl
    \glpreamble Avanya ku-lenoya //
    \gla avan -ya ku= leno -ya //
    \glb earth -\Loc{} like= heaven -\Loc{} //
    \glft `On earth as in heaven' //
\endgl
\xe

\ex % 6
\begingl
    \glpreamble Le ilu dabas vadisān diyan nana nyam //
    \gla le il -u dabas vadisān -Ø diyan nana nyam //
    \glb \PatT{}.\Inan{} give -\Imp{} today bread -\Top{} daily \Fpl{}.\Gen{} \Fpl{}.\Dat{} //
    \glft `Our daily bread give us today' //
\endgl
\xe

\ex % 7
\begingl
    \glpreamble Nay tatamu semānjas nana //
    \gla nay tatam -u semān -j -as nana //
    \glb and forgive -\Imp{} failure -\Pl{} -\Parg{} \Fpl{}.\Gen{} //
    \glft `And forgive our failures' //
\endgl
\xe

\ex % 8
\begingl
    \glpreamble Ku-tatamnang adayam //
    \gla ku= tatam -nang ada -yam //
    \glb like= forgive -\Fpl{}.\Aarg{} those -\Dat{} //
    \glft `As we forgive those' //
\endgl
\xe

\ex % 9
\begingl
    \glpreamble Si ang semayan nya ran //
    \gla si ang sema -yan -Ø nya ran //
    \glb \Rel{} \AgtT{} fail -\Tpl{}.\M{} -\Top{} \Fpl{}.\Loc{} against //
    \glft `Who fail against us' //
\endgl
\xe

\ex % 10
\begingl
    \glpreamble Nay lantoyu nas nembangya //
    \gla nay lant -oy -u nas nembang -ya //
    \glb and lead -\Neg{} -\Imp{} \Fpl{}.\Parg{} temptation -\Loc{} //
    \glft `And do not lead us into temptation' //
\endgl
\xe

\ex % 11
\begingl
    \glpreamble Nārya na madu nas nerau //
    \gla nārya na mad -u nas nerau -Ø //
    \glb but \GenT{} save -\Imp{} \Fpl{}.\Parg{} evil -\Top{} //
    \glft `But from evil save us' //
\endgl
\xe

\ex % 12
\begingl
    \glpreamble Yanoyam vana lagasimas //
    \gla Yanoyam vana lagasim -as //
    \glb because \Ssg{}.\Gen{} kingdom -\Parg{} //
    \glft `For thine is the kingdom' //
\endgl
\xe

\ex % 13
\begingl
    \glpreamble Nay litās nay vomayas asayam //
    \gla nay lita -as nay vomay -as asayam //
    \glb and might -\Parg{} and glory -\Parg{} forever //
    \glft `And the might and the glory forever' //
\endgl
\xe

%% BIBLIOGRAPHY %%%%%%%%%%%%%%%%%%%%%%%%%%%%%%%%%%%%%%%%%%%%%%%%%%%%%%%%%%%%%%%%

%\newpage
\vfill
\printbibliography

\end{document}
