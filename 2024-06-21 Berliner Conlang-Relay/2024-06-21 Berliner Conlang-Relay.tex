\documentclass[
	12pt,
	ngerman,
]{scrartcl}

% Author, Title, Subtitle etc.
\author{Carsten Becker}
\title{Eine hundsgemeine Geschichte}
\subtitle{Beitrag zum Relay des Berliner Conlang-Stammtischs 2024}
\date{\DTMdate{2024-06-21}} % Format: YYYY/MM/DD (rev. N, YYYY/MM/DD)

% Provide running author and date
\makeatletter
\let\runauthor\@author
\let\rundate\@date
\makeatother

% Preferences of float placement
% cf. https://tex.stackexchange.com/a/167864/60686
\makeatletter
\renewcommand\fps@figure{htbp}
\renewcommand\fps@table{htbp}
\def \@floatboxreset {%
        \reset@font
        \normalsize
        \@setminipage
		\centering
}
\makeatother


% Handle language and quotation marks
\usepackage{babel}
\usepackage{csquotes} % Put quotations in \enquote{}!
\SetBlockEnvironment{quotation}
\renewcommand*{\mkccitation}[1]{ (#1)}
\let\q\textquote

% Quotation style for word definitions
\DeclareQuoteStyle{wdef}
	{\textquoteleft}{\textquoteright}
	{\textquotedblleft}{\textquotedblright}
\newcommand{\wdef}[1]{{\setquotestyle{wdef}\enquote{#1}}}

% Set all margins to 2.54 cm
\usepackage[margin=1in]{geometry}
\widowpenalty10000 % Avoid widows like the plague!
\clubpenalty10000 % Avoid orphans like the plage, too!

% Extended formatting of lists
\usepackage{enumitem}
\newlist{glossdefs}{itemize}{1}
\setlist[glossdefs]{nosep, leftmargin=3em, labelwidth=2.5em, align=left}
\setlist[itemize]{noitemsep}

% Make multiple columns available in single-column document
\usepackage{multicol}

% Make text colors and color names available
\usepackage{xcolor}

% Load font stuff for XeTeX
\usepackage{fontspec}

% Set main fonts
\usepackage{microtype}

\newfontfamily{\Tagati}[
	Renderer=Graphite,
	Scale=1.0,
	BoldFont={* Italic},
% 	HyphenChar=·,
]{Tagati Book G}

\setmainfont[
	Ligatures=TeX,
	Numbers={OldStyle,Proportional},
	BoldFont=*-Bold,
	ItalicFont=*-Italic,
	BoldItalicFont=*-BoldItalic,
]{Junicode}

\setsansfont[
	Ligatures=TeX,
	Numbers=Lowercase,
	Scale=MatchUppercase,
	ItalicFont={Open Sans Condensed Light Italic},
	BoldFont={Open Sans Condensed Bold},
	BoldItalicFont={Open Sans Condensed Bold Italic},
]{Open Sans Condensed Light}

% Load BibLaTeX (using Biber), configure citation styles
\usepackage[
	authordate-trad,
	backend=biber,
	alldates=terse,
	safeinputenc,
]{biblatex-chicago}

% To make \textcite look like "Doe (2014: 213)"
\renewcommand*{\postnotedelim}{\addcolon\addspace}
\DeclareFieldFormat{postnote}{#1}
\DeclareFieldFormat{multipostnote}{#1}
\addbibresource{bibliography.bib}

% Date etc.
\usepackage[
	useregional=numeric,
]{datetime2}

% % Verse
% \usepackage{verse}

% Ability to include graphics and dealing with footnotes in descriptions
\usepackage{graphicx}
\usepackage[font={small,sf},labelfont={small,sf},format=plain]{caption}
\usepackage{subcaption}
\usepackage{wrapfig}
\setlength{\columnsep}{2\baselineskip}

% General headers and footers
\usepackage{fancyhdr}
\pagestyle{fancy}

\fancyhead[L]{} % empty
\fancyhead[C]{} % empty
\fancyhead[R]{\thepage}

\fancyfoot[L]{} % empty
\fancyfoot[C]{} % empty
\fancyfoot[R]{} % empty

\renewcommand{\headrulewidth}{0pt}
\renewcommand{\footrulewidth}{0pt}

% First page headers and footers are different
\fancypagestyle{firstpage}{
	\fancyhead[L]{\sffamily \footnotesize \textbf{Benung. The Ayeri Language Resource}}
	\fancyhead[C]{} % empty
	\fancyhead[R]{\sffamily \footnotesize \runauthor{} · \rundate{}}
	
	\fancyfoot[L]{} % empty
	\fancyfoot[C]{} % empty
	\fancyfoot[R]{\sffamily \footnotesize 
	\href{https://ayeri.de}{https://ayeri.de} · 
	\href{https://github.com/carbeck/benung-pdfs}{https://github.com/carbeck/benung-pdfs} · 
	\href{https://creativecommons.org/licenses/by-sa/4.0/}{CC~BY-SA~4.0}
	}
	
	\renewcommand{\headrulewidth}{0.5pt}
	\setlength\footskip{0.5in}
}

\usepackage{ifthen}
\ifthenelse{\value{page}=1}{\thispagestyle{firstpage}{\pagestyle{fancy}}}

% Line spacing
\usepackage{setspace}
\onehalfspacing

% Avoid pagebreaks right after sections and subsections
\usepackage{needspace}
\usepackage{etoolbox}
\preto\section{\needspace{6\baselineskip}}
\preto\subsection{\needspace{6\baselineskip}}

% Things for tables
\usepackage{tabularx}
\usepackage{booktabs}
\usepackage{makecell}
% \usepackage{rotating}
\newcolumntype{C}{>{\centering\arraybackslash}X}

% Formatting of table of glossing abbreviations from Leipzig package manual
\usepackage[
	acronym,
	nomain,
	nonumberlist,
	nopostdot,
	numberedsection=nameref,
	toc=true,
	xindy={codepage=utf8}, % language=english,
]{glossaries}
\usepackage{glossary-inline}

\newglossarystyle{threecolumn}{%
	\renewenvironment{theglossary}{%
		\begin{multicols}{3}
		\begin{glossdefs}%
	}{%
		\end{glossdefs}%
		\end{multicols}%
	}%
	\renewcommand*{\glossaryheader}{}%
	\renewcommand*{\glsgroupheading}[1]{}%
	\newcommand*{\glossaryentryfield}[5]{%
		\item[\glsentryitem{##1}\glstarget{##1}{##2}]
		% \makefirstuc{##3}\glspostdescription{}
		##3\glspostdescription{}
	}%
	\renewcommand*{\glsgroupskip}{}%
}%

% Formatting of glosses
\usepackage{langsci-gb4e}
\usepackage[glosses]{leipzig}
\renewcommand{\eachwordone}{\itshape}
\renewcommand{\eachwordtwo}{\rule[-.5\baselineskip]{0pt}{0pt}}
% leipzig.def
% Glossing abbreviations

% German redefinitions
\renewleipzig{Aarg}{a}{Agens}
\renewleipzig{Abs}{abs}{Absolutiv}
\renewleipzig{Acc}{acc}{Akkusativ}
\renewleipzig{All}{all}{Allativ}
\renewleipzig{Caus}{caus}{Kausativ}
\renewleipzig{Dat}{dat}{Dativ}
\renewleipzig{Dem}{dem}{Demonstrativ}
\renewleipzig{First}{1}{erste Person}
\renewleipzig{F}{f}{Femininum}
\renewleipzig{Gen}{gen}{Genitiv}
\renewleipzig{Imp}{imp}{Imperativ}
\renewleipzig{Ins}{ins}{Instrumentalis}
\renewleipzig{Loc}{loc}{Lokativ}
\renewleipzig{M}{m}{Maskulinum}
\renewleipzig{Neg}{neg}{Negativ}
\renewleipzig{Nmlz}{nmlz}{Nominalisierer}
\renewleipzig{Nom}{nom}{Nominativ}
\renewleipzig{N}{n}{Neutrum}
\renewleipzig{Parg}{p}{Patiens}
\renewleipzig{Pl}{pl}{Plural}
\renewleipzig{Poss}{poss}{Possessiv}
\renewleipzig{Pst}{pst}{Präteritum}
\renewleipzig{Ptcp}{ptcp}{Partizip}
\renewleipzig{Refl}{refl}{Reflexiv}
\renewleipzig{Rel}{rel}{Relativ}
\renewleipzig{Second}{2}{zweite Person}
\renewleipzig{Sg}{sg}{Singular}
\renewleipzig{Third}{3}{dritte Person}
\renewleipzig{Top}{top}{Topik}

% Other common abbreviations, undefined by default
\newleipzig{Dim}{dim}{Diminutiv}

% ɮɛ̃̂.kɔ̌ʔ
\newleipzig{Acv}{acv}{Aktiv}
\newleipzig{Clf}{clf}{Klassifizierer}
\newleipzig{Elv}{elv}{Elativ}
\newleipzig{Inch}{inch}{Inchoativ}
\newleipzig{Ipfv}{ipfv}{Imperfektiv}
\newleipzig{Nmz}{nmz}{Substantivierung}
\newleipzig{Npst}{npst}{Nicht-Präteritum}
\newleipzig{Pfv}{pfv}{Perfektiv}
\newleipzig{Stv}{stv}{Stativ}

% Ayeri
\newleipzig{AgtT}{at}{Agenstopik}
\newleipzig{Anim}{anim}{belebt}
\newleipzig{Ayr}{ayr}{Ayeri}
\newleipzig{CauT}{caut}{Kausativtopik}
\newleipzig{Conj}{conj}{konjunktiv}
\newleipzig{DatT}{datt}{Dativtopik}
\newleipzig{Dir}{dir}{direktiv}
\newleipzig{GenT}{gent}{Genitivtopik}
\newleipzig{Hab}{hab}{Habitativ}
\newleipzig{Hort}{hort}{Hortativ}
\newleipzig{Inan}{inan}{unbelebt}
\newleipzig{InsT}{inst}{Instrumentaltopik}
\newleipzig{Ints}{ints}{intensiv}
\newleipzig{Iter}{iter}{iterativ}
\newleipzig{LocT}{loct}{Lokativtopik}
\newleipzig{Oblig}{oblig}{obligativ}
\newleipzig{PatT}{pt}{Patienstopik}
\newleipzig{Sup}{sup}{Superlativ}


% Trees
\usepackage[linguistics]{forest}

% Underline, strikeout
\usepackage{soul}

% Nicer footnotes
\usepackage[bottom,hang,norule]{footmisc}
\setlength{\footnotesep}{0.75\baselineskip}

% Smaller font in block quotes
\usepackage{relsize}
\AtBeginEnvironment{quote}{\noindent\smaller}
\AtBeginEnvironment{quotation}{\smaller}

% Clickable links in footnotes, TOC, etc.
\usepackage[
% 	xetex,
	bookmarks=true,
	colorlinks=false,
	linktoc=section,
	hidelinks,
	pdfusetitle,
]{hyperref}

% We want URLs to be italic and with regular uppercase numerals
\renewcommand{\UrlFont}{%
	\normalfont%
	\itshape%
	\addfontfeature{RawFeature=-onum}%
}

% In-text references
% cf. https://tex.stackexchange.com/a/139051
% Since German plural formation is not as regular as in English (-e, -en for
% Beispiel), we will define the label as empty
\usepackage[sort&compress,noabbrev]{cleveref}
\newcommand{\crefrangeconjunction}{--}
\crefname{xnumi}{}{}
\creflabelformat{xnumi}{(#2#1#3)}
\crefname{xnumii}{}{}
\creflabelformat{xnumii}{(#2#1#3)}
\crefname{xnumiii}{}{}
\creflabelformat{xnumiii}{(#2#1#3)}
\crefname{xnumiv}{}{}
\creflabelformat{xnumiv}{(#2#1#3)}
\crefrangeformat{xnumi}{(#3#1#4)--(#5#2#6)}
\crefrangeformat{xnumii}{(#3#1#4--#5\crefstripprefix{#1}{#2}#6)}
\crefrangemultiformat{xnumii}{(#3\arabic{xnumi}#1#4--#5#2#6)}
{ and~(#3\arabic{xnumi}#1#4--#5#2#6)}{, (#3\arabic{xnumi}#1#4--#5#2#6)}
{ and~(#3\arabic{xnumi}#1#4--#5#2#6)}

% Subsubsubsection
% cf. https://tex.stackexchange.com/a/356574
\DeclareNewSectionCommand[
  style=section,
  counterwithin=subsubsection,
  afterskip=1.5ex plus .2ex,
  beforeskip=3.25ex plus 1ex minus .2ex,
  afterindent=false,
  level=\paragraphnumdepth,
  tocindent=10em,
  tocnumwidth=5em
]{subsubsubsection}
\setcounter{secnumdepth}{\subsubsubsectionnumdepth}
\setcounter{tocdepth}{\subparagraphtocdepth}
\addto\extrasngerman{%
  \let\subsubsubsectionautorefname\subsubsectionautorefname
}
\crefalias{subsubsubsection}{subsubsection}

% Make glossaries
\makeglossaries

% Macros
\newcommand{\fw}[1]{\textit{#1}} % Foreign Word
\newcommand{\til}{\char`\~} % Literal tilde in text mode % {$\sim$}
\newcommand{\tit}[1]{\textit{#1}} % Title of a work
\newcommand{\tsub}[1]{\textsubscript{#1}} % Subscript
\newcommand{\tsup}[1]{\textsuperscript{#1}} % Superscript
\newcommand{\markyellow}[1]{\colorbox{yellow}{#1}} % Yellow highlighter
\newcommand{\ques}{\textsuperscript{?}} % raised question mark
\newcommand{\zwsp}{\mbox{​}} % Zero-width space (ZWSP)

\newcommand{\ayr}[1]{\zwsp\smash{{\Tagati #1}}} % Plain Ayeri orthography
\newcommand{\rayr}[2]{\zwsp\smash{{\Tagati #1}} \emph{#2}} % Ayeri orthography + *r*omanization
\newcommand{\tayr}[2]{#1 `#2'} % Romanization + *t*ranslation
\newcommand{\xayr}[3]{\zwsp\smash{\Tagati #1} \emph{#2} `#3'} % Ayeri orthography + romanization + translation

\newenvironment{mytitle}{
	\hfill
	\begin{minipage}{0.667\textwidth}
	\vspace{\baselineskip}
	\begin{center}
		\Large
		\sffamily\bfseries
		\makeatletter
}{
		\makeatother
	\end{center}
	\vspace{1em}
	\end{minipage}
	\hfill
}

% Change abstract font
\AtBeginEnvironment{abstract}{\small\sffamily}

%% END OF PREAMBLE %%%%%%%%%%%%%%%%%%%%%%%%%%%%%%%%%%%%%%%%%%%%%%%%%%%%%%%%%%%%

\begin{document}

%% MAIN PART %%%%%%%%%%%%%%%%%%%%%%%%%%%%%%%%%%%%%%%%%%%%%%%%%%%%%%%%%%%%%%%%%%

\begin{mytitle}
	\@title: \@subtitle
\end{mytitle}

\begin{abstract}
A conlang relay was held in May and June 2024 by conlang enthusiasts at the
Department of German studies and Linguistics at Humboldt University of Berlin
in association with a few non-Humboldtian friends, together comprising the
informal Berlin Conlangers Regulars' Table. Documentation here is on analyzing
my forerunner Dominique's torch in \tit{Hoan} and notes on translating the text
into \tit{Ayeri} as the fourth runner, as well as what lexical and grammatical
information I provided for my successor Henrik, who in turn translated the
Ayeri text into \tit{\scalebox{.75}{\textit{R}}u.lu} based on those brief
notes. Due to this round's small and very local scope of five active
participants, all of which are German speakers, it was naturally run in German.
\end{abstract}

In der Vorweihnachtszeit 2023 kam unter den Conlangern am Institut für deutsche
Sprache und Linguistik der Humboldt-Universität zu Berlin die Idee auf, ein
Relay durchzuführen: Ein Staffellauf mit Übersetzungen zwischen konstruierten
Sprachen (\fw{constructed languages}). Umgesetzt wurde der Plan allerdings erst
im Mai 2024. Insgesamt gab es fünf Teilnehmende, wobei ich mit \tit{Ayeri} den
vierten Platz in der Runde einnahm. Mein Vorgänger war Dominique mit
\tit{Hoan},%
%
	\footnote{Die Sprache der \q{Gehörnten} aus der auf Wolfgang und Heike
	Holbeins Roman \tit{Der Greif} (1989) basierten, gleichnamigen Serie auf
	Amazon Prime (2023). Den Text auf Hoan sowie meine Interpretation davon
	basierend auf ein paar kurzen Notizen zur Sprache, die sich auf die zum
	Verständnis des Texts notwendigen Informationen beschränkten, drucke ich hier
	mit Genehmigung des Autors ab (pers.~Mittlg., \DTMDate{2024-06-21}), welcher
	die Sprache für die Serie entwickelt hat.}
%
mein Nachfolger Henrik mit \tit{ʀu.lu}. Als Basis dienten die Spielregeln der
sechsten Language Creation Conference (2015).%
%
	\footnote{Siehe
	\url{https://conlang.org/language-creation-conference/lcc6/lcc6-relay/}
	(\DTMdate{2024-06-21}).}
%
Alle Teilnehmenden hatten allerdings statt zwei Tagen mindestens fünf Tage
Zeit, weil das Relay mitten im Semester stattfand. Das Spiel wurde außerdem
aufgrund des kleinen, lokalen Teilnehmerkreises auf Deutsch statt auf Englisch
durchgeführt.

%%%%%%%%%%%%%%%%%%%%%%%%%%%%%%%%%%%%%%%%%%%%%%%%%%%%%%%%%%%%%%%%%%%%%%%%%%%%%%%

\section{Analyse der Vorlage auf Hoan}
\label{sec:hoanalys}

Der im Folgenden zitierte Text auf Hoan wurde von Dominique an mich
weitergereicht. Meine Aufgabe bestand darin, ihn anhand einer beigegebenen
Skizze zur Grammatik und einer Wortliste im gleichen Umfang wie die Materialien
in \cref{sec:suppl} zu entschlüsseln und ins Deutsche zu übersetzen.%
%
	\footnote{Zum Zeitpunkt der Übersetzung natürlich mir unbekannt, handelt es
	sich um \tit{Das Hündlein von Bretta} aus den \tit{Deutschen Sagen} der
	Brüder Grimm \autocite[154--155]{grimm:dtsagen}.}

\begin{quote}
\textsc{\MakeLowercase{Φutliqe ma Φerm}}

Ḥooq \tsup{ə}nt’ekšaqkask č'oaqa zim, šentuʀ Φerm. Likuʀ ɸutliqe ʀlat'. Φutliq
qairam prenšaqkar. Iš ḥooq \tsup{ə}z'leetšankas hu sabaḥḥol, quumar te naqanat
\tsup{ə}xraqmu nšerik' iš \tsup{ə}nšer. K'uškark eax ma iičaq, iš tilqu
t'rikušam iš kaspos sxengam mat. Iš ɸutliq ʀamkar \tsup{ə}xraqmu nšerik' iš
\tsup{ə}nšer hu ḥooq, man ḥooq p'a ʀamšat'kasač \tsup{ə}nšer hu ɸutliq.

Roošu Luter \tsup{ə}z'leetšankask ɸutliq hu sabaḥḥol ma ralšu kʀai. Φutliq
xeksampkar hauku šar: \enquote{Z'om se xoɸa ma rangč'as ralšu kʀai.} Sabaḥḥol
qaraazat tilqu t'rikušam qun nšerik'u haukušam ku šar ma eaxu ka, iš qeem
troqšat'kas sikas ɸutliq. Hiɸ pʀuuruk'kas sika ma eaxu ku ɸutliq. Iš karkas:
\enquote{Xoɸa haa likuɸ \tsup{ə}nšer!}

Φutliqe ʀlat' \tsup{ə}knoštošam ta eax auškark \tsup{ə}nt'ekenu ku akraq.
\tsup{a}Rɸeem iš \tsup{ə}kzuš\tsup{u}m. Hiɸ hoanu ma č'oaq \tsup{ə}xp'aanšaq.
Se xo ʀtuklas t'rikkas z'almu ɸutle kil sika.
\end{quote}

Zunächst folgt die satzweise Glossierung mit kurzer Diskussion von Fragen, die
sich beim Versuch ergeben haben, das Übermittelte zu verstehen und sinnvoll zu
übersetzen, sowie mit Kommentaren zu grammatischen und inhaltlichen
Auf"|fälligkeiten diesbezüglich. Los geht es mit der Überschrift in
\cref{ex:hoan1}, zu der nichts weiter anzumerken ist.%
%
	\footnote{Der grammatischen Annotation der Beispiele liegen die
	\tit{\citetitle{lgr}} \autocite{lgr} zugrunde, vgl.~außerdem den
	\namecref{leipzig} \tit{\nameref{leipzig}}. Übersetzungen und
	Bedeutungsangaben stehen in Hochkommata.}

\begin{exe}
\ex \label{ex:hoan1}
	\gll Φutliq-e ma Φerm \\
		Hündchen-\Cls{2} in Fērm \\
	\trans \wdef{Das Hündchen zu Fērm}
\end{exe}

Der erste Satzteil in \cref{ex:hoan2} ist interessant, weil er im Perfektiv
steht und daher ergativ aligniert ist. Das syntaktische Subjekt des Satzes ist
also \fw{č'oaqa} \wdef{Stadt}, was anhand des verbalen Kongruenzsuf"|fixes
\fw{-šaq} sichtbar wird. Dennoch steht \fw{ḥooq} \wdef{Mensch} als Agens in der
Topikposition, es geht also im Folgenden um ihn. Die Markierung der
reportativen Evidenz (\fw{-k}) habe ich im Deutschen unübersetzt gelassen, da
durch den Kontext klar ist, dass es sich um eine Erzählung handelt.

\begin{exe}
\ex \label{ex:hoan2}
	\gll Ḥooq \tsup{ə}nt’ek-šaq-kas-k č'oaq-a zim, šent-u-ʀ Φerm. \\
		Mensch leben-\Pfv.\Cls{1}.\Abs-\Pfv.\Cls{1}.\Erg-\Hrs{} Stadt-\Cls{1a}
		klein Name-\Cls{1b}-\Cls{1}.\Pert{} Fērm \\
	\trans \wdef{Ein Mensch bewohnte eine kleine Stadt namens Fērm.}
\end{exe}

Die Form \fw{šentuʀ} im zweiten Satzteil von \cref{ex:hoan2} hat mich zunächst
verwirrt, weil die Endung \fw{-uʀ} wie eine imperfektive Verbendung aussieht,
dann allerdings mit \q{falschem} Themavokal. Tatsächlich aber müssen gemäß den
Grammatiknotizen Köpfe nach ihrer Nominalklasse markiert werden, wenn sie ein
Attribut besitzen.%
%
	\footnote{Ich habe auf eine explizite Glossierung als Attributivmarkierung
	(\Attr) verzichtet, außer bei der Attributivmarkierung mit \fw{-s}, weil
	dieses Suf"|fix alle Nominalklassen ohne Unterschied betrifft. Um die
	Glossierung nicht unnötig in die Länge zu treiben, habe ich außerdem bei
	der Kongruenzmarkierung von Verben die explizite Bezeichnung der dritten
	Person (\Third) ausgespart. Die Markierung der Nominalklasse (\Cls{\#})
	impliziert diese also immer.}
%
Da Hoan kopfmarkierend ist, besitzt es statt eines Genitivs einen Pertensiv,
der hier durch das imperfektive Objekt-Pronominalsuf"|fix
\fw{-ʀ} ausgedrückt wird.

Der Satz in \cref{ex:hoan3} enthält prädikative Possession, bei welcher die
Präposition \fw{liku} \wdef{bei} im Prinzip als imperfektives Verb fungiert.
Die Grammatik unterscheidet kaum zwischen Adjektiven und Substantiven, der Satz
bedeutet wörtlich also in etwa \wdef{Bei ihm ist ein Hündchen von Treue}. Da
der Mensch, \fw{ḥooq}, aus \cref{ex:hoan2} lediglich durch die pronominale
Endung \fw{-ʀ} repräsentiert ist, gehe ich davon aus, dass Hoan eine
\fw{Pro-Drop}-Sprache ist.

\begin{exe}
\ex \label{ex:hoan3}
	\gll Liku-ʀ ɸutliq-e ʀlat'. \\
		bei.sein-\Ipfv.\Cls{1} Hündchen-\Cls{2} treu \\
	\trans \wdef{Er hatte ein treues Hündchen.}
\end{exe}

Auch im nächsten Satz \cref{ex:hoan4} wird der Mensch nur durch das perfektive
Pronominalsuf"|fix \fw{-šaq} referenziert, das ihn als Subjekt im Absolutiv
markiert. Das Hündchen steht jetzt in der Topikposition. Im Folgenden ist also
damit zu rechnen, dass es seinerseits lediglich durch ein Pronominalsuf"|fix am
Verb vertreten erscheint.

\begin{exe}
\ex \label{ex:hoan4}
	\gll Φutliq qairam pren-šaq-kar. \\
		Hündchen stets gehorchen-\Pfv.\Cls{1}.\Abs-\Pfv.\Cls{2}.\Erg{} \\
	\trans \wdef{Das Hündchen gehorchte ihm stets.}
\end{exe}

Diese Annahme wird in \cref{ex:hoan5a} auch direkt bestätigt, insofern der
Bezug auf das Hündchen dort mit \fw{-šan} als Subjektsuf"|fix geschieht. Durch
den Wechsel zu einer imperfektiven Handlung erscheinen am Verb in
\cref{ex:hoan5b} andere Kongruenzaf"|fixe als zuvor, weil Hoan im Imperfekt
akkusatives Alignment aufweist. Entsprechend wird das Hündchen, \fw{ɸutliq},
jetzt mit dem pronominalen Präfix \fw{na-} im Nominativ aufgenommen. Der
Quantor \fw{\tsup{ə}xraqmu} \wdef{eine kleine Menge} zeigt ein
Attributivsuf"|fix \fw{-u}, da er vom Quantifizierten komplementiert wird. Den
Ausdruck \fw{quumar te}, der Obligation ausdrückt, habe ich sowohl in der
Glossierung als auch in der Übersetzung durch \wdef{auf dass} mit Konjunktiv
Präsens wiedergegeben. Diese Lösung schien mir am elegantesten.

\begin{exe}
\ex \label{ex:hoan5}
	\begin{xlist}
	\ex \label{ex:hoan5a}
		\gll Iš ḥooq \tsup{ə}z'leet-šan-kas hu sabaḥḥol, \\
			und Mensch schicken-\Pfv.\Cls{2}.\Abs-\Pfv.\Cls{1}.\Erg{}
			zu Metzger, \\
	\ex \label{ex:hoan5b}
		\gll {quumar te} na-qan-at \tsup{ə}xraqm-u nšerik' iš \tsup{ə}nšer. \\
			{auf dass} \Ipfv.\Cls{2}.\Nom-kaufen-\Ipfv.\Cls{3}.\Acc{}
			etwas-\Cls{3} Wurst und Fleisch \\
	\end{xlist}
	\trans \wdef{Und der Mensch schickte es zum Metzger, auf dass es etwas
		Wurst und Fleisch kaufe.}
\end{exe}

In \cref{ex:hoan6a} tritt auf"|fälligerweise auch innerhalb eines Absatzes das
Reportativsuf"|fix \fw{-k} auf. Dies ist den Notizen zur Grammatik zufolge dem
Topikwechsel vom Menschen zum Hündchen geschuldet, bei dem diese Möglichkeit
besteht. Auf"|fällig ist weiterhin, dass das Verb nicht für das Subjekt \fw{eax}
\wdef{Korb} markiert ist, sondern nur für die Topik beziehungsweise das Objekt,
\fw{-kar} \wdef{es}. Laut der Grammatikskizze ist das Subjektsuf"|fix
\fw{-šat'} (3.~Pers.~Kl.~3~Abs.~Pfv.), das in diesem Kontext erscheinen müsste,
optional.

\begin{exe}
\ex \label{ex:hoan6}
	\begin{xlist}
	\ex \label{ex:hoan6a}
		\gll K'uš-kar-k eax ma iičaq, \\
			nehmen-\Pfv.\Cls{2}.\Erg-\Hrs{} Korb in Maul \\
	\ex \label{ex:hoan6b}
		\gll iš tilq-u t'rik-uš-am iš kasp-o-s sxeng-am ma-t. \\
			und Wort-\Cls{3} ritzen-\Antip-\Ptcp{} und Silber-\Sgv-\Attr{}
			genug-\Ptcp{} in-\Cls{3} \\
	\end{xlist}
	\trans \wdef{Es nahm den Korb ins Maul, und darin war ein Schriftstück
		 sowie ausreichend Geld.}
\end{exe}

Das Suf"|fix \fw{-uš} in \cref{ex:hoan6b} habe ich als (verbalisierenden)
Antipassivmarker interpretiert. Die Kurzgrammatik gibt diesen mit
\fw{-aš} an. Ich vermute also, dass der Vokal sich in seiner Höhe an den
Verbstamm anpasst (\fw{u} \til{} \fw{u}), wie auch bei den Endungen der
imperfektiven Objektkongruenz.%
%
	\footnote{Wenn im ergativen Alignment \emph{x}\tsub{i} (O\tsub{A}, \Erg{})
	\emph{ein Wort}\tsub{j} (S\tsub{P}, \Abs) \emph{ritzt}\tsub{j}, wird im
	entsprechenden antipassiven Satz \emph{x} zum intransitiven Subjekt:
	\emph{x}\tsub{i} (S\tsub{A},
	\Abs) \emph{ritzt}\tsub{i}. Normalerweise müsste ein antipassives Partizip
	also mit einer imperfektiven Lesart einhergehen (\wdef{ritzend},
	\wdef{befehlend}; vgl.~\cite[{316 und die Referenzen dort}]{polinsky2017}),
	was im Gegensatz zu meiner aus dem Kontext abgeleiteten Annahme steht, dass
	die Partizipformen hier perfektiv im Sinne eines Zustandspassivs
	aufzufassen sind (\wdef{geritzt sein}, \wdef{befohlen sein}).
	% Möglicherweise ist auch ein \q{unechtes} Antipassiv intendiert, bei dem
	% das weniger belebte Argument eines transitiven Verbs als Agens auftritt,
	% aber keine Demotion des transitiven Subjekts erfolgt
	% \autocite[vgl.][{327--328 und Referenzen dort}]{polinsky2017}?
}
%
Die Präposition \fw{mat} am Ende des Halbsatzes wird wohl am besten als
\wdef{darin} zu interpretieren sein. Ihr Objekt besteht im Prinzip in der
pronominalen Markierung mit \fw{-t}. Zunächst hatte mich die vorliegende
Konstruktion sehr verwirrt, weil mir nicht klar war, welche \textsc{np} der
Klasse~3 die Präposition im Kontext modifiziert. \fw{Eax} \wdef{Korb} stellt
bei einer Lesart als Pronominaladverb aber eine sinnvolle Referenz dar;
\cref{ex:hoan12} weist dieses Lexem explizit als zur Klasse~3 gehörig aus. Da
Präpositionen als Adverbiale ganze Sätze modifizieren können, bietet sich eine
Interpretation als prädikative Konstruktion an. Hoan besitzt keine overte
Kopula, daher stehen Subjekt und Prädikat nebeneinander.

Der nächste Satz in \cref{ex:hoan7} bereitete keine Probleme beim Übersetzen.
Allein, dass der Mensch kein Fleisch für das Hündchen erhält, erscheint an
dieser Stelle inhaltlich auf"|fällig. Durch den weiteren Verlauf der Geschichte
wird deutlich, warum der Mensch kein Fleisch bekommt -- \emph{vom} Hündchen.

\begin{exe}
\ex \label{ex:hoan7}
	\begin{xlist}
	\ex \label{ex:hoan7a}
		\gll Iš ɸutliq ʀam-kar \tsup{ə}xraqm-u nšerik' iš \tsup{ə}nšer hu
			ḥooq, \\
			und Hündchen erhalten-\Pfv.\Cls{2}.\Erg{} etwas-\Cls{3}
			Wurst und Fleisch für Mensch \\
	\ex \label{ex:hoan7b}
		\gll man ḥooq p'a ʀam-šat'-kas=ač \tsup{ə}nšer hu ɸutliq. \\
			aber Mensch
			\Neg{} erhalten-\Pfv.\Cls{3}.\Abs-\Pfv.\Cls{1}.\Erg=\Neg{}
			Fleisch für Hündchen \\
	\end{xlist}
	\trans \wdef{Und das Hündchen erhielt etwas Wurst und Fleisch für den
	Menschen, doch der Mensch erhielt kein Fleisch \textins{vom} Hündchen.}
\end{exe}

Mit \cref{ex:hoan8} beginnt ein neuer Absatz, entsprechend findet sich hier
wieder das Reportativsuf"|fix \fw{-k} am Verb. Was es mit dem fehlenden Fleisch
auf sich hat, wird hier noch nicht erklärt, jedoch scheint der Herr des
Hündchens Luther zu heißen. Das Glossar gibt seinen Namen als \q{(Martin)
Luther} an. Handelt es sich beim ursprünglichen Text also um eine Parabel des
oder über den Reformations\-theologen? Dass es sich bei \fw{ralšu kʀai}
\wdef{sechster Tag} um den Freitag handelt, ist aus dem Glossar zu entnehmen.%
%
	\footnote{Vgl. auch lat. \fw{fēria sexta}, port. \fw{sexta-feira}
	\wdef{Freitag}.}

\begin{exe}
\ex \label{ex:hoan8}
	\gll Rooš-u Luter \tsup{ə}z'leet-šan-kas-k ɸutliq hu sabaḥḥol ma ralš-u
		kʀai. \\
		Herr-\Cls{1b} Luther
		schicken-\Pfv.\Cls{2}.\Abs-\Pfv.\Cls{1}.\Erg-\Hrs{} Hündchen zu Metzger
		an Tag-\Cls{3} sechs \\
	\trans \wdef{Herr Luther schickte das Hündchen an einem Freitag zum
		Metzger.}
\end{exe} 

Wie zuvor bei \cref{ex:hoan6a} beobachtet, enthält auch das Verb in
\cref{ex:hoan9} keine Markierung für das Subjekt \fw{hauku šar} \wdef{Befehl
des Vaters/des Greifen/des Papstes}. Das optionale Suf"|fix würde auch in diesem
Fall \fw{-šat'} lauten.%
%
	\footnote{Warum \fw{hauku} \wdef{Befehl} anders als \fw{šentuʀ}
	\wdef{Name} in \cref{ex:hoan2} nur für ein Attribut markiert ist, nicht
	jedoch für den Possessor \fw{šar} \wdef{Papst}, geht aus den Notizen zur
	Grammatik augenscheinlich nicht hervor. Die Form *\fw{haukuʀ} wäre ohne
	tiefere Kenntnis der Sprache zu erwarten gewesen. Wörtlich müsste \fw{hauku
	šar} dem Kontext nach wohl mit \wdef{Papstbefehl} zu übersetzen sein.}

\begin{exe}
\ex \label{ex:hoan9}
	\gll Φutliq xeksamp-kar hauk-u šar: \\
		Hündchen bringen-\Pfv.\Cls{2}.\Erg{} Befehl-\Cls{3} Papst \\
	\trans \wdef{Das Hündchen brachte den Befehl des Papstes: \dots}
\end{exe}

Auch \cref{ex:hoan10} ließ sich problemlos übersetzen. Die unmarkierte Verbform
\fw{z'om} \wdef{fasten} ist hier als Imperativ \wdef{faste(t)!} aufzufassen. Da
eine solche Anordnung des Papstes ihrer Intention nach an die ganze
Christenheit gerichtet ist, bin ich bei der Übersetzung ins Deutsche vom Plural
ausgegangen. Numerus wird in Hoan nicht obligatorisch markiert.

\begin{exe}
\ex \label{ex:hoan10}
	\gll Z'om se xoɸa ma rangč'a-s ralš-u kʀai. \\
		fasten.\Imp{} von jetzt an Gesamtheit-\Attr{} Tag-\Cls{3} sechs \\
	\trans \wdef{Fastet ab jetzt den ganzen Freitag lang.}
\end{exe}

In \cref{ex:hoan11a} wechselt der Modus zum Imperfektiv, sodass dieser Teilsatz
wieder akkusativ aligniert ist. Der Teilsatz enthält zwei Partizipien,
\fw{t'rikušam} \wdef{geritzt}, und \fw{haukušam} \wdef{befohlen}. Auch hier
scheint sich beim Vergleich mit \cref{ex:hoan6} der Vokal im Suf"|fix \fw{-aš}
in beiden Fällen dem Öffnungsgrad des vorhergehenden Vokals anzupassen (\fw{i,
u} \til{} \fw{u}).

\begin{exe}
\ex \label{ex:hoan11}
	\begin{xlist}
	\ex \label{ex:hoan11a}
		\gll Sabaḥḥol qa-raaz-at tilq-u t'rik-uš-am qun nšerik'-u hauk-uš-am ku
			šar ma eax-u ka, \\
			Metzger \Ipfv.\Cls{2}.\Nom-erblicken-\Ipfv.\Cls{3}.\Acc{}
			Wort-\Cls{3} ritzen-\Antip-\Ptcp{} betreffs Wurst-\Cls{3}
			Befehl-\Antip-\Ptcp{} von Papst in Korb-\Cls{3} \Hrs{} \\
	\ex \label{ex:hoan11b}
		\gll iš qeem troq-šat'-kas sika-s ɸutliq. \\
			und \Mir{} abhauen-\Pfv.\Cls{3}.\Abs-\Pfv.\Cls{1}.\Erg{}
			Schwanz-\Attr{} Hündchen \\
	\end{xlist}
	\trans \wdef{Der Metzger erblickte das Schriftstück darüber, was der
	Papst zur Wurst befohlen hatte, im Korb und hieb doch dem Hündchen den
	Schwanz ab!}
\end{exe}

Für die Mirativpartikel \fw{qeem} in \cref{ex:hoan11b} gibt es keine direkte
deutsche Entsprechung, am ehesten übernimmt \fw{doch} diese Funktion. Daher
habe ich sie gemäß der Grammatikskizze ihrer Funktion nach mit \Mir{} glossiert
und diesen Satzteil in der Übersetzung als Ausrufesatz formuliert.

Die letzten beiden Sätze des Absatzes, \cref{ex:hoan12} und \cref{ex:hoan13},
waren vollkommen problemlos übersetzbar und beinhalten keine weiteren
Auf"|fälligkeiten, abgesehen vielleicht von der Possessivkonstruktion in
\cref{ex:hoan13}, welche der in \cref{ex:hoan3} gleicht.

\begin{exe}
\ex \label{ex:hoan12}
	\gll Hiɸ pʀuuruk'-kas sika ma eaxu ku ɸutliq. \\
		dann put-\Pfv.\Cls{1}.\Erg{} Schwanz in Korb-\Cls{3} von Hündchen \\
	\trans \wdef{Dann legte er den Schwanz in den Korb des Hündchens.}
\end{exe}

\begin{exe}
\ex \label{ex:hoan13}
	\gll Iš kar-kas: Xoɸa haa liku-ɸ \tsup{ə}nšer! \\
		und sagen-\Pfv.\Cls{1}.\Erg{} jetzt \Prsv{} bei-\Second{} Fleisch \\
	\trans \wdef{Und er sprach: Da hast du das Fleisch!}
\end{exe}

Die Annahme zu \cref{ex:hoan11}, dass sich das Antipassivsuf"|fix \fw{-aš} nach
dem Öffnungsgrad des vorhergehenden Vokals richtet, bestätigt sich in
\cref{ex:hoan14}, wo es in \fw{\tsup{ə}knošt-oš-am} \wdef{verletzt} die Form
\fw{-oš} aufweist (\fw{o} \til{} \fw{o}). Das Partizip ist dem Kontext dieses
Satzes nach unmissverständlich mit perfektiver Bedeutung zu verstehen: Der
boshafte Metzger hat dem Hündchen den Schwanz abgehauen \cref{ex:hoan11}, nun
ist es verletzt. Die Formulierung \fw{\tsup{ə}nt'eken-u ku akraq}
\wdef{Behausung(en) der Straße(n)} mit alienabler Possession ist mir nicht ganz
klar. Ich vermute, dass damit allgemein Häuser an der Straße gemeint sind.
Weiterhin ist interessant, dass das Ziel des Laufens als direktes Objekt
ausgedrückt wird. Die Grammatikskizze benennt diese Möglichkeit explizit.

\begin{exe}
\ex \label{ex:hoan14}
	\gll Φutliq-e ʀlat' \tsup{ə}knošt-oš-am ta eax auš-kar-k \tsup{ə}nt'eken-u
		ku akraq. \\
		Hündchen-\Cls{2} treu verletzen-\Antip-\Ptcp{} mit Korb
		laufen-\Pfv.\Cls{2}.\Erg{}-\Hrs{} Behausung-\Cls{3} von Straße \\
	\trans \wdef{Das verletzte treue Hündchen mit dem Korb lief zu den Häusern
		\textins{an der} Straße.}
\end{exe}

Die unflektierten Verbformen in \cref{ex:hoan15} weisen zunächst auf Imperative
hin, die hier allerdings nicht in den Kontext von Figurenrede eingebunden sind.
Dass der Erzähler, der vorher noch mit dem Hündchen sympathisiert, ihm
plötzlich den Tod wünscht, ist unwahrscheinlich. Im Kontext werden diese Verben
wohl am besten als Verbalabstrakta in einem Nominalsatz zu übersetzen sein,
welche die weitere Handlung stichwortartig raffen.

\begin{exe}
\ex \label{ex:hoan15}
	\gll \tsup{a}Rɸeem iš \tsup{ə}kzuš\tsup{u}m. \\
		stürzen und verenden \\
	\trans \wdef{Sturz und Tod.}
\end{exe}

Auch die allerletzten zwei Sätze in \cref{ex:hoan16} und \cref{ex:hoan17} sind
unauf"|fällig und leicht zu übersetzen. Da in \cref{ex:hoan17} eine allgemeine
Aussage gemacht wird, habe ich mich entschieden, für das Subjekt des Satzes,
\fw{z'almu ɸutle kil sika} \wdef{Bilder von Hunden ohne Schwanz} den Plural zu
benutzen.

\begin{exe}
\ex \label{ex:hoan16}
	\gll Hiɸ hoan-u ma č'oaq \tsup{ə}xp'aan-šaq. \\
		dann Leute-\Cls{1b} in Stadt weinen-\Pfv.\Cls{1}.\Abs{} \\
	\trans \wdef{Da weinten die Leute in der Stadt.}
\end{exe}

\begin{exe}
\ex \label{ex:hoan17}
	\gll Se xo ʀtuklas t'rik-kas z'alm-u ɸutl-e kil sika. \\
		von \Dem.\Cls{3b} Grund meißeln-\Pfv.\Cls{1}.\Erg{} Bild-\Cls{3}
		Hund-\Cls{2} ohne Schwanz \\
	\trans \wdef{Darum fertigten sie Statuen von Hunden ohne Schwanz an.}
\end{exe}

%%%%%%%%%%%%%%%%%%%%%%%%%%%%%%%%%%%%%%%%%%%%%%%%%%%%%%%%%%%%%%%%%%%%%%%%%%%%%%%

\section{Gegenüberstellung der Übersetzungen ins Deutsche}
\label{sec:dtuebers}

Der folgende deutschsprachige Text resultiert aus meiner Interpretation von
Dominiques Text auf Hoan (\cref{sec:hoanalys}). Der Text ist teilweise etwas
inkohärent, wahrscheinlich, weil er bereits drei Übersetzungen durchlaufen hat:
Deutsch~-- Sal Qīnaion~-- Kèramkaq~-- Hoan.

\begin{quote}
\textsc{\MakeLowercase{Das Hündchen zu Fērm}}

Ein Mensch bewohnte eine kleine Stadt namens Fērm. Er hatte ein treues
Hündchen. Das Hündchen gehorchte ihm stets. Und der Mensch schickte es zum
Metzger, auf dass es etwas Wurst und Fleisch kaufe. Es nahm den Korb ins Maul,
und darin war ein Schriftstück sowie ausreichend Geld. Und das Hündchen erhielt
etwas Wurst und Fleisch für den Menschen, doch der Mensch erhielt kein Fleisch
vom Hündchen.%
%
	\footnote{\fw{vom Hündchen}] wörtlich: \fw{für das Hündchen}.}

Herr Luther schickte das Hündchen an einem Freitag zum Metzger. Das Hündchen
brachte den Befehl des Papstes: \q{Fastet ab jetzt den ganzen Freitag lang.}
Der Metzger erblickte das Schriftstück darüber, was der Papst zur Wurst
befohlen hatte, im Korb und hieb doch dem Hündchen den Schwanz ab! Dann legte
er den Schwanz in den Korb des Hündchens. Und er sprach: \q{Da hast du das
Fleisch!}

Das verletzte treue Hündchen mit dem Korb lief zu den Häusern an der Straße.%
%
	\footnote{\fw{Häusern an der Straße}] wörtlich: \fw{Behausung(en) der
	Straße(n)}.}
%
Sturz und Tod. Da weinten die Leute in der Stadt. Darum fertigten sie Statuen
von Hunden ohne Schwanz an.
\end{quote}

Auf Grundlage des obigen Textes habe ich die Übersetzung auf Ayeri angefertigt
(\cref{sec:ayruebers}) und die nachstehende Rückübersetzung ins Deutsche
vorgenommen. Ich habe versucht, den Textsinn etwas zu verbessern, ohne aber zu
große Eingriffe vorzunehmen. Der Text sollte nun also wieder an Kohärenz
gewonnen haben, freilich ohne das Original zu kennen. Ich konnte mir nicht
verkneifen, die Namen anzupassen beziehungsweise zu übersetzen: \fw{Fērm} wird
also zu \fw{Peram} und aus \fw{Herrn Luther} wird \fw{Apitschan dijan}.%
%
	\footnote{\rayr{ApitYnF}{Apican}, \rayr{Apitu}{Apitu}, zu
	\xayr{Apitu}{apitu}{rein}, \xayr{ApitF/}{apit-}{reinigen};
	vgl.~mittelhochdeutsch \fw{lūter} \wdef{hell, rein, klar, lauter}
	\autocite[{s.\,v.~\textit{lûter}}]{lexer:mhdhwb}, neuhochdeutsch
	\fw{lauter}.}

\begin{quote}
\textsc{\MakeLowercase{Das Hündchen zu Peram}}

Ein Mann wohnte in einer kleinen Stadt mit Namen Peram. Er besaß ein treues
Hündchen, das ihm immer zu gehorchen pflegte. Eines Tages schickte der Mann das
Hündchen zum Metzger, auf dass es ein wenig Wurst und Fleisch kaufe. Da packte
es mit dem Maul den Korb, worin sich ausreichend Geld sowie ein Schriftstück
befanden, und machte sich auf den Weg. Und das Hündchen bekam etwas Wurst und
Fleisch für den Mann, doch der Mann bekam nichts vom Hündchen zurück.

An einem Freitag nämlich hatte Apitschan dijan das Hündchen zum Metzger
geschickt. Das Hündchen brachte den Befehl des Hohepriesters: \q{Fastet ab
heute regelmäßig den ganzen Freitag.} Der Metzger erblickte im Korb den Brief
darüber, was der Hohepriester bezüglich der Wurst befohlen hatte, und hieb dem
Hündchen den Schwanz ab! Dann legte er den Schwanz in den Korb des Hündchens.
Und er sprach: \q{Da ist dein Fleisch!}

Das verletzte treue Hündchen lief mit dem \mbox{Korb} zurück zur Straße, doch
es stürzte und kam zu Tode. Da vergossen die Leute in der Stadt zehntausend
Tränen. Seither werden dort Statuen von Hunden ohne Schwanz
angefertigt.
\end{quote}

%%%%%%%%%%%%%%%%%%%%%%%%%%%%%%%%%%%%%%%%%%%%%%%%%%%%%%%%%%%%%%%%%%%%%%%%%%%%%%%

\section{Übersetzung auf Ayeri}
\label{sec:ayruebers}

Nach der eingehenden Analyse des Textes auf Hoan (\cref{sec:hoanalys}) und der
Rückübersetzung ins Deutsche (\cref{sec:dtuebers}) habe ich versucht, den so
gewonnenen Text Satz für Satz auf Ayeri zu übersetzen. Der komplette Text ist
unten sowohl in lateinischer Schrift als auch in Tahano abgedruckt und bildete
die Grundlage für Henriks Übersetzung auf ʀu.lu. Der Text ist vergleichsweise
lang, sodass es durchaus von Vorteil war, fünf Tage Zeit zu haben. Andererseits
ist er lang genug, um einige der interessanteren grammatischen Merkmale der
Sprache unterzubringen und zur Schau zu stellen.

\begin{quote}
\textsc{\MakeLowercase{Veney-veney ya Peram}}

Ang mitanya ayon ayronya kivo garaneri Peram. Ang tahisaya veney-veneyas nasi
si yam rodasa\-yong ya tadayen. Sa turaya bahisya men ayonang veney-veney
baryatiyam, kadāre ang mya inco tubayley nay bariley-kay. Ang da-kacisayo
bantari yona kasuley, siyā yomāran pangisreng-ma, pan\-yanreng naynay, nay ang
sitang-payo sasānya. Nay ang tavyo veney-veney tubayley nay bariley-kay
ayonyam, nārya ang ta-tavya ayon ranyaley veney-veneyena.

Ya turaya mayisa ang Apican diyan veney-veneyas baryatiyam Miyan da-cuyam. Ang
anlyo veney-veney nosānas natrayonena visam: \q{Cunu dabas gutasayam ya Miyan
ikan.} Ang silvya baryati kasuya tamanley minena si ang nosaya mayisa natrayon
visam barina nay ang hayarya māy sitramas veney-veneyena. Ang tapyya epang
sitramas kasuya veney-veneyena. Da-narayāng: \q{Adaya barireng vana!}

Ang sa-sarayo veney-veney nasi nupisa kirinya kayvo kasuya, nārya lesayong nay
ang pen\-gal\-yo tenyanas. Ang da-teryon keynam ayronya simbeyley samang. Sa
tiyasayo masahatay adaya gebisanye yelang veneyyeri si sitramya kayvay.
\end{quote}

\begin{quote}
\hyphenpenalty=10000
\small\Tagati
\begin{multicols}{2}[\textbf{venej/venej y permF}]
ANF mitnFy AyonF AjronFy kivo grneri permF. ANF thisy venej/veneysF nsi si ymF
rodsyoNF y tdyenF. s tury bhisFy menF AyonNF venej/venej brFytiymF, kdaare ANF
mY IMtFyo tubjlej nj brilej/kj. ANF d/ktYisyo bMtri yon ksulej, siyaa yomaarnF
pNisFreNF/m, pnFynFreNF njnj, nj ANF sitNF/pyo ssaanFy. nj ANF tvFyo
venej/venej tubjlej nj brilej/kj AyonFymF, naarFy ANF t/tvFy AyonF rnFylej
venej/veneyen.

y tury myis ANF ApitYnF diynF venej/veneysF brFytiymF miynF d/kYuymF. ANF
AnFlFyo venej/venej nosaansF ntFryonen vismF~– kYunu dbsF gutsymF y miynF IknF.
ANF silvFy brFyti ksuy tmnFlej minen si ANF nosy myis ntFryon vismF brin nj ANF
hyrFy maaj sitFrmsF venej/veneyen. ANF tpYFy EpNF sitFrmsF ksuy venej/veneyen.
d/nryaanF~– Ady brireNF vn!

ANF s/sryo venej/venej nsi nupis kirinFy kjvo ksuy, naarFy lesyoNF nj ANF
peNlFyo tenFynsF. ANF d/terFyonF kejnmF AjronFy siMbejlej smNF. s tiysyo
mshtj Ady gebisFnFye yelNF venejyeri si sitFrmFy kjvj.
\end{multicols}
\end{quote}

Analog zu \cref{sec:hoanalys} werde ich im Folgenden eine kommentierte Analyse
meiner Übersetzung auf Ayeri geben. Dies half auch dabei, die Notizen zur
Grammatik für meinen Nachfolger vorzubereiten, indem daraus deutlich wurde,
welche grammatischen Phänomene neben den absoluten Grundlagen der Flexion und
Syntax zu erklären sind. Auch hier soll der Kommentar chronologisch mit der
Überschrift in \cref{ex:ayr1} beginnen.

Anders als in Hoan spielt Kasusmarkierung in Ayeri eine große Rolle. Wo
\textsc{np}s nicht in Sätze eingebettet sind, erfolgt keine explizite
Markierung von Subjekten, wie \xayr{venej/venej}{veney-veney}{das Hündchen}
illustriert. Der Lokativ bei der Ortsangabe \xayr{y permF}{ya Peram}{in Peram}
ist hingegen semantisch motiviert und darum auch in diesem Kontext notwendig.
Die Kasusmarkierung erscheint bei Namen als Proklitikum, bei Appellativa
dagegen als Suf"|fix. Ayeri besitzt keine Markierung von Definitheit, eine
unspezifische Lesart (\wdef{irgend-}) kann allerdings optional durch ein Präfix
markiert werden, das in diesem Text jedoch nicht vorkommt.

\begin{exe}
\ex \label{ex:ayr1}
	\gll Veney\til{}veney ya=Peram \\
		Hund-\Dim{} \Loc=Peram \\
	\trans \wdef{Das Hündchen zu Peram}
\end{exe}

Weitere Besonderheiten der Kasusmarkierung in Ayeri werden in \cref{ex:ayr2}
deutlich. Auch \xayr{AyonF}{\mbox{ayon}}{Mann} ist oberflächlich unflektiert.
Der Kasusmarker befindet sich in seiner klitischen Form stattdessen links vom
Verb, um anzuzeigen, dass diese \textsc{np} die Topik des Satzes bildet.
Topikmarkierung ist obligatorisch in transitiven Sätzen, Tempusmarkierung
dagegen immer fakultativ.

\begin{exe}
\ex \label{ex:ayr2}
	\gll Ang=mitan-ya ayon ayron-ya kivo garan-eri Peram. \\
		\AgtT=wohnen-\Tsg.\M{} Mann[\Top] Stadt-\Loc{} klein Name-\Ins{}
		Peram \\
	\trans \wdef{Ein Mann wohnte in einer kleinen Stadt mit Namen Peram.}
\end{exe}

Überdies beschreibt \xayr{grneri}{garaneri}{mit Namen} natürlich nicht das
Mittel, mit dem gewohnt wird, sondern stellt den Kopf des Komplements von
\xayr{AjronFy}{ayronya}{in der Stadt} dar. Der Instrumentalis ist in diesem
Kontext also Strukturkasus, der Komplementierung anzeigt. Beinhaltet die
übergeordnete \textsc{np} ein Adjektiv, wird das Komplement zur Desambiguierung
der Modifikationsrelation rechtsversetzt. Daher erscheint das auf
\rayr{Ajrony}{ayronya} bezogene Adjunkt
\xayr{kivo}{kivo}{klein} hier vor dem Komplement \rayr{grneri permF}{garaneri
Peram} statt hinter \rayr{permF}{Peram}, das es so explizit nicht modifiziert.

% \begin{figure}
% \begin{forest}
% [DP
% 	[D′
% 		[NP
% 			[N′
% 				[N⁰
% 					[ayronya]
% 				]
% 				[$t$]
% 			]
% 			[AP
% 				[A⁰
% 					[kivo]
% 				]
% 			]
% 		]
% 	]
% 	[S
% 		[NP
% 			[N⁰
% 				[garaneri]
% 			]
% 		]
% 		[NP
% 			[N⁰
% 				[Peram]
% 			]
% 		]
% 	]
% ]
% \end{forest}
% \caption{Konstituentenstruktur von \fw{ayronya kivo garaneri Peram}}
% \label{fig:konststructayron}
% \end{figure}

Die Markierung der Topik durch Kongruenz betrifft nicht nur Nomen, sondern auch
Pronomen, wie die Verbform \xayr{ANF thisy}{ang tahisaya}{er besaß} in
\cref{ex:ayr3a} zeigt. Ayeri ist insofern eine \fw{Pro-Drop}-Sprache, als
Subjektpronomen enklitisch an die Verbform treten und die Kongruenzmarkierung
ersetzen. Topikalisierte Subjektpronomen haben allerdings die gleiche Form wie
reguläre Kongruenz\-endungen.%
%
	\footnote{Das Adjektiv \xayr{nsi}{nasi}{treu, loyal} wurde für diesen Zweck
	neu gebildet, nämlich als Ableitung vom Verbstamm
	\xayr{nsFyF/}{nasy-}{folgen}.}
\begin{exe}
\ex \label{ex:ayr3}
	\begin{xlist}
	\ex \label{ex:ayr3a}
		\gll Ang=tahisa=ya veney\til{}veney-as nasi \\
			\AgtT=besitzen=\Tsg.\M.\Top{} Hund-\Dim{}-\Parg{} treu \\
	\ex \label{ex:ayr3b}
		\gll si yam=roda-asa=yong ya tadayen. \\
			\Rel{} \DatT=gehorchen-\Hab=\Tsg.\N.\Aarg{} \Tsg.\M.\Top{} immer \\
	\end{xlist}
	\trans \wdef{Er besaß ein treues Hündchen, das ihm immer zu gehorchen
		pflegte.}
\end{exe}

Das Verb im Relativsatz in \cref{ex:ayr3b} hat als Subjekt das Hündchen, das im
Matrixsatz eingeführt wurde, wechselt aber im abhängigen Satz noch nicht die
Topik, weshalb das Verb \xayr{rodsyoNF}{rodasayong}{es gehorcht (gewöhnlich)}
Topikmarkierung für das Pronomen im Dativ mit Referenz auf den Mann zeigt. Eine
Besonderheit beim Habitativsuf"|fix \rayr{/As}{-asa} ist, dass das erste
\fw{-a} einen vorhergehenden Vokal tilgt. Daher erscheint
\rayr{rodsyoNF}{rodasayong} zu \xayr{rod/}{roda-}{gehorchen} mit kurzem \fw{a}
trotz der phonotaktischen Regel, dass zwei aufeinander folgende Vokale mit
gleicher Qualität einen Langvokal bilden. Darüber hinaus ist anzumerken, dass
Relativsätze, die mit einfachem \rayr{si}{si} ohne sekundäre Kasusmarkierung
eingeleitet werden, einen inneren Kopf enthalten, wenn eine overte Verbform
vorhanden ist. Daher weist das Verb das Pronominalklitikum
\xayr{/yoNF}{-yong}{es} auf. \rayr{si}{si} ist also eher als Subjunktion
(\textsc{c⁰}) mit attribuierender Funktion denn als Relativpronomen
(\textsc{d⁰} in \textsc{spec-cp}) zu verstehen.

Nachdem das Hündchen, \rayr{venej/venej}{veney-veney}, nun im Diskurs etabliert
ist und die weiteren Sätze vom Hündchen handeln, wechselt in \cref{ex:ayr5a}
auch die Topikmarkierung entsprechend.

\begin{exe}
\ex \label{ex:ayr5}
	\begin{xlist}
	\ex \label{ex:ayr5a}
		\gll Sa=tura-ya bahis-ya men ayon-ang veney\til{}veney baryati-yam, \\
			\PatT=schicken-\Tsg.\M{} Tag-\Loc{} ein Mann-\Aarg{}
			Hund-\Dim{}[\Top] Metzger-\Dat{} \\
	\ex \label{ex:ayr5b}
		\gll kadāre ang=mya=int=yo tubay-ley nay bari-ley=kay. \\
			sodass \AgtT=\Oblig=kaufen=\Tsg.\N.\Top{} Wurst-\Parg.\Inan{} und
			Fleisch-\Parg.\Inan=wenig \\
	\end{xlist}
	\trans \wdef{Eines Tages schickte der Mann das Hündchen zum Metzger, auf
	dass es ein wenig Wurst und Fleisch kaufe.}
\end{exe}

Da der \xayr{AyonNF}{ayonang}{der Mann} als handelnde Instanz trotzdem das
syntaktische Subjekt darstellt, handelt es sich bei der vorliegenden
Satzkonstruktion um ein \q{Pseudopassiv}. Die Dativmarkierung bei
\xayr{brFytiymF}{baryatiyam}{zum Metzger} ist semantisch. Ob das Hündchen
\fw{dem Metzger} oder \fw{zum Metzger} geschickt wird, ist als Ambiguität in
diesem Kontext vernachlässigbar.%
%
	\footnote{Das belebte Nomen \xayr{brFyti}{baryati}{Metzger, Fleischer}
	wurde neu eingeführt. Es ist eine Tätigkeitsbildung zu
	\xayr{bri}{bari}{Fleisch}. Wollte man wirklich eindeutig sein, könnte man
	auch die Formulierung \xayr{mNsh brFytiy}{mangasaha baryatiya}{zum Metzger
	hin} (hin.zu Metzger-\Loc) verwenden.}

Dass es sich beim Verb \xayr{IMtFyo}{inco}{(es) kauft} im Nebensatz
\cref{ex:ayr5b} um eine zu erfüllende Pflicht oder eine Anweisung handelt,
markiert die Modalpartikel \rayr{mY}{mya}, die ich als Konjunktiv Präsens ins
Deutsche übertragen habe. Grammatikalisierte Modalpartikeln sind in der Regel
von Modalverben abgeleitet, hier von \xayr{mY/}{mya-}{sollen}, und stehen im
präverbalen Klitikcluster zwischen Topikpartikel und Verbstamm. Des Weiteren
beinhaltet der Satz ein quantifizierendes Klitikum \xayr{/kj}{kay}{ein wenig,
ein bisschen}, das als solches an eine \textsc{np} oder \textsc{vp} angehängt
wird.

Der nächste Hauptsatz in \cref{ex:ayr6a} enthält eine weitere Modalpartikel im
Klitikcluster, nämlich \xayr{d/}{da-}{so}, das hier eine präsentative Funktion
ausübt, ähnlich wie französisch \fw{voilà} \wdef{da (ist)}. Daher habe ich
\rayr{d/ktYisyo}{da-kacisayo} mit \wdef{da packte (es)} übersetzt. Ayeri
unterscheidet im Grunde nicht zwischen alienabler und inalienabler Possession,
daher erscheint \xayr{bMtri}{bantari}{mit dem Maul} hier mit dem Possessivum
\xayr{yon}{yona}{sein}. Mit \xayr{ksulej}{kasuley}{den Korb} ist nun auch ein
als solches markiertes Inanimatum in den Text eingeführt.%
%
	\footnote{Da Inanimata im Text seltener vorkommen, habe ich nur diese in der
	Glossierung explizit markiert. Fehlende Bezeichnung der Kategorie impliziert
	also Belebtheit.}

\begin{exe}
\ex \label{ex:ayr6}
	\begin{xlist}
	\ex \label{ex:ayr6a}
		\gll Ang=da=kacisa=yo banta-ri yona kasu-ley, \\
			\AgtT=so=packen=\Tsg.\N.\Top{} Maul-\Ins{} \Tsg.\N.\Gen{}
			Korb-\Parg.\Inan{} \\
	\ex \label{ex:ayr6b}
		\gll si-ya<a> yoma-aran pangis-reng=ma, panyan-reng naynay, \\
			\Rel<\Parg.\Inan>-\Loc{} sein-\Tsg.\Inan{} Geld-\Aarg.\Inan{}=genug
			Schriftstück-\Aarg.\Inan{} außerdem \\
	\ex \label{ex:ayr6c}
		\gll nay ang=sitang=pa=yo sasān-ya. \\
			und \AgtT=\Refl=nehmen=\Tsg.\N.\Top{} Weg-\Loc{} \\
	\end{xlist}
	\trans \wdef{Da packte es mit dem Maul den Korb, worin sich ausreichend
		 Geld sowie ein Schriftstück befanden, und machte sich auf den Weg.}
\end{exe}

Der Relativsatz in \cref{ex:ayr6b} wird mit \rayr{siyaa}{siyā} nun durch ein
\q{echtes} Relativpronomen (\textsc{d⁰}) eingeleitet, insofern die sekundäre
Markierung mit dem Lokativ \rayr{/y}{-ya} pronominalisierende Wirkung hat. Das
Relativpronomen hat im Relativsatz also die korrelative Bedeutung \wdef{worin}.
Dass es sich um eine Kurzform des Pronomens mit sekundärer Kasusmarkierung
handelt, wird durch den Langvokal im Kasussuf"|fix explizit gemacht. Die
Langform lautet \rayr{silejy}{sileyya}. In diesem Kontext fehlt
\rayr{/lej}{-ley} als Kongruenzsuf"|fix wie üblich, weil der
Relativsatz nicht rechtsversetzt ist. Obwohl \rayr{siyaa}{siyā} im Relativsatz
Konstituentenstatus hat, ist es nicht für die Topikalisierung verfügbar, daher
wird dieser Relativsatz wie ein intransitiver Satz behandelt. Das Verb
\xayr{yomaarnF}{yomāran}{(sie) befanden sich} weist darum keine Topikmarkierung
auf. Darüber hinaus enthält \cref{ex:ayr6c} einen idiomatischen Ausdruck,
\xayr{sitNF/p– ssaanFy}{sitang-pa- sasānya}{sich auf den Weg machen}. Die
Partikel \rayr{sitNF/}{sitang-} gibt dem Verb hier eine reflexive Bedeutung;
wörtlich \q{nimmt} man sich auf den Weg.

In \cref{ex:ayr7} erscheinen keine neuen grammatischen Merkmale oder
Konstruktionen, abgesehen von der partiellen Reduplikation bei der Verbform
\xayr{t/tvFy}{ta-tavya}{(er) bekommt wieder/zurück} in \cref{ex:ayr7b}.
Reduplikation der ersten zwei Silbensegmente eines Verbstamms drückt eine
iterative oder reversive Handlung aus, im Kontext des vorliegenden Satzes
die Letztere.

\begin{exe}
\ex \label{ex:ayr7}
	\begin{xlist}
	\ex \label{ex:ayr7a}
		\gll Nay ang=tav-yo veney\til{}veney tubay-ley nay
			bari-ley=kay ayon-yam, \\
			und \AgtT=bekommen-\Tsg.\N{} Hund-\Dim{}[\Top]
			Wurst-\Parg.\Inan{} und Fleisch-\Parg.\Inan{}=etwas Mann-\Dat{} \\
	\ex \label{ex:ayr7b}
		\gll nārya ang=ta\til{}tav-ya ayon ranya-ley veney\til{}veney-ena. \\
			aber \AgtT=bekommen-\Iter-\Tsg.\M{} Mann[\Top] nichts-\Parg.\Inan{}
				Hund-\Dim-\Gen{} \\
	\end{xlist}
	\trans \wdef{Und das Hündchen bekam etwas Wurst und Fleisch für den
	Mann, doch der Mann bekam nichts vom Hündchen zurück.}
\end{exe}

Der erste Satz des neuen Absatzes in \cref{ex:ayr8} wechselt die Topik zunächst
zum Wochentag, \xayr{miynF}{Miyan}{der Sechste}, als Kontext für die Aussage.
Betreffend des Adverbs \xayr{myis}{mayisa}{fertig} ist anzumerken, dass Ayeri
keine obligatorische Tempusmarkierung besitzt, vergleiche auch \cref{ex:ayr2}.
\rayr{myis}{mayisa} betont hier die Abgeschlossenheit der Handlung,
\xayr{tury}{turaya}{(er) schickte, sandte}, und damit deren Vorzeitigkeit im
Erzählkontext: \fw{(er) hatte geschickt}.%
%
	\footnote{Im Grunde also ähnlich wie zum Beispiel im afroamerikanischen
	Englisch, wo die Aspektpartikel \fw{dən} (<~engl.~\fw{done} \wdef{getan,
	erledigt, fertig}) typischerweise eine abgeschlossene Handlung kennzeichnet
	\autocite[60--63]{green2002}.}
%
Um die initiale VP nicht mit Adjunkten zu überladen, habe ich das Adverb
\xayr{d/kYuymF}{da-cuyam}{nämlich, und zwar} ans Ende gestellt.%
%
	\footnote{Das Adverb \xayr{d/kYuymF}{da-cuyam}{nämlich} wurde neu gebildet
	aus \xayr{d/}{da-}{so} und \xayr{kYuymF}{cuyam}{tatsächlich}.}

\begin{exe}
\ex \label{ex:ayr8}
	\gll Ya=tura-ya mayisa ang=Apican diyan veney\til{}veney-as baryati-yam
		Miyan da-cuyam. \\
		\LocT{}=schicken-\Tsg.\M{} \Pfv{} \Aarg=Apican ehrenwert
		Hund-\Dim-\Parg{} Metzger-\Dat{} Sechster[\Top] nämlich \\
	\trans \wdef{An einem Freitag nämlich hatte Apitschan dijan das Hündchen
		zum Metzger geschickt.}
\end{exe} 

In \cref{ex:ayr9} kehrt die Topik wieder zum Hündchen als Protagonisten zurück.
Im Text auf Hoan ist der Ursprung des Befehls der \fw{šar} \wdef{Vater, der
Greif; Papst}. Da ich bisher nichts zur Kultur der Ayeri ausgearbeitet habe,
habe ich diese Bezeichnung mit \xayr{ntFryonF vismF}{natrayon
visam}{Oberpriester, Hohepriester} adaptiert.

\begin{exe}
\ex \label{ex:ayr9}
	\gll Ang=anl-yo veney\til{}veney nosān-as natrayon-ena visam: \\
		\AgtT=bringen-\Tsg.\N{} Hund-\Dim{}[\Top] Befehl-\Parg{}
		Priester-\Gen{} oberster \\
	\trans \wdef{Das Hündchen brachte den Befehl des Hohepriesters: \dots}
\end{exe}

Ein weiterer Kontext neben dem intransitiven in \cref{ex:ayr6b}, in dem regulär
keine Topikmarkierung auftritt, ist in \cref{ex:ayr10} gegeben. Imperative wie
\xayr{kYunu}{cunu}{beginne, fang an!} sind speziell markiert, weisen also trotz
Referenz auf eine zweite Person keine Personenmarkierung auf. Das abhängige
Verb \xayr{gutsymF}{gutasayam}{(regelmäßig) zu fasten} ist mit dem
Partizipsuf"|fix \rayr{/ymF}{-yam} markiert, das seinen infiniten Status
anzeigt.%
%
	\footnote{Das Verb \xayr{gut/}{guta-}{verzichten} wurde um die Bedeutung
	\wdef{fasten} erweitert. Man hätte vielleicht einfacher, jedoch
	langweiliger formulieren können: \xayr{gutsu y miynF IknF mNsr dbsF}{Gutasu
	ya Miyan ikan mangasara dabas}{faste(t) ab heute (regelmäßig) den ganzen
	Freitag} (fasten-\Hab-\Imp{} \Loc=Sechster komplett weg.von heute). Die
	Verbform \rayr{gutsymF}{gutasayam} hat Aspektmarkierung, daher ist es
	möglicherweise nicht ganz korrekt, hier von einer infiniten Form zu sprechen
	-- oder von einem Partizip.}
%
Den Freitag mit \wdef{sechster Wochentag} zu übersetzen, habe ich aus der
Vorlage entnommen, siehe \cref{ex:hoan8}. Dass es sich beim Fastengebot um eine
regelmäßige Übung handelt, impliziert die Habitativmarkierung am Verb.

\begin{exe}
\ex \label{ex:ayr10}
	\gll Cun-u dabas guta-asa-yam ya=Miyan ikan. \\
		beginnen-\Imp{} heute fasten-\Hab-\Ptcp{} \Loc=Sechster komplett \\
	\trans \wdef{Fastet ab heute regelmäßig den ganzen Freitag.}
	% 
\end{exe}

Die Topik in \cref{ex:ayr11a} wandert zum Metzger, zu dem sich jetzt der
narrative Fokus verschiebt. Relativsätze in Ayeri müssen immer attributiv
gebunden sein, deshalb braucht der Satz in \cref{ex:ayr11b} ein semantisch mehr
oder weniger leeres Antezedens, das es mit \xayr{minen}{minena}{Angelegenheit,
Sache} erhält. Der Genitiv hat hier semantische Funktion und erzeugt die
Bedeutung \wdef{über die Angelegenheit}. Ein weiteres Beispiel für einen
solchen freien Genitiv ist \xayr{brin}{barina}{über das Fleisch}.
\xayr{myis}{mayisa}{fertig} tritt hier wie zuvor in \cref{ex:ayr8} in
perfektivierender Funktion auf. \xayr{maaj}{māy}{ja, doch} wird in
\cref{ex:ayr11c} als Modalpartikel mit verstärkender oder bekräftigender
Bedeutung eingesetzt.

\begin{exe}
\ex \label{ex:ayr11}
	\begin{xlist}
	\ex \label{ex:ayr11a}
		\gll Ang=silv-ya baryati kasu-ya taman-ley mine-na \\
			\AgtT=sehen-\Tsg.\M{} Metzger[\Top] Korb-\Loc{} Brief-\Parg.\Inan{}
			Angelegenheit-\Gen{} \\
	\ex \label{ex:ayr11b}
		\gll si ang=nosa-ya mayisa natrayon visam bari-na \\
			\Rel{} \AgtT=befehlen-\Tsg.\M{} \Pfv{} Priester[\Top] oberst
			Fleisch-\Gen{} \\
	\ex \label{ex:ayr11c}
		\gll nay ang=hayar=ya māy sitram-as veney\til{}veney-ena. \\
			und \AgtT=hauen=\Tsg.\Top{} \Ints{} Schwanz-\Parg{}
			Hund-\Dim-\Gen{} \\
	\end{xlist}
	\trans \wdef{Der Metzger erblickte im Korb den Brief darüber, was der
		Hohepriester bezüglich der Wurst befohlen hatte, und hieb dem Hündchen
		den Schwanz ab!}
\end{exe}

Der Satz in \cref{ex:ayr12} bietet im Grunde nichts Neues. Der semantische
Rahmen des Verbs \xayr{tpYFy}{tapyya}{(er) legte} beinhaltet die räumliche
Verschiebung des Themas zum Ziel durch eine Agens, daher erscheint das Ziel,
\xayr{ksuy}{kasuya}{in den Korb}, als einfache \textsc{np} im Lokativ, nicht als
\textsc{pp}.%
%
	\footnote{Expliziter könnte man auch hier formulieren: \xayr{mN koNF
	ksuy}{manga kong kasuya}{in den Korb hinein} (\Dir=in Korb-\Loc).}

\begin{exe}
\ex \label{ex:ayr12}
	\gll Ang=tapy=ya epang sitram-as kasu-ya veney\til{}veney-ena. \\
		\AgtT=legen=\Tsg.\M.\Top{} danach Schwanz-\Parg{} Korb-\Loc{}
		Hund-\Dim-\Gen{} \\
	\trans \wdef{Dann legte er den Schwanz in den Korb des Hündchens.}
\end{exe}

In der Inquitformel in \cref{ex:ayr13} hat \xayr{d/}{da-}{so} eine präsentative
Funktion, wie oben in \cref{ex:ayr6}. Die wörtliche Rede zeigt, dass nicht nur
Adjektive und Nomen, sondern auch Lokaladverbien prädikativ vorkommen können.
Das Adverb \xayr{Ady}{adaya}{da, dort} ist zur Betonung nach vorne gezogen.
Formeller wäre eine Konstruktion mit \xayr{yom/}{yoma-}{(da) sein, sich
befinden}, was aber im Erzählkontext nicht zur verbalen Grobheit des Sprechers
passen würde. Dass die Kasusmarkierung in Ayeri zwar eine semantische Grundlage
hat, die Sprache es aber damit nicht zu genau nimmt, wenn es der Syntax
dienlich ist, sieht man daran, dass \xayr{brireNF}{barireng}{das Fleisch} als
explizit unbelebte Instanz als Agens markiert ist, da es ein Subjekt darstellt.

\begin{exe}
\ex \label{ex:ayr13}
	\gll Da=nara=yāng: Adaya bari-reng vana! \\
		so=sprechen=\Tsg.\M.\Top{} dort Fleisch-\Aarg.\Inan{}
		\Second.\Gen{} \\
	\trans \wdef{Und er sprach: Da ist dein Fleisch!}
\end{exe}

Der erste Halbsatz in \cref{ex:ayr14a} enthält mit
\xayr{s/sryo}{sa-sarayo}{(es) geht zurück} wieder eine Iterativform mit
reversiver Bedeutung, wie zuvor in \cref{ex:ayr7b} gezeigt. Dass Ayeri zwischen
komitativem und instrumentalem \fw{mit} unterscheidet, wird hier durch die
\textsc{pp} \xayr{kjvo ksuy}{kayvo kasuya}{mit dem Korb} deutlich, die erstere
Funktion hat.

\begin{exe}
\ex \label{ex:ayr14}
	\begin{xlist}
		\ex \label{ex:ayr14a}
		\gll Ang=sa\til{}sara-yo veney\til{}veney nasi nupisa kirin-ya kayvo
			kasu-ya, \\
			\AgtT=gehen-\Iter-\Tsg.\N{} Hund-\Dim{}[\Top] treu verletzt
			Straße-\Loc{} mit Korb-\Loc{} \\
		\ex \label{ex:ayr14b}
		\gll nārya lesa=yong nay ang=pengal=yo tenyan-as. \\
			doch fallen=\Tsg.\N.\Aarg{} und \AgtT=treffen=\Tsg.\N.\Top{}
			Tod-\Parg{}	\\
	\end{xlist}
	\trans \wdef{Das verletzte treue Hündchen lief mit dem Korb
		zurück zur Straße, doch es stürzte und kam zu Tode.}%
		%
			\footnote{Das Adjektiv \xayr{nupis}{nupisa}{verletzt} wurde neu
			gebildet aus dem Verb \xayr{nup/}{nupa-}{verletzen, wehtun},
			vergleiche auch die dazugehörige deverbale Substantivableitung
			\xayr{nupaanF}{nupān}{Schaden, Mangel}.}
\end{exe}

Grammatisch ist \cref{ex:ayr16} unauf"|fällig. Lexikalisch interessant ist das
Numeral \xayr{smNF}{samang}{zehntausend}, denn Ayeri verwendet Potenzen von
Hundert, um höhere Zahlenstufen zu bilden: \rayr{smF}{sam} bedeutet
\wdef{zwei}, \rayr{smNF}{samang} ist das Quadrat von Hundert. Anzumerken ist
weiterhin, dass als Basis zwölf verwendet wird, \rayr{smNF}{samang} entspricht
also eigentlich 20\,736. \q{Zehntausend} Tränen zu vergießen ist ein
idiomatischer Ausdruck für große Trauer.%
%
	\footnote{Vgl.~\tit{Translation Challenge: The Sugar Fairies}, siehe unter
	\url{https://ayeri.de/examples} (\DTMdate{2024-06-21}).}

\begin{exe}
\ex \label{ex:ayr16}
	\gll Ang=da=ter-yon keynam ayron-ya simbey-ley samang. \\
		\AgtT=so=vergießen-\Tpl.\N{} Leute[\Top] Stadt-\Loc{}
		Träne-\Parg.\Inan{} zehntausend \\
	\trans \wdef{Da vergossen die Leute in der Stadt zehntausend Tränen.}
\end{exe}

Der Satz in \cref{ex:ayr17} enthält zu guter Letzt gleich mehrere syntaktische
Auf"|fälligkeiten. \q{Echte} Passive werden gebildet, indem die
Agens-\textsc{np} einfach wegfällt. Das Verb kongruiert dann stattdessen mit
der Patiens-\textsc{np} als syntaktischem Subjekt.
\xayr{tiysyo}{tiyasayo}{(sie) werden (gewöhnlich) gemacht} bezieht sich also
auf \xayr{gebisnFy}{gebisanye}{Bildnisse}. Da es sich um eine markierte
Konstruktion handelt, ist es meines Erachtens nicht angebracht, hier von
Ergativität zu sprechen, wenn das Subjekt in diesem Kontext auch wie ein
Absolutiv markiert sein mag.

\begin{exe}
\ex \label{ex:ayr17}
	\gll Sa=tiya-asa-yo masahatay adaya gebisan-ye yelang veney-ye-ri si
		sitram-ya kayvay. \\
		\PatT=machen-\Hab-\Tpl.\N{} seither dort Bildnis-\Pl{}[\Top] Stein
		Hund-\Pl-\Ins{} \Rel{} Schwanz-\Loc{} ohne \\
	\trans \wdef{Seither werden dort Statuen von Hunden ohne Schwanz
		angefertigt.}
\end{exe}

Eine weitere Auf"|fälligkeit ist das Kompositum \xayr{gebisnFy yelNF}{gebisanye
yelang}{Steinbildnisse, Statuen}, das nicht univerbiert ist. Die
Pluralmarkierung (und auch die overte Kasusmarkierung) tritt an den Kopf des
Kompositums, während das modifizierende Nomen wie ein Adjektiv unmarkiert
bleibt. Das nominale Komplement dazu, \xayr{venejyeri}{veneyyeri}{von Hunden}
wird von einem Relativsatz modifiziert, welcher der Anbindung der \textsc{pp}
\xayr{sitFrmFy kjvj}{sitramya kayvay}{ohne Schwanz} als komplexem Attribut von
\rayr{venejyeri}{veneyeri} dient. Im Prinzip enthält der Relativsatz selbst
einen Kopulasatz mit Subjektellipse. \xayr{kjvj}{kayvay}{ohne} ist eine der
Handvoll Postpositionen, die Ayeri besitzt.

%%%%%%%%%%%%%%%%%%%%%%%%%%%%%%%%%%%%%%%%%%%%%%%%%%%%%%%%%%%%%%%%%%%%%%%%%%%%%%%

\section{Beigegebenes Material}
\label{sec:suppl}

Hobbymäßig entwickelte Sprachen in der Regel nur der Person bekannt sind, die
sie erfunden hat, deshalb ist es normalerweise notwendig, der Übersetzung ein
Glossar und ein paar kurze Notizen zur Grammatik beizugeben, um der nächsten
Person im Kreis eine Grundlage zur Interpretation des Erhaltenen zu geben.
Neben der Grammatik und dem Wörterbuch auf der Webseite%
%
	\footnote{Siehe \url{https://ayeri.de} (\DTMdate{2024-06-21}).}
%
habe ich also folgende Informationen dem Text auf Ayeri in \cref{sec:ayruebers}
hinzugefügt (kleine Fehler habe ich korrigiert und ein paar Formulierungen
verbessert).

\subsection{Glossar}

% \setcounter{unbalance}{2}
\begin{multicols}{2}
\raggedright
\begin{description}[nosep]
	\item[adaya]
		\ayr{Ady}
		\emph{Adv.},
		da, dort
	\item[anl-]
		\ayr{AnFlF/}
		\emph{Vb.},
		bringen, liefern
	\item[Apican]
		\ayr{ApitFynF}
		\emph{N.~mask.},
		\textins{Personenname (\q{der Reine})}
	\item[ayon]
		\ayr{AyonF}
		\emph{N.~mask.},
		Mann, Mensch
	\item[ayron]
		\ayr{AjronF}
		\emph{N.~neut.},
		Stadt, Burg
	\item[bahis]
		\ayr{bhisF}
		\emph{N.~inan.},
		Tag
	\item[banta]
		\ayr{bMt}
		\emph{N.~neut.},
		Mund, Maul
	\item[bari]
		\ayr{bri}
		\emph{N.~inan.},
		Fleisch (als Lebensmittel)
	\item[baryati]
		\ayr{brFyti}
		\emph{N.~mask.},
		Metzger, Fleischer
	\item[cun-]
		\ayr{kYunF/}
		\emph{Vb.},
		anfangen, beginnen
	\item[dabas]
		\ayr{dbsF}
		\emph{Adv.},
		heute
	\item[da-cuyam]
		\ayr{d/kYuymF}
		\emph{Adv.}
		nämlich, und zwar
	\item[diyan]
		\ayr{diynF}
		\emph{Adj.},
		wertvoll, lieb; üppig (Wuchs); höf"|licher Anredetitel
	\item[epang]
		\ayr{EpNF}
		\emph{Adv.},
		danach, dann, als nächstes
	\item[garan]
		\ayr{grnF}
		\emph{N.~neut.},
		Name
	\item[gebisan]
		\ayr{gebisnF}
		\emph{N.~neut.},
		Bild, Abbild, Bildnis
	\item[gebisan yelang]
		\ayr{gebisnF yelNF}
		\emph{N.~neut.}
		Statue
	\item[guta-]
		\ayr{gut/}
		\emph{Vb.},
		verzichten; fasten
	\item[hayar-]
		\ayr{hyrF/}
		\emph{Vb.},
		(ab)hacken, (ab)hauen, fällen
	\item[-hen]
		\ayr{/henF}
		\emph{Qnt.},
		alle, jeder
	\item[ikan]
		\ayr{IknF}
		\emph{Adj.}
		ganz, komplett
	\item[int-]
		\ayr{IMtF/}
		\emph{Vb.},
		kaufen
	\item[kacisa-]
		\ayr{ktFyis/}
		\emph{Vb.},
		ergreifen, packen
	\item[kadāre]
		\ayr{kdaare}
		\emph{Konj.},
		sodass, damit, auf dass
	\item[kasu]
		\ayr{ksu}
		\emph{N.~inan.},
		Korb
	\item[-kay]
		\ayr{/kj}
		\emph{Qnt.},
		etwas, wenig(er), ein bisschen
	\item[kayvay]
		\ayr{kjvj}
		\emph{Adp.},
		ohne
	\item[kayvo]
		\ayr{kjvo}
		\emph{Adp.},
		neben, mit, an der Seite von; entlang
	\item[keynam]
		\ayr{kejnmF}
		\emph{N.~neut.},
		Leute, Menschen
	\item[kirin]
		\ayr{kirinF}
		\emph{N.~inan.},
		Straße, breiter Weg
	\item[kivo]
		\ayr{kivo}
		\emph{Adj.},
		klein; kurz (Zeit)
	\item[-ma]
		\ayr{/m}
		\emph{Qnt.},
		ausreichend, genug
	\item[masahatay]
		\ayr{mshtj}
		\emph{Adp., Adv.},
		seit; seitdem, seither
	\item[mayisa]
		\ayr{myis}
		\emph{Adj., Adv.},
		fertig, abgeschlossen (Adj.); generell perfektivierende Bedeutung (Adv.)
	\item[men]
		\ayr{menF}
		\emph{Num., Indef.},
		eins; ein
	\item[mine]
		\ayr{mine}
		\emph{N.~inan.},
		Angelegenheit, Sache, Frage, Argument
	\item[mitan-]
		\ayr{mitnF/}
		\emph{Vb.},
		leben (an einem Ort), wohnen
	\item[Miyan]
		\ayr{miynF}
		\emph{N.~inan.},
		sechster Tag der Woche, Freitag
	\item[māy]
		\ayr{maaj}
		\emph{Adv.},
		ja, doch; generell verstärkende Bedeutung
	\item[mya-]
		\ayr{mY/}
		\emph{Vb.} sollen; als Modalpartikel \rayr{mFy}{mya} mit obligativer
		oder instruktiver Bedeutung
	\item[nara-]
		\ayr{nr/}
		\emph{Vb.},
		sprechen, reden, sagen
	\item[nasi]
		\ayr{nsi}
		\emph{Adj.},
		treu, loyal, ergeben
	\item[natrayon]
		\ayr{ntFryonF}
		\emph{N.~mask.},
		Priester, Mönch
	\item[natrayon visam]
		\ayr{ntFryonF vismF}
		\emph{N.~mask.},
		Hohepriester; Papst
	\item[nay]
		\ayr{nj}
		\emph{Konj.},
		und
	\item[naynay]
		\ayr{njnj}
		\emph{Konj., Adv.},
		auch, ebenfalls, darüber hinaus, und so weiter
	\item[nosa-]
		\ayr{nos/}
		\emph{Vb.},
		anordnen, befehlen, gebieten
	\item[nosān]
		\ayr{nosaanF}
		\emph{N.~neut.},
		Anordnung, Befehl, Gebot
	\item[nupisa]
		\ayr{nupis}
		\emph{Adj.},
		verletzt, verwundet
	\item[nārya]
		\ayr{naarFy}
		\emph{Konj., Adv.},
		aber, außer, doch, obwohl, trotzdem
	\item[pa-]
		\ayr{p/}
		\emph{Vb.}
		nehmen
	\item[pangis]
		\ayr{pNisF}
		\emph{N.~inan.},
		Geld, Zahlungsmittel
	\item[panyan]
		\ayr{pnFynF}
		\emph{N.~inan.},
		Notiz, Zettel
	\item[Peram]
		\ayr{permF}
		\emph{N.~neut.},
		\textins{Ortsname}
	\item[ranya]
		\ayr{rnFy}
		\emph{Indef.},
		niemand, nichts
	\item[roda-]
		\ayr{rod/}
		\emph{Vb.},
		gehorchen
	\item[samang]
		\ayr{smNF}
		\emph{Num.},
		zehntausend ($10_{12}^{2^2}$)
	\item[sara-]
		\ayr{sr/}
		\emph{Vb.},
		gehen, weggehen; aufhören
	\item[sasān]
		\ayr{ssaanF}
		\emph{N.~inan.}
		Weg, Straße
	\item[silv-]
		\ayr{silFvF/}
		\emph{Vb.},
		sehen; ansehen (+~Pat.), zusehen (+~Dat.); einsehen, erkennen
	\item[simbey]
		\ayr{siMbej}
		\emph{N.~inan.},
		Träne
	\item[sitram]
		\ayr{sitFrmF}
		\emph{N.~inan.},
		Schwanz
	\item[tadayen]
		\ayr{tdyenF}
		\emph{Adv.},
		immer, jedes Mal
	\item[tahisa-]
		\ayr{this/}
		\emph{Vb.},
		besitzen
	\item[taman]
		\ayr{tmnF}
		\emph{N.~inan.},
		Brief
	\item[tapy-]
		\ayr{tpFyF/}
		\emph{Vb.},
		setzen, stellen, legen
	\item[tav-]
		\ayr{tvF/}
		\emph{Vb.},
		bekommen
	\item[ter-]
		\ayr{terF/}
		\emph{Vb.},
		verschütten, vergießen; streuen, verstreuen
	\item[tiya-]
		\ayr{tiy/}
		\emph{Vb.},
		schaffen, erschaffen, machen, herstellen
	\item[tubay]
		\ayr{tubj}
		\emph{N.~inan.},
		Wurst
	\item[tura-]
		\ayr{tur/}
		\emph{Vb.},
		senden, übersenden, übermitteln
	\item[veney]
		\ayr{venej}
		\emph{N.~neut.},
		Hund
	\item[visam]
		\ayr{vismF}
		\emph{Adj.},
		Haupt-\dots, Ober-\dots{}
	\item[yelang]
		\ayr{yelNF}
		\emph{N.~inan.},
		Stein (auch als Material)
	\item[yoma-]
		\ayr{yom/}
		\emph{Vb.},
		(da) sein, sich befinden, existieren
\end{description}
\end{multicols}

\subsection{Notizen zur Grammatik}
\label{subsec:gramnot}

\subsubsection{Allophonie}

Bei den Konsonantenphonemen löst /j/ nach /t k/ und /d ɡ/ allophonisch
Palatalisierung zu [t͡ʃ] und [d͡ʒ] aus, die in der Romanisierung mit ⟨c⟩ und
⟨j⟩ wiedergegeben werden.
% Außerdem resultiert [d͡ʒ] regelmäßig durch die
% Kontraktion der nominalen Pluralendung \rayr{/ye}{-ye}, wenn das
% darauf"|folgende Suf"|fix mit Vokal oder /j/ beginnt, beispielsweise
% \rayr{/ye}{-ye}~+~\rayr{/AsF}{-as}~>~\rayr{/ye\_asF}{-jas}.
Zwei adjazente Vokale der gleichen Qualität produzieren einen Langvokal, also
zum Beispiel /a/~+ /a/~>~/aː/ ⟨ā⟩, mit Ausnahme der verbalen Aspekt- und
Modussuf"|fixe, die einen vorangehenden Vokal typischerweise tilgen.
% ; /uː/ ⟨ū⟩ existiert aber nur in wenigen Lexemen, zum Beispiel
% \xayr{bbuu}{babū}{barbarisch}.

\subsubsection{Syntax}

Ayeri (\,\ayr{Ayeri}\,) verwendet Verberststellung (\textsc{vso}) als
unmarkierte Konstituentenfolge. Da die Sprache eine Variante des
\textsc{vo}-Typus darstellt, folgen Modifikatoren ihren Köpfen in der Regel.
Dies bedeutet, dass Adjektive, Possessiva und Relativsätze ihrem Nomen folgen;
genauso folgen Possessoren auch dem Possessum.

Darüber hinaus ist Ayeri im Grunde eine Akkusativsprache (\textsc{s~=~a~≠~o}).
\q{Echte} Passivsubjekte behalten allerdings ihre Patiensmarkierung, während
das Agensargument dann fehlt. In diesen Fällen von Ergativität zu sprechen,
würde die Beschreibung nur unnötig verkomplizieren. Obwohl Belebtheit sogar
eine Flexionskategorie in der Sprache darstellt, bleibt diese Unterscheidung
syntaktisch ungenutzt. Demotion der Agens zu einem obliquen Argument gibt es
aufgrund der semantischen Kasusmarkierung nicht. Es ist aber möglich, ein
\q{unechtes} Passiv zu bilden, bei welchem das Patiensargument logisch die
Topik bildet aber das Verb weiterhin mit dem Agensargument als syntaktischem
Subjekt kongruiert.
% Auch bei kausativen Sätzen bildet der Auslöser, als solcher gesondert
% markiert, logisch die Topik, wird aber ebenfalls nicht zum syntaktischen
% Subjekt. Die anderen Argumente des Verbs werden entsprechend auch in diesem
% Fall nicht herabgestuft.

% In ditransitiven Sätzen wird der Donor als Agens markiert
% (\textsc{s~=~a~=~d}), das Thema als Patiens (\textsc{o~=~t}). Der Rezipient
% (\textsc{r}) erhält Dativmarkierung. Prädikative \textsc{np}s werden
% abweichend als Patiens markiert, um Subjekt (Agens) und Prädikat
% (\q{Patiens}) zu unterscheiden, da Ayeri keine overte Kopula besitzt und
% doppelte Kernrollenmarkierung im gleichen Satz vermeidet.

Neben regulären Verbalsätzen gibt es auch Kopulasätze, allerdings besitzt Ayeri
eine Null-Kopula. Eine Besonderheit ist, dass das Prädikatsnomen in diesem Fall
als Patiens markiert wird, obwohl es mit dem Subjekt (mit Agensmarkierung)
gleichbedeutend ist. Das Prädikat kann zum Zweck der Betonung an die Spitze des
Satzes gestellt werden.

Ayeri macht keinen Unterschied zwischen restriktiven und nicht-restriktiven
Relativsätzen. Relativsätze brauchen allerdings immer ein Antezedens, freie
Relativsätze sind also nicht erlaubt. Relativsätze sind im Grunde eigenständige
Sätze, insofern die Relativpartikel \rayr{si}{si} die Funktion einer
Subjunktion hat, die ein komplexes Attribut an eine \textsc{np} bindet oder mit
deren Hilfe Attribute in ihrem Bezug desambiguiert werden können. Relativsätze
haben daher normalerweise einen internen Kopf. Wenn ein Relativsatz einen
Kopulasatz enthält, kann dessen Subjekt ausfallen.

Komplemente von \textsc{np}s werden zur Vermeidung von Ambiguität in der
Modifikationsrelation rechtsversetzt, wenn die \textsc{np} ein Adjunkt enthält,
welches das Kopfnomen modifiziert.

\subsubsection{Morphosyntax}
\label{subsubsec:morphsyn}

Die Topik wird durch ein Proklitikum am Verb markiert, das im Grunde der
Kasusendung der Topik-\textsc{np} entspricht, während die Topik-\textsc{np}
selbst nullmarkiert ist. Es handelt sich bei Ayeri also um eine sogenannte
\fw{trigger conlang}. Es bestehen nahezu keine Restriktionen für die Wahl der
Topik-\textsc{np}. Pronomen können in gleicher Weise topikalisiert werden.
Topikmarkierung ist obligatorisch in transitiven Sätzen, während intransitive
Sätze normalerweise keine Topik markieren. Auch imperative Verben tragen
normalerweise keine Topikmarkierung.

Die Relativpartikel \rayr{si}{si} zeigt optional Kasuskongruenz mit der
\textsc{np}, welche der Relativsatz modifiziert. Dies geschieht vor allem dann,
wenn der Relativsatz rechtsversetzt ist.

Neben den verschiedenen Pronomenarten ist die einzige Kongruenz zeigende
Wortart das Verb. Grundsätzlich kongruieren Verben mit dem Agensargument, es
sei denn, es fehlt durch echte Passivierung. Ersatzweise kongruiert das Verb
dann mit dem Patiensargument als syntaktischem Subjekt.

\subsubsection{Morphologie}

Ayeri ist eine agglutinierende Sprache und dabei sehr regelmäßig. Entsprechend
dem \textsc{vo}-Typus werden hauptsächlich Suf"|fixe zur Flexion benutzt.
Darüber hinaus besitzt die Sprache etliche Klitika, die sich insbesondere bei
finiten Verben in einem Klitikcluster vor dem Verb zeigen.

\subsubsubsection{Nomen}
\label{subsubsec:nom}

Ayeri hat ein zweistufiges Genussystem: Nomen können entweder belebt (\Anim)
oder unbelebt (\Inan) sein. Zu den belebten Nomen zählen zum Beispiel lebende
Personen und Tiere, Personifizierungen, Gefühle und mentale Prozesse sowie
Dinge, die Anzeichen von Leben zeigen (z.\,B.~Pflanzen) oder die eng mit
Menschen assoziiert sind (z.\,B.~Wohnungen). Menschen sowie Haus- und Nutztiere
können entsprechend ihrem sozialen respektive ihrem biologischen Geschlecht
maskulin (\M) oder feminin (\F) sein. Als belebt klassifizierte Dinge und
Abstrakta sind dagegen neutral (\N). Genus ist dem Lexikon inhärent und kovert,
darum gibt das Glossar es als Hilfsstellung explizit an. Es gibt keine
Markierung von Definit- und Indefinitheit, doch existiert ein optionales Präfix,
das Unspezifizität anzeigt (\xayr{me/}{mə-}{irgendein}), im Text aber nicht
vorkommt.

Nomen flektieren in der Regel nach Numerus und Kasus, können in bestimmten
Kontexten aber auch ohne overte Kasusflexion auftreten. Der Singular ist
unmarkiert, der Plural wird mit dem Suf"|fix \rayr{/ye}{-ye} gekennzeichnet.

Ayeri unterscheidet sieben Kasus: Agens (\Aarg), Patiens (\Parg), Dativ (\Dat),
Genitiv (\Gen), Lokativ (\Loc), Kausativ (\Caus) und Instrumentalis (\Ins),
siehe~\cref{tab:decl}. Die Vokale in Klammern in der Tabelle fallen weg, wenn
der Stamm auf einen Vokal endet, was also auch dann der Fall ist, wenn an die
Wurzel ein Pluralsuf"|fix angehängt ist.

\begin{table}
\caption{Kasusmarkierung der Nomen}
\begin{tabularx}{\linewidth}{l l l c c X}
\toprule
Kasus
	& \multicolumn{2}{c}{Suf"|fixform}
	& \multicolumn{2}{c}{proklitische Form}
	& Funktion
	\\

\cmidrule(lr){2-3}
\cmidrule(lr){4-5}

%
	& \multicolumn{1}{c}{\Anim}
	& \multicolumn{1}{c}{\Inan}
	& \multicolumn{1}{c}{\Anim}
	& \multicolumn{1}{c}{\Inan}
	\\

\midrule

\Aarg
	& -ang
	& -reng
	& ang
	& eng
	& prototypische Agens (Agens, Experiencer, Force); transitive und intransitive Subjekte im Aktiv; Subjekt des \q{unechten} Passivs; Subjekt in Kopulasätzen
	\\

\Parg
	& -as
	& -ley
	& sa
	& le
	& prototypische Patiens (Patiens, Thema); transitive und intransitive Objekte im Aktiv, direktes Objekt; Subjekt des \q{echten} Passivs; Prädikatsnomen in Kopulasätzen
	\\

\midrule

\Dat
	& \multicolumn{2}{c}{-yam}
	& \multicolumn{2}{c}{yam}
	& Rezipient; Ziel, Richtung; indirektes Objekt; sekundäres Prädikatsnomen
	\\

\Gen
	& \multicolumn{2}{c}{-(e)na}
	& \multicolumn{2}{c}{na}
	& Possessor, Quelle; worüber etwas geht bzw. wovon etwas handelt
	\\

\Loc
	& \multicolumn{2}{c}{-ya}
	& \multicolumn{2}{c}{ya}
	& Ort; typisch assoziiertes Ziel von Bewegungsverben
	\\

\Caus
	& \multicolumn{2}{c}{-isa}
	& \multicolumn{2}{c}{sā}
	& Verursacher (nur adverbiale Verwendung)
	\\

\Ins
	& \multicolumn{2}{c}{-(e)ri}
	& \multicolumn{2}{c}{ri}
	& Instrument, Helfer; Komplement einer \textsc{np}
	\\

\bottomrule
\end{tabularx}
\label{tab:decl}
\end{table}

Topikalisierte \textsc{np}s sind nullmarkiert, stattdessen wird der
entsprechende Kasus mit der in \cref{tab:decl} angegebenen klitischen Form
links vom Verb markiert. Eigennamen verwenden ebenfalls die klitische Form bei
der Kasusmarkierung, zum Beispiel \xayr{n bliinF}{na Balīn}{von Berlin}.

Der Diminutiv von Nomen wird durch vollständige Reduplikation angezeigt. Bei
Komposita wird nur das Kopfnomen redupliziert und flektiert. Komposita sind in
der Regel univerbiert, sodass grammatische Endungen an das letzte Element
angehängt werden. Daneben gibt es losere Verbindungen von Nomen, bei denen
ebenfalls nur das Kopfnomen flektiert wird und das modifizierende Nomen als
Attribut folgt.

\subsubsubsection{Pronomen}

Ayeri besitzt durch die Menge an \st{Kazoos} Kasus und Genera eine Fülle von
(ziemlich regelmäßig gebildeten) Personalpronomen, wobei für den Kontext des
vorliegenden Textes nur ein Teil derjenigen in \cref{tab:persproagr} relevant
ist, die ihrerseits nur einen Ausschnitt darstellt. Für dritte Personen werden
auch häufig Demonstrativpronomen verwendet, allerdings kommt dieser Fall im
Text nicht vor. Indefinitpronomen sind im Glossar aufgeführt, sofern sie im
Text vorkommen.

\begin{table}
\caption{Personalpronomen und Personenendungen der Verben}
\begin{tabularx}{\linewidth}{l l C C C C C C c c C C}
\toprule
%
	& %
	& \multicolumn{2}{c}{\makecell[tc]{Kongruenz-/\\Topikform}}
	& \multicolumn{2}{c}{\Aarg}
	& \multicolumn{2}{c}{\Parg}
	& \multicolumn{2}{c}{\Dat}
	& \multicolumn{2}{c}{\Gen}
	\\

\cmidrule(lr){3-4}
\cmidrule(lr){5-6}
\cmidrule(lr){7-8}
\cmidrule(lr){9-10}
\cmidrule(lr){11-12}

%
	& %
	& \multicolumn{1}{c}{\Sg}
	& \multicolumn{1}{c}{\Pl}
	& \multicolumn{1}{c}{\Sg}
	& \multicolumn{1}{c}{\Pl}
	& \multicolumn{1}{c}{\Sg}
	& \multicolumn{1}{c}{\Pl}
	& \multicolumn{1}{c}{\Sg}
	& \multicolumn{1}{c}{\Pl}
	& \multicolumn{1}{c}{\Sg}
	& \multicolumn{1}{c}{\Pl}
	\\

\midrule

\First
	& %
	& ay
	& ayn
	& yang
	& nang
	& yas
	& nas
	& yām
	& nyam
	& nā
	& nana
	\\

\Second
	& %
	& va
	& va
	& vāng
	& vāng
	& vās
	& vās
	& vayam
	& vayam
	& vana
	& vana
	\\

\Third
	& \M
	& ya
	& yan
	& yāng
	& tang
	& yās
	& tas
	& yayam
	& cam
	& yana
	& tan
	\\

%
	& \F
	& ye
	& yen
	& yeng
	& teng
	& yes
	& tes
	& yeyam
	& teyam
	& yena
	& ten
	\\

%
	& \N
	& yo
	& yon
	& yong
	& tong
	& yos
	& tos
	& yoyam
	& toyam
	& yona
	& ton
	\\

%
	& \Inan
	& ara
	& aran
	& reng
	& teng
	& rey
	& tey
	& rayam
	& racam
	& ran
	& ten
	\\

\bottomrule
\end{tabularx}
\label{tab:persproagr}
\end{table}

% Demonstrativpronomen werden mit \rayr{d/}{da-} (indefinit), \rayr{Ed/}{eda-}
% (proximal) und \rayr{Ad/}{ada-} (distal) gebildet. Gerade beim belebten
% Agens- und Patiens-Demonstrativum tritt daran das Element \rayr{/nY}{-nya}
% (z.\,B.~\xayr{AdnYaaNF}{adanyāng}{jener, der da};
% vgl.~\xayr{nYaanF}{nyān}{Person}), in jedem Fall folgt am Schluss die
% Kasusendung, die dieselbe wie bei der Deklination der Nomina ist
% (\cref{tab:decl}).

In \cref{subsubsec:morphsyn} wurde erklärt, dass Relativpartikeln keine
Pronomen im engen Sinn darstellen, allerdings können sie durch sekundäre
Kasusmarkierung pronominalisiert werden. Das Relativ\-pronomen trägt dann eine
zweite Kasusendung, die seine grammatische Funktion als Konstituente innerhalb
des Relativsatzes markiert. Wenn die Relativpartikel keine primäre
Kasuskongruenz aufweist (z.\,B. \rayr{sin}{sina} mit Bezug auf eine
Genitiv-\textsc{np}) und so die sekundäre Endung an das einfache \rayr{si}{si}
tritt, wird der Vokal der sekundären Endung zur Desambiguierung gedehnt, zum
Beispiel \xayr{sinaa}{sinā}{von welchem}. Sekundär markierte Relativa können
jedoch innerhalb des Relativsatzes nicht selbst als Topiken fungieren, insofern
sie ihre Kasusmarkierung nicht ans Verb abgeben können.

\subsubsubsection{Verben}

Verben kongruieren nach Person (\First, \Second, \Third) und Numerus (\Sg,
\Pl) ihres Subjekts, siehe \cref{tab:persproagr}. Bei dritten Personen kommen
noch Genus und Belebtheit (\M, \F, \N, \Inan) als Flexionskategorien hinzu. Bei
pronominalen Subjekten ersetzt das Personalpronomen das Kongruenzsuf"|fix am
Verb, indem es als Enklitikum ans Ende des Verbstamms tritt. Die
Personenendungen der regulären Kongruenz mit dem Subjekt und die
topikalisierten pronominalen Klitika sind homophon, zum Beispiel korrespondiert
die Vollform \xayr{/yaaNF}{-yāng}{er} mit der topikalisierten Form
\rayr{ANF—/y}{ang \dots\ -ya}. \rayr{/y}{-ya} ist gleichzeitig auch die
Kongruenzendung für den Bezug auf eine Subjekt-\textsc{np} im Singular
Maskulinum.

Finite Verben weisen darüber hinaus optional Flexion für Tempus auf, ansonsten
für Aspekt und Modus. Dafür werden verschiedene Markierungsstrategien
verwendet. Im Rahmen des Texts sind habitualer und iterativer Aspekt sowie der
Imperativ als Modus relevant. Der Imperativ der zweiten Person wird mit der
Quasi-Personenendung \rayr{/U}{-u} markiert, die einen vorhergehenden Vokal
tilgt. Habitualer Aspekt wird mit der Endung \rayr{/As}{-asa} markiert, die an
den Verbstamm tritt und ebenfalls einen vorhergehenden Vokal tilgt. Aspekt kann
darüber hinaus durch Adverbien ausgedrückt werden, zum Beispiel
\xayr{myis}{mayisa}{fertig sein}, welches die Abgeschlossenheit einer Handlung
betont.

Iterativer Aspekt drückt aus, dass eine Handlung mehrfach geschieht, kann aber
auch reversive Bedeutung haben, zum Beispiel
\xayr{t/tpYnNF}{ta-tapyanang}{wir legen immer wieder} oder \wdef{wir legen
wieder zurück}. Wie das Beispiel zeigt, wird iterativer Aspekt durch
Reduplikation der ersten beiden Silbensegmente des Verbstamms angezeigt.

Modalität wird in der Regel durch Modalpartikeln ausgedrückt, die im
präverbalen Klitikcluster nach dem Topikmarker stehen. Diese haben
typischerweise die Form von unflektierten Verbstämmen, zum Beispiel
korrespondiert \xayr{miNF/}{ming-}{können} mit der Partikel \rayr{miNF}{ming}
und \xayr{mY/}{mya-}{sollen} mit der Partikel \rayr{mY}{mya}.

Bei \xayr{d/}{da-}{so} handelt es sich um eine Partikel, die zum einen
pronominal verwendet werden kann, zum Beispiel \xayr{d/kilyNF}{da-kilayang}{ich
darf das} oder \xayr{d/IMtYyeNF}{da-incyeng}{sie kauft eins}. Zum anderen kann
sie auch präsentative Funktion haben, beispielsweise in
\xayr{d/shyaaNF}{da-sahayāng}{da kommt er}.

Eine weitere Partikel stellt \rayr{sitNF}{sitang-} dar, das anstelle eines
vollständigen Reflexivpronomens auftreten kann.
\xayr{sitNF/ketFtNF}{sitang-kettang}{sie waschen sich} ist also äquivalent zu
\rayr{ANF ketFynF sitNF/tsF}{ang kecan sitang-tas}.

Wenn ein Verb ein verbales Komplement besitzt, zum Beispiel bei Kontroll- und
Raisingverben, weist das abhängige Verb eine im Prinzip infinite Form auf, die
mit \rayr{/ymF}{-yam} gekennzeichnet und als \q{Partizip} bezeichnet wird. Mit
\rayr{/AnF}{-an} nominalisiert kann diese Form als Gerundium verwendet werden.
Infinite Verben dieser Art können trotzdem Modus- und Aspektmarkierung
aufweisen.

\subsubsubsection{Adjektive, Adverbien \& Co.}

Adjektive weisen keine Kongruenz auf, können aber negiert und gesteigert
werden, genauso wie auch Adverbien. Sie stehen immer direkt hinter ihrem Bezug.

Neben Adjektiven im engeren Sinn besitzt Ayeri eine Reihe von Quantoren, die in
der Regel am Ende der \textsc{np} (determinierende Quantoren), \textsc{vp} oder
\textsc{ap} (adverbiale Quantoren) hängen. Der Text enthält mehrere solcher
Partikeln, zum Beispiel \xayr{/kj}{-kay}{wenig, etwas, ein bisschen}.

\subsubsubsection{Präpositionen}

Freie Dative und Genitive kennzeichnen eine Bewegung zu etwas hin
beziehungsweise von etwas her (vgl.~\cref{subsubsec:nom}). Freie Lokative
kennzeichnen eine Position, vor allem eine, die prototypisch mit dem Verb im
Satz assoziiert wird. Dies kommt insbesondere bei Positions- und
Bewegungsverben zum Tragen.

Ayeri verwendet darüber hinaus in der Regel Präpositionen, die größtenteils von
Nomen abgeleitet sind. Daneben gibt es eine Reihe von Postpositionen, von denen
die meisten jüngere, sekundäre Bildungen etwa aus Adverbialen darstellen. Das
Präpositionalobjekt steht in der Regel im Lokativ. Steht es im Dativ,
kennzeichnet dieser bei manchen Präpositionen eine Bewegung in Richtung des
Objekts statt eines Ruhens an dem Ort, welchen das Objekt bezeichnet.

%% BIBLIOGRAPHY %%%%%%%%%%%%%%%%%%%%%%%%%%%%%%%%%%%%%%%%%%%%%%%%%%%%%%%%%%%%%%%

% \vfill
% \pagebreak

\begingroup\multicolsep=0pt
\printglossary[
	style=threecolumn,
	type=leipzig,
	title={Abkürzungen der Glossierung},
]
\endgroup

% % \nocite{*} % returns all entries from the bibliography database
\printbibliography[heading=bibintoc]

\end{document}
