\documentclass[
	12pt,
	ngerman,
]{scrartcl}

% Author, Title, Subtitle etc.
\author{Carsten Becker}
\title{Ein fruchtloses Bemühen}
\subtitle{Beitrag zum 2.~Relay des Berliner Conlang-Stammtischs}
\date{\DTMdate{2024-12-00}} % Format: YYYY/MM/DD (rev. N, YYYY/MM/DD)

% Provide running author and date
\makeatletter
\let\runauthor\@author
\let\rundate\@date
\makeatother

% Preferences of float placement
% cf. https://tex.stackexchange.com/a/167864/60686
\makeatletter
\renewcommand\fps@figure{htbp}
\renewcommand\fps@table{htbp}
\def \@floatboxreset {%
	\reset@font
	\normalsize
	\@setminipage
	\centering
}
\makeatother


% Handle language and quotation marks
\usepackage{babel}
\usepackage{csquotes} % Put quotations in \enquote{}!
\SetBlockEnvironment{quotation}
\renewcommand*{\mkccitation}[1]{ (#1)}
\let\q\textquote

% Quotation style for word definitions
\DeclareQuoteStyle{wdef}
	{\textquoteleft}{\textquoteright}
	{\textquotedblleft}{\textquotedblright}
\newcommand{\wdef}[1]{{\setquotestyle{wdef}\enquote{#1}}}

% Set all margins to 2.54 cm
\usepackage[margin=1in]{geometry}
\widowpenalty10000 % Avoid widows like the plague!
\clubpenalty10000 % Avoid orphans like the plage, too!

% Extended formatting of lists
\usepackage{enumitem}
\newlist{glossdefs}{itemize}{1}
\setlist[glossdefs]{nosep, leftmargin=3em, labelwidth=2.5em, align=left}
\setlist[itemize]{noitemsep}

% Make multiple columns available in single-column document
\usepackage{multicol}

% Make text colors and color names available
\usepackage{xcolor}

% Load font stuff for XeTeX
\usepackage{fontspec}

% Set main fonts
\usepackage{microtype}

\newfontfamily{\Tagati}[
	Renderer=Graphite,
	Scale=1.0,
	BoldFont={* Italic},
% 	HyphenChar=·,
]{Tagati Book G}

% \setmainfont[
% 	Ligatures=TeX,
% 	Numbers={OldStyle,Proportional},
% 	BoldFont=*-Bold,
% 	ItalicFont=*-Italic,
% 	BoldItalicFont=*-BoldItalic,
% ]{Junicode}

\setmainfont[
	Ligatures=TeX,
	Numbers={OldStyle,Proportional},
]{Junicode VF}

\setsansfont[
	Ligatures=TeX,
	Numbers=Lowercase,
	Scale=MatchUppercase,
	ItalicFont={Noto Sans Condensed Light Italic},
	BoldFont={Noto Sans Condensed Bold},
	BoldItalicFont={Noto Sans Condensed Bold Italic},
]{Noto Sans Condensed Light}

% Load BibLaTeX (using Biber), configure citation styles
\usepackage[
	authordate-trad,
	backend=biber,
	alldates=terse,
	safeinputenc,
]{biblatex-chicago}

% To make \textcite look like "Doe (2014: 213)"
\renewcommand*{\postnotedelim}{\addcolon\addspace}
\DeclareFieldFormat{postnote}{#1}
\DeclareFieldFormat{multipostnote}{#1}
\addbibresource{bibliography.bib}

% Date etc.
\usepackage[
	useregional=numeric,
]{datetime2}

% % Verse
% \usepackage{verse}

% Ability to include graphics and dealing with footnotes in descriptions
\usepackage{graphicx}
\usepackage[font={small,sf},labelfont={small,sf},format=plain]{caption}
\usepackage{subcaption}
\usepackage{wrapfig}
\setlength{\columnsep}{2\baselineskip}

% General headers and footers
\usepackage{fancyhdr}
\pagestyle{fancy}

\fancyhead[L]{} % empty
\fancyhead[C]{} % empty
\fancyhead[R]{\thepage}

\fancyfoot[L]{} % empty
\fancyfoot[C]{} % empty
\fancyfoot[R]{} % empty

\renewcommand{\headrulewidth}{0pt}
\renewcommand{\footrulewidth}{0pt}

% First page headers and footers are different
\fancypagestyle{firstpage}{
	\fancyhead[L]{\sffamily \footnotesize \textbf{Benung. The Ayeri Language Resource}}
	\fancyhead[C]{} % empty
	\fancyhead[R]{\sffamily \footnotesize \runauthor{} · \rundate{}}
	
	\fancyfoot[L]{} % empty
	\fancyfoot[C]{} % empty
	\fancyfoot[R]{\sffamily \footnotesize 
	\href{https://ayeri.de}{https://ayeri.de} · 
	\href{https://github.com/carbeck/benung-pdfs}{https://github.com/carbeck/benung-pdfs} · 
	\href{https://creativecommons.org/licenses/by-sa/4.0/}{CC~BY-SA~4.0}
	}
	
	\renewcommand{\headrulewidth}{0.5pt}
	\setlength\footskip{0.5in}
}

\usepackage{ifthen}
\ifthenelse{\value{page}=1}{\thispagestyle{firstpage}{\pagestyle{fancy}}}

% Line spacing
\usepackage{setspace}
\onehalfspacing

% Avoid pagebreaks right after sections and subsections
\usepackage{needspace}
\usepackage{etoolbox}
\preto\section{\needspace{6\baselineskip}}
\preto\subsection{\needspace{6\baselineskip}}

% Things for tables
\usepackage{tabularx}
\usepackage{booktabs}
\usepackage{makecell}
% \usepackage{rotating}
\newcolumntype{C}{>{\centering\arraybackslash}X}

% Formatting of table of glossing abbreviations from Leipzig package manual
\usepackage[
	acronym,
	nomain,
	nonumberlist,
	nopostdot,
	numberedsection=nameref,
	toc=true,
	xindy={codepage=utf8}, % language=english,
]{glossaries}
\usepackage{glossary-inline}

\newglossarystyle{threecolumn}{%
	\renewenvironment{theglossary}{%
		\begin{multicols}{3}
		\begin{glossdefs}%
	}{%
		\end{glossdefs}%
		\end{multicols}%
	}%
	\renewcommand*{\glossaryheader}{}%
	\renewcommand*{\glsgroupheading}[1]{}%
	\newcommand*{\glossaryentryfield}[5]{%
		\item[\glsentryitem{##1}\glstarget{##1}{##2}]
		% \makefirstuc{##3}\glspostdescription{}
		##3\glspostdescription{}
	}%
	\renewcommand*{\glsgroupskip}{}%
}%

% Formatting of glosses
\usepackage{langsci-gb4e}
\usepackage[glosses]{leipzig}
\renewcommand{\eachwordone}{\itshape}
\renewcommand{\eachwordtwo}{\rule[-.5\baselineskip]{0pt}{0pt}}
% leipzig.def
% Glossing abbreviations

% German redefinitions
\renewleipzig{Aarg}{a}{Agens}
\renewleipzig{Abs}{abs}{Absolutiv}
\renewleipzig{Acc}{acc}{Akkusativ}
\renewleipzig{All}{all}{Allativ}
\renewleipzig{Caus}{caus}{Kausativ}
\renewleipzig{Dat}{dat}{Dativ}
\renewleipzig{Dem}{dem}{Demonstrativ}
\renewleipzig{First}{1}{erste Person}
\renewleipzig{F}{f}{Femininum}
\renewleipzig{Gen}{gen}{Genitiv}
\renewleipzig{Imp}{imp}{Imperativ}
\renewleipzig{Ins}{ins}{Instrumentalis}
\renewleipzig{Loc}{loc}{Lokativ}
\renewleipzig{M}{m}{Maskulinum}
\renewleipzig{Neg}{neg}{Negativ}
\renewleipzig{Nmlz}{nmlz}{Nominalisierer}
\renewleipzig{Nom}{nom}{Nominativ}
\renewleipzig{N}{n}{Neutrum}
\renewleipzig{Parg}{p}{Patiens}
\renewleipzig{Pl}{pl}{Plural}
\renewleipzig{Poss}{poss}{Possessiv}
\renewleipzig{Pst}{pst}{Präteritum}
\renewleipzig{Ptcp}{ptcp}{Partizip}
\renewleipzig{Refl}{refl}{Reflexiv}
\renewleipzig{Rel}{rel}{Relativ}
\renewleipzig{Second}{2}{zweite Person}
\renewleipzig{Sg}{sg}{Singular}
\renewleipzig{Third}{3}{dritte Person}
\renewleipzig{Top}{top}{Topik}

% Other common abbreviations, undefined by default
\newleipzig{Dim}{dim}{Diminutiv}

% ɮɛ̃̂.kɔ̌ʔ
\newleipzig{Acv}{acv}{Aktiv}
\newleipzig{Clf}{clf}{Klassifizierer}
\newleipzig{Elv}{elv}{Elativ}
\newleipzig{Inch}{inch}{Inchoativ}
\newleipzig{Ipfv}{ipfv}{Imperfektiv}
\newleipzig{Nmz}{nmz}{Substantivierung}
\newleipzig{Npst}{npst}{Nicht-Präteritum}
\newleipzig{Pfv}{pfv}{Perfektiv}
\newleipzig{Stv}{stv}{Stativ}

% Ayeri
\newleipzig{AgtT}{at}{Agenstopik}
\newleipzig{Anim}{anim}{belebt}
\newleipzig{Ayr}{ayr}{Ayeri}
\newleipzig{CauT}{caut}{Kausativtopik}
\newleipzig{Conj}{conj}{konjunktiv}
\newleipzig{DatT}{datt}{Dativtopik}
\newleipzig{Dir}{dir}{direktiv}
\newleipzig{GenT}{gent}{Genitivtopik}
\newleipzig{Hab}{hab}{Habitativ}
\newleipzig{Hort}{hort}{Hortativ}
\newleipzig{Inan}{inan}{unbelebt}
\newleipzig{InsT}{inst}{Instrumentaltopik}
\newleipzig{Ints}{ints}{intensiv}
\newleipzig{Iter}{iter}{iterativ}
\newleipzig{LocT}{loct}{Lokativtopik}
\newleipzig{Oblig}{oblig}{obligativ}
\newleipzig{PatT}{pt}{Patienstopik}
\newleipzig{Sup}{sup}{Superlativ}


% Trees
\usepackage[linguistics]{forest}

% Underline, strikeout
\usepackage{soul}

% Nicer footnotes
\usepackage[bottom,hang,norule]{footmisc}
\setlength{\footnotesep}{0.75\baselineskip}

% Smaller font in block quotes
\usepackage{relsize}
\AtBeginEnvironment{quote}{\noindent\smaller}
\AtBeginEnvironment{quotation}{\smaller}

% Clickable links in footnotes, TOC, etc.
\usepackage[
% 	xetex,
	bookmarks=true,
	colorlinks=false,
	linktoc=section,
	hidelinks,
	pdfusetitle,
]{hyperref}

% We want URLs to be italic and with regular uppercase numerals
\renewcommand{\UrlFont}{%
	\normalfont%
	\itshape%
	\addfontfeature{RawFeature=-onum}%
}

% In-text references
% cf. https://tex.stackexchange.com/a/139051
% Since German plural formation is not as regular as in English (-e, -en for
% Beispiel), we will define the label as empty
\usepackage[sort&compress,noabbrev]{cleveref}
\newcommand{\crefrangeconjunction}{--}
\crefname{xnumi}{}{}
\creflabelformat{xnumi}{(#2#1#3)}
\crefname{xnumii}{}{}
\creflabelformat{xnumii}{(#2#1#3)}
\crefname{xnumiii}{}{}
\creflabelformat{xnumiii}{(#2#1#3)}
\crefname{xnumiv}{}{}
\creflabelformat{xnumiv}{(#2#1#3)}
\crefrangeformat{xnumi}{(#3#1#4)--(#5#2#6)}
\crefrangeformat{xnumii}{(#3#1#4--#5\crefstripprefix{#1}{#2}#6)}
\crefrangemultiformat{xnumii}{(#3\arabic{xnumi}#1#4--#5#2#6)}
{ and~(#3\arabic{xnumi}#1#4--#5#2#6)}{, (#3\arabic{xnumi}#1#4--#5#2#6)}
{ and~(#3\arabic{xnumi}#1#4--#5#2#6)}

% Subsubsubsection
% cf. https://tex.stackexchange.com/a/356574
\DeclareNewSectionCommand[
	style=section,
	counterwithin=subsubsection,
	afterskip=1.5ex plus .2ex,
	beforeskip=3.25ex plus 1ex minus .2ex,
	afterindent=false,
	level=\paragraphnumdepth,
	tocindent=10em,
	tocnumwidth=5em
]{subsubsubsection}
\setcounter{secnumdepth}{\subsubsubsectionnumdepth}
\setcounter{tocdepth}{\subparagraphtocdepth}
\addto\extrasngerman{%
	\let\subsubsubsectionautorefname\subsubsectionautorefname
}
\crefalias{subsubsubsection}{subsubsection}

% Make glossaries
\makeglossaries

% Macros
\newcommand{\fw}[1]{\textit{#1}} % Foreign Word
\newcommand{\til}{\char`\~} % Literal tilde in text mode % {$\sim$}
\newcommand{\tit}[1]{\textit{#1}} % Title of a work
\newcommand{\tsub}[1]{\textsubscript{#1}} % Subscript
\newcommand{\tsup}[1]{\textsuperscript{#1}} % Superscript
\newcommand{\markyellow}[1]{\colorbox{yellow}{#1}} % Yellow highlighter
\newcommand{\ques}{\textsuperscript{?}} % raised question mark
\newcommand{\zwsp}{\mbox{​}} % Zero-width space (ZWSP)

\newcommand{\ayr}[1]{\zwsp\smash{{\Tagati #1}}} % Plain Ayeri orthography
\newcommand{\rayr}[2]{\zwsp\smash{{\Tagati #1}} \emph{#2}} % Ayeri orthography + *r*omanization
\newcommand{\tayr}[2]{#1 `#2'} % Romanization + *t*ranslation
\newcommand{\xayr}[3]{\zwsp\smash{\Tagati #1} \emph{#2} `#3'} % Ayeri orthography + romanization + translation

\newenvironment{mytitle}{
	\hfill
	\begin{minipage}{0.667\textwidth}
	\vspace{\baselineskip}
	\begin{center}
		\Large
		\sffamily\bfseries
		\makeatletter
}{
		\makeatother
	\end{center}
	\vspace{1em}
	\end{minipage}
	\hfill
}

% Change abstract font
\AtBeginEnvironment{abstract}{\small\sffamily}

% Source language name
\newcommand{\Lhengko}{ɮɛ̃̂.kɔ̌ʔ}

%% END OF PREAMBLE %%%%%%%%%%%%%%%%%%%%%%%%%%%%%%%%%%%%%%%%%%%%%%%%%%%%%%%%%%%%

\begin{document}

%% MAIN PART %%%%%%%%%%%%%%%%%%%%%%%%%%%%%%%%%%%%%%%%%%%%%%%%%%%%%%%%%%%%%%%%%%

\begin{mytitle}
	\@title: \@subtitle
\end{mytitle}

\begin{abstract}
Another local relay was held in Berlin during November and a little into early
December 2024. This time, I had the pleasure to translate from Henrik's
\emph{\Lhengko} as the last person in a circle of seven participants. Thus, I
returned my torch back to Bruno, who had started the game, and who translated
my \emph{Ayeri} text into his \emph{Kèramkaq} to conclude. Again, the game was
run in German due to the limited, local scope.
\end{abstract}

Nachdem das Relay im Mai Anklang fand, hatten wir Lust auf eine weitere Runde
und konnten zwei weitere Teilnehmende dazugewinnen. Insgesamt haben also sieben
Personen an diesem Relay teilgenommen, wobei ich mit \emph{Ayeri} der letzte im
Kreis war, bevor der Staffelstab an die erste Person zurückging, um das Spiel
zu beenden. Mein Vorgänger war Henrik mit \emph{\Lhengko}, mein Nachfolger
Bruno mit \emph{Kèramkaq}. Natürlich wurde auch diesmal nach den bewährten
Spielregeln gespielt.%
%
	\footnote{Siehe
	\url{https://conlang.org/language-creation-conference/lcc6/lcc6-relay/}
	(\DTMdate{2024-11-30}).}
%
Die Teilnehmenden hatten dieses Mal aufgrund der größeren Anzahl vier Tage Zeit
für ihre Etappe, daher ist der zu übersetzende Text im Vergleich zum letzten
Mal kürzer ausgefallen. Das Spiel wurde wieder auf Deutsch durchgeführt.

%%%%%%%%%%%%%%%%%%%%%%%%%%%%%%%%%%%%%%%%%%%%%%%%%%%%%%%%%%%%%%%%%%%%%%%%%%%%%%%

\section{Analyse der Vorlage auf \Lhengko}
\label{sec:lhenganalys}

Den nachfolgenden Text habe ich von Henrik neben einer Wortliste und ein paar
Notizen zur Grammatik analog zum hier gezeigten Material erhalten, wie üblich.
Henrik scheint eine Vorliebe für komplexe Silben und \textsc{ipa} als
Transkriptionssystem anstelle der lateinischen Schrift zu haben, was bereits
beim letzten Spiel mit \fw{ʀu.lu} deutlich wurde, aus dem Kierán übersetzt hat.%
%
	\footnote{Vermutlich unvergesslich für alle, die beim stammtischgemäßen
	Abschlusstreffen dabei waren, ist die aufgrund der überbordenden
	Kongruenzmorphologie und häufig mehrsilbigen grammatischen Morphemen
	siebzehn\-minütige Lesung des Texts von Kierán, weil Henrik an diesem Abend
	verhindert war.}
%

\begin{quote}
\begin{sloppypar}
	ɮʷǐ˞.ʐʷɔ̃̂ kxʉ̂.ʐˤɒ̌~|
	ⁿfˠâ ɮʷǐ˞.ʐʷɔ̃̂ kxʉ̂.ʐˤɒ̌~|
	ⁿzʲɨû˞.ʐʷɔ̃ kxʉ̂.ʐˤɒ̌~|
	ⁿfˠã̂.fˠã zʲɨu˞.ʐʷɔ̃ kxʉ̂.ʐˤɒ̌~|
	zʲaô ɻy̌ˠ.lʲœ̃ ⁿsʲɔ̌ ɣɨʔ ⁿχî˞.ʂʷǒ nɛœʔ kxʉ̂.ʐʲœ.nœ̌.kxʲɨũ~|
	nœ̌.ɻʷɛ̃.f"|i.lʲœ̃ ⁿχî˞.ʂʷǒ ɣɨʔ jɔ̌ˠ ⁿχî˞.sʲɛʔ ɬʷy̟̌ ˀɬʷʉ̌ zʲɔ̃ ɣʉ χî˞ ˀnœ̌.tʃʰɛʔ~|
	zʲaô ɻy̌ˠ.lʲœ̃ kxʉ̂.ʁɨ̌ʔ~|
	ⁿʂǎʔ kiỹ̌.χûˠ.ʁĩ.ɕǒ~|
	
	zʲaô ɻy̌ˠ.lʲœ̃ pɛœ̌ˠ.ʋˤû˞ kxʉ̂.n̪yˠ.mo.qχo˞~|
	ɣiy̌.f"|i.ʋˤû˞ ˀnœ̌ʔ.sˠê~|
	ʐɛœ̃̂ zʷê.χʲœ̌.mʷœ̂.pʲy~|
	zʲaô ɻy̌ˠ.lʲœ̃ ʈʷɔ̂ˠ tʃɛ̃ kxʉ̂.n̪yˠ~|
	ɣiy̌.f"|i.ʋˤû˞ ˀnœ̌ʔ.sˠê~|
	ʐɛœ̃̂ pʲœ̌.f"|i.ʐʷɔ̃̂ zʷê.pʲœ̌~|
	ʈɔ̌ˠ ɴiˠ tʃɛ̃̂ kxʉ̂.n̪yˠ.n̪yˠ.ʑʷœ̌ʔ.ʑʷœ.teœ̃̂~|
	ˀxû ɣɨ̌ʔ kxʉ̂.pʷy̟̌~|
	zʲaô ɻy̌ˠ.lʲœ̃ ⁿfɛœ̌.ʐʷɔ̃̂ kxʉ̂.nɑːʔ.zʲɛœ.kxʲɔ̃̂ː~|
	tʃʷiŷ n̪ʲĩ̌ kxʉ̂.χʲœ̌~|
	ⁿʂǎʔ ⁿχî˞.ʐʷɔ̃ kiỹ̌.ŋɑ~|
	kiỹ̌.ɕɑ̂ː~|
	ˀʂŷ˞ tʲœ tʲɛœ̂ kiỹ̌.ɕaɔ~|
\end{sloppypar}
\end{quote}

Das Lautinventar der Sprache, wie es sich in der Transkription präsentiert,
stellt freilich eine Herausforderung für die Romanisierung dar, vor allem
aufgrund der zahlreichen Sekundärartikulationen. Allerdings scheint die
Transkription recht eng zu sein. Ob zum Beispiel \emph{n} und \emph{n̪}
kontrastiert werden, lässt sich nicht erschließen, da die Grammatiknotizen
keine Auskunft zum Phoneminventar geben.

Die folgende morphologische Analyse mit Glossierung des Texts in \Lhengko{} war
der erste Schritt bei der Bearbeitung des Staffelstabs.%
%
	\footnote{Der grammatischen Annotation der Beispiele liegen die
	\tit{\citetitle{lgr}} \autocite{lgr} zugrunde, vgl.~außerdem den
	\namecref{leipzig} \tit{\nameref{leipzig}}. Übersetzungen und
	Bedeutungsangaben stehen in Hochkommata.}
%
Der Text war erfreulich unproblematisch, was die Annotation betraf, abgesehen
von Satz \cref{ex:lheng6}, den ich nicht recht zu interpretieren wusste und
daher für meine Übersetzung aufgrund des Kontexts deuten musste.

\begin{exe}
\ex \label{ex:lheng1}
	\gll ɮʷǐ˞.ʐʷɔ̃̂ kxʉ̂.ʐˤɒ̌ \\
		Tag-\Stv{} \Pst-enden.\Pfv{} \\
	\trans \wdef{Ein Tag hatte geendet.}
\end{exe}

\begin{exe}
\ex \label{ex:lheng2}
	\gll ⁿfˠâ ɮʷǐ˞.ʐʷɔ̃̂ kxʉ̂.ʐˤɒ̌ \\
		alt Tag-\Stv{} \Pst-enden.\Pfv{} \\
	\trans \wdef{Ein alter Tag hatte geendet.}
\end{exe}

\begin{exe}
\ex \label{ex:lheng3}
	\gll ⁿzʲɨû˞.ʐʷɔ̃ kxʉ̂.ʐˤɒ̌ \\
		Nacht-\Stv{} \Pst-enden.\Pfv{} \\
	\trans \wdef{Eine Nacht hatte geendet.}
\end{exe}

\begin{exe}
\ex \label{ex:lheng4}
	\gll ⁿfˠã̂.fˠã zʲɨu˞.ʐʷɔ̃ kxʉ̂.ʐˤɒ̌ \\
		\Elv\til{}alt Nacht-\Stv{} \Pst-enden.\Pfv{} \\
	\trans \wdef{Eine ältere Nacht hatte geendet.}
\end{exe}

\begin{exe}
\ex \label{ex:lheng5}
	\gll zʲaô ɻy̌ˠ.lʲœ̃ ⁿsʲɔ̌ ɣɨʔ ⁿχî˞.ʂʷǒ nɛœʔ kxʉ̂.ʐʲœ.nœ̌.kxʲɨũ \\
		\Fpl.\Poss{} Fuchs-\Acv{} Sommer in Wein-Hügel
		hinein \Pst-\Inch-\All-laufen \\
	\trans \wdef{Im Sommer kam unser Fuchs in den Weinberg gelaufen.}
\end{exe}

Inchoativ (\q{Ingressiv}) markiert hier aufgrund der Direktivmarkierung
(\fw{nœ̌-} \wdef{hin zu}) das Imperfektiv.

\begin{exe}
\ex \label{ex:lheng6}
	\gll nœ̌.ɻʷɛ̃.f"|i.lʲœ̃ ⁿχî˞.ʂʷǒ ɣɨʔ jɔ̌ˠ ⁿχî˞.sʲɛʔ ɬʷy̟̌ ˀɬʷʉ̌ zʲɔ̃ ɣʉ χî˞ ˀnœ̌.tʃʰɛʔ \\
		\All-kommen-\Nmlz-\Acv{} Wein-Hügel in hoch Wein-Strauch an wenig
		\Clf:klein.rund reif Wein nach.oben-sehen.\Pfv{} \\
	\trans \wdef{Als er kam, schaute er auf und sah, dass hoch oben an den Reben des Weinbergs ein paar Trauben reif waren.} (?)
\end{exe}

Warum \fw{ɣʉ χî˞}, nicht \fw{ɣʉ ⁿχî˞} (V\textsubscript{\leipzigfont{[+\,tense]}} blockt \fw{onset feature spread}). Möglicherweise koordiniertes Verb? Verstehe den Satz nicht, aber interpretiere ihn gemäß dem Kontext so gut wie möglich.

\begin{exe}
\ex \label{ex:lheng7}
	\gll zʲaô ɻy̌ˠ.lʲœ̃ kxʉ̂.ʁɨ̌ʔ \\
		\Fpl.\Poss{} Fuchs-\Acv{} \Pst-sagen.\Pfv{} \\
	\trans \wdef{Unser Fuchs sprach: \dots}
\end{exe}

\begin{exe}
\ex \label{ex:lheng8}
	\gll ⁿʂǎʔ kiỹ̌.χûˠ.ʁĩ.ɕǒ \\
		wahrlich \Npst-durstig.sein-hungrig.sein-\Fsg{} \\
	\trans \wdef{Ich habe wirklich Hunger und Durst.}
\end{exe}

(Ende 1. Absatz)

\begin{exe}
\ex \label{ex:lheng9}
	\gll zʲaô ɻy̌ˠ.lʲœ̃ pɛœ̌ˠ.ʋˤû˞ kxʉ̂.n̪yˠ.mo.qχo˞ \\
		\Fpl.\Poss{} Fuchs-\Acv{} Kraft-\Ins{} \Pst-versuchen-her-greifen \\
	\trans \wdef{Unser Fuchs versuchte mit Kraft, nach \textins{den Trauben} zu greifen.}
\end{exe}

\begin{exe}
\ex \label{ex:lheng10}
	\gll ɣiy̌.f"|i.ʋˤû˞ ˀnœ̌ʔ.sˠê \\
		schwingen-\Nmlz-\Ins{} nach.oben-springen \\
	\trans \wdef{Er sprang mit Schwung hoch,}
\end{exe}

\begin{exe}
\ex \label{ex:lheng11}
	\gll ʐɛœ̃̂ zʷê.χʲœ̌.mʷœ̂.pʲy \\
		aber \Pst.\Neg-können.\Pfv-hin-erreichen \\
	\trans \wdef{doch er konnte \textins{sie} nicht erreichen.}
\end{exe}

\begin{exe}
\ex \label{ex:lheng12}
	\gll zʲaô ɻy̌ˠ.lʲœ̃ ʈʷɔ̂ˠ tʃɛ̃ kxʉ̂.n̪yˠ \\
		\Fpl.\Poss{} Fuchs-\Acv{} zwei Mal \Pst-versuchen \\
	\trans \wdef{Unser Fuchs versuchte es zweimal,}
\end{exe}

\begin{exe}
\ex \label{ex:lheng13}
	\gll ɣiy̌.f"|i.ʋˤû˞ ˀnœ̌ʔ.sˠê \\
		schwingen-\Nmlz-\Ins{} nach.oben-springen \\
	\trans \wdef{mit Schwung hochzuspringen,}
\end{exe}

\begin{exe}
\ex \label{ex:lheng14}
	\gll ʐɛœ̃̂ pʲœ̌.f"|i.ʐʷɔ̃̂ zʷê.pʲœ̌ \\
		aber erreichen.\Pfv-\Nmlz-\Stv{} \Pst.\Neg-erreichen.\Pfv{} \\
	\trans \wdef{aber sein Versuch war erfolglos.}
\end{exe}

\begin{exe}
\ex \label{ex:lheng15}
	\gll ʈɔ̌ˠ ɴiˠ tʃɛ̃̂ kxʉ̂.n̪yˠ.n̪yˠ.ʑʷœ̌ʔ.ʑʷœ.teœ̃̂ \\
		viel \Clf:wiederkehrend Mal \Pst-\Iter\til{}versuchen-\Iter\til{}sich.sehnen-essen \\
	\trans \wdef{Viele Male versuchte er es und sehnte sich wieder und wieder, \textins{sie} zu essen.}
\end{exe}

\begin{exe}
\ex \label{ex:lheng16}
	\gll ˀxû ɣɨ̌ʔ kxʉ̂.pʷy̟̌ \\
		Ende in \Pst-aufgeben \\
	\trans \wdef{Am Ende gab er auf.}
\end{exe}

\begin{exe}
\ex \label{ex:lheng17}
	\gll zʲaô ɻy̌ˠ.lʲœ̃ ⁿfɛœ̌.ʐʷɔ̃̂ kxʉ̂.nɑːʔ.zʲɛœ.kxʲɔ̃̂ː \\
		\Fpl.\Poss{} Fuchs-\Acv{} Nase-\Stv{} \Pst-heben.\Pfv-weg-laufen.\Pfv \\
	\trans \wdef{Unser Fuchs hob die Nase und lief davon.}
\end{exe}

\begin{exe}
\ex \label{ex:lheng18}
	\gll tʃʷiŷ n̪ʲĩ̌ kxʉ̂.χʲœ̌ \\
		Grund aus \Pst-können.\Pfv{} \\
	\trans \wdef{Vielleicht deshalb:}
\end{exe}

Warum nicht \fw{kxʉ̂ʔ.χʲœ̌} \wdef{(er) konnte} (\leipzigfont{[+\,tense]} blockt \fw{onset feature spread}).

\begin{exe}
\ex \label{ex:lheng19}
	\gll ⁿʂǎʔ ⁿχî˞.ʐʷɔ̃ kiỹ̌.ŋɑ \\
		wahrlich Wein-\Stv{} \Npst-sauer.sein.\Pfv{} \\
	\trans \wdef{Der Wein war wirklich sauer.}
\end{exe}

\begin{exe}
\ex \label{ex:lheng20}
	\gll kiỹ̌.ɕɑ̂ː \\
		\Npst-erzählen.\Pfv{} \\
	\trans \wdef{Es ist auserzählt.}
\end{exe}

\begin{exe}
\ex \label{ex:lheng21}
	\gll ˀʂŷ˞ tʲœ tʲɛœ̂ kiỹ̌.ɕaɔ \\
		mancher \Clf:menschlich Mensch \Npst-erzählen \\
	\trans \wdef{So erzählte man sich unter den Menschen.}
\end{exe}

%%%%%%%%%%%%%%%%%%%%%%%%%%%%%%%%%%%%%%%%%%%%%%%%%%%%%%%%%%%%%%%%%%%%%%%%%%%%%%%

\section{Gegenüberstellung der Übersetzungen ins Deutsche}
\label{sec:dtuebers}

Der folgende deutschsprachige Text resultiert aus meiner Interpretation von
Henriks Text auf \Lhengko.

\begin{quote}
Ein Tag hatte geendet.
Ein alter Tag hatte geendet.
Eine Nacht hatte geendet.
Eine ältere Nacht hatte geendet.
Im Sommer kam unser Fuchs in den Weinberg gelaufen.
Als er kam, schaute er auf und sah, dass hoch oben an den Reben des Weinbergs ein paar Trauben reif waren.
Unser Fuchs sprach:
\enquote{Ich habe wirklich Hunger und Durst.}

Unser Fuchs versuchte mit Kraft, nach \textins{den Trauben} zu greifen.
Er sprang mit Schwung hoch,
doch er konnte \textins{sie} nicht erreichen.
Unser Fuchs versuchte es zweimal,
mit Schwung hochzuspringen,
aber sein Versuch war erfolglos.
Viele Male versuchte er es und sehnte sich wieder und wieder, \textins{sie} zu essen.
Am Ende gab er auf.
Unser Fuchs hob die Nase und lief davon.
Vielleicht deshalb:
Der Wein war wirklich sauer.
Es ist auserzählt.
So erzählte man sich unter den Menschen.
\end{quote}

Auf Grundlage des obigen Textes habe ich die Übersetzung in Ayeri angefertigt
und die nachstehende Rückübersetzung ins Deutsche vorgenommen.

\begin{quote}
Die Sonne hat den Mond schon zehntausend Zehntausendmale gejagt.
Ein Fuchs kam damals im Sommer regelmäßig in den Weinberg.
Er bemerkte, dass sich oben an den Reben einige reife Trauben befanden.
Und der Fuchs sprach zu sich:
\q{Ich habe wirklich Hunger und großen Durst.
Lasst uns versuchen, die saftigen Trauben zu erbeuten.}

Der Fuchs rannte und sprang, doch er kam nicht nah genug an die Trauben heran.
Er versuchte es noch einmal, kam aber nicht heran.
Er versuchte es immer und immer wieder.
Er sehnte sich so, sie zu essen.
Endlich musste er doch aufgeben.
Da hob der Fuchs die Nase und lief davon.
Vielleicht war dies der Grund:
Der Wein war wirklich sauer.
So erzählen die Menschen einander.
\end{quote}

%%%%%%%%%%%%%%%%%%%%%%%%%%%%%%%%%%%%%%%%%%%%%%%%%%%%%%%%%%%%%%%%%%%%%%%%%%%%%%%

\section{Übersetzung auf Ayeri}
\label{sec:ayruebers}

\begin{quote}
Ang kimbyo iri perin kolunas samanganyam samang.
Ang sahasayo adauyi runay nimpurivanya matayya.
Kengyong, ya yomayo ling nusan betayjang-aril vilay.
Da-sitang-ningyo runayang:
\q{Mabyang ancu nay tapanyang māy.
Linku-linku vitryam betayjas gali.}

Nimpyo runayang nay pucong, nārya ya sahoyyong nasay-ma betayye.
Linkayong palunganyam, sahoy\-yong nārya.
Li-linkayong ikananyam.
Ang tunyo māy konjam rey.
Rua subryong panca nārya.
Ang da-ringyo runay vinās yona nay sarayong.
Yamanreng yoming edaley:
Nimpurang prasi ancu.
Ang da-ningyan keynamye sitanyayam.
\end{quote}

\begin{quote}
\hyphenpenalty=10000
\small\Tagati
\setcounter{unbalance}{2}
\begin{multicols}{2}
ANF kiMbFyo Iri perinF kolunsF smNnFymF smNF.
ANF shsyo Adauyi runj niMpurivnFy mtjy.
keNFyoNF, y yomyo liNF nusnF betjye\_aNF/ArilF vilj.
da/sitaNF/niNFyo runyNF:
mbFyNF AMkYu nj tpnFyaNF maaj.
liMku/liMku vitFrFymF betjye\_asF gli.

niMpFyo runyNF nj pukFyoNF, naarY y shojyoNF nsj/m betjye.
liMkyoNF pluNnFymF, shjyoNF naarY.
li/liMkyoNF IknnFymF.
ANF tunFyo maaj koMdFymF rej.
ru\_a subFrFyoNF pMtY naarY.
ANF d/riNFyo runj vinaasF yon nj sryoNF.
ymnFreNF yomiNF Edlej:
nimpurNF pFrsi AMkYu.
ANF d/niNFynF kejnmFye sitnYymF.
\end{multicols}
\end{quote}

% Ayeri has a well-behaved romanization
% \renewcommand{\eachwordone}{\itshape}

\begin{exe}
\ex \label{ex:ayr1}
	\gll Ang kimbyo iri perin kolunas samanganyam samang. \\
		\AgtT= jagen-\Tpl.\N{} schon Sonne[\Top] Mond-\Parg{}
		zehntausendmal zehntausend \\
	\trans \wdef{Die Sonne hat den Mond schon zehntausend Zehntausendmale
		gejagt.}
\end{exe}

\begin{exe}
\ex \label{ex:ayr2}
	\gll Ang sahasayo adauyi runay nimpurivanya matayya. \\
		\AgtT= kommen-\Hab-\Tsg.\N{} damals Fuchs[\Top] Weinberg-\Loc{}
		Sommer-\Loc{} \\
	\trans \wdef{Ein Fuchs kam damals im Sommer regelmäßig in den Weinberg.}
\end{exe}

\begin{exe}
\ex \label{ex:ayr3}
	\gll Kengyong, ya yomayo ling nusan betayjang-aril vilay. \\
		merken=\Tsg.\N.\Aarg{} \LocT= sich.befinden-\Tsg.\N{} oben Busch[\Top] Beere-\Pl-\Aarg=einige reif \\
	\trans \wdef{Er bemerkte, dass sich oben an den Reben einige reife
		Trauben befanden.}
\end{exe}

\begin{exe}
\ex \label{ex:ayr4}
	\gll Da-sitang-ningyo runayang: \\
		so=\Refl=sprechen-\Tpl.\N{} Fuchs-\Aarg{} \\
	\trans \wdef{Und der Fuchs sprach zu sich:}
\end{exe}

\begin{exe}
\ex \label{ex:ayr5}
	\gll Mabyang ancu nay tapanyang māy. \\
		hungrig.sein=\Fsg.\Aarg{} wirklich und durstig.sein=\Fsg.\Aarg{}
		\Ints{} \\
	\trans \wdef{Ich habe wirklich Hunger und großen Durst.}
\end{exe}

\begin{exe}
\ex \label{ex:ayr6}
	\gll Linku-linku vitryam betayjas gali. \\
		\Hort\til{}versuchen-\Imp{} erbeuten-\Ptcp{} Beere-\Pl-\Parg.\Inan{}
		saftig \\
	\trans \wdef{Lasst uns versuchen, die saftigen Trauben zu erbeuten.}
\end{exe}

(Ende 1. Absatz)

\begin{exe}
\ex \label{ex:ayr7}
	\gll Nimpyo runayang nay pucong, nārya ya sahoyyong nasay-ma betayye. \\
		rennen-\Tsg.\N{} Fuchs-\Aarg{} und springen-\Tsg.\N{} aber \LocT= kommen-\Neg=\Tsg.\N.\Aarg{} in.Nähe=genug Beere-\Pl{}[\Top] \\
	\trans \wdef{Der Fuchs rannte und sprang, doch er kam nicht nah genug an die Trauben heran.}
\end{exe}

\begin{exe}
\ex \label{ex:ayr8}
	\gll Linkayong palunganyam, sahoyyong nārya. \\
		versuchen=\Tsg.\N.\Aarg{} nochmal kommen-\Neg=\Tsg.\N.\Aarg{} aber \\
	\trans \wdef{Er versuchte es noch einmal, kam aber nicht heran.}
\end{exe}

\begin{exe}
\ex \label{ex:ayr9}
	\gll Li-linkayong ikananyam. \\
		\Iter\til{}versuchen=\Tsg.\N.\Aarg{} vielmals \\
	\trans \wdef{Er versuchte es immer und immer wieder.}
\end{exe}

\begin{exe}
\ex \label{ex:ayr10}
	\gll Ang tunyo māy konjam rey. \\
		\AgtT= verlangen=\Tsg.\N.\Top{} \Ints{} essen-\Ptcp{}
		\Tpl.\Inan.\Parg \\
	\trans \wdef{Er sehnte sich so, sie zu essen.}
\end{exe}

\begin{exe}
\ex \label{ex:ayr11}
	\gll Rua subryong panca nārya. \\
		müssen= aufgeben=\Tsg.\N.\Aarg{} endlich aber \\
	\trans \wdef{Endlich musste er doch aufgeben.}
\end{exe}

\begin{exe}
\ex \label{ex:ayr12}
	\gll Ang da-ringyo runay vinās yona nay sarayong. \\
		\AgtT= so=heben-\Tsg.\N{} Fuchs[\Top] Nase-\Parg{} \Tsg.\N.\Gen{} und
		gehen=\Tsg.\N.\Aarg{} \\
	\trans \wdef{Da hob der Fuchs die Nase und lief davon.}
\end{exe}

\begin{exe}
\ex \label{ex:ayr13}
	\gll Yamanreng yoming edaley: \\
		Grund-\Aarg.\Inan{} vielleicht dies-\Parg.\Inan{} \\
	\trans \wdef{Vielleicht war dies der Grund:}
\end{exe}

\begin{exe}
\ex \label{ex:ayr14}
	\gll Nimpurang prasi ancu. \\
		Wein-\Aarg{} sauer wirklich \\
	\trans \wdef{Der Wein war wirklich sauer.}
\end{exe}

\begin{exe}
\ex \label{ex:ayr15}
	\gll Ang da-ningyan keynamye sitanyayam. \\
		\AgtT= so=erzählen-\Tpl.\Aarg{} Mensch-\Pl{}[\Top] einander-\Dat{} \\
	\trans \wdef{So erzählen die Menschen einander.}
\end{exe}

%%%%%%%%%%%%%%%%%%%%%%%%%%%%%%%%%%%%%%%%%%%%%%%%%%%%%%%%%%%%%%%%%%%%%%%%%%%%%%%

\section{Beigegebenes Material}
\label{sec:suppl}

...

\subsection{Glossar}

% \setcounter{unbalance}{2}
\begin{multicols}{2}
\raggedright
\begin{description}[nosep]
	% \item[Lemma]
	% 	\ayr{lemma}
	% 	\emph{Wortart},
	% 	Übersetzung

	\item[-aril]
		\ayr{/ArilF}
		\emph{Adv.},
		etwas, ein paar, manche
	\item[-ma]
		\ayr{/m}
		\emph{Adv.},
		genug, genügend
	\item[adauyi]
		\ayr{Adauyi}
		\emph{Pron.-Adv.},
		dann, damals
	\item[ancu]
		\ayr{AMkYu}
		\emph{Adv.},
		wirklich
	\item[betay]
		\ayr{betj}
		\emph{N., inan.},
		Beere
	\item[gali]
		\ayr{gli}
		\emph{Adj.},
		saftig
	\item[ikananyam]
		\ayr{IknnFymF}
		\emph{Adv.},
		vielfach, vielmals
	\item[iri]
		\ayr{Iri}
		\emph{Adv.},
		schon
	\item[keng-]
		\ayr{keNF/}
		\emph{Vb.},
		bemerken
	\item[keynam]
		\ayr{kejnmF}
		\emph{N., anim.},
		Mensch
	\item[kimb-]
		\ayr{kiMbF/}
		\emph{Vb.},
		jagen
	\item[kolun]
		\ayr{kolunF}
		\emph{N., anim.},
		Mond
	\item[kond-]
		\ayr{koMdF/}
		\emph{Vb.},
		essen
	\item[ling]
		\ayr{liNF}
		\emph{Präp.},
		oben (an), auf; während (parallel geschehend zu)
	\item[linka-]
		\ayr{liMk/}
		\emph{Vb.},
		versuchen
	\item[mab-]
		\ayr{mbF/}
		\emph{Vb.},
		hungern, hungrig sein
	\item[matay]
		\ayr{mtj}
		\emph{N., inan.},
		Sommer
	\item[māy]
		\ayr{maaj}
		\emph{Adv.},
		ja, doch
	\item[nasay]
		\ayr{nsj}
		\emph{Präp.},
		in der Nähe von
	\item[nay]
		\ayr{nj}
		\emph{Konj.},
		und
	\item[nimp-]
		\ayr{niMpF/}
		\emph{Vb.},
		laufen, rennen
	\item[nimpur]
		\ayr{niMpurF}
		\emph{N., anim.},
		Wein
	\item[nimpurivan]
		\ayr{niMpurivnF}
		\emph{N., inan.},
		Weinberg
	\item[ning-]
		\ayr{niNF/}
		\emph{Vb.},
		erzählen
	\item[nusan]
		\ayr{nusnF}
		\emph{N., anim.},
		Busch, Strauch
	\item[nārya]
		\ayr{naarY}
		\emph{Adv.},
		aber, doch
	\item[palunganyam]
		\ayr{pluNnFymF}
		\emph{Adv.},
		noch einmal
	\item[panca]
		\ayr{pMtY}
		\emph{Adv.},
		schließlich, endlich
	\item[perin]
		\ayr{perinF}
		\emph{N., anim.},
		Sonne
	\item[prasi]
		\ayr{pFrsi}
		\emph{Adj.},
		sauer
	\item[puk-]
		\ayr{pukF/}
		\emph{Vb.},
		springen, hüpfen
	\item[rey]
		\ayr{rej}
		\emph{Pers.-Pron.},
		es
	\item[ring-]
		\ayr{riNF/}
		\emph{Vb.},
		wachsen; heben
	\item[rua-]
		\ayr{ru\_a/}
		\emph{Vb.},
		müssen
	\item[runay]
		\ayr{runj}
		\emph{N., anim.},
		Fuchs
	\item[saha-]
		\ayr{sh/}
		\emph{Vb.},
		kommen; passieren
	\item[samang]
		\ayr{smNF}
		\emph{Num.},
		zehntausend
	\item[samanganyam]
		\ayr{smNnFymF}
		\emph{Adv.},
		zehntausendmal
	\item[sara-]
		\ayr{sr/}
		\emph{Vb.},
		gehen, verlassen; aufhören
	\item[sitanya]
		\ayr{sitnFy}
		\emph{Indef.-Pron.},
		einander
	\item[subr-]
		\ayr{subFrF/}
		\emph{Vb.},
		aufgeben, einbüßen
	\item[tapan-]
		\ayr{tpnF/}
		\emph{Vb.},
		dürsten, durstig sein
	\item[tun-]
		\ayr{tunF/}
		\emph{Vb.},
		wünschen, begehren
	\item[vilay]
		\ayr{vilj}
		\emph{Adj.},
		reif
	\item[vina]
		\ayr{vin}
		\emph{N., anim.},
		Nase
	\item[vitr-]
		\ayr{vitFrF/}
		\emph{Vb.},
		ergreifen, (ein)fangen
	\item[yaman]
		\ayr{ymnF}
		\emph{N., inan.},
		Grund, Anlass, Ursache
	\item[yoma-]
		\ayr{yom/}
		\emph{Vb.},
		da sein, vorhanden sein, sich befinden
	\item[yoming]
		\ayr{yomiNF}
		\emph{Adv.},
		vielleicht
	\item[yona]
		\ayr{yon}
		\emph{Pers.-Pron.},
		sein
\end{description}
\end{multicols}

\subsection{Notizen zur Grammatik}
\label{subsec:gramnot}

\subsubsection{Allophonie}

Bei den Konsonantenphonemen löst /j/ nach /t k/ und /d ɡ/ allophonisch
Palatalisierung zu [t͡ʃ] und [d͡ʒ] aus, die in der Romanisierung mit ⟨c⟩ und
⟨j⟩ wiedergegeben werden.
Zwei adjazente Vokale der gleichen Qualität produzieren einen Langvokal, also
zum Beispiel /a/~+ /a/~>~/aː/ ⟨ā⟩, mit Ausnahme der verbalen Aspekt- und
Modussuf"|fixe, die einen vorangehenden Vokal typischerweise tilgen.
% ; /uː/ ⟨ū⟩ existiert aber nur in wenigen Lexemen, zum Beispiel
% \xayr{bbuu}{babū}{barbarisch}.

\subsubsection{Syntax}

Ayeri (\,\ayr{Ayeri}\,) verwendet Verberststellung (\textsc{vso}) als
unmarkierte Konstituentenfolge. Da die Sprache eine Variante des
\textsc{vo}-Typus darstellt, folgen Modifikatoren ihren Köpfen in der Regel.
Dies bedeutet, dass Adjektive, Possessiva und Relativsätze ihrem Nomen folgen;
genauso folgen Possessoren auch dem Possessum.

% Darüber hinaus ist Ayeri im Grunde eine Akkusativsprache (\textsc{s~=~a~≠~o}).
% \q{Echte} Passivsubjekte behalten allerdings ihre Patiensmarkierung, während
% das Agensargument dann fehlt. In diesen Fällen von Ergativität zu sprechen,
% würde die Beschreibung nur unnötig verkomplizieren. Obwohl Belebtheit sogar
% eine Flexionskategorie in der Sprache darstellt, bleibt diese Unterscheidung
% syntaktisch ungenutzt. Demotion der Agens zu einem obliquen Argument gibt es
% aufgrund der semantischen Kasusmarkierung nicht. Es ist aber möglich, ein
% \q{unechtes} Passiv zu bilden, bei welchem das Patiensargument logisch die
% Topik bildet aber das Verb weiterhin mit dem Agensargument als syntaktischem
% Subjekt kongruiert.
% Auch bei kausativen Sätzen bildet der Auslöser, als solcher gesondert
% markiert, logisch die Topik, wird aber ebenfalls nicht zum syntaktischen
% Subjekt. Die anderen Argumente des Verbs werden entsprechend auch in diesem
% Fall nicht herabgestuft.

% In ditransitiven Sätzen wird der Donor als Agens markiert
% (\textsc{s~=~a~=~d}), das Thema als Patiens (\textsc{o~=~t}). Der Rezipient
% (\textsc{r}) erhält Dativmarkierung. Prädikative \textsc{np}s werden
% abweichend als Patiens markiert, um Subjekt (Agens) und Prädikat
% (\q{Patiens}) zu unterscheiden, da Ayeri keine overte Kopula besitzt und
% doppelte Kernrollenmarkierung im gleichen Satz vermeidet.

Neben regulären Verbalsätzen gibt es auch Kopulasätze, allerdings besitzt Ayeri
eine Null-Kopula. Eine Besonderheit ist, dass das Prädikatsnomen in diesem Fall
als Patiens markiert wird, obwohl es mit dem Subjekt (mit Agensmarkierung)
gleichbedeutend ist. Das Prädikat kann zum Zweck der Betonung an die Spitze des
Satzes gestellt werden.

% Ayeri macht keinen Unterschied zwischen restriktiven und nicht-restriktiven
% Relativsätzen. Relativsätze brauchen allerdings immer ein Antezedens, freie
% Relativsätze sind also nicht erlaubt. Relativsätze sind im Grunde eigenständige
% Sätze, insofern die Relativpartikel \rayr{si}{si} die Funktion einer
% Subjunktion hat, die ein komplexes Attribut an eine \textsc{np} bindet oder mit
% deren Hilfe Attribute in ihrem Bezug desambiguiert werden können. Relativsätze
% haben daher normalerweise einen internen Kopf. Wenn ein Relativsatz einen
% Kopulasatz enthält, kann dessen Subjekt ausfallen.

% Komplemente von \textsc{np}s werden zur Vermeidung von Ambiguität in der
% Modifikationsrelation rechtsversetzt, wenn die \textsc{np} ein Adjunkt enthält,
% welches das Kopfnomen modifiziert.

\subsubsection{Morphosyntax}
\label{subsubsec:morphsyn}

Die Topik wird durch ein Proklitikum am Verb markiert, das im Grunde der
Kasusendung der Topik-\textsc{np} entspricht, während die Topik-\textsc{np}
selbst nullmarkiert ist. Es handelt sich bei Ayeri also um eine sogenannte
\fw{trigger conlang}. Es bestehen nahezu keine Restriktionen für die Wahl der
Topik-\textsc{np}. Pronomen können in gleicher Weise topikalisiert werden.
Topikmarkierung ist obligatorisch in transitiven Sätzen, während intransitive
Sätze normalerweise keine Topik markieren. Auch imperative Verben tragen
normalerweise keine Topikmarkierung.

% Die Relativpartikel \rayr{si}{si} zeigt optional Kasuskongruenz mit der
% \textsc{np}, welche der Relativsatz modifiziert. Dies geschieht vor allem dann,
% wenn der Relativsatz rechtsversetzt ist.

Neben den verschiedenen Pronomenarten ist die einzige Kongruenz zeigende
Wortart das Verb. Grundsätzlich kongruieren Verben mit dem Agensargument, es
sei denn, es fehlt durch
% echte
Passivierung. Ersatzweise kongruiert das Verb dann mit dem Patiensargument als
syntaktischem Subjekt.

\subsubsection{Morphologie}

Ayeri ist eine agglutinierende Sprache und dabei sehr regelmäßig. Entsprechend
dem \textsc{vo}-Typus werden hauptsächlich Suf"|fixe zur Flexion benutzt.
Darüber hinaus besitzt die Sprache etliche Klitika, die sich insbesondere bei
finiten Verben in einem Klitikcluster vor dem Verb zeigen.

\subsubsubsection{Nomen}
\label{subsubsec:nom}

Ayeri hat ein zweistufiges Genussystem: Nomen können entweder belebt (\Anim)
oder unbelebt (\Inan) sein. Zu den belebten Nomen zählen zum Beispiel lebende
Personen und Tiere, Personifizierungen, Gefühle und mentale Prozesse sowie
Dinge, die Anzeichen von Leben zeigen (z.\,B.~Pflanzen) oder die eng mit
Menschen assoziiert sind (z.\,B.~Wohnungen). Menschen sowie Haus- und Nutztiere
können entsprechend ihrem sozialen respektive ihrem biologischen Geschlecht
maskulin (\M) oder feminin (\F) sein. Als belebt klassifizierte Dinge und
Abstrakta sind dagegen neutral (\N). Genus ist dem Lexikon inhärent und kovert,
darum gibt das Glossar es als Hilfsstellung explizit an. Es gibt keine
Markierung von Definit- und Indefinitheit, doch existiert ein optionales Präfix,
das Unspezifizität anzeigt (\xayr{me/}{mə-}{irgendein}), im Text aber nicht
vorkommt.

Nomen flektieren in der Regel nach Numerus und Kasus, können in bestimmten
Kontexten aber auch ohne overte Kasusflexion auftreten. Der Singular ist
unmarkiert, der Plural wird mit dem Suf"|fix \rayr{/ye}{-ye} gekennzeichnet.
Dieses Suf"|fix hat ein Allomorph \rayr{/ye}{-j} (in der eigenen Schrift nicht
graphisch differenziert), das erscheint, wenn das darauf"|folgende Suf"|fix mit
Vokal oder /j/ beginnt, beispielsweise
\rayr{/ye}{-ye}~+~\rayr{/AsF}{-as}~>~\rayr{/ye\_asF}{-jas}.

Ayeri unterscheidet sieben Kasus: Agens (\Aarg), Patiens (\Parg), Dativ (\Dat),
Genitiv (\Gen), Lokativ (\Loc), Kausativ (\Caus) und Instrumentalis (\Ins),
siehe~\cref{tab:decl}. Die Vokale in Klammern in der Tabelle fallen weg, wenn
der Stamm auf einen Vokal endet, was also auch dann der Fall ist, wenn an die
Wurzel ein Pluralsuf"|fix angehängt ist.

\begin{table}
\caption{Kasusmarkierung der Nomen}
\begin{tabularx}{\linewidth}{l l l c c X}
\toprule
Kasus
	& \multicolumn{2}{c}{Suf"|fixform}
	& \multicolumn{2}{c}{proklitische Form}
	& Funktion
	\\

\cmidrule(lr){2-3}
\cmidrule(lr){4-5}

%
	& \multicolumn{1}{c}{\Anim}
	& \multicolumn{1}{c}{\Inan}
	& \multicolumn{1}{c}{\Anim}
	& \multicolumn{1}{c}{\Inan}
	\\

\midrule

\Aarg
	& -ang
	& -reng
	& ang
	& eng
	& prototypische Agens (Agens, Experiencer, Force); transitive und intransitive Subjekte im Aktiv; Subjekt des \q{unechten} Passivs; Subjekt in Kopulasätzen
	\\

\Parg
	& -as
	& -ley
	& sa
	& le
	& prototypische Patiens (Patiens, Thema); transitive und intransitive Objekte im Aktiv, direktes Objekt; Subjekt des \q{echten} Passivs; Prädikatsnomen in Kopulasätzen
	\\

\midrule

\Dat
	& \multicolumn{2}{c}{-yam}
	& \multicolumn{2}{c}{yam}
	& Rezipient; Ziel, Richtung; indirektes Objekt; sekundäres Prädikatsnomen
	\\

\Gen
	& \multicolumn{2}{c}{-(e)na}
	& \multicolumn{2}{c}{na}
	& Possessor, Quelle; worüber etwas geht bzw. wovon etwas handelt
	\\

\Loc
	& \multicolumn{2}{c}{-ya}
	& \multicolumn{2}{c}{ya}
	& Ort; typisch assoziiertes Ziel von Bewegungsverben
	\\

\Caus
	& \multicolumn{2}{c}{-isa}
	& \multicolumn{2}{c}{sā}
	& Verursacher (nur adverbiale Verwendung)
	\\

\Ins
	& \multicolumn{2}{c}{-(e)ri}
	& \multicolumn{2}{c}{ri}
	& Instrument, Helfer; Komplement einer \textsc{np}
	\\

\bottomrule
\end{tabularx}
\label{tab:decl}
\end{table}

Topikalisierte \textsc{np}s sind nullmarkiert, stattdessen wird der
entsprechende Kasus mit der in \cref{tab:decl} angegebenen klitischen Form
links vom Verb markiert. Eigennamen verwenden ebenfalls die klitische Form bei
der Kasusmarkierung, zum Beispiel \xayr{n bliinF}{na Balīn}{von Berlin}.

Der Diminutiv von Nomen wird durch vollständige Reduplikation angezeigt. Bei
Komposita wird nur das Kopfnomen redupliziert und flektiert. Komposita sind in
der Regel univerbiert, sodass grammatische Endungen an das letzte Element
angehängt werden. Daneben gibt es losere Verbindungen von Nomen, bei denen
ebenfalls nur das Kopfnomen flektiert wird und das modifizierende Nomen als
Attribut folgt.

\subsubsubsection{Pronomen}

Ayeri besitzt durch die Menge an Kasus und Genera eine Fülle von (ziemlich
regelmäßig gebildeten) Personalpronomen, wobei für den Kontext des vorliegenden
Textes nur ein Teil derjenigen in \cref{tab:persproagr} relevant ist, die
ihrerseits nur einen Ausschnitt darstellt. Für dritte Personen werden auch
häufig Demonstrativpronomen verwendet%
% , allerdings kommt dieser Fall im Text nicht vor
. Indefinitpronomen sind im Glossar aufgeführt, sofern sie im Text vorkommen.

\begin{table}
\caption{Personalpronomen und Personenendungen der Verben (relevanter Ausschnitt)}
\begin{tabularx}{\linewidth}{
	l l
	C C
	C C
	C C
	% c c
	C C
}
\toprule
%
	& %
	& \multicolumn{2}{c}{\makecell[tc]{Kongruenz-/\\Topikform}}
	& \multicolumn{2}{c}{\Aarg}
	& \multicolumn{2}{c}{\Parg}
	% & \multicolumn{2}{c}{\Dat}
	& \multicolumn{2}{c}{\Gen}
	\\

\cmidrule(lr){3-4}
\cmidrule(lr){5-6}
\cmidrule(lr){7-8}
\cmidrule(lr){9-10}
% \cmidrule(lr){11-12}

%
	& %
	& \multicolumn{1}{c}{\Sg}
	& \multicolumn{1}{c}{\Pl}
	& \multicolumn{1}{c}{\Sg}
	& \multicolumn{1}{c}{\Pl}
	& \multicolumn{1}{c}{\Sg}
	& \multicolumn{1}{c}{\Pl}
	% & \multicolumn{1}{c}{\Sg}
	% & \multicolumn{1}{c}{\Pl}
	& \multicolumn{1}{c}{\Sg}
	& \multicolumn{1}{c}{\Pl}
	\\

\midrule

\First
	& %
	& ay
	& ayn
	& yang
	& nang
	& yas
	& nas
	% & yām
	% & nyam
	& nā
	& nana
	\\

\Second
	& %
	& va
	& va
	& vāng
	& vāng
	& vās
	& vās
	% & vayam
	% & vayam
	& vana
	& vana
	\\

\Third
	& \M
	& ya
	& yan
	& yāng
	& tang
	& yās
	& tas
	% & yayam
	% & cam
	& yana
	& tan
	\\

%
	& \F
	& ye
	& yen
	& yeng
	& teng
	& yes
	& tes
	% & yeyam
	% & teyam
	& yena
	& ten
	\\

%
	& \N
	& yo
	& yon
	& yong
	& tong
	& yos
	& tos
	% & yoyam
	% & toyam
	& yona
	& ton
	\\

%
	& \Inan
	& ara
	& aran
	& reng
	& teng
	& rey
	& tey
	% & rayam
	% & racam
	& ran
	& ten
	\\

\bottomrule
\end{tabularx}
\label{tab:persproagr}
\end{table}

Demonstrativpronomen werden mit \rayr{d/}{da-} (indefinit), \rayr{Ed/}{eda-}
(proximal) und \rayr{Ad/}{ada-} (distal) gebildet. Gerade beim belebten
Agens- und Patiens-Demonstrativum tritt daran das Element \rayr{/nY}{-nya}
(z.\,B.~\xayr{AdnYaaNF}{adanyāng}{jener, der da};
vgl.~\xayr{nYaanF}{nyān}{Person}), in jedem Fall folgt am Schluss die
Kasusendung, die dieselbe wie bei der Deklination der Nomina ist
(\cref{tab:decl}).

% In \cref{subsubsec:morphsyn} wurde erklärt, dass Relativpartikeln keine
% Pronomen im engen Sinn darstellen, allerdings können sie durch sekundäre
% Kasusmarkierung pronominalisiert werden. Das Relativ\-pronomen trägt dann eine
% zweite Kasusendung, die seine grammatische Funktion als Konstituente innerhalb
% des Relativsatzes markiert. Wenn die Relativpartikel keine primäre
% Kasuskongruenz aufweist (z.\,B. \rayr{sin}{sina} mit Bezug auf eine
% Genitiv-\textsc{np}) und so die sekundäre Endung an das einfache \rayr{si}{si}
% tritt, wird der Vokal der sekundären Endung zur Desambiguierung gedehnt, zum
% Beispiel \xayr{sinaa}{sinā}{von welchem}. Sekundär markierte Relativa können
% jedoch innerhalb des Relativsatzes nicht selbst als Topiken fungieren, insofern
% sie ihre Kasusmarkierung nicht ans Verb abgeben können.

\subsubsubsection{Verben}

Verben kongruieren nach Person (\First, \Second, \Third) und Numerus (\Sg,
\Pl) ihres Subjekts, siehe \cref{tab:persproagr}. Bei dritten Personen kommen
noch Genus und Belebtheit (\M, \F, \N, \Inan) als Flexionskategorien hinzu. Bei
pronominalen Subjekten ersetzt das Personalpronomen das Kongruenzsuf"|fix am
Verb, indem es als Enklitikum ans Ende des Verbstamms tritt. Die
Personenendungen der regulären Kongruenz mit dem Subjekt und die
topikalisierten pronominalen Klitika sind homophon, zum Beispiel korrespondiert
die Vollform \xayr{/yaaNF}{-yāng}{er} mit der topikalisierten Form
\rayr{ANF—/y}{ang \dots\ -ya}. \rayr{/y}{-ya} ist gleichzeitig auch die
Kongruenzendung für den Bezug auf eine Subjekt-\textsc{np} im Singular
Maskulinum.

Finite Verben weisen darüber hinaus optional Flexion für Tempus auf, ansonsten
für Aspekt und Modus. Dafür werden verschiedene Markierungsstrategien verwendet.
Im Rahmen des Texts sind habitualer und iterativer Aspekt sowie der Imperativ
als Modus relevant. Der Imperativ der zweiten Person wird mit der
Quasi-Personenendung \rayr{/U}{-u} markiert, die einen vorhergehenden Vokal
tilgt, bei Hortativen wird die Verbform zusätzlich redupliziert. Habitualer
Aspekt wird mit der Endung \rayr{/As}{-asa} markiert, die an den Verbstamm tritt
und ebenfalls einen vorhergehenden Vokal tilgt. Aspekt kann darüber hinaus durch
Adverbien ausgedrückt werden, zum Beispiel
\xayr{myis}{mayisa}{fertig sein}, welches die Abgeschlossenheit einer Handlung
betont.

Iterativer Aspekt drückt aus, dass eine Handlung mehrfach geschieht, kann aber
auch reversive Bedeutung haben, zum Beispiel
\xayr{t/tpYnNF}{ta-tapyanang}{wir legen immer wieder} oder \wdef{wir legen
wieder zurück}. Wie das Beispiel zeigt, wird iterativer Aspekt durch
Reduplikation der ersten beiden Silbensegmente des Verbstamms angezeigt.

Modalität wird in der Regel durch Modalpartikeln ausgedrückt, die im
präverbalen Klitikcluster nach dem Topikmarker stehen. Diese haben
typischerweise die Form von unflektierten Verbstämmen, zum Beispiel
korrespondiert \xayr{miNF/}{ming-}{können} mit der Partikel \rayr{miNF}{ming}
und \xayr{mY/}{mya-}{sollen} mit der Partikel \rayr{mY}{mya}.

Bei \xayr{d/}{da-}{so} handelt es sich um eine Partikel, die zum einen
pronominal verwendet werden kann, zum Beispiel \xayr{d/kilyNF}{da-kilayang}{ich
darf das} oder \xayr{d/IMtYyeNF}{da-incyeng}{sie kauft eins}. Zum anderen kann
sie auch präsentative Funktion haben, beispielsweise in
\xayr{d/shyaaNF}{da-sahayāng}{da kommt er}.

Eine weitere Partikel stellt \rayr{sitNF}{sitang-} dar, das anstelle eines
vollständigen Reflexivpronomens auftreten kann.
\xayr{sitNF/ketFtNF}{sitang-kettang}{sie waschen sich} ist also äquivalent zu
\rayr{ANF ketFynF sitNF/tsF}{ang kecan sitang-tas}.

Wenn ein Verb ein verbales Komplement besitzt, zum Beispiel bei Kontroll- und
Raisingverben, weist das abhängige Verb eine im Prinzip infinite Form auf, die
mit \rayr{/ymF}{-yam} gekennzeichnet und als \q{Partizip} bezeichnet wird. Mit
\rayr{/AnF}{-an} nominalisiert kann diese Form als Gerundium verwendet werden.
Infinite Verben dieser Art können trotzdem Modus- und Aspektmarkierung
aufweisen.

\subsubsubsection{Adjektive, Adverbien \& Co.}

Adjektive weisen keine Kongruenz auf, können aber negiert und gesteigert
werden, genauso wie auch Adverbien. Sie stehen immer direkt hinter ihrem Bezug.

Neben Adjektiven im engeren Sinn besitzt Ayeri eine Reihe von Quantoren, die in
der Regel an die flektierte Form des Nomens (determinierende Quantoren), Verbs,
ein Adjektiv oder eine Präposition (adverbiale Quantoren) angehängt werden.
% Der Text enthält mehrere solcher Partikeln, zum Beispiel
% \xayr{/kj}{-kay}{wenig, etwas, ein bisschen}.

Numeralia sind duodezimal. Größere Potenzen werden mit dem Derivationssuffix
\rayr{/nNF}{-nang} gebildet: \xayr{menNF}{menang}{100} (zu
\xayr{menF}{men}{eins}), \xayr{smNF}{samang}{1\,00\,00} (zu
\xayr{smF}{sam}{zwei}), \xayr{kjnNF}{kaynang}{1\,00\,00\,00\,00}, etc. Diese
Einheitswörter fungieren als Köpfe, die von Numeralia attribuiert werden, zum
Beispiel \xayr{menNF yo}{menang yo}{400} (zu \xayr{yo}{yo}{vier}).
Ordinalzahlen werden durch Nominalisierung der Kardinalzahlen gebildet, also
zum Beispiel \xayr{tmnF koMkYnFyen}{iran koncanyena}{der fünfte Monat} (zu
\xayr{Iri}{iri}{fünf}). Multiplikativzahlen verwenden davon die Dativform, also
zum Beispiel \xayr{miynFymF}{miyanyam}{sechsmal} (zu \xayr{miye}{miye}{sechs}).
Distributivzahlen verwenden stattdessen den Instrumental, zum Beispiel
\xayr{Itneri}{itaneri}{zu je sieben} (zu \xayr{Ito}{ito}{sieben}),
allerdings kommt dieser Fall im Text nicht vor. Ordinal-, Multiplikativ- und
Distributivzahlen können prinzipiell genauso wie Ordinalzahlen von anderen
Numeralia attribuiert werden, und zwar in ihrer ordinalen Form.

\subsubsubsection{Präpositionen}

Freie Dative und Genitive können eine Bewegung zu etwas hin beziehungsweise von
etwas her kennzeichnen (vgl.~\cref{subsubsec:nom}). Freie Lokative kennzeichnen
eine Position, vor allem eine, die prototypisch mit dem Verb im Satz assoziiert
wird. Dies kommt insbesondere bei Positions- und Bewegungsverben zum Tragen.

Ayeri verwendet darüber hinaus in der Regel Präpositionen, die größtenteils von
Nomen abgeleitet sind. Daneben gibt es eine Reihe von Postpositionen, von denen
die meisten jüngere, sekundäre Bildungen etwa aus Adverbialen darstellen. Das
Präpositionalobjekt steht in der Regel im Lokativ. Steht es im Dativ,
kennzeichnet dieser bei manchen Präpositionen eine Bewegung in Richtung des
Objekts statt eines Ruhens an dem Ort, welchen das Objekt bezeichnet.

%% BIBLIOGRAPHY %%%%%%%%%%%%%%%%%%%%%%%%%%%%%%%%%%%%%%%%%%%%%%%%%%%%%%%%%%%%%%%

% \vfill
% \pagebreak

\begingroup\multicolsep=0pt
\printglossary[
	style=threecolumn,
	type=leipzig,
	title={Abkürzungen der Glossierung},
]
\endgroup

% % \nocite{*} % returns all entries from the bibliography database
\printbibliography[heading=bibintoc]

\end{document}
