\documentclass[
	12pt,
	ngerman,
]{scrartcl}

% Author, Title, Subtitle etc.
\author{Carsten Becker}
\title{Ein fruchtloses Bemühen}
\subtitle{Beitrag zum zweiten Relay des Berliner Conlang-Stammtischs}
\date{\DTMdate{2025-01-00}} % Format: YYYY/MM/DD (rev. N, YYYY/MM/DD)

% Provide running author and date
\makeatletter
\let\runauthor\@author
\let\rundate\@date
\makeatother

% Preferences of float placement
% cf. https://tex.stackexchange.com/a/167864/60686
\makeatletter
\renewcommand\fps@figure{htbp}
\renewcommand\fps@table{htbp}
\def \@floatboxreset {%
	\reset@font
	\normalsize
	\@setminipage
	\centering
}
\makeatother


% Handle language and quotation marks
\usepackage[
	english,
	main=ngerman
]{babel}
\usepackage[autostyle]{csquotes} % Put quotations in \enquote{}!
\SetBlockEnvironment{quotation}
\renewcommand*{\mkccitation}[1]{ (#1)}
\let\q\textquote

% Quotation style for word definitions
\DeclareQuoteStyle{wdef}
	{\textquoteleft}{\textquoteright}
	{\textquotedblleft}{\textquotedblright}
\newcommand{\wdef}[1]{{\setquotestyle{wdef}\enquote{#1}}}

% Set all margins to 2.54 cm
\usepackage[margin=1in]{geometry}
\widowpenalty10000 % Avoid widows like the plague!
\clubpenalty10000 % Avoid orphans like the plage, too!

% Extended formatting of lists
\usepackage{enumitem}
\newlist{glossdefs}{itemize}{1}
\setlist[glossdefs]{nosep, leftmargin=3em, labelwidth=2.5em, align=left}
\setlist[itemize]{noitemsep}

% Make multiple columns available in single-column document
\usepackage{multicol}

% Make text colors and color names available
\usepackage{xcolor}

% Load font stuff for XeTeX
\usepackage{fontspec}

% Set main fonts
\usepackage{microtype}

\newfontfamily{\Tagati}[
	Renderer=Graphite,
	Scale=1.0,
	BoldFont={* Italic},
% 	HyphenChar=·,
]{Tagati Book G}

% \setmainfont[
% 	Ligatures=TeX,
% 	Numbers={OldStyle,Proportional},
% 	BoldFont=*-Bold,
% 	ItalicFont=*-Italic,
% 	BoldItalicFont=*-BoldItalic,
% ]{Junicode}

\setmainfont[
	Ligatures=TeX,
	Numbers={OldStyle,Proportional},
]{Junicode}

\setsansfont[
	Ligatures=TeX,
	Numbers=Lowercase,
	Scale=MatchUppercase,
	ItalicFont={Noto Sans Condensed Light Italic},
	BoldFont={Noto Sans Condensed Bold},
	BoldItalicFont={Noto Sans Condensed Bold Italic},
]{Noto Sans Condensed Light}

% Load BibLaTeX (using Biber), configure citation styles
\usepackage[
	style=langsci-unified,
	backend=biber,
	alldates=terse,
	safeinputenc,
	natbib,
]{biblatex}
\addbibresource{bibliography.bib}

% biblatex
% Use sane date format as specified in LangSci Generic Style Rules:
% (Accessed YYYY-MM-DD) => (Zugriff am DD.MM.YYYY)
% Essentially, undefine redefinition from langsci-unified.bbx. Copied from
% biblatex/biblatex.def and modified from biblatex/lbx/german.lbx.
\DefineBibliographyStrings{german}{%
  % Defaults to "gesehen am" per biblatex/lbx/german.lbx
  urlseen = {Zugriff am},
}
\renewbibmacro*{url+urldate}{%
  \usebibmacro{url}%
  \iffieldundef{urlyear}
    {}
    {\setunit*{\addspace}%
     \usebibmacro{urldate}}}

% AVMs
\usepackage[lfg]{langsci-avm}
\avmsetup{values=\normalfont}

% Date etc.
\usepackage[
	useregional=numeric,
]{datetime2}

% % Verse
% \usepackage{verse}

% Ability to include graphics and dealing with footnotes in descriptions
\usepackage{graphicx}
\usepackage[font={small,sf},labelfont={small,sf},format=plain]{caption}
\usepackage{subcaption}
\usepackage{wrapfig}
\setlength{\columnsep}{2\baselineskip}

% General headers and footers
\usepackage{fancyhdr}
\pagestyle{fancy}

\fancyhead[L]{} % empty
\fancyhead[C]{} % empty
\fancyhead[R]{\thepage}

\fancyfoot[L]{} % empty
\fancyfoot[C]{} % empty
\fancyfoot[R]{} % empty

\renewcommand{\headrulewidth}{0pt}
\renewcommand{\footrulewidth}{0pt}

% First page headers and footers are different
\fancypagestyle{firstpage}{
	\fancyhead[L]{\sffamily \footnotesize \textbf{Benung. The Ayeri Language Resource}}
	\fancyhead[C]{} % empty
	\fancyhead[R]{\sffamily \footnotesize \runauthor{} · \rundate{}}
	
	\fancyfoot[L]{} % empty
	\fancyfoot[C]{} % empty
	\fancyfoot[R]{\sffamily \footnotesize 
	\href{https://ayeri.de}{https://ayeri.de} · 
	\href{https://github.com/carbeck/benung-pdfs}{https://github.com/carbeck/benung-pdfs} · 
	\href{https://creativecommons.org/licenses/by-sa/4.0/}{CC~BY-SA~4.0}
	}
	
	\renewcommand{\headrulewidth}{0.5pt}
	\setlength\footskip{0.5in}
}

\usepackage{ifthen}
\ifthenelse{\value{page}=1}{\thispagestyle{firstpage}{\pagestyle{fancy}}}

% Line spacing
\usepackage{setspace}
\onehalfspacing

% Avoid pagebreaks right after sections and subsections
\usepackage{needspace}
\usepackage{etoolbox}
\preto\section{\needspace{6\baselineskip}}
\preto\subsection{\needspace{6\baselineskip}}

% Things for tables
\usepackage{tabularx}
\usepackage{booktabs}
\usepackage{makecell}
% \usepackage{rotating}
\newcolumntype{C}{>{\centering\arraybackslash}X}

% Formatting of table of glossing abbreviations from Leipzig package manual
\usepackage[
	acronym,
	nomain,
	nonumberlist,
	nopostdot,
	numberedsection=nameref,
	toc=true,
	xindy={codepage=utf8}, % language=english,
]{glossaries}
\usepackage{glossary-inline}

\newglossarystyle{threecolumn}{%
	\renewenvironment{theglossary}{%
		\begin{multicols}{3}
		\begin{glossdefs}%
	}{%
		\end{glossdefs}%
		\end{multicols}%
	}%
	\renewcommand*{\glossaryheader}{}%
	\renewcommand*{\glsgroupheading}[1]{}%
	\newcommand*{\glossaryentryfield}[5]{%
		\item[\glsentryitem{##1}\glstarget{##1}{##2}]
		% \makefirstuc{##3}\glspostdescription{}
		##3\glspostdescription{}
	}%
	\renewcommand*{\glsgroupskip}{}%
}%

% Formatting of glosses
\usepackage{langsci-gb4e}
\usepackage[glosses]{leipzig}
\renewcommand{\eachwordone}{\itshape}
\renewcommand{\eachwordtwo}{\rule[-.5\baselineskip]{0pt}{0pt}}
% leipzig.def
% Glossing abbreviations

% German redefinitions
\renewleipzig{Aarg}{a}{Agens}
\renewleipzig{Abs}{abs}{Absolutiv}
\renewleipzig{Acc}{acc}{Akkusativ}
\renewleipzig{All}{all}{Allativ}
\renewleipzig{Caus}{caus}{Kausativ}
\renewleipzig{Dat}{dat}{Dativ}
\renewleipzig{Dem}{dem}{Demonstrativ}
\renewleipzig{First}{1}{erste Person}
\renewleipzig{F}{f}{Femininum}
\renewleipzig{Gen}{gen}{Genitiv}
\renewleipzig{Imp}{imp}{Imperativ}
\renewleipzig{Ins}{ins}{Instrumentalis}
\renewleipzig{Loc}{loc}{Lokativ}
\renewleipzig{M}{m}{Maskulinum}
\renewleipzig{Neg}{neg}{Negativ}
\renewleipzig{Nmlz}{nmlz}{Nominalisierer}
\renewleipzig{Nom}{nom}{Nominativ}
\renewleipzig{N}{n}{Neutrum}
\renewleipzig{Parg}{p}{Patiens}
\renewleipzig{Pl}{pl}{Plural}
\renewleipzig{Poss}{poss}{Possessiv}
\renewleipzig{Pst}{pst}{Präteritum}
\renewleipzig{Ptcp}{ptcp}{Partizip}
\renewleipzig{Refl}{refl}{Reflexiv}
\renewleipzig{Rel}{rel}{Relativ}
\renewleipzig{Second}{2}{zweite Person}
\renewleipzig{Sg}{sg}{Singular}
\renewleipzig{Third}{3}{dritte Person}
\renewleipzig{Top}{top}{Topik}

% Other common abbreviations, undefined by default
\newleipzig{Dim}{dim}{Diminutiv}

% ɮɛ̃̂.kɔ̌ʔ
\newleipzig{Acv}{acv}{Aktiv}
\newleipzig{Clf}{clf}{Klassifizierer}
\newleipzig{Elv}{elv}{Elativ}
\newleipzig{Inch}{inch}{Inchoativ}
\newleipzig{Ipfv}{ipfv}{Imperfektiv}
\newleipzig{Nmz}{nmz}{Substantivierung}
\newleipzig{Npst}{npst}{Nicht-Präteritum}
\newleipzig{Pfv}{pfv}{Perfektiv}
\newleipzig{Stv}{stv}{Stativ}

% Ayeri
\newleipzig{AgtT}{at}{Agenstopik}
\newleipzig{Anim}{anim}{belebt}
\newleipzig{Ayr}{ayr}{Ayeri}
\newleipzig{CauT}{caut}{Kausativtopik}
\newleipzig{Conj}{conj}{konjunktiv}
\newleipzig{DatT}{datt}{Dativtopik}
\newleipzig{Dir}{dir}{direktiv}
\newleipzig{GenT}{gent}{Genitivtopik}
\newleipzig{Hab}{hab}{Habitativ}
\newleipzig{Hort}{hort}{Hortativ}
\newleipzig{Inan}{inan}{unbelebt}
\newleipzig{InsT}{inst}{Instrumentaltopik}
\newleipzig{Ints}{ints}{intensiv}
\newleipzig{Iter}{iter}{iterativ}
\newleipzig{LocT}{loct}{Lokativtopik}
\newleipzig{Oblig}{oblig}{obligativ}
\newleipzig{PatT}{pt}{Patienstopik}
\newleipzig{Sup}{sup}{Superlativ}


% Trees
\usepackage[linguistics]{forest}
\usetikzlibrary{tikzmark, positioning}

% Underline, strikeout
\usepackage{soul}

% Nicer footnotes
\usepackage[bottom,hang,norule]{footmisc}
\setlength{\footnotesep}{0.75\baselineskip}

% Smaller font in block quotes
\usepackage{relsize}
\AtBeginEnvironment{quote}{\noindent\smaller}
\AtBeginEnvironment{quotation}{\smaller}

% Clickable links in footnotes, TOC, etc.
\usepackage[
% 	xetex,
	bookmarks=true,
	colorlinks=false,
	linktoc=section,
	hidelinks,
	pdfusetitle,
]{hyperref}

% We want URLs to be italic and with regular uppercase numerals
\renewcommand{\UrlFont}{%
	\normalfont%
	\itshape%
	\addfontfeature{RawFeature=-onum}%
}

% In-text references
% cf. https://tex.stackexchange.com/a/139051
% Since German plural formation is not as regular as in English (-e, -en for
% Beispiel), we will define the label as empty
\usepackage[sort&compress,noabbrev]{cleveref}
\newcommand{\crefrangeconjunction}{--}
\crefname{xnumi}{}{}
\creflabelformat{xnumi}{(#2#1#3)}
\crefname{xnumii}{}{}
\creflabelformat{xnumii}{(#2#1#3)}
\crefname{xnumiii}{}{}
\creflabelformat{xnumiii}{(#2#1#3)}
\crefname{xnumiv}{}{}
\creflabelformat{xnumiv}{(#2#1#3)}
\crefrangeformat{xnumi}{(#3#1#4)--(#5#2#6)}
\crefrangeformat{xnumii}{(#3#1#4--#5\crefstripprefix{#1}{#2}#6)}
\crefrangemultiformat{xnumii}{(#3\arabic{xnumi}#1#4--#5#2#6)}
{ and~(#3\arabic{xnumi}#1#4--#5#2#6)}{, (#3\arabic{xnumi}#1#4--#5#2#6)}
{ and~(#3\arabic{xnumi}#1#4--#5#2#6)}

% Subsubsubsection
% cf. https://tex.stackexchange.com/a/356574
\DeclareNewSectionCommand[
	style=section,
	counterwithin=subsubsection,
	afterskip=1.5ex plus .2ex,
	beforeskip=3.25ex plus 1ex minus .2ex,
	afterindent=false,
	level=\paragraphnumdepth,
	tocindent=10em,
	tocnumwidth=5em
]{subsubsubsection}
\setcounter{secnumdepth}{\subsubsubsectionnumdepth}
\setcounter{tocdepth}{\subparagraphtocdepth}
\addto\extrasngerman{%
	\let\subsubsubsectionautorefname\subsubsectionautorefname
}
\crefalias{subsubsubsection}{subsubsection}

% Make glossaries
\makeglossaries

% Macros
\newcommand{\fw}[1]{\textit{#1}} % Foreign Word
\newcommand{\til}{\char`\~} % Literal tilde in text mode % {$\sim$}
\newcommand{\tit}[1]{\textit{#1}} % Title of a work
\newcommand{\tsub}[1]{\textsubscript{#1}} % Subscript
\newcommand{\tsup}[1]{\textsuperscript{#1}} % Superscript
\newcommand{\markyellow}[1]{\colorbox{yellow}{#1}} % Yellow highlighter
\newcommand{\ques}{\textsuperscript{?}} % raised question mark
\newcommand{\zwsp}{\mbox{​}} % Zero-width space (ZWSP)

\newcommand{\ayr}[1]{\zwsp\smash{{\Tagati #1}}} % Plain Ayeri orthography
\newcommand{\rayr}[2]{\zwsp\smash{{\Tagati #1}} \emph{#2}} % Ayeri orthography + *r*omanization
\newcommand{\tayr}[2]{#1 `#2'} % Romanization + *t*ranslation
\newcommand{\xayr}[3]{\zwsp\smash{\Tagati #1} \emph{#2} `#3'} % Ayeri orthography + romanization + translation

\newenvironment{mytitle}{
	\hfill
	\begin{minipage}{0.667\textwidth}
	\vspace{\baselineskip}
	\begin{center}
		\Large
		\sffamily\bfseries
		\makeatletter
}{
		\makeatother
	\end{center}
	\vspace{1em}
	\end{minipage}
	\hfill
}

% Change abstract font
\AtBeginEnvironment{abstract}{\small\sffamily}

% Source language name
\newcommand{\Lhengko}{ɮɛ̃̂.kɔ̌ʔ}

%% END OF PREAMBLE %%%%%%%%%%%%%%%%%%%%%%%%%%%%%%%%%%%%%%%%%%%%%%%%%%%%%%%%%%%%

\begin{document}

%% MAIN PART %%%%%%%%%%%%%%%%%%%%%%%%%%%%%%%%%%%%%%%%%%%%%%%%%%%%%%%%%%%%%%%%%%

\begin{mytitle}
	\@title: \@subtitle
\end{mytitle}

\begin{abstract}
Another local relay was held in Berlin during November 2024. This time, I had
the pleasure to translate from Henrik's \emph{\Lhengko} as the last person in a
circle of seven participants. Thus, I returned my torch back to Bruno, who had
started the game, and who translated my \emph{Ayeri} text into his
\emph{Paksuta} to conclude. Again, the game was run in German due to the
limited, local scope. The text that reached me was about a fox's repeated but
unsuccessful attempts to reach and eat ripe grapes growing on top of a vine in
a vinyard during summer. The fox gives up after a number of tries and trots
off. The story seems to conclude with an enigmatic prediction that the wine will
really be sour. The base text was a version of Aesop's fable \tit{The fox and
the grapes} \autocite[76--77]{jacobs1894}.
\end{abstract}

Nachdem das Relay im Mai Anklang fand, bestand der Wunsch nach einer weiteren
Runde. Zwei weitere Teilnehmende konnten dazugewonnen werden, sodass insgesamt
sieben Personen an diesem Relay teilgenommen haben. Mit \emph{Ayeri} war ich
der letzte im Kreis, bevor der Staffelstab zum Ausgangspunkt zurückging, um das
Spiel zu beenden. Mein Vorgänger war Henrik mit \emph{\Lhengko}, mein
Nachfolger Bruno mit \emph{Paksuta}. Natürlich wurde auch diesmal nach den
bewährten Spielregeln gespielt.%
%
	\footnote{Siehe
	\url{https://conlang.org/language-creation-conference/lcc6/lcc6-relay/}
	(\DTMdate{2024-11-30}).}
%
Die Teilnehmenden hatten dieses Mal vier Tage Zeit für ihre Etappe, daher ist
der zu übersetzende Text im Vergleich zum letzten Mal kürzer ausgefallen. Das
Spiel wurde wieder auf Deutsch durchgeführt. Der ursprüngliche Text wurde von
Bruno organisiert. Es handelte sich um eine englischsprachige Version der
Äsop-Fabel \tit{Der Fuchs und die Trauben} \autocite[76--77]{jacobs1894}:

\begin{quote}
\selectlanguage{english}
\tit{The fox and the grapes}

One hot summer's day a Fox was strolling through an orchard till he came to a
bunch of Grapes just ripening on a vine which had been trained over a lofty
branch. \q{Just the thing to quench my thirst} quoth he. Drawing back a few
paces, he took a run and a jump, and just missed the bunch. Turning round again
with a One, Two, Three, he jumped up, but with no greater success. Again
and again he tried after the tempting morsel, but at last had to give it up,
and walked away with his nose in the air, saying: \q{I am sure they are sour.}
\end{quote}

Die Moral, \textquote{It is easy to despise what you cannot get}
\autocite[77]{jacobs1894}, hatte Bruno schon bei der ersten Übersetzung
weggelassen. Der Inhalt des Texts ist die ganze Runde hindurch weitestgehend
erhalten geblieben. Durch con-kulturelle Erzählpraktiken sind allerdings
Einleitungs- und Schlusssätze dazugekommen.

%%%%%%%%%%%%%%%%%%%%%%%%%%%%%%%%%%%%%%%%%%%%%%%%%%%%%%%%%%%%%%%%%%%%%%%%%%%%%%%

\section{Analyse der Vorlage auf \Lhengko}
\label{sec:lhenganalys}

Den nachfolgenden Text habe ich von Henrik neben einer Wortliste und ein paar
Notizen zur Grammatik analog zum Material in \cref{sec:suppl} erhalten, wie
üblich. Henrik scheint eine Vorliebe für komplexe Silben und IPA als
Transkriptionssystem anstelle der lateinischen Schrift zu haben, was bereits
beim letzten Spiel an \fw{ʀu.lu} deutlich wurde, aus dem Kierán übersetzt hat.

\begin{quote}
\begin{sloppypar}
	ɮʷǐ˞.ʐʷɔ̃̂ kxʉ̂.ʐˤɒ̌~|
	ⁿfˠâ ɮʷǐ˞.ʐʷɔ̃̂ kxʉ̂.ʐˤɒ̌~|
	ⁿzʲɨû˞.ʐʷɔ̃ kxʉ̂.ʐˤɒ̌~|
	ⁿfˠã̂.fˠã zʲɨu˞.ʐʷɔ̃ kxʉ̂.ʐˤɒ̌~|
	zʲaô ɻy̌ˠ.lʲœ̃ ⁿsʲɔ̌ ɣɨʔ ⁿχî˞.ʂʷǒ nɛœʔ kxʉ̂.ʐʲœ.nœ̌.kxʲɨũ~|
	nœ̌.ɻʷɛ̃.f"|i.lʲœ̃ ⁿχî˞.ʂʷǒ ɣɨʔ jɔ̌ˠ ⁿχî˞.sʲɛʔ ɬʷy̟̌ ˀɬʷʉ̌ zʲɔ̃ ɣʉ χî˞ ˀnœ̌.tʃʰɛʔ~|
	zʲaô ɻy̌ˠ.lʲœ̃ kxʉ̂.ʁɨ̌ʔ~|
	ⁿʂǎʔ kiỹ̌.χûˠ.ʁĩ.ɕǒ~|
	
	zʲaô ɻy̌ˠ.lʲœ̃ pɛœ̌ˠ.ʋˤû˞ kxʉ̂.n̪yˠ.mo.qχo˞~|
	ɣiy̌.f"|i.ʋˤû˞ ˀnœ̌ʔ.sˠê~|
	ʐɛœ̃̂ zʷê.χʲœ̌.mʷœ̂.pʲy~|
	zʲaô ɻy̌ˠ.lʲœ̃ ʈʷɔ̂ˠ tʃɛ̃ kxʉ̂.n̪yˠ~|
	ɣiy̌.f"|i.ʋˤû˞ ˀnœ̌ʔ.sˠê~|
	ʐɛœ̃̂ pʲœ̌.f"|i.ʐʷɔ̃̂ zʷê.pʲœ̌~|
	ʈɔ̌ˠ ɴiˠ tʃɛ̃̂ kxʉ̂.n̪yˠ.n̪yˠ.ʑʷœ̌ʔ.ʑʷœ.teœ̃̂~|
	ˀxû ɣɨ̌ʔ kxʉ̂.pʷy̟̌~|
	zʲaô ɻy̌ˠ.lʲœ̃ ⁿfɛœ̌.ʐʷɔ̃̂ kxʉ̂.nɑːʔ.zʲɛœ.kxʲɔ̃̂ː~|
	tʃʷiŷ n̪ʲĩ̌ kxʉ̂.χʲœ̌~|
	ⁿʂǎʔ ⁿχî˞.ʐʷɔ̃ kiỹ̌.ŋɑ̌~|
	kiỹ̌.ɕɑ̂ː~|
	ˀʂŷ˞ tʲœ tʲɛœ kiỹ̌.ɕaɔ̌~|
\end{sloppypar}
\end{quote}

Das Lautinventar der Sprache, wie es sich in der Transkription präsentiert,
stellt allerdings eine Herausforderung für die Romanisierung dar, vor allem
aufgrund der zahlreichen Sekundärartikulationen. Allerdings scheint die
Transkription recht eng zu sein. Ob zum Beispiel alveolares \emph{n} und
dentales \emph{n̪} tatsächlich kontrastiert werden -- was ungewöhnlich wäre --
lässt sich nicht erschließen, da die Grammatiknotizen keine Auskunft zum
Phoneminventar geben.

Die folgende morphologische Analyse mit Glossierung des Texts in \Lhengko{} war
der erste Schritt bei der Bearbeitung des Staffelstabs.%
%
	\footnote{Der grammatischen Annotation der Sätze liegen die
	\tit{\citetitle{lgr}} zugrunde \autocite{lgr}, vgl.~außerdem den
	\namecref{leipzig} \tit{\nameref{leipzig}}. Übersetzungen und
	Bedeutungsangaben stehen zur Abgrenzung von Zitaten in Hochkommata.}
%
Der Text war erfreulich unproblematisch, was die morphologische Annotation
betraf. Bei der Übersetzung haben hauptsächlich die Sätze~\cref{ex:lheng6}
und~\cref{ex:lheng19} Verständnis\-schwierigkeiten bereitet, sodass ich sie
aufgrund des Kontexts deuten musste.

Die Fabel öffnet mit einer rhetorischen Figur, die sich auf die ersten vier
Sätze \crefrange{ex:lheng1}{ex:lheng4} erstreckt und die narrative
Vergangenheit der Erzählung etabliert.

\begin{exe}
\ex \label{ex:lheng1}
	\gll ɮʷǐ˞.ʐʷɔ̃̂ kxʉ̂.ʐˤɒ̌ \\
		Tag-\Stv{} \Pst-enden.\Pfv{} \\
	\trans \wdef{Ein Tag hatte geendet.}
\end{exe}

\Lhengko{} markiert Perfektivität zusätzlich zum Tempus. Perfektivität wird --
genauso wie Pluralität bei stark belebten Nomina -- durch Vokalgradation
(\q{Ablaut}) angezeigt. Dabei werden die meisten Vokale um eine oder zwei
Stufen geöffnet; zusätzlich wird der Ton der Silbe beim Stammwechsel
invertiert. Die perfektive Form des Verbs \fw{ʐˤɔ̂} \wdef{enden} lautet daher
\fw{ʐˤɒ̌} \wdef{beendet}. Die Verbform wird hier zusätzlich mit dem
Tempuspräfix \fw{kxʉ̂-} als Präteritum markiert, wodurch das Ende der Handlung
in der Vergangenheit explizit gemacht wird.

\begin{exe}
\ex \label{ex:lheng2}
	\gll ⁿfˠâ ɮʷǐ˞.ʐʷɔ̃̂ kxʉ̂.ʐˤɒ̌ \\
		alt Tag-\Stv{} \Pst-enden.\Pfv{} \\
	\trans \wdef{Ein alter Tag hatte geendet.}
\end{exe}

\begin{exe}
\ex \label{ex:lheng3}
	\gll ⁿzʲɨû˞.ʐʷɔ̃ kxʉ̂.ʐˤɒ̌ \\
		Nacht-\Stv{} \Pst-enden.\Pfv{} \\
	\trans \wdef{Eine Nacht hatte geendet.}
\end{exe}

\begin{exe}
\ex \label{ex:lheng4}
	\gll ⁿfˠã̂.fˠã zʲɨu˞.ʐʷɔ̃ kxʉ̂.ʐˤɒ̌ \\
		\Elv\til{}alt Nacht-\Stv{} \Pst-enden.\Pfv{} \\
	\trans \wdef{Eine ältere Nacht hatte geendet.}
\end{exe}

Die Sätze \crefrange{ex:lheng1}{ex:lheng4} zeigen weiterhin, dass \Lhengko{}
eine Aktiv/Stativ-Unterscheidung besitzt (\Acv/\Stv). Auch, wenn \fw{ɮʷǐ˞}
\wdef{Tag} und \fw{ⁿzʲɨû˞} \wdef{Nacht} im Stativ stehen (\fw{-ʐʷɔ̃}), sind
sie die einzigen overten Argument des Verbs, weswegen ich sie als Subjekte
aufgefasst habe. Verben haben overte Personenmarkierung nur für die erste und
zweite Person. Objektkongruenz ist prinzipiell möglich, doch erscheint
immer nur eine Kongruenzendung, da overten Subjektendungen der Vorrang über
Kongruenz mit dem Objekt gegeben wird. Die Verben in diesen vier Sätzen tragen
allerdings keine overten Personenmarkierungen, weil nur Referenten der dritten
Person auftreten. Ich habe mich entschieden, Nullmarkierung nicht zu
glossieren.

Satz \cref{ex:lheng4} weist gleich zwei interessante Merkmale am Adjektiv
\fw{ⁿfˠã̂.fˠa} \wdef{älter, ältest} auf. Die Zitationsform des Adjektivs lautet
\fw{ⁿfˠâ} \wdef{alt}, vergleiche \cref{ex:lheng2}. Dieses Adjektiv wird
einerseits zur Komparation redupliziert, wobei \Lhengko{} anscheinend nur eine
Steigerungsstufe unterscheidet, die in den Notizen als Elativ (\Elv) benannt
wird. Durch \fw{onset feature spread} wird andererseits die Pränasalisierung
des Anlauts auf die vorhergehende Silbe übertragen und zeigt sich dort in der
Nasalierung des Vokals im Silbenkern.

Dass das Phänomen über Wortgrenzen hinweg operiert, zeigt sich an der Form
\fw{ⁿfˠã̂.fˠã} zu Beginn von  \cref{ex:lheng4}, deren Wurzel \fw{ⁿzʲɨû˞}
\wdef{Nacht} ihre Pränasalierung auf die Vorgängersilbe überträgt. Diese lautet
dadurch zunächst \fw{ⁿfˠã̂}. Wenn Silben, aufeinander folgen, die zur selben
phonologischen Phrase gehören und denselben Ton besitzen, bleiben Folgesilben
der ersten mit phonemisch fallendem Ton phonetisch tief bis zur nächsten
Tonänderung oder zum Ende der Phrase. Bei der reduplizierten Form
\fw{ⁿfˠã̂.fˠã} ist es daher nicht nötig, den fallenden beziehungsweise
tiefen Ton auf der zweiten Silbe gesondert zu markieren.

Weil Verben nur ein Präfix haben dürfen, können Verben mit Direktivpräfix wie
\fw{nœ̌.kxʲɨũ} \wdef{hineinlaufen} in \cref{ex:lheng5} kein Tempuspräfix
erhalten. Stattdessen wird das Tempus indirekt über Aspekt ausgedrückt,
beziehungsweise hier über eine Periphrase mit \fw{ʐʲœ̂} (zu \fw{ⁿʐʲɛœ̂}
\wdef{anfangen}), die imperfekte Bedeutung hat.

\begin{exe}
\ex \label{ex:lheng5}
	\gll zʲaô ɻy̌ˠ.lʲœ̃ ⁿsʲɔ̌ ɣɨʔ ⁿχî˞.ʂʷǒ nɛœʔ kxʉ̂.ʐʲœ.nœ̌.kxʲɨũ \\
		\Fpl.\Poss{} Fuchs-\Acv{} Sommer in Wein-Hügel
		hinein \Pst-\Inch-\All-laufen \\
	\trans \wdef{Den Sommer über kam unser Fuchs in den Weinberg gelaufen.}
\end{exe}

Zusammen mit \fw{kxʉ̂-} als Markierung des Präteritums müsste die
zusammengesetzte Verbform \fw{kxʉ̂.ʐʲœ.nœ̌.kxʲɨũ} in \cref{ex:lheng5} also so
etwas wie \wdef{lief (für unbestimmte Zeit) hin zu} bedeuten, was ich im
Kontext der Angabe \fw{ⁿsʲɔ̌ ɣɨʔ} \wdef{im Sommer} als habituelle Handlung
während dieser Zeit aufgefasst habe. Die Glossierung von \fw{ʐʲœ̂} als
inchoativ, beziehungsweise nach der Terminologie der Notizen ingressiv, rührt
von der ursprünglichen Bedeutung des Hilfsverbs her.

Das Subjekt von \cref{ex:lheng6}, \fw{nœ̌.ɻʷɛ̃.f"|i.lʲœ̃}, ist eine
nominalisierte Verbform, die dem Kontext nach wörtlich vermutlich am besten
als \wdef{der Hinkommende} oder \wdef{der Ankommende} zu übersetzen ist, im
deutschsprachigen Kontext idiomatisch als temporaler Nebensatz \wdef{als er
ankam}.

\begin{exe}
\ex \label{ex:lheng6}
	\gll nœ̌.ɻʷɛ̃.f"|i.lʲœ̃ ⁿχî˞.ʂʷǒ ɣɨʔ jɔ̌ˠ ⁿχî˞.sʲɛʔ ɬʷy̟̌ ˀɬʷʉ̌ zʲɔ̃ ɣʉ χî˞ ˀnœ̌.tʃʰɛʔ \\
		\All-kommen-\Nmlz-\Acv{} Wein-Hügel in hoch Wein-Strauch an wenig
		\Clf:klein.rund reif Wein nach.oben-sehen.\Pfv{} \\
	\trans \wdef{Als er ankam, schaute er auf und sah, dass hoch oben an den
	Reben des Weinbergs ein paar Trauben reif waren.}
\end{exe}

Da \Lhengko{} sehr durchgängig OV-Merkmale zeigt, befindet sich nicht nur das
Verb \fw{ˀnœ̌.tʃʰɛʔ} \wdef{sah nach oben} am Ende des Satzes \cref{ex:lheng6},
sondern es zeigen sich auch durchgängig Postpositionen wie \fw{ɣǐʔ} \wdef{in}
und \fw{ɬʷy̟̌} \wdef{an}. Genau wie Adjektive vorangehen, besitzen die
Komposita \fw{ⁿχî˞.ʂʷǒ} \wdef{Weinberg} und \fw{ⁿχî˞.sʲɛʔ}
\wdef{Weinrebe} die Abfolge \textsc{modifizierer}~--~\textsc{kopf}.%
%
	\footnote{Dies passt perfekt zu
	\citeauthor{greenberg1966}s~(\citeyear[77--79, 85--86]{greenberg1966})
	Universalien 4 (SOV korreliert mit Postpositionen) und 18 (vorangestellte
	Adjektive korrelieren mit vorangestellten Demonstrativa und Numeralia), sowie
	zum Umkehrschluss aus Universalie~5, demzufolge vorangestellte Genitive mit
	vorangestellten Adjektiven korrelieren sollten.}

Ein weiteres auf"|fälliges Merkmal des \Lhengko{} sind Klassifizierer, die mit
Mengenangaben und Kardinalzahlen auftreten, wie bei \fw{ˀɬwʉ̌ zjɔ̃ ɣʉ χî˞} in
\cref{ex:lheng6}. Diese NP habe ich als \wdef{ein paar reife Trauben}
interpretiert. Entweder benötigt das Objekt von \fw{ˀnœ̌.tʃʰɛʔ} keine
Markierung mit dem Stativ (\tsup{?}\fw{ⁿχî˞.ʐʷɔ̃}) oder meine Analyse ist nicht
ganz korrekt. Dieselbe NP zeigt außerdem eine Komplikation in der Regel zum
\fw{onset feature spread}. Im Text heißt es \fw{ɣʉ́ χî˞}, nicht *\fw{ɣʉ́
ⁿχî˞}, weil \fw{ʉ} das Auslautmerkmal \leipzigfont{[+\,gespannt]} besitzt. Das
Auslautmerkmal blockiert die Abgabe der Nasalierung von \fw{ⁿχî˞}~\wdef{Wein}
an \fw{ɣʉ̌} \wdef{rein (sein)}. Als Resultat fällt die Pränasalierung weg.

Genauso wie Adjektive ihren Nomen vorangehen, geht auch das Possessivpronomen
\fw{zʲaô} \wdef{unser} in \cref{ex:lheng7} dem Nomen voran. Darüber hinaus
begegnet in \cref{ex:lheng8} die komplexe Verbform \fw{kiỹ̌.χûˠ.ʁĩ.ɕǒ}
\wdef{ich habe Hunger und Durst}. Diese setzt sich aus den beiden Verbstämmen
\fw{χûˠ} \wdef{durstig sein} und \fw{ʁĩ̂} \wdef{hungrig sein} zusammen, die
hier in einer seriellen Verbkonstruktion mit koordinativer Bedeutung stehen.%
%
	\footnote{Serielle Verben, Tonakzent und Klassifizierer in einer
	Silbensprache lassen kontinental-ostasiatische Sprachen als Vorbild vermuten.
	Beim Nachtreffen hat sich herausgestellt, dass Tangut
	(\url{https://glottolog.org/resource/languoid/id/tang1334}; Zugriff am
	\DTMdate{2024-12-13}) die hauptsächliche Inspirationsquelle war.}

\begin{exe}
\ex \label{ex:lheng7+8}
	\begin{xlist}
	\ex \label{ex:lheng7}
		\gll zʲaô ɻy̌ˠ.lʲœ̃ kxʉ̂.ʁɨ̌ʔ \\
			\Fpl.\Poss{} Fuchs-\Acv{} \Pst-sagen.\Pfv{} \\

	\ex \label{ex:lheng8}
		\gll ⁿʂǎʔ kiỹ̌.χûˠ.ʁĩ.ɕǒ \\
			wahrlich \Npst-durstig.sein-hungrig.sein-\Fsg{} \\
		\trans \wdef{Unser Fuchs sprach: \q{Ich habe wirklich Hunger und Durst.}}
\end{xlist}
\end{exe}

Während \Lhengko{} bei den Tempora einen Unterschied zwischen Präteritum und
Nicht-Präteritum macht, ist letztere Form in \cref{ex:lheng8} vermutlich
aufgrund des Zusammenspiels von Tempus, Negation und Direktion overt mit
\fw{kiỹ̌-} markiert, vergleiche dazu auch \cref{ex:lheng11} mit Kombination
von Negation und Tempus. Allein an dieser Stelle zeigt das Verb mit \fw{-ɕǒ}
ein Pronominalsuf"|fix der 1.~Person Singular.

Auch Satz \cref{ex:lheng9} macht sich eine serielle Verbkonstruktion zunutze.
Anders als in \cref{ex:lheng8}, ist das zweite Verb, \fw{mo.qχo˞}, wörtlich
\wdef{hergreifen}, dem ersten, \fw{kxʉ.n̪yˠ} \wdef{versuchte}, dem Sinn nach
untergeordnet. Die Sätze \cref{ex:lheng9,ex:lheng10} sind darüber hinaus zwei
der drei, die den Instrumental zeigen, nämlich bei \fw{pɛœ̌ˠ.ʋˤû˞} \wdef{mit
Kraft} und \fw{ɣiy̌.f"|i.ʋˤû˞} \wdef{mit Schwung}.

\begin{exe}
\ex \label{ex:lheng9}
	\gll zʲaô ɻy̌ˠ.lʲœ̃ pɛœ̌ˠ.ʋˤû˞ kxʉ̂.n̪yˠ.mo.qχo˞ \\
		\Fpl.\Poss{} Fuchs-\Acv{} Kraft-\Ins{} \Pst-versuchen-her-greifen \\
	\trans \wdef{Unser Fuchs versuchte mit Kraft, nach \textins{den Trauben} zu greifen.}
\end{exe}

Der Teilsatz in \cref{ex:lheng11} zeigt noch eine weitere Besonderheit des
\Lhengko{} beim Verb \fw{zʷê.χʲœ̌.mʷœ̂.pʲy} \wdef{konnte nicht erreichen}. Das
Verb hat ein Präfix \fw{zʷê-}, das als Portmanteaumorphem Negation und
Präteritum vereint, siehe auch \cref{ex:lheng8}.

\begin{exe}
\ex \label{ex:lheng10+11}
	\begin{xlist}
	\ex \label{ex:lheng10}
		\gll ɣiy̌.f"|i.ʋˤû˞ ˀnœ̌ʔ.sˠê \\
			schwingen-\Nmlz-\Ins{} nach.oben-springen \\

	\ex \label{ex:lheng11}
		\gll ʐɛœ̃̂ zʷê.χʲœ̌.mʷœ̂.pʲy \\
			aber \Pst.\Neg-können.\Pfv-hin-erreichen \\
		\trans \wdef{Er sprang mit Schwung hoch, doch er konnte \textins{sie} nicht erreichen.}
\end{xlist}
\end{exe}

Nachfolgend wird in \cref{ex:lheng13} der Satz in \cref{ex:lheng10} wörtlich
wiederholt. Auch die Formulierung \fw{pjœ̌.f"|i.ʐwɔ̃ zwê.pjœ̌} zu \fw{pʲŷ}
\wdef{erreichen} in \cref{ex:lheng14}, wörtlich \wdef{das Erreichte hat nichts
erreicht}, sieht etwas wie ein Wortspiel basierend auf dem Wortlaut von
\cref{ex:lheng11} aus. Im Zusammenhang bezieht sie sich \cref{ex:lheng14} wohl
auf die erfolglosen Versuche, an die Trauben zu gelangen.

\begin{exe}
\ex \label{ex:lheng12+13+14}
	\begin{xlist}
	\ex \label{ex:lheng12}
		\gll zʲaô ɻy̌ˠ.lʲœ̃ ʈʷɔ̂ˠ tʃɛ̃ kxʉ̂.n̪yˠ \\
			\Fpl.\Poss{} Fuchs-\Acv{} zwei Mal \Pst-versuchen \\
	
	\ex \label{ex:lheng13}
		\gll ɣiy̌.f"|i.ʋˤû˞ ˀnœ̌ʔ.sˠê \\
			schwingen-\Nmlz-\Ins{} nach.oben-springen \\
	
	\ex \label{ex:lheng14}
		\gll ʐɛœ̃̂ pʲœ̌.f"|i.ʐʷɔ̃̂ zʷê.pʲœ̌ \\
			aber erreichen.\Pfv-\Nmlz-\Stv{} \Pst.\Neg-erreichen.\Pfv{} \\
		\trans \wdef{Unser Fuchs versuchte es zweimal, mit Schwung hochzuspringen, aber sein Versuch war erfolglos.}
\end{xlist}
\end{exe}

Was die Zahlenangabe \fw{ʈʷɔ̂ˠ tʃɛ̃} \wdef{zweimal} in \cref{ex:lheng12}
betrifft, gehen die Grammatiknotizen nur auf den Unterschied zwischen Kardinal-
und Ordinalzahlen ein; \fw{tʃɛ̃̂} \wdef{Mal} wurde der Vokabelliste entnommen.
Warum in \cref{ex:lheng15} der Klassifizierer \fw{ɴiˠ} in der Angabe \fw{ʈɔ̌ˠ
ɴiˠ tʃɛ̃̂} \wdef{viele Male} auftritt, in \cref{ex:lheng12} aber nicht, wird
aus den Notizen nicht deutlich. Möglicherweise wird ein Unterschied zwischen
bestimmten (\wdef{zwei}) und unbestimmten Mengenangaben (\wdef{viel}) gemacht.

\begin{exe}
\ex \label{ex:lheng15}
	\gll ʈɔ̌ˠ ɴiˠ tʃɛ̃̂ kxʉ̂.n̪yˠ.n̪yˠ.ʑʷœ̌ʔ.ʑʷœ.teœ̃̂ \\
		viel \Clf:wiederkehrend Mal \Pst-\Iter\til{}versuchen-\Iter\til{}sich.sehnen-essen \\
	\trans \wdef{Viele Male versuchte er es und sehnte sich wieder und wieder, \textins{sie} zu essen.}
\end{exe}

Darüber hinaus bildet \Lhengko{} iterative Verben wie in der Kompositform
\fw{kxʉ̂.n̪yˠ.n̪yˠ.ʑʷœ̌ʔ.ʑʷœ.teœ̃̂} in \cref{ex:lheng15} durch Reduplikation,
was eine kleine Gemeinsamkeit mit Ayeri darstellt, siehe \cref{ex:ayr9} und
\cref{iter}. Interessant ist, dass beide Verben in der Kette, \fw{n̪ŷˠ}
\wdef{versuchen} und \fw{ˀʑwœ̌} \wdef{sich sehnen}, für diesen Aspekt markiert
sind. Zu \cref{ex:lheng16} ist nichts weiter anzumerken.

\begin{exe}
\ex \label{ex:lheng16}
	\gll ˀxû ɣɨ̌ʔ kxʉ̂.pʷy̟̌ \\
		Ende in \Pst-aufgeben \\
	\trans \wdef{Am Ende gab er auf.}
\end{exe}

Die Form \fw{ⁿfɛœ̌.ʐʷɔ̃̂} \wdef{Nase} in \cref{ex:lheng17} betreffend, ist im
Vergleich mit \fw{(ⁿ)χî˞} \wdef{Wein} in \cref{ex:lheng6} festzustellen, dass
erstere als Objekt von \fw{kxʉ̂.nɑːʔ} \wdef{hob} Stativmarkierung aufweist.
Auch \fw{ⁿχî˞.ʐʷɔ̃} \wdef{Wein} in \cref{ex:lheng19} zeigt explizite
Markierung im Vergleich zur Form ohne Kasussuf"|fix in \cref{ex:lheng6}.

\begin{exe}
\ex \label{ex:lheng17}
	\gll zʲaô ɻy̌ˠ.lʲœ̃ ⁿfɛœ̌.ʐʷɔ̃̂ kxʉ̂.nɑːʔ.zʲɛœ.kxʲɔ̃̂ː \\
		\Fpl.\Poss{} Fuchs-\Acv{} Nase-\Stv{} \Pst-heben.\Pfv-weg-laufen.\Pfv \\
	\trans \wdef{Unser Fuchs hob die Nase und lief davon.}
\end{exe}

Optionalität der Stativmarkierung für wenig oder weniger belebte Referenten
entsprechend der Regel für die Pluralmarkierung scheint also kein Faktor zu
sein, insbesondere auch, wenn man das Abstraktum \fw{pʲœ̌.f"|i.ʐʷɔ̃̂} \wdef{das
Erreichte} in \cref{ex:lheng14} mitbetrachtet. Möglicherweise handelt es sich
beim Fehlen der Markierung an früherer Stelle also um eine Nachlässigkeit.

Die beiden Teilsätze in \cref{ex:lheng18+19} scheinen aufeinander bezogen zu
sein, insofern \cref{ex:lheng19} scheinbar den in \cref{ex:lheng18} genannten
Grund ausführt. Dies ist allerdings einem kleinen aber folgereichen
Schreibfehler geschuldet: \label{konnte} Bei der Nachbesprechung hat sich
herausgestellt, dass bei der Übersetzung in \fw{Qunlat} als fünfter Etappe eine
Verwechslung zwischen dem Verbstamm für \wdef{sagen} und dem für
\wdef{können} passiert ist, nämlich bei der Wortform \fw{raithwarakishqu}
\wdef{deshalb konnte (er)} zu \fw{ithwa} \wdef{können, vermögen} statt
intendiertem \fw{itha} \wdef{sagen, sprechen}. Pauls Rückübersetzung lautet: \q{Daher sagte er: Wahrlich, Weintrauben sind sauer.}

\begin{exe}
\ex \label{ex:lheng18+19}
	\begin{xlist}
	\ex \label{ex:lheng18}
		\gll tʃʷiŷ n̪ʲĩ̌ kxʉ̂.χʲœ̌ \\
			Grund aus \Pst-können.\Pfv{} \\

	\ex \label{ex:lheng19}
		\gll ⁿʂǎʔ ⁿχî˞.ʐʷɔ̃ kiỹ̌.ŋɑ̌ \\
			wahrlich Wein-\Stv{} \Npst-sauer.sein.\Pfv{} \\
		\trans \textins{\wdef{Aus diesem Grund konnte er: Wahrlich, der Wein wird sauer werden.}} (Rückübersetzung aus \Lhengko{} von Henrik)
\end{xlist}
\end{exe}

Die Tempus- und Aspektmarkierung von \fw{kxʉ̂.χʲœ̌} zum Stamm \fw{ˀχʲŷ}
\wdef{können, vermögen} in \cref{ex:lheng18} deutet eine abgeschlossene
Handlung in der Vergangenheit an, die von \fw{kiỹ̌.ŋɑ̌} zum Stamm
\fw{ŋâ} \wdef{sauer sein} in \cref{ex:lheng19} dagegen in der Zukunft, da die
Kombination von Nicht-Präteritum mit Perfektivmarkierung eine futurische
Bedeutung hat. Bei der Übersetzung für den Staffelstab hatte ich aus dem
Zeitdruck geschuldeter Unachtsamkeit die Tempusmarkierung in \cref{ex:lheng19}
ignoriert, was zunächst keine allzu großen Konsequenzen hat, weil im Ayeri
Tempusmarkierung nicht obligatorisch ist, siehe \cref{ex:ayr14} und
\cref{tempus}. Ich hatte allerdings fälschlich übersetzt: \q{Vielleicht war
dies der Grund: Der Wein war wirklich sauer.} Diese Schlussfolgerung, die
die tatsächliche Säure der Trauben dafür verantwortlich macht, dass der Fuchs
die begehrten Früchte nicht erreichen kann, passt nicht in den
Textzusammenhang.

Auch bei dem Satz in \cref{ex:lheng20} zeigt die Tempus-Aspekt-Kombination das
Futur an, während ich für den Staffelstab vom Präteritum ausgegangen bin:
\q{Es ist auserzählt}. Für den Schluss einer Geschichte scheint mir dies
allerdings auch logischer. In Pauls Rückübersetzung heißt es: \q{So wird es
erzählt. Und so erzähle ich es.} Henrik hatte diesen Satz als Teil der
wörtlichen Rede des Fuchses aufgefasst. Bei der Übersetzung ins Ayeri habe ich
ihn ausgelassen.

\begin{exe}
\ex \label{ex:lheng20}
	\gll kiỹ̌.ɕɑ̂ː \\
		\Npst-erzählen.\Pfv{} \\
	\trans \wdef{\textins{Man} wird \textins{dies} erzählen.}
\end{exe}

Im Schlusssatz \cref{ex:lheng21} tritt wie zuvor in \cref{ex:lheng15} ein
Klassifizierer mit einer unbestimmten Mengenangabe auf. Die Grammatiknotizen
vermerken zu seriellen Verbkonstruktionen, dass alle Verben normalerweise
dasselbe Subjekt haben oder reziproke Handlungen vorliegen.

\begin{exe}
\ex \label{ex:lheng21}
	\gll ˀʂŷ˞ tʲœ tʲɛœ̂ kiỹ̌.ɕaɔ̌ \\
		mancher \Clf:menschlich Mensch \Npst-erzählen \\
	\trans \wdef{Manche Menschen erzählen \textins{dies}.}
\end{exe}

In \cref{ex:lheng21} fehlt bei \fw{tʲɛœ̂} wieder die Kasusmarkierung.
Abweichend vom Wortlaut des Staffelstabs habe ich für den Schluss die
Überstezung \q{So erzählt man sich unter den Menschen} gewählt, weil ich
aufgrund einer Aussage in den Grammatiknotizen nicht sicher war, ob das Verb
hier möglicherweise reziprok gebraucht sein könnte.

%%%%%%%%%%%%%%%%%%%%%%%%%%%%%%%%%%%%%%%%%%%%%%%%%%%%%%%%%%%%%%%%%%%%%%%%%%%%%%%

\section{Gegenüberstellung der Übersetzungen ins Deutsche}
\label{sec:dtuebers}

Der folgende deutschsprachige Text resultiert aus meiner Interpretation von
Henriks Text auf \Lhengko. Nachträglich sind mir noch Fehler bei der
Übersetzung ins Deutsche aufgefallen, die nicht in die Textversion im
Staffelstab eingeflossen sind.

\begin{quote}
Ein Tag hatte geendet.
Ein alter Tag hatte geendet.
Eine Nacht hatte geendet.
Eine ältere Nacht hatte geendet.
Im Sommer kam unser Fuchs in den Weinberg gelaufen.
Als er kam, schaute er auf und sah, dass hoch oben an den Reben des Weinbergs ein paar Trauben reif waren.
Unser Fuchs sprach:
\enquote{Ich habe wirklich Hunger und Durst.}

Unser Fuchs versuchte mit Kraft, nach \textins{den Trauben} zu greifen.
Er sprang mit Schwung hoch,
doch er konnte \textins{sie} nicht erreichen.
Unser Fuchs versuchte es zweimal,
mit Schwung hochzuspringen,
aber sein Versuch war erfolglos.
Viele Male versuchte er es und sehnte sich wieder und wieder, \textins{sie} zu essen.
Am Ende gab er auf.
Unser Fuchs hob die Nase und lief davon.
Vielleicht deshalb:
Der Wein war wirklich sauer.
Es ist auserzählt.
So erzählt man sich unter den Menschen.%
%
	\footnote{\emph{Vielleicht deshalb: \dots{}}] Henriks eigene Rückübersetzung:
		\q{Aus diesem Grund konnte er: \q{Wahrlich, der Wein wird
		sauer werden. \textins{Das} wird \textins{man} erzählen.} Manche Menschen
		erzählen \textins{es}.} Zur kontextuell zweifelhaften Form \fw{konnte}
		vgl.~Kommentar zu \cref{ex:lheng18+19}.}%
\end{quote}

Auf Grundlage des obigen Textes habe ich die Übersetzung im Ayeri angefertigt
und die nachstehende Rückübersetzung ins Deutsche vorgenommen.

\begin{quote}
Die Sonne hat den Mond schon zehntausend Zehntausendmale gejagt.
Ein Fuchs kam damals im Sommer regelmäßig in den Weinberg.
Er bemerkte, dass sich oben an den Reben einige reife Trauben befanden.
Und der Fuchs sprach zu sich:
\q{Ich habe wirklich Hunger und großen Durst.
Lasst uns versuchen, die saftigen Trauben zu erbeuten.}

Der Fuchs rannte und sprang, doch er kam nicht nah genug an die Trauben heran.
Er versuchte es noch einmal, kam aber nicht heran.
Er versuchte es immer und immer wieder.
Er sehnte sich so, sie zu essen.
Endlich musste er doch aufgeben.
Da hob der Fuchs die Nase und lief davon.
Vielleicht war dies der Grund:
Der Wein war wirklich sauer.
So erzählen die Menschen einander.
\end{quote}

%%%%%%%%%%%%%%%%%%%%%%%%%%%%%%%%%%%%%%%%%%%%%%%%%%%%%%%%%%%%%%%%%%%%%%%%%%%%%%%

\section{Übersetzung ins Ayeri}
\label{sec:ayruebers}

Die Übersetzung ins Ayeri basiert auf der Rückübersetzung von Henriks Text auf
\Lhengko{} ins Deutsche in \cref{sec:dtuebers}. Dabei habe ich mir an wenigen
Stellen Freiheiten bei der Adaptation erlaubt.

\begin{quote}
Ang kimbyo iri perin kolunas samanganyam samang.
Ang sahasayo adauyi runay nimpurivanya matayya.
Kengyong, ya yomayo ling nusan betayjang-aril vilay.
Da-sitang-ningyo runayang:
\q{Mabyang ancu nay tapanyang māy.
Linku-linku vitryam betayjas gali.}

Nimpyo runayang nay pucong, nārya ya sahoyyong nasay-ma betayye.
Linkayong palunganyam, sahoy\-yong nārya.
Li-linkayong ikananyam.
Ang tunyo māy konjam rey.
Rua subryong panca nārya.
Ang da-ringyo runay vinās yona nay sarayong.
Yamanreng yoming edaley:
Nimpurang prasi ancu.
Ang da-ningyan keynamye sitanyayam.
\end{quote}

\begin{quote}
\hyphenpenalty=10000
\small\Tagati
\setcounter{unbalance}{2}
\begin{multicols}{2}
ANF kiMbFyo Iri perinF kolunsF smNnFymF smNF.
ANF shsyo Adauyi runj niMpurivnFy mtjy.
keNFyoNF, y yomyo liNF nusnF betjye\_aNF/ArilF vilj.
d/sitNF/niNFyo runyNF~:
mbFyNF AMkYu nj tpnFyNF maaj.
liMku/liMku vitFrFymF betjye\_asF gli.

niMpFyo runyNF nj pukFyoNF, naarY y shojyoNF nsj/m betjye.
liMkyoNF pluNnFymF, shjyoNF naarY.
li/liMkyoNF IknnFymF.
ANF tunFyo maaj koMdFymF rej.
ru\_a subFrFyoNF pMtY naarY.
ANF d/riNFyo runj vinaasF yon nj sryoNF.
ymnFreNF yomiNF Edlej~:
nimpurNF pFrsi AMkYu.
ANF d/niNFynF kejnmFye sitnYymF.
\end{multicols}
\end{quote}

Der Text auf \Lhengko{} beginnt mit einer aus mehreren Sätzen bestehenden
rhetorischen Figur, die die Erzählzeit als Vergangenheit etabliert. Ich habe
mich entschlossen, diese Passage (die von Dominique stammt, der allerdings
diesmal nicht ins \fw{Hoan} übersetzt hat) nicht wörtlich zu übersetzen,
sondern in dem Satz in \cref{ex:ayr1} zusammenzufassen.

\begin{exe}
\ex \label{ex:ayr1}
	\gll Ang kimbyo iri perin kolunas samanganyam samang. \\
		\AgtT= jagen-\Tpl.\N{} schon Sonne[\Top] Mond-\Parg{}
		zehntausendmal zehntausend \\
	\trans \wdef{Die Sonne hat den Mond schon zehntausend Zehntausendmale
		gejagt.}
\end{exe}

Um die märchenhafte Vergangenheit der Fabel zu beschwören, habe ich mit
1\,0000\,0000\tsub{12} Nächten~-- was wörtlich 429\,981\,696\tsub{10} Nächten
entspricht~-- versucht, eine metaphorisch sehr lange Zeitdistanz anzudeuten. Das
Zahlwort \xayr{smNF}{samang}{zehntausend} ist von der Kardinalzahl
\xayr{smF}{sam}{zwei} abgeleitet. Um die multiplikative Bedeutung abzuleiten,
wird die Kardinalzahl nominalisiert (\xayr{smNnF}{samangan}{zehntausendste})
und dann in den Dativ gesetzt: \xayr{smNnFymF}{samanganyam}{zehntausendmal}.
Der Dativ markiert hier also weder einen Rezipienten noch ein Ziel
(vgl.~\cref{subsubsec:nom,numeralia}).

Dass in der Fabel immer von \fw{unserem Fuchs} die Rede ist, hatte ich als
stilistische Eigenheit des Erzählens in \Lhengko\ aufgefasst und kurzerhand in
\cref{ex:ayr2} konventioneller mit \xayr{runj}{runay}{Fuchs} ohne
Possessivpronomen \xayr{nn}{nana}{unser} übersetzt. Die Formulierung \fw{unser
Fuchs} in der Vorlage auf \Lhengko{} ist aber eigentlich dem Genussystem des
Qunlat von Paul geschuldet, das Substantive pragmatisch als für die
Sprecherinstanz zur Gruppe oder nicht zur Gruppe gehörig markiert. Henrik hatte
sich dazu entschieden, die Assoziativ-Markierung in diesem Kontext mit
\fw{zʲaô ɻy̌ˠ.lʲœ̃} \wdef{unser Fuchs} zu übersetzen.

\begin{exe}
\ex \label{ex:ayr2}
	\gll Ang sahasayo adauyi runay nimpurivanya matayya. \\
		\AgtT= kommen-\Hab-\Tsg.\N{} damals Fuchs[\Top] Weinberg-\Loc{}
		Sommer-\Loc{} \\
	\trans \wdef{Ein Fuchs kam damals im Sommer regelmäßig in den Weinberg.}
\end{exe}

Anders als im Deutschen habe ich mich dazu entschieden, den Fuchs als Neutrum
zu behandeln, weil über sein Geschlecht als Figur in der Erzählung nichts
bekannt ist und dieses letztlich auch keine Rolle spielt. Neutrale Animata
stellen im Ayeri eine Restkategorie dar, in die solche belebten Referenten
fallen, die sich weder eindeutig dem Maskulinum noch dem Femininum zuordnen
lassen.

Satz \cref{ex:lheng5} in der Vorlage greift nachfolgend auf, dass der Fuchs die
Trauben bemerkt, als er in den Weinberg gelaufen kommt. Diese relative
Redundanz habe ich in seiner Entsprechung in \cref{ex:ayr3} getilgt, zumal
Partizipialkonstruktionen wie \fw{nœ̌.ɻʷɛ̃.f"|i.lʲœ̃} \wdef{der Hinkommende}
untypisch für Ayeri sind.

\begin{exe}
\ex \label{ex:ayr3}
	\gll Kengyong, ya yomayo ling nusan betayjang-aril vilay. \\
		merken=\Tsg.\N.\Aarg{} \LocT= sich.befinden-\Tsg.\N{} oben Busch[\Top] Beere-\Pl-\Aarg=einige reif \\
	\trans \wdef{Er bemerkte, dass sich oben an den Reben einige reife
		Trauben befanden.}
\end{exe}

Ayeri besitzt eine Nullkopula, trotzdem tritt im Kontext von \cref{ex:ayr3} das
Verb \rayr{yom/}{yoma-} auf -- nicht als Kopula, sondern als Vollverb mit der
Bedeutung \wdef{da sein, sich befinden}, siehe auch \cref{ex:ayr13+14}. Darüber
hinaus sind im Ayeri Präpositionen häufig von Substantiven abgeleitet, so im
Grunde auch \xayr{liNF}{ling}{auf, oben an}, das sich auf die Oberseite von
etwas bezieht \autocite[173]{becker:ayrgram}. Was im Deutschen mit dem
Lokaladverb \fw{oben} übersetzt wird, ist daher im Ayeri eine Präposition, und
dass die Traube \fw{an} der Weinrebe hängen, wird durch die Wahl des Lokativs
als Kasus des Präpositionalobjekts ausgedrückt.

In \cref{ex:ayr4+5+6} folgt ein längerer Block von wörtlicher Rede, wobei ich
den Satz in \cref{ex:ayr6} entgegen der Vorlage zur Rede hinzugenommen habe, um
die Erzählung etwas lebendiger zu gestalten und im Erzählduktus von Tierfabeln
zu bleiben.

\begin{exe}
\ex \label{ex:ayr4+5+6}
	\begin{xlist}
	\ex \label{ex:ayr4}
		\gll Da-sitang-ningyo runayang: \\
			so=\Refl=sprechen-\Tpl.\N{} Fuchs-\Aarg{} \\
		\trans \wdef{Und der Fuchs sprach zu sich:}

	\ex \label{ex:ayr5}
		\gll Mabyang ancu nay tapanyang māy. \\
			hungrig.sein=\Fsg.\Aarg{} wirklich und durstig.sein=\Fsg.\Aarg{}
			\Ints{} \\
		\trans \wdef{Ich habe wirklich Hunger und großen Durst.}

	\ex \label{ex:ayr6}
		\gll Linku-linku vitryam betayjas gali. \\
			\Hort\til{}versuchen-\Imp{} erbeuten-\Ptcp{} Beere-\Pl-\Parg.\Inan{}
			saftig \\
		\trans \wdef{Lasst uns versuchen, die saftigen Trauben zu erbeuten.}
	\end{xlist}
\end{exe}

Die Redeeinleitung in \cref{ex:ayr4} zeigt die Verwendung des Reflexivmarkers
\rayr{sitNF/}{sitang-} als Proklitikum, das direkt an den Verbstamm tritt,
statt ein separates koindiziertes Objektpronomen zu modifizieren. Der erste
Satz der Figurenrede in \cref{ex:ayr5} macht aus dem Satz der Vorlage einen
Parallelismus. Der Fuchs beschließt seinen Plan in \cref{ex:ayr6} mit einer
Hortativform, die Reduplikation als morphologisches Bildungsmuster im Ayeri
illustriert. Der Satz enthält mit \xayr{liMkF/}{link-}{versuchen} ein
Subjektkontrollverb, dessen Imperativform Topikmarkierung blockiert, für die
die Argumente des Komplementsatzes in diesem syntaktischen Kontext andernfalls
verfügbar wären, weil die infinite Verbform \xayr{vitFrFymF}{vitryam}{zu
erbeuten} ebenfalls Topikmarkierung blockiert \autocite[vgl.][212,
375--377]{becker:ayrgram}.

Darüber hinaus ist im Ausgangstext keine Entsprechung zu \cref{ex:ayr6} zu
finden. Die Interpretation, dass \wdef{mit Kraft greifen} in der Vorlage als
\wdef{erbeuten, erhaschen} zu verstehen ist, habe ich zu verantworten.
Letztlich geht dieser Satz auf Dominique zurück, dem dritten im Kreis, dessen
Text in \fw{Orogalanne} an dieser Stelle lautet: \fw{Ibit bobakez tuššuk tuzzum
tuššišmašažarba.} \wdef{Unter Getrampel sammelte der Fuchs all seine Kraft und
sprang.} Die Interpretation, dass der Fuchs alles daransetzt, die Trauben zu
pflücken, geht auf Paul zurück, der schreibt: \fw{Peshaki esratharasherah
issasherah path qokalakarkith sakasharak sanaasit.} \wdef{Der Fuchs kam und
versuchte mit aller Kraft (ihn) zu pflücken.}

Durch die Hinzunahme des Satzes in \cref{ex:ayr6} verschiebt sich der
Absatzwechsel im Vergleich zur direkten Vorlage an dieser Stelle um einen Satz
nach hinten. Dass der Fuchs in \cref{ex:ayr5} mit den Trauben nicht nur seinen
Durst, sondern auch seinen Hunger stillen möchte, geht auf Kieráns Übersetzung
in \fw{Laajaa} (Etappe 4) zurück, in der es an dieser Stelle heißt:
% काल लतालब। ईं अनः आटिश उ जै।
\fw{Kāl latā́lab: Iṁ anaḥ ā̀ṭiś u jai.}
\wdef{Der Fuchs sagte: Wahrlich, ich bin durstig und hungrig!}


Der zweite Absatz, der mit Satz \cref{ex:ayr7} beginnt, erzählt von der
Umsetzung des Plans und dessen Scheitern aus Erzählerperspektive. Wie
\xayr{liNF}{ling}{auf, oben an} in \cref{ex:ayr3} kann auch
\xayr{nsj}{nasay}{Nähe} in \cref{ex:ayr7} als Präposition mit der Bedeutung
\wdef{in der Nähe von} verwendet werden.

\begin{exe}
\ex \label{ex:ayr7}
	\gll Nimpyo runayang nay pucong, nārya ya sahoyyong nasay-ma betayye. \\
		rennen-\Tsg.\N{} Fuchs-\Aarg{} und springen-\Tsg.\N{} aber \LocT= kommen-\Neg=\Tsg.\N.\Aarg{} in.Nähe=genug Beere-\Pl{}[\Top] \\
	\trans \wdef{Der Fuchs rannte und sprang, doch er kam nicht nah genug an die Trauben heran.}
\end{exe}

Die Bewegungsrichtung hin zu den Trauben wird durch das Verb
\xayr{sh/}{saha-}{kommen} angegeben, daher erscheint \xayr{y~—~betjye}{ya
\dots\ betayye}{an die Trauben} mit einfachem Lokativ, ohne Direktivpartikel
\rayr{mN}{manga} oder Dativmarkierung. Die Präposition wird um die adverbiale
Gradpartikel \xayr{/m}{-ma}{genug} erweitert, um die Bedeutung \wdef{nah genug
an} zu erfassen.

Satz \cref{ex:ayr8} bezieht sich auf den vorhergehenden Satz und spart dessen
Lokaladverbial als Ellipse aus, zumal dieser Teil in \cref{ex:ayr7} bereits als
Topik markiert war und daher im Zusammenhang als bekannt und mitgedacht
vorausgesetzt werden kann.

\begin{exe}
\ex \label{ex:ayr8}
	\gll Linkayong palunganyam, sahoyyong nārya. \\
		versuchen=\Tsg.\N.\Aarg{} nochmal kommen-\Neg=\Tsg.\N.\Aarg{} aber \\
	\trans \wdef{Er versuchte es noch einmal, kam aber nicht heran.}
\end{exe}

Die Verbform in \cref{ex:ayr9} zeigt wie die Form
\xayr{liMku/liMku}{linku-linku}{lasst uns versuchen} in \cref{ex:ayr6}
Reduplikation, allerdings in einem anderen grammatischen Kontext und mit
anderer Bildung. Während Hortative die komplette imperative Verbform
reduplizieren, werden zur Bildung des Iterativs nur die ersten zwei
Silbensegmente redupliziert. Die Iterativform zum Stamm
\xayr{liMk/}{linka-}{versuchen} lautet daher \xayr{li/liMk/}{li-linka-}{(immer)
wieder versuchen}.

\begin{exe}
\ex \label{ex:ayr9}
	\gll Li-linkayong ikananyam. \\
		\Iter\til{}versuchen=\Tsg.\N.\Aarg{} vielmals \\
	\trans \wdef{Er versuchte es immer und immer wieder.}
\end{exe}

Weiterhin zeigt die Form \xayr{IknnFymF}{ikananyam}{vielmals}, dass nicht nur
von Kardinalzahlen, sondern auch von Quantoren wie \xayr{IknF}{ikan}{viel,
sehr} multiplikative Formen gebildet werden können, die als Adverbien
Verwendung finden, wie hier gezeigt.

Wie zuvor zu \cref{ex:ayr6} angemerkt, stehen Kontrollverben unter bestimmten
Umständen die Argumente des Komplementsatzes zur Topikmarkierung zur Verfügung.
Der nächste Satz in \cref{ex:ayr10} markiert daher eine Agenztopik, obwohl das
Matrixprädikat \xayr{tunFyo}{tunyo}{verlangt, sehnt sich} in diesem Kontext als
nominales Argument nur ein Subjekt vorsieht. Durch Hinzunahme des Objekts
\xayr{rej}{rey}{sie} des Verbs \xayr{koMdFymF}{konjam}{zu essen} im
Komplementsatz wird das Matrixverb transitiv.

\begin{exe}
\ex \label{ex:ayr10}
	\gll Ang tunyo māy konjam rey. \\
		\AgtT= sich.sehnen=\Tsg.\N.\Top{} \Ints{} essen-\Ptcp{}
		\Tpl.\Inan.\Parg \\
	\trans \wdef{Er sehnte sich so, sie zu essen.}
\end{exe}

Im Sinne der \fw{Lexical-functional Grammar} könnte vielleicht die funktionale
Struktur in \cref{fig:ayr10_fstruct} angenommen werden, nach der das Objekt
(\leipzigfont{obj}) des Komplementsatzes (\leipzigfont{xcomp}) in $g$ ein
Objekt des Prädikators (\leipzigfont{pred}) in $f$ wird, allerdings ein
externes (Auf"|listung hinter den Spitzklammern), weil es in diesem Kontext
kein eigenes Argument des Verbs in $f$ darstellt \autocite[vgl.][304--308,
319--323]{bresnanetal2016}. Meine Annahme ist also, dass gleichzeitig
Subjektkontrolle und Objektraising vorliegen.

\begin{figure}
	\avm{
		\id{f}{[
			pred	& \wdef{sich-sehnen <(↑~subj) (↑~xcomp)> (↑~obj)}\smallskip\\
			adj		& \{
				[
					pred	& \wdef{so}
				]
			\}\smallskip\\
			%
			top	& [
				pred	& \textit{pro}\\
				pers	& \Third\\
				anim	& $+$\\
				gend	& \N\\
				num		& \Sg\\
				case	& \Aarg
			]\tikzmark{upst_top}\smallskip\\
			%
			subj	& \tikzmark{upst_subj}\\
			%
			obj		& [
				pred	& \textit{pro}\\
				pers	& \Third\\
				anim	& $-$\\
				num		& \Pl\\
				case	& \Parg
			]\tikzmark{upst_obj}\smallskip\\
			%
			xcomp	& \id{g}{[
				pred	& \wdef{essen <(↑~subj) (↑~obj)>}\\
				subj	& \tikzmark{dnst_subj}\\
				obj		& \tikzmark{dnst_obj}\\
			]}\\
		]}
	}
	\begin{tikzpicture}[remember picture, overlay]
		\draw [rounded corners=1ex, black!50] ([yshift=.5ex]{pic cs:upst_top})
			-- ++(east:10.05em) |- ([yshift=.5ex]{pic cs:upst_subj});
		%
		\draw [rounded corners=1ex, black!50] ([yshift=.5ex]{pic cs:upst_subj})
			-- ++(east:15em) |- ([yshift=.5ex]{pic cs:dnst_subj});
		%
		\draw [rounded corners=1ex, black!50] ([yshift=.5ex]{pic cs:upst_obj})
			-- ++(east:9.25em) |- ([yshift=.5ex]{pic cs:dnst_obj});
	\end{tikzpicture}
	\caption{Angenommene funktionale Struktur des Satzes \fw{Ang tunyo māy konjam
		rey} \wdef{Er sehnte sich so, sie zu essen}}
	\label{fig:ayr10_fstruct}
\end{figure}

In \cref{ex:ayr11} steht ein Modalverb \xayr{ru\_a/}{rua-}{müssen} zusammen mit
dem Hauptverb \xayr{subFrF/}{subr-}{aufgeben}. In diesem grammatischen
Zusammenhang wird das Modalverb zur Partikel, die vor dem finiten Hauptverb
steht. Daher lautet die zusammengesetzte Verbform \xayr{ru\_a subFrFyoNF}{rua
subryong}{er gab auf}. Das Adverb \xayr{naarFy}{nārya}{aber, doch} tritt
gewöhnlich als Konjunktionaladverb am Anfang eines Satzes auf. Da es hier aber
eher satzadverbialen Charakter hat und \xayr{pMtY}{panca}{endlich} dem Sinn
nach enger zum Verb gehört, habe ich mich entschieden, \rayr{naarY}{nārya} ans
Ende zu stellen.

\begin{exe}
\ex \label{ex:ayr11}
	\gll Rua subryong panca nārya. \\
		müssen= aufgeben=\Tsg.\N.\Aarg{} endlich doch \\
	\trans \wdef{Endlich musste er doch aufgeben.}
\end{exe}

Der erfolglose Fuchs gibt beleidigt auf und läuft davon. Die Partikel
\xayr{d/}{da-}{so} kann mit Verben je nach Zusammenhang eine unterschiedliche
Funktionen haben (vgl.~\cref{da}). Typisch ist im Erzählkontext aber eine
Satzverknüpfende beziehungweise präsentative Bedeutung wie in \cref{ex:ayr12}.

\begin{exe}
\ex \label{ex:ayr12}
	\gll Ang da-ringyo runay vinās yona nay sarayong. \\
		\AgtT= so=heben-\Tsg.\N{} Fuchs[\Top] Nase-\Parg{} \Tsg.\N.\Gen{} und
		gehen=\Tsg.\N.\Aarg{} \\
	\trans \wdef{Da hob der Fuchs die Nase und lief davon.}
\end{exe}

Aufgrund der Verwechslung von \wdef{sagen} und \wdef{können} ergibt
\cref{ex:ayr13+14} im Kontext der Fabel nicht viel Sinn, allerdings hatte ich
in der Kürze der Zeit auch nicht eingegriffen, um den Text wieder etwas
\q{geradezubiegen}. Die zu Grunde liegende Fabel war mir unbekannt, aber selbst
mit Textkenntnis scheint es mir in diesem Fall nicht angebracht, zum
ursprünglichen Wortlaut zurückzukehren. Henrik hatte in seiner Rückübersetzung
den Satz in \cref{ex:ayr13} zur wörtlichen Rede des Fuchses gezählt, allerdings
war dies aufgrund der Übermittlung der Vorlage als reine IPA-Transkription
nicht ersichtlich (vgl.~\cref{sec:lhenganalys}).

\begin{exe}
\ex \label{ex:ayr13+14}
	\begin{xlist}
	\ex \label{ex:ayr13}
		\gll Yamanreng yoming edaley: \\
			Grund-\Aarg.\Inan{} vielleicht dies-\Parg.\Inan{} \\
		\trans \wdef{Vielleicht war dies der Grund:}

	\ex \label{ex:ayr14}
		\gll Nimpurang prasi ancu. \\
			Wein-\Aarg{} sauer wirklich \\
		\trans \wdef{Der Wein war wirklich sauer.}
	\end{xlist}
\end{exe}

In meiner Analyse habe ich \cref{ex:ayr13,ex:ayr14} aufeinander bezogen und als
eine enigmatische Deutung des Geschehens durch den Erzähler interpretiert. Da
\Lhengko{} dritte Personen nicht am Verb markiert und unpersönliche Aussagen
nicht gesondert markiert werden, ging aus dem Text nicht genau hervor, worauf
sich \fw{kxʉ̂.χʲœ̌} \wdef{konnte} in \cref{ex:lheng18} bezieht.

In morphosyntaktischer Hinsicht zeigen beide Sätze den bereits zu
\cref{ex:ayr3} angemerkten Sachverhalt, dass Ayeri eine Nullkopula besitzt.
Dass eine prädikative Konstruktion vorliegt, wird in \cref{ex:ayr13} aus der
unterschiedlichen Kasusmarkierung der Glieder deutlich, weil das Pronomen als
Teil des Prädikats abweichend Patiensmarkierung aufweist
(vgl.~\cref{subsubsec:nom}). In \cref{ex:ayr14} mit prädikativem Adjektiv ist
die Gliederung allerdings nicht so deutlich, weil Adjektive keine Kongruenz
aufweisen.

Den Satz in \cref{ex:lheng20} hatte ich ausgelassen, ob aus Unachtsamkeit oder
um Redundanz mit \cref{ex:ayr15} zu vermeiden, kann ich im Nachhinein nicht
mehr sagen. Im Schlusssatz habe ich mich entschieden, explizit das
Reziprokpronomen \xayr{sitnY}{sitanya}{einander} in Abgrenzung zum
Reflexivpronomen \xayr{sitNF}{sitang}{sich} zu verwenden.

\begin{exe}
\ex \label{ex:ayr15}
	\gll Ang da-ningyan keynamye sitanyayam. \\
		\AgtT= so=erzählen-\Tpl.\Aarg{} Mensch-\Pl{}[\Top] einander-\Dat{} \\
	\trans \wdef{So erzählen die Menschen einander.}
\end{exe}

Insgesamt hat es wie immer Spaß gemacht, sich in einen Text in einer
unbekannten Sprache einzuarbeiten und diesen zu analysieren. \Lhengko\ hat die
Glossierung des Texts aufgrund der sehr agglutinierenden Morphologie recht
einfach gemacht, trotz der phonologischen Tücken bezüglich der regressiven
Ausbreitung von Anlautmerkmalen und der sehr komplexen Silbenstruktur.
Überraschend war, dass dieses Mal zur Übersetzung keine Wortstämme neu gebildet
oder alte in ihrer Bedeutung erweitert werden mussten.

%%%%%%%%%%%%%%%%%%%%%%%%%%%%%%%%%%%%%%%%%%%%%%%%%%%%%%%%%%%%%%%%%%%%%%%%%%%%%%%

\section{Beigegebenes Material}
\label{sec:suppl}

Um die Analyse des Texts auf Ayeri durch die nächste Person -- in diesem Fall
Bruno -- zu ermöglichen, müssen ein Glossar und Hinweise zur Grammatik als
Hilfsmittel dazugegeben werden, beides in einem Umfang, der dem Text
gerecht wird aber nicht überfordert. Dies erfordert, Kernkonzepte der Grammatik
der jeweiligen Sprache so prägnant wie möglich zusammenzufassen und auf den
Teil zu reduzieren, der zur Analyse des Texts nötig ist. Praktischerweise war
es möglich, den Großteil der Notizen zur Grammatik vom letzten Relay mit
kleineren Verbesserungen wiederzuverwenden.

\subsection{Glossar}

% \setcounter{unbalance}{2}
\begin{multicols}{2}
\raggedright
\begin{description}[nosep]
	% \item[Lemma]
	% 	\ayr{lemma}
	% 	\emph{Wortart},
	% 	Übersetzung

	\item[-aril]
		\ayr{/ArilF}
		\emph{Adv.},
		etwas, ein paar, manche
	\item[-ma]
		\ayr{/m}
		\emph{Adv.},
		genug, genügend
	\item[adauyi]
		\ayr{Adauyi}
		\emph{Pron.-Adv.},
		dann, damals
	\item[ancu]
		\ayr{AMkYu}
		\emph{Adv.},
		wirklich
	\item[betay]
		\ayr{betj}
		\emph{N., inan.},
		Beere
	\item[gali]
		\ayr{gli}
		\emph{Adj.},
		saftig
	\item[ikananyam]
		\ayr{IknnFymF}
		\emph{Adv.},
		vielfach, vielmals
	\item[iri]
		\ayr{Iri}
		\emph{Adv.},
		schon
	\item[keng-]
		\ayr{keNF/}
		\emph{Vb.},
		bemerken
	\item[keynam]
		\ayr{kejnmF}
		\emph{N., anim.},
		Mensch
	\item[kimb-]
		\ayr{kiMbF/}
		\emph{Vb.},
		jagen
	\item[kolun]
		\ayr{kolunF}
		\emph{N., anim.},
		Mond
	\item[kond-]
		\ayr{koMdF/}
		\emph{Vb.},
		essen
	\item[ling]
		\ayr{liNF}
		\emph{Präp.},
		oben (an), auf; während (parallel geschehend zu)
	\item[linka-]
		\ayr{liMk/}
		\emph{Vb.},
		versuchen
	\item[mab-]
		\ayr{mbF/}
		\emph{Vb.},
		hungern, hungrig sein
	\item[matay]
		\ayr{mtj}
		\emph{N., inan.},
		Sommer
	\item[māy]
		\ayr{maaj}
		\emph{Adv.},
		ja, doch
	\item[nasay]
		\ayr{nsj}
		\emph{Präp.},
		in der Nähe von
	\item[nay]
		\ayr{nj}
		\emph{Konj.},
		und
	\item[nimp-]
		\ayr{niMpF/}
		\emph{Vb.},
		laufen, rennen
	\item[nimpur]
		\ayr{niMpurF}
		\emph{N., anim.},
		Wein
	\item[nimpurivan]
		\ayr{niMpurivnF}
		\emph{N., inan.},
		Weinberg
	\item[ning-]
		\ayr{niNF/}
		\emph{Vb.},
		erzählen
	\item[nusan]
		\ayr{nusnF}
		\emph{N., anim.},
		Busch, Strauch
	\item[nārya]
		\ayr{naarY}
		\emph{Adv.},
		aber, doch
	\item[palunganyam]
		\ayr{pluNnFymF}
		\emph{Adv.},
		noch einmal
	\item[panca]
		\ayr{pMtY}
		\emph{Adv.},
		schließlich, endlich
	\item[perin]
		\ayr{perinF}
		\emph{N., anim.},
		Sonne
	\item[prasi]
		\ayr{pFrsi}
		\emph{Adj.},
		sauer
	\item[puk-]
		\ayr{pukF/}
		\emph{Vb.},
		springen, hüpfen
	\item[rey]
		\ayr{rej}
		\emph{Pers.-Pron.},
		es
	\item[ring-]
		\ayr{riNF/}
		\emph{Vb.},
		wachsen; heben
	\item[rua-]
		\ayr{ru\_a/}
		\emph{Vb.},
		müssen
	\item[runay]
		\ayr{runj}
		\emph{N., anim.},
		Fuchs
	\item[saha-]
		\ayr{sh/}
		\emph{Vb.},
		kommen; passieren
	\item[samang]
		\ayr{smNF}
		\emph{Num.},
		zehntausend
	\item[samanganyam]
		\ayr{smNnFymF}
		\emph{Adv.},
		zehntausendmal
	\item[sara-]
		\ayr{sr/}
		\emph{Vb.},
		gehen, verlassen; aufhören
	\item[sitanya]
		\ayr{sitnFy}
		\emph{Indef.-Pron.},
		einander
	\item[subr-]
		\ayr{subFrF/}
		\emph{Vb.},
		aufgeben, einbüßen
	\item[tapan-]
		\ayr{tpnF/}
		\emph{Vb.},
		dürsten, durstig sein
	\item[tun-]
		\ayr{tunF/}
		\emph{Vb.},
		wünschen, begehren
	\item[vilay]
		\ayr{vilj}
		\emph{Adj.},
		reif
	\item[vina]
		\ayr{vin}
		\emph{N., anim.},
		Nase
	\item[vitr-]
		\ayr{vitFrF/}
		\emph{Vb.},
		ergreifen, (ein)fangen
	\item[yaman]
		\ayr{ymnF}
		\emph{N., inan.},
		Grund, Anlass, Ursache
	\item[yoma-]
		\ayr{yom/}
		\emph{Vb.},
		da sein, vorhanden sein, sich befinden
	\item[yoming]
		\ayr{yomiNF}
		\emph{Adv.},
		vielleicht
	\item[yona]
		\ayr{yon}
		\emph{Pers.-Pron.},
		sein
\end{description}
\end{multicols}

\subsection{Notizen zur Grammatik}
\label{subsec:gramnot}

\subsubsection{Allophonie}

Bei den Konsonantenphonemen löst /j/ nach /t k/ und /d ɡ/ allophonisch
Palatalisierung zu [t͡ʃ] und [d͡ʒ] aus, die in der Romanisierung mit ⟨c⟩ und
⟨j⟩ wiedergegeben werden.
Zwei adjazente Vokale der gleichen Qualität produzieren einen Langvokal, also
zum Beispiel /a/~+ /a/~>~/aː/ ⟨ā⟩, mit Ausnahme der verbalen Aspekt- und
Modussuf"|fixe, die einen vorangehenden Vokal typischerweise tilgen.
% ; /uː/ ⟨ū⟩ existiert aber nur in wenigen Lexemen, zum Beispiel
% \xayr{bbuu}{babū}{barbarisch}.

\subsubsection{Syntax}

Ayeri (\mbox{\ayr{Ayeri}}) verwendet Verberststellung (VSO) als unmarkierte
Konstituentenfolge. Da die Sprache eine Variante des VO-Typus darstellt, folgen
Modifikatoren ihren Köpfen in der Regel. Dies bedeutet, dass Adjektive und
Relativsätze ihrem Nomen folgen; genauso folgen Possessoren auch dem Possessum.

% Darüber hinaus ist Ayeri im Grunde eine Akkusativsprache (S~=~A~≠~O).
% \q{Echte} Passivsubjekte behalten allerdings ihre Patiensmarkierung, während
% das Agensargument dann fehlt. In diesen Fällen von Ergativität zu sprechen,
% würde die Beschreibung nur unnötig verkomplizieren. Obwohl Belebtheit sogar
% eine Flexionskategorie in der Sprache darstellt, bleibt diese Unterscheidung
% syntaktisch ungenutzt. Demotion der Agens zu einem obliquen Argument gibt es
% aufgrund der semantischen Kasusmarkierung nicht. Es ist aber möglich, ein
% \q{unechtes} Passiv zu bilden, bei dem das Patiensargument logisch die
% Topik bildet aber das Verb weiterhin mit dem Agensargument als syntaktischem
% Subjekt kongruiert.
% Auch bei kausativen Sätzen bildet der Auslöser, als solcher gesondert
% markiert, logisch die Topik, wird aber ebenfalls nicht zum syntaktischen
% Subjekt. Die anderen Argumente des Verbs werden entsprechend auch in diesem
% Fall nicht herabgestuft.

% In ditransitiven Sätzen wird der Donor als Agens markiert
% (S~=~A~=~D), das Thema als Patiens (O~=~T). Der Rezipient
% (R) erhält Dativmarkierung. Prädikative NPs werden
% abweichend als Patiens markiert, um Subjekt (Agens) und Prädikat
% (\q{Patiens}) zu unterscheiden, da Ayeri keine overte Kopula besitzt und
% doppelte Kernrollenmarkierung im gleichen Satz vermeidet.

Neben regulären Verbalsätzen gibt es auch Kopulasätze, allerdings besitzt Ayeri
eine Null-Kopula. Eine Besonderheit ist, dass das Prädikatsnomen in diesem Fall
als Patiens markiert wird, obwohl es auf das Subjekt (mit Agensmarkierung)
bezogen ist. Das Prädikat kann zum Zweck der Betonung an die Spitze des Satzes
gestellt werden.

% Ayeri macht keinen Unterschied zwischen restriktiven und nicht-restriktiven
% Relativsätzen. Relativsätze brauchen allerdings immer ein Antezedens, freie
% Relativsätze sind also nicht erlaubt. Relativsätze sind im Grunde eigenständige
% Sätze, insofern die Relativpartikel \rayr{si}{si} die Funktion einer
% Subjunktion hat, die ein komplexes Attribut an eine NP bindet oder mit
% deren Hilfe Attribute in ihrem Bezug desambiguiert werden können. Relativsätze
% haben daher normalerweise einen internen Kopf. Wenn ein Relativsatz einen
% Kopulasatz enthält, kann dessen Subjekt ausfallen.

% Komplemente von NPs werden zur Vermeidung von Ambiguität in der
% Modifikationsrelation rechtsversetzt, wenn die NP ein Adjunkt enthält,
% das das Kopfnomen modifiziert.

\subsubsection{Morphosyntax}
\label{subsubsec:morphsyn}

Die Topik wird durch ein Proklitikum am Verb markiert, das im Grunde der
Kasusendung der Topik-NP entspricht, während die Topik-NP
selbst nullmarkiert ist. Es handelt sich bei Ayeri also um eine sogenannte
\fw{trigger conlang}. Es bestehen nahezu keine Restriktionen für die Wahl der
Topik-NP. Pronomen können in gleicher Weise topikalisiert werden.
Topikmarkierung ist obligatorisch in transitiven Sätzen, während intransitive
Sätze normalerweise keine Topik markieren. Auch imperative Verben tragen
normalerweise keine Topikmarkierung.

% Die Relativpartikel \rayr{si}{si} zeigt optional Kasuskongruenz mit der
% NP, die der Relativsatz modifiziert. Dies geschieht vor allem dann,
% wenn der Relativsatz rechtsversetzt ist.

Neben den verschiedenen Pronomenarten ist die einzige Kongruenz zeigende
Wortart das Verb. Grundsätzlich kongruieren Verben mit dem Agensargument, es
sei denn, es fehlt durch
% echte
Passivierung. Ersatzweise kongruiert das Verb dann mit dem Patiensargument als
syntaktischem Subjekt.

\subsubsection{Morphologie}

Ayeri ist eine agglutinierende Sprache und dabei sehr regelmäßig. Entsprechend
dem VO-Typus werden hauptsächlich Suf"|fixe zur Flexion benutzt.
Darüber hinaus besitzt die Sprache etliche Klitika, die sich insbesondere bei
finiten Verben in einem Klitikcluster vor dem Verb zeigen.

\subsubsubsection{Nomen}
\label{subsubsec:nom}

Ayeri hat ein zweistufiges Genussystem: Nomen können entweder belebt (\Anim)
oder unbelebt (\Inan) sein. Zu den belebten Nomen zählen zum Beispiel lebende
Personen und Tiere, Personifizierungen, Gefühle und mentale Prozesse sowie
Dinge, die Anzeichen von Leben zeigen (z.\,B.~Pflanzen) oder die eng mit
Menschen assoziiert sind (z.\,B.~Wohnungen). Menschen sowie Haus- und Nutztiere
können entsprechend ihrem sozialen respektive ihrem biologischen Geschlecht
maskulin (\M) oder feminin (\F) sein. Als belebt klassifizierte Dinge und
Abstrakta sind dagegen neutral (\N). Genus ist dem Lexikon inhärent und kovert,
darum gibt das Glossar es als Hilfsstellung explizit an. Es gibt keine
Markierung von Definit- und Indefinitheit, doch existiert ein optionales Präfix,
das Unspezifizität anzeigt (\xayr{me/}{mə-}{irgendein}), im Text aber nicht
vorkommt.

Nomen flektieren in der Regel nach Numerus und Kasus, können in bestimmten
Kontexten aber auch ohne overte Kasusflexion auftreten. Der Singular ist
unmarkiert, der Plural wird mit dem Suf"|fix \rayr{/ye}{-ye} gekennzeichnet.
Dieses Suf"|fix hat ein Allomorph \rayr{/ye}{-j} (in der eigenen Schrift nicht
graphisch differenziert), das erscheint, wenn das darauf"|folgende Suf"|fix mit
Vokal oder /j/ beginnt, beispielsweise
\rayr{/ye}{-ye}~+~\rayr{/AsF}{-as}~>~\rayr{/ye\_asF}{-jas}.

Ayeri unterscheidet sieben Kasus: Agens (\Aarg), Patiens (\Parg), Dativ (\Dat),
Genitiv (\Gen), Lokativ (\Loc), Kausativ (\Caus) und Instrumentalis (\Ins),
siehe~\cref{tab:decl}. Die Vokale in Klammern in der Tabelle fallen weg, wenn
der Stamm auf einen Vokal endet, was also auch dann der Fall ist, wenn an die
Wurzel ein Pluralsuf"|fix angehängt ist.

\begin{table}
\caption{Kasusmarkierung der Nomen}
\begin{tabularx}{\linewidth}{l l l c c X}
\toprule
Kasus
	& \multicolumn{2}{c}{Suf"|fixform}
	& \multicolumn{2}{c}{proklitische Form}
	& Funktion
	\\

\cmidrule(lr){2-3}
\cmidrule(lr){4-5}

%
	& \multicolumn{1}{c}{\Anim}
	& \multicolumn{1}{c}{\Inan}
	& \multicolumn{1}{c}{\Anim}
	& \multicolumn{1}{c}{\Inan}
	\\

\midrule

\Aarg
	& -ang
	& -reng
	& ang
	& eng
	& prototypische Agens (Agens, Experiencer, Force); transitive und intransitive Subjekte im Aktiv; Subjekt des \q{unechten} Passivs; Subjekt in Kopulasätzen
	\\

\Parg
	& -as
	& -ley
	& sa
	& le
	& prototypische Patiens (Patiens, Thema); transitive und intransitive Objekte im Aktiv, direktes Objekt; Subjekt des \q{echten} Passivs; Prädikatsnomen in Kopulasätzen
	\\

\midrule

\Dat
	& \multicolumn{2}{c}{-yam}
	& \multicolumn{2}{c}{yam}
	& Rezipient; Ziel, Richtung; indirektes Objekt; sekundäres Prädikatsnomen
	\\

\Gen
	& \multicolumn{2}{c}{-(e)na}
	& \multicolumn{2}{c}{na}
	& Possessor, Quelle; worüber etwas geht bzw. wovon etwas handelt
	\\

\Loc
	& \multicolumn{2}{c}{-ya}
	& \multicolumn{2}{c}{ya}
	& Ort; typisch assoziiertes Ziel von Bewegungsverben
	\\

\Caus
	& \multicolumn{2}{c}{-isa}
	& \multicolumn{2}{c}{sā}
	& Verursacher (nur adverbiale Verwendung)
	\\

\Ins
	& \multicolumn{2}{c}{-(e)ri}
	& \multicolumn{2}{c}{ri}
	& Instrument, Helfer; Komplement einer NP
	\\

\bottomrule
\end{tabularx}
\label{tab:decl}
\end{table}

Topikalisierte NPs sind nullmarkiert, stattdessen wird der
entsprechende Kasus mit der in \cref{tab:decl} angegebenen klitischen Form
links vom Verb markiert. Eigennamen verwenden ebenfalls die klitische Form bei
der Kasusmarkierung, zum Beispiel \xayr{n bliinF}{na Balīn}{von Berlin}.

Der Diminutiv von Nomen wird durch vollständige Reduplikation angezeigt. Bei
Komposita wird nur das Kopfnomen redupliziert und flektiert. Komposita sind in
der Regel univerbiert, sodass grammatische Endungen an das letzte Element
angehängt werden. Daneben gibt es losere Verbindungen von Nomen, bei denen
ebenfalls nur das Kopfnomen flektiert wird und das modifizierende Nomen als
Attribut folgt.

\subsubsubsection{Pronomen}

Ayeri besitzt durch die Menge an Kasus und Genera eine Fülle von (ziemlich
regelmäßig gebildeten) Personalpronomen, wobei für den Kontext des vorliegenden
Textes nur ein Teil derjenigen in \cref{tab:persproagr} relevant ist, die
ihrerseits nur einen Ausschnitt darstellt. Für dritte Personen werden auch
häufig Demonstrativpronomen verwendet%
% , allerdings kommt dieser Fall im Text nicht vor
. Indefinitpronomen sind im Glossar aufgeführt, sofern sie im Text vorkommen.

\begin{table}
\caption{Personalpronomen und Personenendungen der Verben (relevanter Ausschnitt)}
\begin{tabularx}{\linewidth}{
	l l
	C C
	C C
	C C
	% c c
	C C
}
\toprule
%
	& %
	& \multicolumn{2}{c}{\makecell[tc]{Kongruenz-/\\Topikform}}
	& \multicolumn{2}{c}{\Aarg}
	& \multicolumn{2}{c}{\Parg}
	% & \multicolumn{2}{c}{\Dat}
	& \multicolumn{2}{c}{\Gen}
	\\

\cmidrule(lr){3-4}
\cmidrule(lr){5-6}
\cmidrule(lr){7-8}
\cmidrule(lr){9-10}
% \cmidrule(lr){11-12}

%
	& %
	& \multicolumn{1}{c}{\Sg}
	& \multicolumn{1}{c}{\Pl}
	& \multicolumn{1}{c}{\Sg}
	& \multicolumn{1}{c}{\Pl}
	& \multicolumn{1}{c}{\Sg}
	& \multicolumn{1}{c}{\Pl}
	% & \multicolumn{1}{c}{\Sg}
	% & \multicolumn{1}{c}{\Pl}
	& \multicolumn{1}{c}{\Sg}
	& \multicolumn{1}{c}{\Pl}
	\\

\midrule

\First
	& %
	& ay
	& ayn
	& yang
	& nang
	& yas
	& nas
	% & yām
	% & nyam
	& nā
	& nana
	\\

\Second
	& %
	& va
	& va
	& vāng
	& vāng
	& vās
	& vās
	% & vayam
	% & vayam
	& vana
	& vana
	\\

\Third
	& \M
	& ya
	& yan
	& yāng
	& tang
	& yās
	& tas
	% & yayam
	% & cam
	& yana
	& tan
	\\

%
	& \F
	& ye
	& yen
	& yeng
	& teng
	& yes
	& tes
	% & yeyam
	% & teyam
	& yena
	& ten
	\\

%
	& \N
	& yo
	& yon
	& yong
	& tong
	& yos
	& tos
	% & yoyam
	% & toyam
	& yona
	& ton
	\\

%
	& \Inan
	& ara
	& aran
	& reng
	& teng
	& rey
	& tey
	% & rayam
	% & racam
	& ran
	& ten
	\\

\bottomrule
\end{tabularx}
\label{tab:persproagr}
\end{table}

Demonstrativpronomen werden mit \rayr{d/}{da-} (indefinit), \rayr{Ed/}{eda-}
(proximal) und \rayr{Ad/}{ada-} (distal) gebildet. Gerade beim belebten
Agens- und Patiens-Demonstrativum tritt daran das Element \rayr{/nY}{-nya}
(z.\,B.~\xayr{AdnYaaNF}{adanyāng}{jener, der da};
vgl.~\xayr{nYaanF}{nyān}{Person}), in jedem Fall folgt am Schluss die
Kasusendung, die dieselbe wie bei der Deklination der Nomina ist
(\cref{tab:decl}).

% In \cref{subsubsec:morphsyn} wurde erklärt, dass Relativpartikeln keine
% Pronomen im engen Sinn darstellen, allerdings können sie durch sekundäre
% Kasusmarkierung pronominalisiert werden. Das Relativ\-pronomen trägt dann eine
% zweite Kasusendung, die seine grammatische Funktion als Konstituente innerhalb
% des Relativsatzes markiert. Wenn die Relativpartikel keine primäre
% Kasuskongruenz aufweist (z.\,B. \rayr{sin}{sina} mit Bezug auf eine
% Genitiv-NP) und so die sekundäre Endung an das einfache \rayr{si}{si}
% tritt, wird der Vokal der sekundären Endung zur Desambiguierung gedehnt, zum
% Beispiel \xayr{sinaa}{sinā}{von welchem}. Sekundär markierte Relativa können
% jedoch innerhalb des Relativsatzes nicht selbst als Topiken fungieren, insofern
% sie ihre Kasusmarkierung nicht ans Verb abgeben können.

\subsubsubsection{Verben}

Verben kongruieren nach Person (\First, \Second, \Third) und Numerus (\Sg,
\Pl) ihres Subjekts, siehe \cref{tab:persproagr}. Bei dritten Personen kommen
noch Genus und Belebtheit (\M, \F, \N, \Inan) als Flexionskategorien hinzu. Bei
pronominalen Subjekten ersetzt das Personalpronomen das Kongruenzsuf"|fix am
Verb, indem es als Enklitikum ans Ende des Verbstamms tritt. Die
Personenendungen der regulären Kongruenz mit dem Subjekt und die
topikalisierten pronominalen Klitika sind homophon, zum Beispiel korrespondiert
die Vollform \xayr{/yaaNF}{-yāng}{er} mit der topikalisierten Form
\rayr{ANF—/y}{ang \dots\ -ya}. \rayr{/y}{-ya} ist gleichzeitig auch die
Kongruenzendung für den Bezug auf eine Subjekt-NP im Singular
Maskulinum.

Finite Verben weisen darüber hinaus optional Flexion für Tempus\label{tempus}
auf, ansonsten für Aspekt und Modus. Dafür werden verschiedene
Markierungsstrategien verwendet. Im Rahmen des Texts sind habitualer und
iterativer Aspekt sowie der Imperativ als Modus relevant. Der Imperativ der
zweiten Person wird mit der Quasi-Personenendung \rayr{/U}{-u} markiert, die
einen vorhergehenden Vokal tilgt, bei Hortativen wird die Verbform zusätzlich
redupliziert. Habitualer Aspekt wird mit der Endung \rayr{/As}{-asa} markiert,
die an den Verbstamm tritt und ebenfalls einen vorhergehenden Vokal tilgt.
Aspekt kann darüber hinaus durch Adverbien ausgedrückt werden, zum Beispiel
\xayr{myis}{mayisa}{fertig sein}, das die Abgeschlossenheit einer Handlung
betont.%
%
	\footnote{Ich hatte vergessen zu erwähnen, dass Verben mit \rayr{/Oj}{-oy}
	negiert werden. Diese Information habe ich auf Anfrage per Messenger
	nachgeliefert.}

Iterativer Aspekt drückt aus, dass eine Handlung mehrfach geschieht, kann aber
auch reversive Bedeutung haben, zum Beispiel
\xayr{t/tpYnNF}{ta-tapyanang}{wir legen immer wieder} oder \wdef{wir legen
wieder zurück}. Wie das Beispiel zeigt, wird iterativer Aspekt\label{iter}
durch Reduplikation der ersten beiden Silbensegmente des Verbstamms angezeigt.

Modalität wird in der Regel durch Modalpartikeln ausgedrückt, die im
präverbalen Klitikcluster nach dem Topikmarker stehen. Diese haben
typischerweise die Form von unflektierten Verbstämmen, zum Beispiel
korrespondiert \xayr{miNF/}{ming-}{können} mit der Partikel \rayr{miNF}{ming}
und \xayr{mY/}{mya-}{sollen} mit der Partikel \rayr{mY}{mya}.

\label{da}
Bei \xayr{d/}{da-}{so} handelt es sich um eine Partikel, die zum einen
pronominal verwendet werden kann, zum Beispiel \xayr{d/kilyNF}{da-kilayang}{ich
darf das} oder \xayr{d/IMtYyeNF}{da-incyeng}{sie kauft eins}. Zum anderen kann
sie auch präsentative Funktion haben, beispielsweise in
\xayr{d/shyaaNF}{da-sahayāng}{da kommt er}.

Eine weitere Partikel stellt \rayr{sitNF}{sitang-} dar, das anstelle eines
vollständigen Reflexivpronomens auftreten kann.
\xayr{sitNF/ketFtNF}{sitang-kettang}{sie waschen sich} ist also äquivalent zu
\rayr{ANF ketFynF sitNF/tsF}{ang kecan sitang-tas}.

Wenn ein Verb ein verbales Komplement besitzt, zum Beispiel bei Kontroll- und
Raisingverben, weist das abhängige Verb eine im Prinzip infinite Form auf, die
mit \rayr{/ymF}{-yam} gekennzeichnet und als \q{Partizip} bezeichnet wird. Mit
\rayr{/AnF}{-an} nominalisiert kann diese Form als Gerundium verwendet werden.
Infinite Verben dieser Art können trotzdem Modus- und Aspektmarkierung
aufweisen.

\subsubsubsection{Adjektive, Adverbien \& Co.}

Adjektive weisen keine Kongruenz auf, können aber negiert und gesteigert
werden, genauso wie auch Adverbien. Sie stehen immer direkt hinter ihrem Bezug.

Neben Adjektiven im engeren Sinn besitzt Ayeri eine Reihe von Quantoren, die in
der Regel an die flektierte Form des Nomens (determinierende Quantoren), Verbs,
ein Adjektiv oder eine Präposition (adverbiale Quantoren) angehängt werden.
% Der Text enthält mehrere solcher Partikeln, zum Beispiel
% \xayr{/kj}{-kay}{wenig, etwas, ein bisschen}.

\label{numeralia}
Numeralia sind duodezimal. Größere Potenzen werden mit dem Derivationssuf"|fix
\rayr{/nNF}{-nang} gebildet: \xayr{menNF}{menang}{100} (zu
\xayr{menF}{men}{eins}), \xayr{smNF}{samang}{1\,00\,00} (zu
\xayr{smF}{sam}{zwei}), \xayr{kjnNF}{kaynang}{1\,00\,00\,00\,00}, etc. Diese
Einheitswörter fungieren als Köpfe, die von Numeralia attribuiert werden, zum
Beispiel \xayr{menNF yo}{menang yo}{400} (zu \xayr{yo}{yo}{vier}).
Ordinalzahlen werden durch Nominalisierung der Kardinalzahlen gebildet, also
zum Beispiel \xayr{tmnF koMkYnFyen}{iran koncanyena}{der fünfte Monat} (zu
\xayr{Iri}{iri}{fünf}). Multiplikativzahlen verwenden davon die Dativform, also
zum Beispiel \xayr{miynFymF}{miyanyam}{sechsmal} (zu \xayr{miye}{miye}{sechs}).
Distributivzahlen verwenden stattdessen den Instrumental, zum Beispiel
\xayr{Itneri}{itaneri}{zu je sieben} (zu \xayr{Ito}{ito}{sieben}),
allerdings kommt dieser Fall im Text nicht vor. Ordinal-, Multiplikativ- und
Distributivzahlen können prinzipiell genauso wie Ordinalzahlen von anderen
Numeralia attribuiert werden, und zwar in ihrer ordinalen Form.

\subsubsubsection{Präpositionen}

Freie Dative und Genitive können eine Bewegung zu etwas hin beziehungsweise von
etwas her kennzeichnen (vgl.~\cref{subsubsec:nom}). Freie Lokative kennzeichnen
eine Position, vor allem eine, die prototypisch mit dem Verb im Satz assoziiert
wird. Dies kommt insbesondere bei Positions- und Bewegungsverben zum Tragen.

Ayeri verwendet darüber hinaus in der Regel Präpositionen, die größtenteils von
Nomen abgeleitet sind. Daneben gibt es eine Reihe von Postpositionen, von denen
die meisten jüngere, sekundäre Bildungen etwa aus Adverbialen darstellen. Das
Präpositionalobjekt steht in der Regel im Lokativ. Steht es im Dativ,
kennzeichnet dieser bei manchen Präpositionen eine Bewegung in Richtung des
Objekts statt eines Ruhens an dem Ort, den das Objekt bezeichnet.

%% BIBLIOGRAPHY %%%%%%%%%%%%%%%%%%%%%%%%%%%%%%%%%%%%%%%%%%%%%%%%%%%%%%%%%%%%%%%

% \vfill
% \pagebreak

\begingroup\multicolsep=0pt
\printglossary[
	style=threecolumn,
	type=leipzig,
	title={Abkürzungen der Glossierung},
]
\endgroup

% % \nocite{*} % returns all entries from the bibliography database
\printbibliography[heading=bibintoc]

\end{document}
