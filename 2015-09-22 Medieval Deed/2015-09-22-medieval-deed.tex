\documentclass[12pt,paper=a4]{scrartcl}

% Author, Title, Subtitle etc.
\author{Carsten Becker}
\title{Fences and Gardens}
\subtitle{An Ayeri Translation of a Medieval Neighborhood Dispute}

% Handle language and quotation marks
\usepackage[ngerman,english]{babel}
\usepackage{csquotes} % Put quotations in \enquote{}!
\SetBlockEnvironment{quotation}
\renewcommand*{\mkccitation}[1]{ (#1)}

% Set all margins to 2.54 cm
\usepackage[margin=1in]{geometry}
\widowpenalty10000 % Avoid widows like the plague!
\clubpenalty10000 % Avoid orphans like the plage, too!

% Make multiple columns available in single-column document
\usepackage{multicol}

% Make text colors and color names available
\usepackage[xetex]{xcolor}

% Load font stuff for XeTeX
\usepackage{xltxtra}
\usepackage{fontspec}

% Set main fonts
\usepackage[config=mt-Junicode]{microtype}

\newfontfamily{\Tagati}[
    Renderer=Graphite,
    Scale=1.1,
    BoldFont={* Italic},
    HyphenChar=¶,
]{Tagati Book G}

\setmainfont[
    Ligatures=TeX,
    Numbers=Lowercase,
]{Junicode}

\setsansfont[
    Ligatures=TeX,
    Numbers=Lowercase,
    Scale=MatchUppercase,
    BoldFont={Open Sans Condensed Bold},
]{Open Sans Condensed Light}

% Load BibLaTeX (using Biber), configure citation styles
% Uses <https://github.com/semprag/biblatex-sp-unified>
\usepackage[
    style=biblatex-sp-unified,
    citestyle=sp-authoryear-comp,
    maxnames=2,
    backend=biber,
    safeinputenc,
    hyperref,
]{biblatex}

% To make \textcite look like "Doe (2014: 213)"
\renewcommand*{\postnotedelim}{\addcolon\addspace}
\DeclareFieldFormat{postnote}{#1}
\DeclareFieldFormat{multipostnote}{#1}

% Enable generating files from this .tex file (for the bibliography) 
\usepackage{filecontents}

% Date etc.
\usepackage{datetime}
\renewcommand{\dateseparator}{/}

% Clickable links in footnotes, TOC, etc.
\usepackage[
    xetex,
    bookmarks=true,
    colorlinks=false,
    linktoc=section,
    hidelinks,
    pdfusetitle,
]{hyperref}

\usepackage{url}
\urlstyle{rm}

% Ability to include graphics and dealing with footnotes in descriptions
\usepackage{graphicx}
\usepackage[font={small,sf},labelfont={small,sf},format=plain]{caption}
\usepackage{subcaption}
\usepackage{wrapfig}
\setlength{\columnsep}{2\baselineskip}

% General headers and footers
\usepackage{fancyhdr}
\pagestyle{fancy}

\fancyhead[L]{} % empty
\fancyhead[C]{} % empty
\fancyhead[R]{\thepage}

\fancyfoot[L]{} % empty
\fancyfoot[C]{} % empty
\fancyfoot[R]{} % empty

\renewcommand{\headrulewidth}{0pt}
\renewcommand{\footrulewidth}{0pt}

% First page headers and footers are different
\fancypagestyle{firstpage}{
    \fancyhead[L]{\sffamily \footnotesize \textbf{Benung. The Ayeri Language Resource}}
    \fancyhead[C]{} % empty
    \fancyhead[R]{\sffamily \footnotesize Carsten Becker · \yyyymmdddate\today}
    
    \fancyfoot[L]{} % empty
    \fancyfoot[C]{} % empty
    \fancyfoot[R]{\sffamily \footnotesize 
	\href{http://benung.nfshost.com}{http://benung.nfshost.com} · 
	\href{https://github.com/carbeck/benung-pdfs}{https://github.com/carbeck/benung-pdfs} · 
	\href{https://creativecommons.org/licenses/by-sa/4.0/}{CC~BY-SA~4.0}
    }
    
    \renewcommand{\headrulewidth}{0.5pt}
    \setlength\footskip{0.5in}
}

\usepackage{ifthen}
\ifthenelse{\value{page}=1}{\thispagestyle{firstpage}{\pagestyle{fancy}}}

% Line spacing
\usepackage{setspace}
\onehalfspacing

% Avoid pagebreaks right after sections and subsections
\usepackage{needspace}
\usepackage{etoolbox}
\preto\section{\needspace{6\baselineskip}}
\preto\subsection{\needspace{6\baselineskip}}

% Things for tables
\usepackage{longtable}
\usepackage{tabu}
\usepackage{booktabs}
% \extrarowsep=2pt
% \usepackage{rotating}

% Formatting of glosses
\usepackage{expex}
\usepackage{leipzig}

\newleipzig{AgtT}{at}{agent topic}
\newleipzig{PatT}{pt}{patient topic}
\newleipzig{DatT}{datt}{dative topic}
\newleipzig{GenT}{gent}{genitive topic}
\newleipzig{LocT}{loct}{locative topic}
\newleipzig{InsT}{inst}{instrumentative topic}
\newleipzig{CauT}{caut}{causative topic}
\newleipzig{An}{an}{animate}
\newleipzig{Inan}{inan}{inanimate}

\newleipzig{Np}{np}{nominal phrase}
\newleipzig{Pp}{pp}{prepositional phrase}

\newleipzig{PGmc}{pgmc}{Proto Germanic}
\newleipzig{Gmc}{gmc}{Germanic}
\newleipzig{Os}{os}{Old Saxon}
\newleipzig{Ohg}{ohg}{Old High German}
\newleipzig{Mhg}{mhg}{Middle High German}
\newleipzig{Nhg}{nhg}{New High German}
\newleipzig{Gk}{gk}{Greek}

% Nicer footnotes
\usepackage[bottom]{footmisc}
\deffootnote{0em}{1.5em}{\textsuperscript\thefootnotemark\enskip}
\renewcommand{\footnoterule}{\rule{0pt}{0pt}{\vspace*{-0pt}}}
\setlength{\footnotesep}{1em}

% Smaller font in block quotes
\usepackage{relsize}
\AtBeginEnvironment{quote}{\noindent\smaller}
\AtBeginEnvironment{quotation}{\smaller}

% Macros
\newcommand{\fw}[1]{\textit{#1}} % Foreign Word
\newcommand{\tit}[1]{\textit{#1}} % Title of a work
\newcommand{\q}[1]{\enquote{#1}} % Context-aware quotation
\newcommand{\qq}[1]{\enquote*{#1}} % Explicit sublevel quotation
\newcommand{\tsup}[1]{\textsuperscript{#1}} % Superscript
\newcommand{\markyellow}[1]{\colorbox{yellow}{#1}} % Yellow highlighter
\newcommand{\ques}{\fakesuperscript{?}} % raised question mark
\newcommand{\norm}[1]{⸢\textit{#1}⸣} % Normalized spelling

\newcommand{\ayr}[1]{{\Tagati #1}}
\newcommand{\xayr}[3]{{\Tagati #1} \emph{#2} \enquote*{#3}}

\newenvironment{ayeri}{
    %\hyphenpenalty=10000
    %\hbadness=10000
    \doublespacing
    %\begin{multicols}{2}
    \Tagati
}{
    %\end{multicols} \par
}

\newenvironment{mytitle}{
    \hfill
    \begin{minipage}{0.667\textwidth}
	\vspace{\baselineskip}
	\begin{center}
	    \Large
	    \sffamily\bfseries
	    \makeatletter
}{
	    \makeatother
	\end{center}
	\vspace{1em}
    \end{minipage}
    \hfill
}

%% BIBLIOGRAPHY DATABASE %%%%%%%%%%%%%%%%%%%%%%%%%%%%%%%%%%%%%%%%%%%%%%%%%%%%%%%

\begin{filecontents*}{\jobname.bib}
@mvcollection{CAO,
	editor = {Friedrich Wilhelm and Richard Newald and Helmut de Boor and Diether Haacke and Bettina Kirschstein},
	gender = {pp},
	title = {{Corpus der altdeutschen Originalurkunden bis zum Jahr 1300}},
	shorttitle = {CAO},
	shorthand = {CAO},
	% sortkey = {CAO0},
	volumes = {5},
	publisher = {Moritz Schauenburg and Erich Schmidt},
	location = {Lahr/Schwarzwald and Berlin},
	year = {1932--2004},
}

@collection{CAO1,
	title = {1200--1282},
	shorttitle = {CAO I},
	shorthand = {CAO I},
	volume = {1},
	% sortkey = {CAO1},
	editor = {Friedrich Wilhelm},
	location = {Lahr (Baden)},
	publisher = {Moritz Schauenburg},
	year = {1932},
	crossref = {CAO},
}

@collection{CAO5,
	booktitle = {Nachtragsurkunden 1261--1297},
	bookshorttitle = {CAO V},
	shorthand = {CAO V},
	volume = {5},
	% sortkey = {CAO5},
	editor = {Helmut de Boor and Diether Haacke and Bettina Kirschstein},
	% titleaddon = {Begründet von Friedrich Wilhelm. Fortgeführt von Richard Newald},
	founder = {Friedrich Wilhelm},
	continuator = {Richard Newald},
	gender = {pp},
	location = {Berlin},
	publisher = {Erich Schmidt},
	year = {2004},
	crossref = {CAO},
}

@incollection{n163,
	title = {N~163 (381~a)},
	origdate = {1279-03-19},
	origyear = {1279},
	pages = {127},
	crossref = {CAO5},
}

@incollection{n163-online,
	title = {N~163 (381~a) 1279 März 19},
	editor = {Kurt Gärtner and Andrea Rapp and Andreas Gniffke},
	url = {http://tcdh01.uni-trier.de/cgi-bin/iCorpus/CorpusIndex.tcl?hea=qf&for=qfcoraltdu&cnt=qfcoraltdu&xid=CW50172},
	urldate = {2015-09-23},
	maintitle = ,
	mainsubtitle = {{Corpus der altdeutschen Originalurkunden bis zum Jahr 1300}},
	publisher = {University of Trier},
	location = {Trier},
	year = {2007},
}

@reference{lexer,
	title = {{Mittelhochdeutsches Handwörterbuch von Matthias Lexer}},
	editor = {Matthias Lexer},
	volumes = {3},
	origlocation = {Leipzig},
	origpublisher = {Hirzel},
	origyear = {1872--1878},
	publisher = {University of Trier},
	location = {Trier},
	year = {1998--2015},
	url = {http://woerterbuchnetz.de/Lexer/},
	urldate = {2015-10-08},
}

@reference{duden:vornamen,
	title = {{Duden Lexikon Vornamen}},
	editor = {Rosa Kohlheim and Volker Kohlheim},
	edition = {6},
	publisher = {Duden},
	location = {Mannheim},
	year = {2013}
}
\end{filecontents*}

\addbibresource{\jobname.bib}

%% END OF PREAMBLE %%%%%%%%%%%%%%%%%%%%%%%%%%%%%%%%%%%%%%%%%%%%%%%%%%%%%%%%%%%%%

\begin{document}

%% MAIN PART %%%%%%%%%%%%%%%%%%%%%%%%%%%%%%%%%%%%%%%%%%%%%%%%%%%%%%%%%%%%%%%%%%%

\begin{mytitle}
    \@title: \@subtitle
\end{mytitle}

\section{Friedrich Wilhelm and His Diplomatic Dream}
Part of the data basis for my upcoming master's thesis is formed 
by \citeauthor{CAO}'s \citetitle*{CAO} (\enquote*{Corpus of Old German Original 
Deeds up to the Year 1300}, henceforth \citetitle{CAO}). This collection 
comprises over 4,000 legal documents in German language from the 13th century 
and was compiled by several generations of researchers over the course of over 
70 years. Most of the deeds come from the southwestern corner of the 
German-speaking area, with centers in Strasbourg, Basel, Zurich and Constance. A 
steady increase in activity over the last quarter of the 13th century can 
generally be observed, however.

In his foreword to the first volume of the \citetitle{CAO}, Wilhelm made a 
strong argument for rendering the modern transcriptions of the handwritten 
documents as closely as possible to the medieval originals in a time when most 
scholars in the field prefered normalized editions of medieval texts for 
literary criticism – he harshly derided the notion of normalized texts as 
\enquote{Esperanto-Mittelhochdeutsch} \autocite[see][VIII--IX]{CAO1}. Wilhelm's 
intended target audience was what we call linguists today, rather than literary 
critics. He explains in his own words:

\foreignblockquote{german}{\begin{multicols}{2}Es ergibt sich daher die weitere 
Notwendigkeit, diese urkundlichen Quellen in einer Ausgabe vor sich zu haben, 
die diese Orthographien nicht durch Normalisieren nach Lachmanns und seiner 
Nachfolger Art verwischt. Denn dieses Normalisieren macht zum guten Teil die 
modernen Urkundenveröffentlichungen, welche Urkunden in deutscher Sprache 
ent\-halten, […] für den Sprachforscher un\-brauch\-bar […]. Gerade das 
\enquote{Normalisieren} also mußte bei diesem Corpus, das in erster Linie dem 
Sprachforscher dienen soll, vermieden werden. Ein möglichst genauer Abdruck, 
soweit ein solcher überhaupt das Original einer Urkunde ersetzen kann, war das 
erste Erfordernis, das erfüllt werden mußte. \autocite[LX]{CAO1}

\columnbreak

{\itshape Thus, the additional necessity arises to have these diplomatic sources 
be laid out in an edition that does not smudge these orthographies by means of 
normalization after the fashion of Lachmann and his followers. For this 
normalizing renders the better part of modern publications of deeds which 
contain deeds in German language […] useless to the language scholar […]. 
Especially this \enquote{normalizing} thus had to be avoided in this corpus, 
which is first and foremost catering to language scholars. The first requirement 
that had to be met was a rendition of the text as close as possible to the 
original, if such a rendition in print can ever be a replacement for the 
original document at all.}\end{multicols}}

The bulk of the texts contained in the \citetitle{CAO} is formed by such mundane 
things as sales and lease contracts, wills, legal settlements, and regulations 
concerning trade and construction. Moreover, the texts are usually not too long 
and not too complex either, which makes some of them suitable for translation 
challenges, I suppose.

Due to the nature of the deeds, there is a fair amount of legal formulas in 
these documents, however, these use very transparent native terms, which ought 
to make finding equivalents reasonably straightforward. An English translation 
based on the annotations by the editors of the \citetitle{CAO} will be provided 
\autocite[see][]{n163-online}.

\section{\enquote{Alle die, die disen brief lesent, horent oder sehent …}}
In the following, I will present the text of the deed number 
\citefield{n163}{title} -- effectively dated Augsburg, March 19, 1279 -- in 
Middle High German language as given in \textcites{n163}{n163-online}. In spite 
of Wilhelm's self-professed firm belief in diplomatic editions, the version of 
this document as it is presented in the \citetitle{CAO} shows signs of modern 
editing, that is, abbreviations are resolved, non-names are changed to 
consistent lower case, and punctuation is adjusted to modern standards.

\blockquote{\textbf{J}nn\emph{omin}e d\emph{omi}n\emph{i} 
am\emph{en}!\footnote{\enquote{Original \emph{Jnnōîe dn̄j am̄}}.} Alle die, die 
diſen brief leſent, horent oder ſehent, die ſuln daz wizzen, daz her Vlrich 
frovn Engeln tohterman verkaufte herm Chvnrad dem Hafener einen halben garten 
mit ſogtaner beſcheidenheit, daz der her Chvnrat der Hafener vf den ſelben 
halben garten weder zvn noch hvſ bwen ſolte noch de cheinen bv dar vf tvͦn 
ſolte, der im ze ſchaden chomen mohte. Daz ſtvnt alſ lange, vnz der her Chvnrat 
der Hafener dar fvr vnde wolte gezvnet haben den ſelben garten. Do deſ her 
Vlrich innan wart, do fvr er fvr gerih\-te vnde clagte hinze dem Hafener, daz er 
da zvnen wolte, da erſ niht tvn ſolte, wande er ez mit gedingede im alſo geben 
hete, daz er weder zvn noh hvſ da bwen ſolte noh nihteſ deſ, daz im ze ſchaden 
chomen mohte. Deſ laugent im der Hafener, daz er mit im alſo iht kauft hete. Dar 
vber wart erteilt, daz der Hafener bereite, daz er mit im alſo iht kauft hete, 
in bezivgte danne her Vlrich, daz ez alſo wæ̂re. Da wolte her Vlrich ſinſ rehteſ 
niht vmbe vnde gerte einſ tageſ vmbe ſinen gezivk. Do der tak chom, do gie her 
Vlrich fvr vnde erzivgte ſelbe dritte, alſ reht waſ, daz er vf den ſelben garten 
chein den bv tvn ſolte, der im ze ſchaden chomen mohte, alſ da vor geſchriben 
ſtat. Vnde do er daz erzivgte, do gert er vrteil, wande die lvte tôtlich wæren 
vnde auh vergæzzen, man ſolte im wol der stet\footnote{\enquote{\label{fn:2} 
Erster Buchstabe dieses Wortes im Original Majuskel} (First letter of this word 
originally majuscule).} brief dar vber geben. Daz wart im erteilt mit geſamenter 
vrteil. Vnde da von, daz zwiſchen in chein kriech mer gewahſen mvge vmbe die 
ſache, dar vmbe wart geben dirre brief verſigelt mit der stet\footref{fn:2} 
jnſigel\footref{fn:2} ze Auſpurk, daz dar an hanget. Vnde ſint deſ ge\-ziu\-ge 
her Volkwin, her Sibot der Stolzhirz, her Vlrich Fundan, her Livpolt der 
Schroter, her Livpolt der Stolzhirz, her Chvnrat Reinbot, meiſter Chvnrat von 
Schoenegge, her Chvnrat Notkauf, her Chvnrat der Bart vnde ander genvge. Do daz 
geſchach vnde auh dirre brief geben wart, do waſ von goteſ\footref{fn:2} geburte 
tuſent\footref{fn:2} iar zwej hvndert iar in dem nivn\footref{fn:2} vnde 
sibenzigoſten iare an dem ſvnnetage vor dem balm tage.
\begin{center}
S[iegel]
\end{center}}

The text goes about like this in English:

\blockquote{\emph{\textbf{I}n nomine domini amen!} All those who read this writ, 
by hearing or seeing it, they shall know that Mr.\,Ulrich, Mrs.\,Engel's 
son-in-law, sold Mr.\,Chunrat der Hafener half of a garden under the provision 
that Mr.\,Chunrat der Hafener build neither fence nor house in this same half 
garden, nor put any other building in it which would cause damage to the other. 
This was set, up until Mr.\,Chunrat der Hafener went there and wanted to fence the 
garden. As Mr.\,Ulrich became aware of this, he went to court and sued Hafener 
since he wanted to set up a fence there even though he was not supposed to, 
since he had pledged by contract that he would neither build a fence nor a house 
in it, nor anything else that would cause damage to the other. Hafener denied 
this: he had not contracted with Ulrich this way. The court assessed that 
Hafener refuted that he had contracted with Ulrich this way, unless Mr.\,Ulrich 
could prove that he had. Mr.\,Ulrich did not want to forfeit his rights, then, 
and asked for a day to testify. When the day came, Mr.\,Ulrich came up and 
testified together with two witnesses, as was the law, that Hafener was not 
supposed to build anything in this same garden that would cause damage to him, 
as written above. And after giving his testimony, he demanded judgement and a 
municipal writ, since people are mortal and forgetful. This was awarded to him 
unanimously. And so that there may not spring any more disputes between them 
from this issue, this writ was issued, sealed with the city of Augsburg's seal 
which is attached. To this are witness Mr.\,Volkwin, Mr.\,Sibot der Stolzhirz, 
Mr.\,Ulrich Fundan, Mr.\,Liupolt der Schroter, Mr.\,Liupolt der Stolzhirz, 
Mr.\,Chunrat Reinbot, Master Chunrat von Schoenegge, Mr.\,Chunrat Notkauf, 
Mr.\,Chunrat der Bart and others enough. When this took place and also this 
writ was issued, the count from God's birth was one thousand years, two hundred 
years, in the seventy-ninth year on the Sunday before Palm Sunday.

\begin{center}
Seal
\end{center}}

\section{Discussion}

\subsection*{In nomine domini …}

% In nomine domini amen!

\ex \label{ex:innomine}
\begingl
	\glpreamble Garāneri na Nahang, āmen! //
	\gla garān @ -eri na Nahang, āmen! //
	\glb name -\Ins{} \Gen{} Lord, amen! //
	\glft \enquote{In the name of the Lord, amen!} //
\endgl \xe

I really wish I had something besides Ayeri ready to draw from, but 
since I don't, I decided to also translate this Latin phrase into Ayeri.
Since I have come up with at least a sketch for another language a while ago 
(the Ayeri exonym for it being \xayr{turyi}{Turayi}{Hillandic}), however, I 
couldn't help also thinking about this:

\ex[exno=\ref{ex:innomine}′] \begingl
	\glpreamble \rightcomment{[Turayi, \emph{tentatively}]} Āsti-n na Pāg 
		ko saț, āmen! //
	\gla āsti @ -n na Pāg ko saț, āmen! //
	\glb name -\Def{} \Def{} Lord of with, amen! //
	\glft \enquote{In the name of the Lord, amen!} //
\endgl
\xe

Turayi in its modern version is supposed to be some kind of anti-Ayeri in being 
fairly isolating with modifier-head word order, which is why there are 
postpositions in the translation above. In a little twist on modifier-head
word order, however, determiners go into the so-called \fw{Wackernagel position}, 
that is, they go into the second position of the \Np{} if nothing else occupies 
this slot. In Turayi, you would thus get \textsc{n}~\textsc{det} if no 
adjectives are involved, but \textsc{adj}~\textsc{det}~\textsc{n} if there are. 
There would, however, be an exception for possessive phrases, in which the (more 
basic?) word order \textsc{det}~\textsc{n}~\emph{of} would be preferred instead 
of \textsc{n det}~\emph{of} as one would otherwise expect. I will have to think 
about whether postpositions in general block movement of determiners into the 
Wackernagel position in spite of different levels of dependency.

\subsection*{Alle die, die …}

% Alle die, die diſen brief leſent, horent oder ſehent, die ſuln daz wizzen, daz 
% her Vlrich frovn Engeln tohterman verkaufte herm Chvnrad dem Hafener einen 
% halben garten mit ſogtaner beſcheidenheit, daz der her Chvnrat der Hafener vf 
% den ſelben halben garten weder zvn noch hvſ bwen ſolte noch de cheinen bv dar 
% vf tvͦn ſolte, der im ze ſchaden chomen mohte.

\pex
	\glpreamble Ang mya koronyan keynam-hen si ang layayan eda-tahang 
		nivayēri soyang tangyēri tan, sa lataya mondo mesam yam 
		Sikontendo Biratayati tiga ang Hasanjan tiga si samyanang na 
		Nina\-vay tiga, dilengeri, ya mya vehoyya ang Sikontendo 
		Biratayati tiga kong eda-mondo mesam lahanley soyang nangās 
		soyang vehanley palung siley eng ming nupisongara yās. //
	\a \begingl
		\gla ang mya koron @ -yan keynam @ =hen si ang laya @ -yan eda= 
		@ tahang @ -ley niva @ -ye @ -eri soyang tang @ -ye @ -eri 
			tan,~… //
		\glb \AgtT{} shall know -\Tpl{}.\M{} people =all \Rel{} 
			\AgtT{} read -\Tpl{}.\M{}.\Top{} this= writ 
			-\Parg{}.\Inan{} eye -\Pl{} -\Ins{} or ear -\Pl{} 
			-\Ins{} \Tpl{}.\Gen{},~… //
		\glft \enquote{All people who read this writ with their eyes or 
			their ears shall know that …} //
	\endgl
	\a \begingl
		\gla sa lata @ -ya mondo mesam yam= @ Sikontendo Biratayati tiga 
			ang= @ Hasanjan tiga si samyan @ -ang na= @ Ninavay 
			tiga, dileng @ -eri,~… //
		\glb \PatT{} sell -\Tsg{} garden.\Top{} half \Dat{}= Chunrat 
			{der Hafener} honorable \Aarg{}= Ulrich honorable \Rel{} 
			son.in.law -\Aarg{} \Gen{}= Engel honorable, rule 
			-\Ins{},~… //
		\glft \enquote{the honorable Ulrich, son-in-law of the honorable 
			Engel, sold half a garden to the honorable Chunrat der 
			Hafener, under the condition that …} //
	\endgl
	\a \begingl
		\gla ya mya veh @ -oy @ -ya ang= @ Sikontendo Biratayati tiga 
			kong eda= @ mondo mesam lahan @ -ley soyang nanga @ -as 
			soyang vehan @ -ley palung si @ -ley eng ming nupa @ 
			-isa @ -ong @ -ara yās. //
		\glb \LocT{} shall build -\Neg{} -\Tsg{}.\M{} \Aarg{}= 
			Chunrat {der Hafener} honorable inside this= 
			garden.\Top{} half fence -\Parg{}.\Inan{} or house 
			-\Parg{} or building -\Parg{}.\Inan{} other \Rel{} 
			-\Parg{}.\Inan{} \AgtT{}.\Inan{} can hurt -\Caus{} 
			-\Irr{} -\Tsg{}.\Inan{}.\Top{} \Tsg{}.\M{}.\Parg{}. //
		\glft \enquote{Chunrat der Hafener is not supposed to build into 
		the garden either fence or house or another building that could 
		cause him harm.} //
	\endgl
\xe

This is one hell of a long sentence, which I nonetheless didn't split up into 
several ones in order to keep the legalese spirit of the original. However, 
for convenience, I split the one long sentence into its constituent parts 
for glossing. The initial passage, \fw{Alle die, die diſen brief leſent, horent 
oder ſehent} `All those who read this writ, by hearing or seeing it' is a 
rhetorical topos of the genre and the way basically all deeds in the 
\citetitle{CAO} begin, with some variations.

Since it's always fun to look up the etymologies of names and trying to calque 
them into Ayeri, I permitted myself to do that here as well and came up with the 
translations you can see in \autoref{tab:names1}.

\begin{table}[t]
\centering
\begin{tabu} to \textwidth {X[40] X[60] X[40] X[60]}
	\rowfont {\bfseries\upshape\footnotesize}
	\everyrow{\rowfont{\footnotesize}}
	\multicolumn2{c}{German}
		& \multicolumn2{c}{Ayeri} \\ %\endhead
	
	\midrule
	
	Chunrat
		& \Ohg{} \fw{kuoni} `brave, strong', \newline
			\Ohg{} \fw{rāt} `counsel'
		& \ayr{sik\_o\_MteMdo} \fw{Sikontendo}
		& \xayr{sikoNF}{sikong}{advice}, \newline
			\xayr{teMdo}{tendo}{courageous} \\
	
	\midrule
	
	Engel
		& \Gk{} ἄγγελος \fw{ángelos} `messenger'
		& \ayr{ninvj} \fw{Ninavay}
		& \xayr{niny}{ninaya}{messenger}, \newline
			\xayr{/vy}{-vaya}{fem. occupation} \\
	
	\midrule
	
	Hafener
		& \Mhg{} \norm{havenære} `potter'\footnotemark
		& \ayr{birtyti} \fw{Biratayati}
		& \xayr{birtj}{biratay}{pot}, \newline
			\xayr{/Ati}{-ati}{-maker} \\
	
	\midrule
	
	Ulrich
		& \Ohg{} \fw{uodal} `heritage, home', \newline
			\Ohg{} \fw{rīhhi} `noble, rich, powerful'
		& \ayr{hsMdYnF} \fw{Hasanjan}
		& \xayr{hsNF}{hasang}{origin}, \newline
			\xayr{IdYnF}{ijan}{rich} \\
\end{tabu}
	\caption{The negotiating parties' names calqued into Ayeri}
	\label{tab:names1}
\end{table}

\footnotetext{Compare \Mhg{} \norm{haven} `pot'; normalized spellings follow 
the headwords in \textcite{lexer} and are indicated by ⸢…⸣ brackets when 
referring to the text using retrofitted spellings.}

Another choice I made is translating the titles \norm{herre} `lord' and 
\norm{vrouwe} `lady' equally as \xayr{tig}{tiga} {honorable}. In the case of 
this deed, I \emph{assume} that the participants in the sale and their families 
are not noble as the titles would suggest, but more or less wealthy townspeople. 
Thus, \norm{herre} and \norm{vrouwe}, respectively, have probably already 
assumed their modern meanings as generic respectful terms of address comparable 
to English \fw{Mr.} and \fw{Mrs.} in this context, which I used in the English 
translation of the deed above as well for the same reason. The dictionary still 
has entries for \xayr{tyonF}{Tayon}{Mr.} and \xayr{tenFvnF}{Tenvan}{Mrs.} 
at the time of writing this, but I've never actually used these since I dropped 
the modifier \ayr{tj} \fw{tay} that these are supposed to be based on.

\subsection*{Daz stunt als lange …}

% Daz ſtvnt alſ lange, vnz der her Chvnrat der Hafener dar fvr vnde wolte 
% gezvnet haben den ſelben garten.

\ex \begingl
	\glpreamble Adareng tono, ang no vehya Sikontendo Biratayati tiga
		lahanley miday eda-mondoya pesan. //
	\gla Ada @ -reng tono, ang sara @ -ya Ø= @ Sikontendo Biratayati tiga
		adaya pesan nay ang no veh @ -ya lahan @ -ley miday eda= @ mondo 
		@ -ya. //
	\glb that -\Aarg{}.\Inan{} certain, \AgtT{} go -\Tsg{}.\M{} \Top{}= 
		Chunrat {der Hafener} honorable there until and \AgtT{} want 
		build -\Tsg{}.\M{}.\Top{} fence -\Parg{}.\Inan{} around this= 
		garden -\Loc{}. //
	\glft \enquote{This was certain until the honorable Chunrat der Hafener 
		went there and wanted to build a fence around this garden.} //
\endgl \xe

Literally, \fw{Daz ſtvnt alſ lange} means `this stood so long', which is 
quite idiomatic, so I didn't want to translate it literally and decided to go 
with \xayr{AdreNF tono}{adareng tono}{this was certain}. What is expressed 
here is that the deal between Chunrat and Ulrich was good until Chunrat wanted 
to build a fence, which of course runs counter their agreement.

\subsection*{Do des her Ulrich …}

% Do deſ her Vlrich innan wart, do fvr er fvr gerihte vnde clagte hinze dem 
% Hafener, daz er da zvnen wolte, da erſ niht tvn ſolte, wande er ez mit 
% gedingede im alſo geben hete, daz er weder zvn noh hvſ da bwen ſolte noh 
% nihteſ deſ, daz im ze ſchaden chomen mohte.

\pex%[glspace=.5em]
	\glpreamble Tadayya si kengya ang Hasanjan tiga, ang sahaya tagātiya
		nay sa tirayyāng Biratayati, ang no vehya lahanley nārya mya
		da-vehoyyāng yanoyam ang tonisaya narāneri ban yana, ang 
		vehoyongya adaya lahanley soyang nangās soyang vehanley palung 
		siley eng ming nupisongara yās. //
	\a \label{ex:do} \begingl
		\gla taday @ -ya si keng @ -ya ang= Hasanjan tiga,~… //
		\glb time -\Loc{} \Rel{} notice -\Tsg{}.\M{} \Aarg{}= Ulrich 
			honorable,~… //
		\glft \enquote{When the honorable Ulrich noticed that, …} //
	\endgl
	\a \begingl
		\gla ang saha @ -ya tagāti @ -ya nay sa tiray @ -yāng Ø= @ 
			{Biratayati},~… //
		\glb \AgtT{} come -\Tsg{}.\M{}.\Top{} judge -\Loc{} and \PatT{} 
			complain.about -\Tsg{}.\M{}.\Aarg{} \Top{}= {der 
			Hafener,}~… //
		\glft \enquote{he came to a judge and complained about der 
			Hafener that …} //
	\endgl
	\a \begingl
		\gla ang no veh @ -ya lahan @ -ley nārya mya da= @ veh @ -oy @ 
			-yāng~… //
		\glb \AgtT{} want build -\Tsg{}.\M{}.\Top{} fence 
			-\Parg{}.\Inan{} although shall such= build 
			-\Neg{} -\Tsg{}.\M{}.\Aarg{}~… //
		\glft \enquote{he wants to build a fence although he was not 
			supposed to do so …} //
	\endgl
	\a \begingl
		\gla yanoyam ang tonisa @ -ya narān @ -eri ban yana,~… //
		\glb because \AgtT{} assure -\Tsg{}.\M{}.\Top{} word -\Ins{} 
			good \Tsg{}.\M{}.\Gen{},~… //
		\glft \enquote{because he had assured with his good word that 
			…} //
	\endgl
	\a \begingl
		\gla ang veh @ -oy @ -ong @ -ya adaya lahan @ -ley soyang nanga 
			@ -as soyang vehan @ -ley palung si @ -ley eng ming nupa 
			@ -isa @ -ong @ -ara yās. //
		\glb \AgtT{} build -\Neg{} -\Irr{} -\Tsg{}.\M{}.\Top{} there 
			fence -\Parg{}.\Inan{} or house -\Parg{} or building 
			-\Parg{}.\Inan{} other \Rel{} -\Parg{}.\Inan{} 
			\AgtT{}.\Inan{} can hurt -\Caus{} -\Irr{} 
			-\Tsg{}.\Inan{}.\Top{} \Tsg{}.\M{}.\Parg{}. //
		\glft \enquote{he would not build there either fence or house or 
			another building that could cause him harm.} //
	\endgl
\xe

I was a little stumped at first about how to translate the original \fw{do} 
`when' in (\ref{ex:do}), since Ayeri uses its equivalent question word 
\xayr{sitdj}{sitaday}{when} only in questions but usually not as a relative 
pronoun. And a headless relative clause is what we have here in both German and 
English. What is left out is something like \fw{ze der zît} or `at the time', 
respectively. This realization made me remember that I could just translate it 
as that, which is how we get \ayr{tdjy si} \fw{tadayya si} here.

I translated \norm{klagen} `sue' as \xayr{tirj/}{tiray-}{complain, bemoan} here 
since I thought that this fits the general situation and action -- maybe also 
not uninfluenced by the German word \Nhg{} \fw{klagen}, which covers both 
meanings, `complain' and `sue'. What is interesting here, though, is that the 
verb \ayr{tirj/} \fw{tiray-} takes a direct object/patient \Np{} as an argument, 
not a \Pp{}, which is why I gave the translation in the gloss as `complain 
about'.

\subsection*{Des laugent im der Hafener …}

% Deſ laugent im der Hafener, daz er mit im alſo iht kauft hete.

\ex
\begingl
	\glpreamble Le garica ang Biratayati adanya, narayāng, rī intoyyāng 
		da-dileng. //
	\gla le garik @ -ya ang= Biratayati adanya, nara @ -yāng, rī int @ 
		-oy @ -yāng eda= @ dileng. //
	\glb \PatT{} deny -\Tsg{}.\M{} \Aarg{}= {der Hafener} that.\Top{}, say 
		-\Tsg{}.\M{}.\Aarg{}, \InsT{} buy -\Neg{} -\Tsg{}.\M{}.\Aarg{} 
		such= rule.\Top{} //
	\glft \enquote{This der Hafener denied, he said he did not buy under 
		such a condition.} //
\endgl \xe

The Ayeri sentence is fairly vanilla here. I purposefully used references to 
the purchase and its conditions as the topic in both cases and introduced the 
reportative \xayr{nryaaNF}{narayāng}{he said} for more clarity. What is a little 
more interesting, though, is the second half of the \Mhg{} sentence,
\fw{daz er mit im alſo iht kauft hete} `that he had not bought with him this 
way'. \enquote{Buy with someone} as an idiomatic expression for `conclude a sale 
with someone' I haven't heard in either modern German or in English so far.

\subsection*{Dar uber wart erteilt …}

% Dar vber wart erteilt, daz der Hafener bereite, daz er mit im alſo iht kauft 
% hete, in bezivgte danne her Vlrich, daz ez alſo wæ̂re.

\ex \begingl
	\glpreamble Maritapya tagātiang, ranya ang Biratayati, rī incāng
		da-dileng, nārya ang ming pukatongya Hasanjan tiga puluyley 
		yayam. //
	\gla maritap @ -ya tagāti @ -ang, ran @ -ya ang= @ Biratayati, rī
		int @ -yāng eda= @ dileng, nārya bata ang ming pukat @ -ong @ 
		-ya Ø= Hasanjan tiga puluy @ -ley yayam. //
	\glb determine -\Tsg{}.\M{} judge -\Aarg{}, contest -\Tsg{}.\M{}
		\Aarg{}= {der Hafener}, \InsT{} buy -\Tsg{}.\M{}.\Aarg{} such= 
		rule.\Top{}, except if \AgtT{} can prove -\Irr{} -\Tsg{}.\M{} 
		\Top{} Ulrich honorable opposite -\Parg{}.\Inan{} 
		\Tsg{}.\M{}.\Dat{}. //
	\glft \enquote{The judge determined that der Hafener contested that 
		he had bought under such a condition, except if the honorable 
		Ulrich could prove the opposite to him.} //
\endgl \xe

\subsection*{Da wolte her Ulrich …}

% Da wolte her Vlrich ſinſ rehteſ niht vmbe vnde gerte einſ tageſ vmbe ſinen 
% gezivk.

\ex \begingl
	\glpreamble Le no da-subroyya ang Hasanjan tiga kaytan yana nay 
		ang da-pinyaya bahisyam siyā ming ha\-kong\-yāng.//
	\gla le no da= @ subr @ -oy @ -ya ang= @ Hasanjan tiga kaytan yana 
		nay ang da= @ pinya @ -ya bahis @ -yam si @ -Ø @ -yā ming haka 
		@ -ong @ -yāng. //
	\glb \PatT{} want thus= give.up -\Neg{} -\Tsg{}.\M{} \Aarg{}= Ulrich 
		honorable right.\Top{} \Tsg{}.\Gen{} and \AgtT{} thus= ask 
		-\Tsg{}.\M{}.\Top{} day -\Dat{} \Rel{} -\Dat{} -\Loc{} can 
		testify -\Irr{} -\Tsg{}.\M{}.\Aarg{}. //
	\glft \enquote{The honorable Ulrich did not want to give up his right 
		there and he thus asked for a day on which he could testify.} //
\endgl \xe

Another remark on interesting idioms here. The \Mhg{} sentence contains the 
phrase \fw{wolte … ſinſ rehteſ niht vmbe}, literally `did not want of his 
right around', meaning that he did not want to forfeit his rights according to 
the annotation in \textcite{n163-online}.

\subsection*{Do der tak chom …}

% Do der tak chom, do gie her Vlrich fvr vnde erzivgte ſelbe dritte, alſ reht 
% waſ, daz er vf den ſelben garten chein den bv tvn ſolte, der im ze ſchaden 
% chomen mohte, alſ da vor geſchriben ſtat.

\ex \begingl
	\glpreamble … //
	\gla … //
	\glb … //
	\glft \enquote{…} //
\endgl \xe

…

\subsection*{Unde do er …}

% Vnde do er daz erzivgte, do gert er vrteil, wande die lvte tôtlich wæren vnde 
% auh vergæzzen, man ſolte im wol der stet brief dar vber geben.

\ex \begingl
	\glpreamble … //
	\gla … //
	\glb … //
	\glft \enquote{…} //
\endgl \xe

…

\subsection*{Daz wart im erteilt …}

% Daz wart im erteilt mit geſamenter vrteil.

\ex \begingl
	\glpreamble … //
	\gla … //
	\glb … //
	\glft \enquote{…} //
\endgl \xe

…

\subsection*{Unde da von …}

% Vnde da von, daz zwiſchen in chein kriech mer gewahſen mvge vmbe die ſache, 
% dar vmbe wart geben dirre brief verſigelt mit der stet jnſigel ze Auſpurk, daz 
% dar an hanget.

\ex \begingl
	\glpreamble … //
	\gla … //
	\glb … //
	\glft \enquote{…} //
\endgl \xe

…

\subsection*{Unde sint des geziuge …}

% Vnde ſint deſ geziuge her Volkwin, her Sibot der Stolzhirz, her Vlrich Fundan, 
% her Livpolt der Schroter, her Livpolt der Stolzhirz, her Chvnrat Reinbot, 
% meiſter Chvnrat von Schoenegge, her Chvnrat Notkauf, her Chvnrat der Bart vnde 
% ander genvge.

\pex 
	\glpreamble Nay ang bengyan eda-mandanya ayonye tiga Ledayyan, Baysikān 
		Ikandesay, Hasanjan Pundan, Keynantendo Vehimati, Keynantendo 
		Ikandesay, Sikontendo Baysāruan, Sikontendo na Hinyanveno 
		baykan, Sikontendo Intamarkan, Sikontendo Piku nay keynam-ma 
		palung. //
	
	\a \begingl
		\gla Nay ang beng @ -yan eda= @ mandan @ -ya ayon @ -ye tiga~… //
		\glb and \AgtT{} attend -\Tpl{}.\M{} this= hearing -\Loc{} man 
			-\Pl{}.\Top{} honorable~… //
		\glft \enquote{And this hearing was attended by the honorable men 
			…} //
	\endgl
	
	\a \begingl
		\gla Ledayyan, Baysikān Ikandesay, Hasanjan Pundan, Keynantendo 
			Vehimati, Keynantendo Ikandesay, Sikontendo Baysāruan, 
			Sikontendo na Hinyanveno baykan, Sikontendo Intamarkan, 
			Sikontendo Piku~… //
		\glb Volkwin, Sibot {der Stolzhirz}, Ulrich Fundan, Liupolt 
			{der Schroter}, Liupolt {der Stolzhirz}, Chunrat 
			Reinbot, Chunrat \Gen{} Schoenegge master, Chunrat 
			Notkauf, Chunrat {der Bart}~… //
		\glft \enquote{Volkwin, Sibot der Stolzhirz, Ulrich Fundan, 
			Liupolt der Schroter, Liupolt der Stolzhirz, Chunrat 
			Reinbot, master Chunrat von Schoenegge, Chunrat Notkauf, 
			Chunrat der Bart} //
	\endgl
	
	\a \begingl
		\gla nay keynam @ =ma palung. //
		\glb and people.\Top{} =enough other. //
		\glft \enquote{and enough other people.} //
	\endgl
\xe

At this point, I wonder if I better shouldn't have calqued the names. The 
problem is that the names \fw{Reinbot}, \fw{Sibot} and \fw{Volkwin} aren't 
common German names anymore. Also, I have no clue about \fw{Fundan}. The rest of 
the names -- especially the surnames -- are transparent enough to venture a 
reasonable guess. For \fw{Chunrat} and \fw{Ulrich}, see \autoref{tab:names1} 
above. While \fw{Chunrat}, \fw{Ulrich}, \fw{Liupolt} and \fw{Volkwin} can simply 
be looked up in a dictionary,\footnote{Such as \textcite{duden:vornamen}, for 
example, though Wiktionary said the same things in all cases.} the remaining two 
names I could only guess at indirectly by looking up similar modern names.

Both \fw{Reinbot} and \fw{Sibot} contain the element \fw{-bot}, which may be 
related to the modern name \fw{Bodo}, which derives from \Os{} \fw{bodo} 
`master, ruler' or \Ohg{} \fw{boto} `messenger'. \fw{Rein-} is probably related 
to the \fw{Rein-} in similar surviving names like \fw{Reinfried}, \fw{Reinhard}, 
\fw{Reinhold}, where it derives from \Gmc{} \fw{*ragina} `counsel, decision (of 
the spirits), fate'. For the \fw{Si-} in \fw{Sibot} I assumed a contraction from 
\fw{Sige-} (cf. \Mhg{} \fw{sige} `victory'), and thus a relation to 
\fw{Siegfried} (in other deeds sometimes abbreviated to \fw{Sifrit} or similar), 
\fw{Si(e)gmar}, \fw{Siegmund}, where the \fw{Sieg-} part is from \Ohg{} 
\fw{sigu} `victory'.

\autoref{tab:names2} gives a summary of the names in the list of witnesses if 
they haven't occurred already earlier and their respective calques. A general 
problem with two-part Germanic given names is that when calquing them, they 
often become at least twice as long since nouns in Ayeri are very commonly 
disyllabic already.

\begin{table}[t]
\centering
\begin{tabu} to \textwidth {X[40] X[60] X[40] X[60]}
	\rowfont {\bfseries\upshape\footnotesize}
	\everyrow{\rowfont{\footnotesize}}
	\multicolumn2{c}{German}
		& \multicolumn2{c}{Ayeri} \\ %\endhead
	
	\midrule
	
	Bart
		& \Mhg{} \fw{bart} `beard'
		& \ayr{piku} \fw{Piku}
		& \xayr{piku}{piku}{beard} \\
	
	\midrule
	
	Liupolt
		& \Ohg{} \fw{liut} `people', \newline
			\Ohg{} \fw{bald} `bold'
		& \ayr{kejnMteMdo} \fw{Keynantendo}
		& \xayr{kejnmF}{keynam}{people}, \newline
			\xayr{teMdo}{tendo}{courageous} \\
	
	\midrule
	
	Notkauf
		& \Mhg{} \norm{nôt} `hardship, trouble', \newline
			\Mhg{} \norm{kouf} `trade, bargain'
		& \ayr{IMtmrFknF} \fw{Intamarkan}
		& \xayr{IMtnF}{intan}{purchase}, \newline
			\xayr{marFknF}{markan}{trouble} \\
	
	\midrule
	
	Reinbot
		& \Gmc{} \fw{*ragina} `counsel, fate'; \newline
			\Os{} \fw{bodo} `master, ruler' or \newline
			\Ohg{} \fw{boto} `messenger'
		& \ayr{bjsaaru\_anF} \fw{Baysāruan}
		& \xayr{bjhi}{bayhi}{ruler}, \newline
			\xayr{shru\_anF}{saharuan}{fate, destiny} \\
	
	\midrule
	
	von Schoenegge
		& \Mhg{} \norm{schœne} `beautiful', \newline
			\Mhg{} \norm{ecke} `corner'
		& \ayr{hinYnFveno} \fw{Hinyanveno}
		& \xayr{hinYnF}{hinyan}{corner}, \newline
			\xayr{veno}{veno}{beautiful} \\
	
	\midrule
	
	Schroter
		& \Mhg{} \norm{schrôtære} `taylor'
		& \ayr{vehimti} \fw{Vehimati}
		& \xayr{vehimF}{vehim}{dress}, \newline
			\xayr{/Ati}{-ati}{-maker} \\
	
	\midrule
	
	Sibot
		& \Ohg{} \fw{sigu} `victory'; \newline
			\Os{} \fw{bodo} `master, ruler' or \newline
			\Ohg{} \fw{boto} `messenger'
		& \ayr{bjsikaanF} \fw{Baysikān}
		& \xayr{bjhi}{bayhi}{ruler}, \newline
			\xayr{siktu}{sikatu}{victorious} \\
	
	\midrule
	
	Stolzhirz
		& \Mhg{} \fw{stolz} `foolish, superb', \newline
			\Mhg{} \fw{hirz} `deer'
		& \ayr{IkMdesj} \fw{Ikandesay}
		& \xayr{IkmF}{ikam}{deer}, \newline
			\xayr{desj}{desay}{noble, distinguished} \\
	
	\midrule
	
	Volkwin
		& \Ohg{} \fw{folc} `people, folk', \newline
			\Ohg{} \fw{wini} `friend'
		& \ayr{ledjynF} \fw{Ledayyan}
		& \xayr{lednF}{ledan}{friend}, \newline
			\xayr{Ajye}{ayye}{people} \\
	
\end{tabu}
	\caption{The witnesses' names calqued into Ayeri}
	\label{tab:names2}
\end{table}

\subsection*{Do daz geschach …}

% Do daz geſchach vnde auh dirre brief geben wart, do waſ von goteſ geburte 
% tuſent iar zwej hvndert iar in dem nivn vnde sibenzigoſten iare an dem 
% ſvnnetage vor dem balm tage.

% $ ayerinumbers.py -s 1279
% 8A7₁₂: menang hen mallan-ito

\ex \begingl
	\glpreamble Tadayya si da-sahareng nay ilisāra eda-tahangley,
		kuranreng vesangya na Pangal masahatay pericanley menang hen, 
		kong mallan-itanya pericanena, bahisya perin marin bahisya 
		adang.//
	\gla taday @ -ya si da= @ saha @ -reng nay ilisa @ -ara eda= @ tahang 
		@ -ley, kuran @ -reng vesang @ -ya na= @ Pangal masahatay 
		perican @ -ley menang hen, kong mallan @ .itan @ -ya perican @ 
		-ena, bahis @ -ya perin marin bahis @ -ya adang. //
	\glb time -\Loc{} \Rel{} such= come -\Tsg{}.\Inan{} and 
		issue -\Tsg{}.\Inan{} this writ -\Parg{}.\Inan{}, count 
		-\Aarg{}.\Inan{} birth -\Loc{} \Gen{}= God since year 
		-\Parg{}.\Inan{} gross eight, within tenty .seven -\Loc{} year 
		-\Gen{}, day -\Loc{} sun before day -\Loc{} palm. //
	\glft \enquote{At the time this happened and this writ was issued,
		the count since the birth of God was eight gross years, in 
		the tenty-seventh year, on the Sunday before the Palm Day.} //
\endgl \xe

The dating formula seems very convoluted with its repetition of \fw{iar} `year', 
but this is the way it works in many of the \citetitle{CAO}'s deeds. What is 
interesting, though not surprising since literality was historically clerical, 
is that religious feasts or name days of saints are used as reference points 
instead of modern calendar days numbered 1 through 31, and dates are given 
relative to those feast days. Ayeri itself uses a base 12 number system, so 
1279 converts to \textsc{8a7}.

This section introduces \ayr{Ilis/} \fw{ilisa-} with the additional meaning 
`to issue' and newly coins \xayr{AdNF}{adang}{palm tree}.

\section{\enquote{Ang mya koronyan keynam-hen …}}

In its native script, the complete translated text looks like this:

\blockquote{

	\begin{ayeri}
		\fw{AAsFti/nF n paagF ko stJ\_F, AAmenF!} %1
			ANF mY koronYnF kejnm/henF si ANF lyynF Ed/thNF nivyeeri 
			soyNF tNYeeri tnF, s lty moMdo mesmF ymF sik\_o\_MteMdo 
			birtyti tig ANF hsMdYnF tig si smYnNF n ninvj tig, 
			dileNeri, y mY vehojy ANF sik\_o\_MteMdo birtyti tig 
			koNF Ed/moMdo mesmF lhnFlej soyNF nNaasF soyNF vehnF\-lej 
			pluNF silej ENF miNF nupisoNr yaasF. %2
		AdreNF tono, ANF no vehY sik\_o\_MteMdo birtyti tig lhnFlej midj 
			Ed/moMdoy pesnF. %3
		tdjy si keNY ANF hsMdYnF tig, ANF shy tgaatiy \& s tirj\-yaaNF 
			birtyti, ANF no vehY lhnF\-lej naarY mY d/vehojyaanF 
			ynoymF ANF tonisy nraaneri bnF yn, ANF veho\-yoNY Ady 
			lhnFlej soyNF nNaasF soyNF vehnFlej pluNF silej ENF miNF 
			nupisoNr yaasF. %4
		le grikY ANF birtyti AdnY, nryaaNF, rii IMtojyaaNF d/dileNF. %5
		mritpY tgaati\_aNF, rnY ANF birtyti, rii IMtYaaNF d/dilenF,
			naarY ANF miNF puktoNY hsMdYnF tig pulujlej yym. %6
		le no d/subFrojy ANF hsMdYnF tig kjtnF yn \& ANF d/pinYy bhisYmF 
			siyaa miNF hkoNYaaNF. %7
		— %8
		— %9
		— %10
		— %11
		\& ANF beNYnF Ed/mMdnY AyonYe tig ledjynF, bjsikaanF IkMdesj, 
			hsMdYnF puMdnF, kej\-nM\-teMdo vehimti, kej\-nM\-teMdo 
			IkMdesj, sik\_o\_MteMdo bjsaaru\_anF, sik\_o\_MteMdo n 
			hinYnFveno bjknF, sik\_o\_MteMdo IMtmrFknF, 
			sik\_o\_MteMdo piku \& kejnmF/m pluNF. %12
		tdjy si d/shreNF \& Ilisaar Ed/thNFlej, kurnreNF vesNY n pNlF 
			mshtj perikYnFlej menNF henF, koNF mlFlnF/ItnY 
			perikYnen, bhisY perinF mrinF bhisY AdNF. %13
		\begin{center}
			melkmF
		\end{center}
	\end{ayeri}
}

And for completeness' sake, this is the complete text again in Roman 
transcription:

\blockquote{
	\fw{\textbf{Ā}sti-n na Pāg ko saț, āmen!}\footnote{Alternatively:
		\ayr{graaneri n nhNF, AAmenF!} \fw{Garāneri na Nahang, āmen!}
		See page \pageref{ex:innomine} for details.} %1
	Ang mya koronyan keynam-hen si ang layayan eda-tahang nivayēri soyang 
		tangyēri tan, sa lataya mondo mesam yam Sikontendo Biratayati 
		tiga ang Hasanjan tiga si samyanang na Nina\-vay tiga, 
		dilengeri, ya mya vehoyya ang Sikontendo Biratayati tiga kong 
		eda-mondo mesam lahanley soyang nangās soyang vehanley palung 
		siley eng ming nupisongara yās. %2
	Adareng tono, ang no vehya Sikontendo Birata\-yati tiga lahanley miday 
		eda-mondoya pesan. %3
	Tadayya si kengya ang Hasanjan tiga, ang sahaya tagātiya nay sa 
		tirayyāng Birata\-yati, ang no vehya lahanley nārya mya 
		da-vehoyyāng yanoyam ang tonisaya narāneri ban yana, ang 
		vehoyongya adaya lahanley soyang nangās soyang vehanley palung 
		siley eng ming nupisongara yās. %4
	Le garica ang Biratayati adanya, narayāng, rī intoyyāng da-dileng. %5
	Maritapya tagātiang, ranya ang Biratayati, rī incāng da-dileng, 
		nārya ang ming pukatongya Hasanjan tiga puluyley yayam. %6
	Le no da-subroyya ang Hasanjan tiga kaytan yana nay ang da-pinyaya 
		bahisyam siyā ming hakongyāng. %7
	… %8
	… %9
	… %10
	… %11
	Nay ang bengyan eda-mandanya ayonye tiga Ledayyan, Baysikān Ikandesay, 
		Hasanjan Pundan, Keynantendo Vehimati, Keynantendo Ikandesay, 
		Sikontendo Baysāruan, Sikontendo na Hinyanveno baykan, 
		Sikontendo Intamarkan, Sikontendo Piku nay keynam-ma palung. %12
	Tadayya si da-sahareng nay ilisāra eda-tahangley, kuranreng vesangya na 
		Pangal masahatay pericanley menang hen, kong mallan-itanya 
		pericanena, bahisya perin marin bahisya adang. %13
	\begin{center}
		Melakam
	\end{center}
}

\section{Epilog}
At 365 words, the text of \citefield{n163}{title} is rather mid-length. It took 
me over a week to translate it, in every sitting doing a sentence or two, in 
spite of it not being especially difficult structurally. Also, since I've been 
working on Ayeri for so long and since this deed uses a lot of basic vocabulary, 
I didn't have to coin a lot of new words. In fact, though, the devil was rather 
in the details of \LaTeX{} formatting than coming up with idiomatic translations 
of the sentences.

What was fun, was trying to come to terms with that initial sentence, 
\fw{Jnn\emph{omin}e d\emph{omi}n\emph{i} am\emph{en}!} So far, I have only dealt 
with monolingual texts to translate from, usually in either English or German. 
\enquote*{Vaporlang} is still only a bullet list of ideas and a word list at 
this point, but it was an interesting experiment trying to tease out something 
concrete from those vague ideas at long last. Maybe something to expand on.

%% BIBLIOGRAPHY %%%%%%%%%%%%%%%%%%%%%%%%%%%%%%%%%%%%%%%%%%%%%%%%%%%%%%%%%%%%%%%%

\vfill
\printbibliography

\end{document}
