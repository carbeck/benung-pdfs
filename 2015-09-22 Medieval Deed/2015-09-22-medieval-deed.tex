\documentclass[12pt,paper=a4]{scrartcl}

% Author, Title, Subtitle etc.
\author{Carsten Becker}
\title{Fences and Gardens}
\subtitle{An Ayeri Translation of a Medieval Neighborhood Dispute}

% Handle language and quotation marks
\usepackage[ngerman,english]{babel}
\usepackage{csquotes} % Put quotations in \enquote{}!
\SetBlockEnvironment{quotation}
\renewcommand*{\mkccitation}[1]{ (#1)}

% Set all margins to 2.54 cm
\usepackage[margin=1in]{geometry}
\widowpenalty10000 % Avoid widows like the plague!
\clubpenalty10000 % Avoid orphans like the plage, too!

% Make multiple columns available in single-column document
\usepackage{multicol}

% Make text colors and color names available
\usepackage[xetex]{xcolor}

% Load font stuff for XeTeX
\usepackage{xltxtra}
\usepackage{fontspec}

% Set main fonts
\usepackage[config=mt-Junicode]{microtype}

\newfontfamily{\Tagati}[
    Renderer=Graphite,
    Scale=1.1,
    BoldFont={* Italic},
    HyphenChar=·,
]{Tagati Book G}

\setmainfont[
    Ligatures=TeX,
    Numbers=Lowercase,
]{Junicode}

\setsansfont[
    Ligatures=TeX,
    Numbers=Lowercase,
    Scale=MatchUppercase,
    BoldFont={Open Sans Condensed Bold},
]{Open Sans Condensed Light}

% Load BibLaTeX (using Biber), configure citation styles
% Uses <https://github.com/semprag/biblatex-sp-unified>
\usepackage[
    style=biblatex-sp-unified,
    citestyle=sp-authoryear-comp,
    maxnames=2,
    backend=biber,
    safeinputenc,
    hyperref,
]{biblatex}

% To make \textcite look like "Doe (2014: 213)"
\renewcommand*{\postnotedelim}{\addcolon\addspace}
\DeclareFieldFormat{postnote}{#1}
\DeclareFieldFormat{multipostnote}{#1}

% Enable generating files from this .tex file (for the bibliography) 
\usepackage{filecontents}

% Date etc.
\usepackage{datetime}
\renewcommand{\dateseparator}{/}

% Clickable links in footnotes, TOC, etc.
\usepackage[
    xetex,
    bookmarks=true,
    colorlinks=false,
    linktoc=section,
    hidelinks,
    pdfusetitle,
]{hyperref}

\usepackage{url}
\urlstyle{rm}

% Ability to include graphics and dealing with footnotes in descriptions
\usepackage{graphicx}
\usepackage[font={small,sf},labelfont={small,sf},format=plain]{caption}
\usepackage{subcaption}
\usepackage{wrapfig}
\setlength{\columnsep}{2\baselineskip}

% General headers and footers
\usepackage{fancyhdr}
\pagestyle{fancy}

\fancyhead[L]{} % empty
\fancyhead[C]{} % empty
\fancyhead[R]{\thepage}

\fancyfoot[L]{} % empty
\fancyfoot[C]{} % empty
\fancyfoot[R]{} % empty

\renewcommand{\headrulewidth}{0pt}
\renewcommand{\footrulewidth}{0pt}

% First page headers and footers are different
\fancypagestyle{firstpage}{
    \fancyhead[L]{\sffamily \footnotesize \textbf{Benung. The Ayeri Language Resource}}
    \fancyhead[C]{} % empty
    \fancyhead[R]{\sffamily \footnotesize Carsten Becker · \yyyymmdddate\today}
    
    \fancyfoot[L]{} % empty
    \fancyfoot[C]{} % empty
    \fancyfoot[R]{\sffamily \footnotesize 
	\href{http://benung.nfshost.com}{http://benung.nfshost.com} · 
	\href{https://github.com/carbeck/benung-pdfs}{https://github.com/carbeck/benung-pdfs} · 
	\href{https://creativecommons.org/licenses/by-sa/4.0/}{CC~BY-SA~4.0}
    }
    
    \renewcommand{\headrulewidth}{0.5pt}
    \setlength\footskip{0.5in}
}

\usepackage{ifthen}
\ifthenelse{\value{page}=1}{\thispagestyle{firstpage}{\pagestyle{fancy}}}

% Line spacing
\usepackage{setspace}
\onehalfspacing

% Things for tables
\usepackage{longtable}
\usepackage{tabu}
\extrarowsep=2.5pt
% \usepackage{rotating}

% Formatting of glosses
\usepackage{expex}
\usepackage{leipzig}

\newleipzig{AgtT}{at}{agent topic}
\newleipzig{PatT}{pt}{patient topic}
\newleipzig{DatT}{datt}{dative topic}
\newleipzig{GenT}{gent}{genitive topic}
\newleipzig{LocT}{loct}{locative topic}
\newleipzig{InsT}{inst}{instrumentative topic}
\newleipzig{CauT}{caut}{causative topic}
\newleipzig{An}{an}{animate}
\newleipzig{Inan}{inan}{inanimate}

% Nicer footnotes
\usepackage[bottom]{footmisc}
\deffootnote{0em}{1.5em}{\textsuperscript\thefootnotemark\enskip}
\renewcommand{\footnoterule}{\rule{0pt}{0pt}{\vspace*{-0pt}}}
\setlength{\footnotesep}{1.5em}

% Smaller font in block quotes
\usepackage{relsize,etoolbox}
\AtBeginEnvironment{quote}{\noindent\smaller}
\AtBeginEnvironment{quotation}{\smaller}

% Macros
\newcommand{\fw}[1]{\textit{#1}} % Foreign Word
\newcommand{\tit}[1]{\textit{#1}} % Title of a work
\newcommand{\q}[1]{\enquote{#1}} % Context-aware quotation
\newcommand{\qq}[1]{\enquote*{#1}} % Explicit sublevel quotation
\newcommand{\tsup}[1]{\textsuperscript{#1}} % Superscript
\newcommand{\markyellow}[1]{\colorbox{yellow}{#1}} % Yellow highlighter
\newcommand{\ques}{\fakesuperscript{?}} % raised question mark

\newcommand{\ayr}[1]{{\Tagati #1}}
\newcommand{\xayr}[3]{{\Tagati #1} \emph{#2} \enquote*{#3}}

\newenvironment{ayeri}{
    %\hyphenpenalty=10000
    %\hbadness=10000
    \doublespacing
    \begin{multicols}{2}
    \Tagati
}{
    \end{multicols} \par
}

\newenvironment{mytitle}{
    \hfill
    \begin{minipage}{0.667\textwidth}
	\vspace{\baselineskip}
	\begin{center}
	    \Large
	    \sffamily\bfseries
	    \makeatletter
}{
	    \makeatother
	\end{center}
	\vspace{1em}
    \end{minipage}
    \hfill
}

% blah
\usepackage{lipsum}

%% BIBLIOGRAPHY DATABASE %%%%%%%%%%%%%%%%%%%%%%%%%%%%%%%%%%%%%%%%%%%%%%%%%%%%%%%

\begin{filecontents*}{\jobname.bib}
@mvcollection{CAO,
	editor = {Friedrich Wilhelm and Richard Newald and Helmut de Boor and Diether Haacke and Bettina Kirschstein},
	gender = {pp},
	title = {{C}orpus der altdeutschen {O}riginalurkunden bis zum {J}ahr 1300},
	shorttitle = {CAO},
	shorthand = {CAO},
	% sortkey = {CAO0},
	volumes = {5},
	publisher = {Moritz Schauenburg and Erich Schmidt},
	location = {Lahr/Schwarzwald and Berlin},
	year = {1932--2004},
}

@collection{CAO1,
	title = {1200--1282},
	shorttitle = {CAO I},
	shorthand = {CAO I},
	volume = {1},
	% sortkey = {CAO1},
	editor = {Friedrich Wilhelm},
	location = {Lahr (Baden)},
	publisher = {Moritz Schauenburg},
	year = {1932},
	crossref = {CAO},
}

@collection{CAO5,
	booktitle = {Nachtragsurkunden 1261--1297},
	bookshorttitle = {CAO V},
	shorthand = {CAO V},
	volume = {5},
	% sortkey = {CAO5},
	editor = {Helmut de Boor and Diether Haacke and Bettina Kirschstein},
	% titleaddon = {Begründet von Friedrich Wilhelm. Fortgeführt von Richard Newald},
	founder = {Friedrich Wilhelm},
	continuator = {Richard Newald},
	gender = {pp},
	location = {Berlin},
	publisher = {Erich Schmidt},
	year = {2004},
	crossref = {CAO},
}

@incollection{n163,
	title = {N~163 (381~a)},
	origdate = {1279-03-19},
	origyear = {1279},
	pages = {127},
	crossref = {CAO5},
}

@incollection{n163-online,
	title = {N~163 (381~a) 1279 März 19},
	editor = {Kurt Gärtner and Andrea Rapp and Andreas Gniffke},
	url = {http://tcdh01.uni-trier.de/cgi-bin/iCorpus/CorpusIndex.tcl?hea=qf&for=qfcoraltdu&cnt=qfcoraltdu&xid=CW50172},
	urldate = {2015-09-23},
	maintitle = ,
	mainsubtitle = {{Corpus der altdeutschen Originalurkunden bis zum Jahr 1300}},
	publisher = {University of Trier},
	location = {Trier},
	year = {2007},
}
\end{filecontents*}

\addbibresource{\jobname.bib}

%% END OF PREAMBLE %%%%%%%%%%%%%%%%%%%%%%%%%%%%%%%%%%%%%%%%%%%%%%%%%%%%%%%%%%%%%

\begin{document}

%% MAIN PART %%%%%%%%%%%%%%%%%%%%%%%%%%%%%%%%%%%%%%%%%%%%%%%%%%%%%%%%%%%%%%%%%%%

\begin{mytitle}
    \@title: \@subtitle
\end{mytitle}

\section{Friedrich Wilhelm and His Diplomatic Dream}
Part of the data basis for my upcoming master's thesis is formed 
by \citeauthor{CAO}'s \citetitle*{CAO} (\enquote*{Corpus of Old German Original 
Deeds up to the Year 1300}, henceforth \citetitle{CAO}). This collection 
comprises over 4,000 legal documents in German language from the 13th century 
and was compiled by several generations of researchers over the course of over 
70 years. Most of the deeds come from the southwestern corner of the 
German-speaking area, with centers in Strasbourg, Basel, Zurich and Constance. A 
steady increase in activity over the last quarter of the 13th century can 
generally be observed, however.

In his foreword to the first volume of the \citetitle{CAO}, Wilhelm made a 
strong argument for rendering the modern transcriptions of the handwritten 
documents as closely as possible to the medieval originals in a time when most 
scholars in the field prefered normalized editions of medieval texts for 
literary criticism – he harshly derided the notion of normalized texts as 
\enquote{Esperanto-Mittelhochdeutsch} \autocite[see][VIII--IX]{CAO1}. Wilhelm's 
intended target audience was what we call linguists today, rather than literary 
critics, as he explains:

\foreignblockquote{german}{\begin{multicols}{2}Es ergibt sich daher die weitere 
Notwendigkeit, diese urkundlichen Quellen in einer Ausgabe vor sich zu haben, 
die diese Orthographien nicht durch Normalisieren nach Lachmanns und seiner 
Nachfolger Art verwischt. Denn dieses Normalisieren macht zum guten Teil die 
modernen Urkundenveröffentlichungen, welche Urkunden in deutscher Sprache 
enthalten, [...] für den Sprachforscher unbrauchbar [...]. Gerade das 
\enquote{Normalisieren} also mußte bei diesem Corpus, das in erster Linie dem 
Sprachforscher dienen soll, vermieden werden. Ein möglichst genauer Abdruck, 
soweit ein solcher überhaupt das Original einer Urkunde ersetzen kann, war das 
erste Erfordernis, das erfüllt werden mußte. \autocite[LX]{CAO1}

\columnbreak

{\itshape Thus, the additional necessity arises to have these diplomatic sources 
be laid out in an edition that does not blur these orthographies by means of 
normalization after the fashion of Lachmann and his followers. For this 
normalizing renders the better part of modern publications of deeds which 
contain deeds in German language [...] useless to the language scholar [...]. 
Especially this \enquote{normalizing} thus had to be avoided in this corpus, 
which is first and foremost catering to language scholars. The first requirement 
that had to be met was a rendition of the text as close as possible to the 
original, if such a rendition in print can ever be a replacement for the 
original document at all.}\end{multicols}}

The bulk of the texts contained in the \citetitle{CAO} is formed by such mundane 
things as sales and lease contracts, wills, legal settlements, and regulations 
concerning trade and construction. Moreover, the texts are usually not too long 
and not too complex either, which makes some of them suitable for translation 
challenges, I suppose.

Due to the nature of the deeds, there is a fair amount of legal formulas in 
these documents, however, these use very transparent native terms, which ought 
to make finding equivalents reasonably straightforward. An English translation 
based on the annotations by the editors of the \citetitle{CAO} will be provided 
\autocite[see][]{n163-online}.

\section{\enquote{Alle die, die disen brief lesent, hœrent oder sehent}}
In the following, I will present the text of the deed number 
\citefield{n163}{title} -- effectively dated Augsburg, March 19, 1279 -- in 
Middle High German language as given in \textcites{n163}{n163-online}. In spite 
of Wilhelm's self-professed firm belief in diplomatic editions, the version of 
this document as it is presented in the \citetitle{CAO} shows signs of modern 
editing, that is, abbreviations are resolved and punctuation is adjusted to 
modern standards.

\blockquote{\textbf{J}nn\emph{omin}e d\emph{omi}n\emph{i} 
am\emph{en}!\footnote{\enquote{Original \emph{Jnnōîe dn̄j am̄}}.} Alle die, die 
diſen brief leſent, horent oder ſehent, die ſuln daz wizzen, daz her Vlrich 
frovn Engeln tohterman verkaufte herm Chvnrad dem Hafener einen halben garten 
mit ſogtaner beſcheidenheit, daz der her Chvnrat der Hafener vf den ſelben 
halben garten weder zvn noch hvſ bwen ſolte noch de cheinen bv dar vf tvͦn 
ſolte, der im ze ſchaden chomen mohte. Daz ſtvnt alſ lange, vnz der her Chvnrat 
der Hafener dar fvr vnde wolte gezvnet haben den ſelben garten. Do deſ her 
Vlrich innan wart, do fvr er fvr gerih\-te vnde clagte hinze dem Hafener, daz er 
da zvnen wolte, da erſ niht tvn ſolte, wande er ez mit gedingede im alſo geben 
hete, daz er weder zvn noh hvſ da bwen ſolte noh nihteſ deſ, daz im ze ſchaden 
chomen mohte. Deſ laugent im der Hafener, daz er mit im alſo iht kauft hete. Dar 
vber wart erteilt, daz der Hafener bereite, daz er mit im alſo iht kauft hete, 
in bezivgte danne her Vlrich, daz ez alſo wæ̂re. Da wolte her Vlrich ſinſ rehteſ 
niht vmbe vnde gerte einſ tageſ vmbe ſinen gezivk. Do der tak chom, do gie her 
Vlrich fvr vnde erzivgte ſelbe dritte, alſ reht waſ, daz er vf den ſelben garten 
chein den bv tvn ſolte, der im ze ſchaden chomen mohte, alſ da vor geſchriben 
ſtat. Vnde do er daz erzivgte, do gert er vrteil, wande die lvte tôtlich wæren 
vnde auh vergæzzen, man ſolte im wol der stet\footnote{\enquote{\label{fn:2} 
Erster Buchstabe dieses Wortes im Original Majuskel} (First letter of this word 
originally majuscule).} brief dar vber geben. Daz wart im erteilt mit geſamenter 
vrteil. Vnde da von, daz zwiſchen in chein kriech mer gewahſen mvge vmbe die 
ſache, dar vmbe wart geben dirre brief verſigelt mit der stet\footref{fn:2} 
jnſigel\footref{fn:2} ze Auſpurk, daz dar an hanget. Vnde ſint deſ ge\-ziu\-ge 
her Volkwin, her Sibot der Stolzhirz, her Vlrich Fundan, her Livpolt der 
Schroter, her Livpolt der Stolzhirz, her Chvnrat Reinbot, meiſter Chvnrat von 
Schoenegge, her Chvnrat Notkauf, her Chvnrat der Bart vnde ander genvge. Do daz 
geſchach vnde auh dirre brief geben wart, do waſ von goteſ\footref{fn:2} geburte 
tuſent\footref{fn:2} iar zwej hvndert iar in dem nivn\footref{fn:2} vnde 
sibenzigoſten iare an dem ſvnnetage vor dem balm tage.
\begin{center}
S[iegel]
\end{center}}

The text goes about like this in English:

\blockquote{\emph{\textbf{I}n nomine domini amen.} All those who read this writ, 
by hearing or seeing it, they shall know that Mr.\,Ulrich, Mrs.\,Engel's 
son-in-law, sold Mr.\,Chunrat der Hafener half of a garden under the provision 
that Mr.\,Chunrat der Hafener build neither fence nor house in this same half 
garden, nor put any other building in it which would cause damage to the other. 
This was set, up until Mr.\,Chunrat der Hafener wanted to fence the garden. As 
Mr.\,Ulrich became aware of this, he went to court and sued Hafener since he 
wanted to set up a fence there even though he was not supposed to, since he had 
pledged by contract that he would neither build a fence nor a house in it, nor 
anything else that would cause damage to the other. Hafener denied this: he had 
not contracted with Ulrich this way. The court assessed that Hafener refuted 
that he had contracted with Ulrich this way, unless Mr.\,Ulrich could prove that 
he had. Mr.\,Ulrich did not want to forfeit his rights, then, and asked for a 
day to testify. When the day came, Mr.\,Ulrich came up and testified together 
with two witnesses, as was the law, that Hafener was not supposed to build 
anything in this same garden that would cause damage to him, as written above. 
And after giving his testimony, he demanded judgement and a municipal writ, 
since people are mortal and forgetful. This was awarded to him unanimously. And 
so that there may not spring any more disputes between them from this issue, 
this writ was issued, sealed with the city of Augsburg's seal which is attached. 
To this are witness Mr.\,Volkwin, Mr.\,Sibot der Stolzhirz, Mr.\,Ulrich Fundan, 
Mr.\,Liupolt der Schroter, Mr.\,Liupolt der Stolzhirz, Mr.\,Chunrat Reinbot, 
Master Chunrat von Schoenegge, Mr.\,Chunrat Notkauf, Mr.\,Chunrat der Bart and 
others enough. When this took place and also this writ was issued, the count 
from God's birth was one thousand years, two hundred years, in the seventy-ninth 
year on the Sunday before Palm Sunday.

\begin{center}
Seal
\end{center}}

\section{Ayeri Translation}

% In nomine domini amen.

\ex \begingl
	\glpreamble Garāneri na Nahang, āmen! //
	\gla garān -eri na Nahang, āmen! //
	\glb name -\Ins{} \Gen{} Lord, amen! //
	\glft \enquote{In the name of the Lord, amen!} //
\endgl \xe

	I really wish I had another language besides Ayeri to draw from right 
	now, but since I don't, I decided to also translate this Latin phrase 
	into Ayeri.

% Alle die, die 
% diſen brief leſent, horent oder ſehent, die ſuln daz wizzen, daz her Vlrich 
% frovn Engeln tohterman verkaufte herm Chvnrad dem Hafener einen halben garten 
% mit ſogtaner beſcheidenheit, daz der her Chvnrat der Hafener vf den ſelben 
% halben garten weder zvn noch hvſ bwen ſolte noch de cheinen bv dar vf tvͦn 
% ſolte, der im ze ſchaden chomen mohte.

\pex
	\glpreamble Ang mya koronyan keynam-hen si ang layayan eda-tahang 
		nivayēri soyang tangyēri tan, sa lataya mondo mesam yam 
		Sikontendo Biratayati tiga ang Litijan tiga si samyanang na 
		Ninavay tiga, dilengeri, ya mya vehoyya ang Sikontendo 
		Biratayati tiga kong eda-mondo mesam lahanley soyang nangās 
		soyang vehanley palung siley eng ming nupisongara yās. //
	\a \begingl
		\gla ang mya koron -yan keynam =hen si ang laya -yan eda= tahang 
			-ley niva -ye -eri soyang tang -ye -eri tan, ... //
		\glb \AgtT{} be.supposed.to know -\Tpl{}.\M{} people =all \Rel{} 
			\AgtT{} read -\Tpl{}.\M{}.\Top{} this= writ 
			-\Parg{}.\Inan{} eye -\Pl{} -\Ins{} or ear -\Pl{} 
			-\Ins{} \Tpl{}.\Gen{}, ... //
		\glft \enquote{All people who read this writ with their eyes or 
			their ears shall know that ...} //
	\endgl
	\a \begingl
		\gla sa lata -ya mondo mesam yam= Sikontendo Biratayati tiga 
			ang= Litijan tiga si samyan -ang na= Ninavay tiga, 
			dileng -eri, ... //
		\glb \PatT{} sell -\Tsg{} garden.\Top{} half \Dat{}= Conrad 
			Potter honorable \Aarg{}= Ulric honorable \Rel{} 
			son.in.law -\Aarg{} \Gen{}= Angelica honorable, rule 
			-\Ins{}, ... //
		\glft \enquote{the honorable Ulrich, son-in-law of the honorable 
			Engel, sold half a garden to the honorable Chunrad der 
			Hafener, under the rule that ...} //
	\endgl
	\a \begingl
		\gla ya mya veh -oy -ya ang= Sikontendo Biratayati tiga kong 
			eda= mondo mesam lahan -ley soyang nanga -as soyang 
			vehan -ley palung si -ley eng ming nupa -isa -ong -ara 
			yās. //
		\glb \LocT{} be.supposed.to build -\Neg{} -\Tsg{}.\M{} \Aarg{}= 
			Conrad Potter honorable inside this= garden.\Top{} half 
			fence -\Parg{}.\Inan{} or house -\Parg{} or building 
			-\Parg{}.\Inan{} \Rel{} -Parg{}.\Inan{} can hurt 
			-\Caus{} -\Irr{} -\Tsg{}.\Inan{} \Tsg{}.\M{}.\Parg{}. //
		\glft \enquote{Chunrad der Hafener is not supposed to build into 
		the garden either fence or house or another building that could 
		damage him.} //
	\endgl
\xe

This is one hell of a long sentence, which I nonetheless did not split up into several ones in order to keep the legalese spirit of the original. Since it is always fun looking up the etymologies of names and trying to calque them into Ayeri, I permitted myself to do so here as well and came up with the following:

\begin{longtabu} to \textwidth {X[37] X[63] | X[40] X[60]}
	\rowfont {\bfseries\upshape\footnotesize}
	\everyrow{\rowfont{\footnotesize}}
	\multicolumn2{c|}{German}					& \multicolumn2{c}{Ayeri} \\ \hline %\endhead
	
	Chunrad (Conrad)	& Gmc \fw{kuoni} `brave', \newline
				  Gmc \fw{rad} `counsel'		& \ayr{sikoMteMdo} \fw{Sikontendo}	& \xayr{sikoNF}{sikong}{advice}, \newline
														  \xayr{teMdo}{tendo}{courageous} \\
	Hafener (Potter)	& MHG \fw{havenære} `potter'		& \ayr{birtyti} \fw{Biratayati}		& \xayr{birtj}{biratay}{pot}, \newline
														  \xayr{/Ati}{-ati}{maker} \\ [2.5pt]
	\hline
	Ulrich (Ulric)		& Gmc \fw{odal} `heritage, prosperity', \newline
				  Gmc \fw{ric} `power'			& \ayr{litidYnF} \fw{Litijan}		& \xayr{lito}{lito}{power}, \newline
														  \xayr{IdYn}{idyan}{rich} \\
	\hline
	Engel (Angel)		& Gk ἄγγελος \fw{ángelos} `messenger'	& \ayr{ninvj} \fw{Ninavay}		& \xayr{niny}{ninaya}{messenger} \newline
														  \xayr{/vy}{-vaya}{fem. occupation} \\
\end{longtabu}

Another choice I made is translating the titles \fw{herre} `lord' and 
\fw{vrouwe} `lady' equally as \xayr{tig}{tiga}{honorable}. In the case of this 
deed, I \emph{assume} that the participants in the sale and their families are 
not noble as the titles would suggest, but more or less wealthy townspeople. 
Thus, \fw{herre} and \fw{vrouwe}, respectively, have probably already assumed 
their modern meanings as generic respectful terms of address comparable to 
English \fw{Mr.} and \fw{Mrs.} here.

% Daz ſtvnt alſ lange, vnz der her Chvnrat der Hafener dar fvr vnde wolte 
% gezvnet haben den ſelben garten.

\ex \begingl
	\glpreamble ... //
	\gla ... //
	\glb ... //
	\glft \enquote{...} //
\endgl \xe

...

% Do deſ her Vlrich innan wart, do fvr er fvr 
% gerihte vnde clagte hinze dem Hafener, daz er da zvnen wolte, da erſ niht tvn 
% ſolte, wande er ez mit gedingede im alſo geben hete, daz er weder zvn noh hvſ 
% da bwen ſolte noh nihteſ deſ, daz im ze ſchaden chomen mohte.

\ex \begingl
	\glpreamble ... //
	\gla ... //
	\glb ... //
	\glft \enquote{...} //
\endgl \xe

...

% Deſ laugent im 
% der Hafener, daz er mit im alſo iht kauft hete. Dar vber wart erteilt, daz 
% der Hafener bereite, daz er mit im alſo iht kauft hete, in bezivgte danne her 
% Vlrich, daz ez alſo wæ̂re.

\ex \begingl
	\glpreamble ... //
	\gla ... //
	\glb ... //
	\glft \enquote{...} //
\endgl \xe

...

% Da wolte her Vlrich ſinſ rehteſ niht vmbe vnde 
% gerte einſ tageſ vmbe ſinen gezivk.

\ex \begingl
	\glpreamble ... //
	\gla ... //
	\glb ... //
	\glft \enquote{...} //
\endgl \xe

...

% Do der tak chom, do gie her Vlrich fvr 
% vnde erzivgte ſelbe dritte, alſ reht waſ, daz er vf den ſelben garten chein 
% den bv tvn ſolte, der im ze ſchaden chomen mohte, alſ da vor geſchriben ſtat.

\ex \begingl
	\glpreamble ... //
	\gla ... //
	\glb ... //
	\glft \enquote{...} //
\endgl \xe

...

% Vnde do er daz erzivgte, do gert er vrteil, wande die lvte tôtlich wæren vnde 
% auh vergæzzen, man ſolte im wol der stet brief dar vber geben.

\ex \begingl
	\glpreamble ... //
	\gla ... //
	\glb ... //
	\glft \enquote{...} //
\endgl \xe

...

% Daz wart im erteilt mit geſamenter vrteil.

\ex \begingl
	\glpreamble ... //
	\gla ... //
	\glb ... //
	\glft \enquote{...} //
\endgl \xe

...

% Vnde da von, daz zwiſchen in chein kriech mer 
% gewahſen mvge vmbe die ſache, dar vmbe wart geben dirre brief verſigelt mit 
% der stet jnſigel ze Auſpurk, daz dar an hanget.

\ex \begingl
	\glpreamble ... //
	\gla ... //
	\glb ... //
	\glft \enquote{...} //
\endgl \xe

...

% Vnde ſint deſ geziuge her 
% Volkwin, her Sibot der Stolzhirz, her Vlrich Fundan, her Livpolt der Schroter, 
% her Livpolt der Stolzhirz, her Chvnrat Reinbot, meiſter Chvnrat von 
% Schoenegge, her Chvnrat Notkauf, her Chvnrat der Bart vnde ander genvge.

\ex \begingl
	\glpreamble ... //
	\gla ... //
	\glb ... //
	\glft \enquote{...} //
\endgl \xe

...

% Do 
% daz geſchach vnde auh dirre brief geben wart, do waſ von goteſ geburte tuſent 
% iar zwej hvndert iar in dem nivn vnde sibenzigoſten iare an dem ſvnnetage vor 
% dem balm tage.

\ex \begingl
	\glpreamble ... //
	\gla ... //
	\glb ... //
	\glft \enquote{...} //
\endgl \xe

...

%% BIBLIOGRAPHY %%%%%%%%%%%%%%%%%%%%%%%%%%%%%%%%%%%%%%%%%%%%%%%%%%%%%%%%%%%%%%%%

%\newpage
\printbibliography

\end{document}
