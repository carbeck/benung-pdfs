\documentclass[12pt,paper=a4]{scrartcl}

% Author, Title, Subtitle etc.
\author{Carsten Becker}
\title{Fences and Gardens}
\subtitle{A Postscript}

% Handle language and quotation marks
\usepackage[ngerman,english]{babel}
\usepackage{csquotes} % Put quotations in \enquote{}!
\SetBlockEnvironment{quotation}
\renewcommand*{\mkccitation}[1]{ (#1)}

% Set all margins to 2.54 cm
\usepackage[margin=1in]{geometry}
\widowpenalty10000 % Avoid widows like the plague!
\clubpenalty10000 % Avoid orphans like the plage, too!

% Make multiple columns available in single-column document
\usepackage{multicol}

% Make text colors and color names available
\usepackage[xetex]{xcolor}

% Load font stuff for XeTeX
\usepackage{xltxtra}
\usepackage{fontspec}

% Set main fonts
\usepackage[config=mt-Junicode]{microtype}

\setmainfont[
  Ligatures=TeX,
  Numbers=Lowercase,
]{Junicode}

\setsansfont[
  Ligatures=TeX,
  Numbers=Lowercase,
  Scale=MatchUppercase,
  BoldFont={Open Sans Condensed Bold},
]{Open Sans Condensed Light}

% Date etc.
\usepackage{datetime}
\renewcommand{\dateseparator}{/}

% Clickable links in footnotes, TOC, etc.
\usepackage[
  xetex,
  bookmarks=true,
  colorlinks=false,
  linktoc=section,
  hidelinks,
  pdfusetitle,
]{hyperref}

\usepackage{url}
\urlstyle{rm}

% Ability to include graphics
\usepackage{graphicx}

% General headers and footers
\usepackage{fancyhdr}
\pagestyle{fancy}

\fancyhead[L]{} % empty
\fancyhead[C]{} % empty
\fancyhead[R]{\thepage}

\fancyfoot[L]{} % empty
\fancyfoot[C]{} % empty
\fancyfoot[R]{} % empty

\renewcommand{\headrulewidth}{0pt}
\renewcommand{\footrulewidth}{0pt}

% First page headers and footers are different
\fancypagestyle{firstpage}{
  \fancyhead[L]{\sffamily \footnotesize \textbf{Benung. The Ayeri Language Resource}}
  \fancyhead[C]{} % empty
  \fancyhead[R]{\sffamily \footnotesize Carsten Becker · \yyyymmdddate\today}
  
  \fancyfoot[L]{} % empty
  \fancyfoot[C]{} % empty
  \fancyfoot[R]{\sffamily \footnotesize 
	\href{http://benung.nfshost.com}{http://benung.nfshost.com} · 
	\href{https://github.com/carbeck/benung-pdfs}{https://github.com/carbeck/benung-pdfs} · 
	\href{https://creativecommons.org/licenses/by-sa/4.0/}{CC~BY-SA~4.0}
  }
  
  \renewcommand{\headrulewidth}{0.5pt}
  \setlength\footskip{0.5in}
}

\usepackage{ifthen}
\ifthenelse{\value{page}=1}{\thispagestyle{firstpage}{\pagestyle{fancy}}}

% Line spacing
\usepackage{setspace}
\onehalfspacing

% Avoid pagebreaks right after sections and subsections
\usepackage{needspace}
\usepackage{etoolbox}
\preto\section{\needspace{6\baselineskip}}
\preto\subsection{\needspace{6\baselineskip}}

% Nicer footnotes
\usepackage[bottom]{footmisc}
\deffootnote{0em}{1.5em}{\textsuperscript\thefootnotemark\enskip}
\renewcommand{\footnoterule}{\rule{0pt}{0pt}{\vspace*{-0pt}}}
\setlength{\footnotesep}{1em}

% Smaller font in block quotes
\usepackage{relsize}
\AtBeginEnvironment{quote}{\noindent\smaller}
\AtBeginEnvironment{quotation}{\smaller}

% Macros
\newcommand{\fw}[1]{\textit{#1}} % Foreign Word
\newcommand{\tit}[1]{\textit{#1}} % Title of a work
\newcommand{\q}[1]{\enquote{#1}} % Context-aware quotation
\newcommand{\qq}[1]{\enquote*{#1}} % Explicit sublevel quotation
\newcommand{\tsup}[1]{\textsuperscript{#1}} % Superscript
\newcommand{\markyellow}[1]{\colorbox{yellow}{#1}} % Yellow highlighter

% Initials
\usepackage{lettrine}

% Some hacking to make that diacritic in "In nomine domini" possible
\usepackage{calc}
\newlength{\charwidth}
\newcommand{\innomdom}[3]{
	\setlength{\charwidth}{\widthof{#2'}}#1#2\mbox{\makebox[-\charwidth]{\raisebox{1.6\depth}{\rotatebox{180}{\fontspec{FreeSerif}ⷹ}}}\hspace{\charwidth}#3}
}
\newcommand\ol[1]{{\setul{-0.9em}{}\ul{#1}}}

\newenvironment{mytitle}{
  \hfill
  \begin{minipage}{0.667\textwidth}
	\vspace{\baselineskip}
	\begin{center}
	  \Large
	  \sffamily\bfseries
	  \makeatletter
}{
	  \makeatother
	\end{center}
	\vspace{1em}
  \end{minipage}
  \hfill
}

\begin{document}

%% MAIN PART %%%%%%%%%%%%%%%%%%%%%%%%%%%%%%%%%%%%%%%%%%%%%%%%%%%%%%%%%%%%%%%%%%%

\begin{mytitle}
  \@title: \@subtitle
\end{mytitle}

We actually have photos of all -- or at least a large number of -- the deeds from 
Wilhelm's \tit{Corpus der altdeutschen Originalurkunden bis zum Jahr 1300} at my 
university's Institute of Medieval German Philology, so I dug up the one for 
the deed I tried translating earlier (N163 (381a), dated Augsburg, March 19, 
1279) and transcribed it:

\begin{quotation}
\innomdom{\lettrine[lines=3,lraise=0.25]{J}{\upshape n}}{n}{}\innomdom{d}{n}{j} 
a͞m. Alle die die diſen brief leſent hoꝛent od͛ ſehent die ſuln daʒ wiʒʒen / daʒ h͛ 
Vlrich frovn Engeln toht͛man v͛kaufte h͛m Chvnrad dem hafener einen halben garten 
mit ſogtaner beſcheidenheit / daʒ d͛ h͛ Chvnrat d͛ hafen͛ vf den ſelben halben garten wed͛ ʒvn noch hvſ bwen ſolte / noch 
de cheinen bv darvf tvͦn ſolte d͛ im ʒe ſchaden chomen mohte. Daʒ ſtvnt alſ lange 
/ vnʒ d͛ h͛ Chvnrat d͛ hafen͛ dar fvr vnde wolte geʒvnet haben den ſelben garten. 
Do deſ h͛ Vlrich innan wart / do fvr er fvr gerihte. vnde chlagte hinʒe dem hafen͛ 
daʒ er da ʒvnen wolte da erſ niht tvn ſolte wande er eʒ mit gedingede im alſo 
geben hete / daʒ er wed͛ ʒvn noh hvſ / da bwen ſolte / noh nihteſ deſ / daʒ im ʒe 
ſchaden chomen mohte. Deſ laugent im d͛ hafener / daʒ er mit im alſo iht kauft 
hete. Darvb͛ wart erteilt / daʒ d͛ hafen͛ bereite daʒ er mit im alſo iht kauft hete 
/ in beʒivte danne h͛ Vlrich daʒ eʒ alſo wæ̂re. Da wolte h͛ Vlrich ſinſ rehteſ niht 
vmbe / vnde gerte einſ tageſ vmbe ſinen geʒivk. Do d͛ tak chom do gie h͛ Vlrich 
fvr / vnde erʒivgte ſelbe dritte alſ reht waſ. daʒ er vf den ſelben garten chein 
den bv tvn ſolte d͛ im ʒe ſchaden chomen mohte alſ da voꝛ geſchriben ſtat / vnde 
do er daʒ erʒivgte / do gert er vrteil wande die lvte tôtlich wæren vnde auh 
vergæʒʒen / man ſolte im wol d͛ Stet brief darvb͛ geben. Daʒ wart im erteilt mit 
geſament͛ vrteil. Vnde da von daʒ ʒwischen in chein kriech mer gewahſen mvge vmbe 
die ſache / dar vmbe wart geben dirre brief v͛ſigelt mit d͛ Stet Jnſigel ʒe Auſpurk 
daʒ dar an hanget. Vnde ſint deſ geʒuige h͛ Volkwin. h͛ Sibot d͛ Stolʒhirʒ. her 
Vlrich Fundan. h͛ Livpolt d͛ Stolʒhirʒ. her Chvnrat Reinbot. Meiſt͛ Chvnrat von 
Schoenegge. her chvnrat notkauf. her Chvnrat d͛ bart / vnde and͛ genvge. Do daʒ 
geſchach vnde auh dirre brief geben wart / do waſ von Goteſ geburte. Tuſent iar. 
ʒwei hundert iar. in dem Nivn vnde Sibenʒigoſten iare / an dem ſvnnetage voꝛ dem 
balmtage.
\end{quotation}

\end{document}
