\documentclass[12pt,paper=letter]{scrartcl}

% Handle language and quotation marks
\usepackage[ngerman,american]{babel}
\usepackage{csquotes} % Put quotations in \enquote{}!
\SetBlockEnvironment{quotation}
\renewcommand*{\mkccitation}[1]{ (#1)}

% Set all margins to 2.54 cm
\usepackage[margin=1in]{geometry}
\widowpenalty10000 % Avoid widows like the plague!
\clubpenalty10000 % Avoid orphans like the plage, too!

% Make multiple columns available in single-column document
\usepackage{multicol}

% Make text colors and color names available
\usepackage[xetex]{xcolor}

% Load font stuff for XeTeX, including Unicode support
\usepackage{fontspec}
\usepackage{xunicode}
\usepackage{xltxtra}

% Set main fonts
\newfontfamily{\Tagati}[
    Scale=1.1,
    BoldFont={* Italic},
]{Tagati Book G}

\setmainfont[
    Ligatures=TeX,
    Numbers=Lowercase,
]{Junicode}

\setsansfont[
    Ligatures=TeX,
    Numbers=Lowercase,
    Scale=MatchUppercase,
    BoldFont={Open Sans Condensed Bold},
]{Open Sans Condensed Light}

% Load BibLaTeX (using Biber), configure citation styles
% Uses <https://github.com/semprag/biblatex-sp-unified>
\usepackage[
    bibstyle=biblatex-sp-unified,
    citestyle=authoryear-comp,
    maxnames=2,
    backend=biber,
    safeinputenc,
]{biblatex}

% To make \textcite look like "Doe (2014: 213)"
\renewcommand*{\postnotedelim}{\addcolon\addspace}
\DeclareFieldFormat{postnote}{#1}
\DeclareFieldFormat{multipostnote}{#1}

% Enable generating files from this .tex file (for the bibliography) 
\usepackage{filecontents}

% Date etc.
\usepackage{datetime}
\renewcommand{\dateseparator}{/}

% Clickable links in footnotes, TOC, etc.
\usepackage[
    xetex,
    bookmarks=true,
    colorlinks=false,
    linktoc=section,
    hidelinks,
]{hyperref}

\usepackage{url}
\urlstyle{rm}

% Ability to include graphics and dealing with footnotes in descriptions
\usepackage{graphicx}
\usepackage[font={small,sf},labelfont={small,sf},format=plain]{caption}
\usepackage{subcaption}
\usepackage{wrapfig}
\setlength{\columnsep}{2\baselineskip}

% General headers and footers
\usepackage{fancyhdr}
\pagestyle{fancy}

\fancyhead[L]{} % empty
\fancyhead[C]{} % empty
\fancyhead[R]{\thepage}

\fancyfoot[L]{} % empty
\fancyfoot[C]{} % empty
\fancyfoot[R]{} % empty

\renewcommand{\headrulewidth}{0pt}
\renewcommand{\footrulewidth}{0pt}

% First page headers and footers are different
\fancypagestyle{firstpage}{
    \fancyhead[L]{\sffamily \footnotesize \textbf{Benung. The Ayeri Language Resource}}
    \fancyhead[C]{} % empty
    \fancyhead[R]{\sffamily \footnotesize Carsten Becker · \yyyymmdddate\today}
    
    \fancyfoot[L]{} % empty
    \fancyfoot[C]{} % empty
    \fancyfoot[R]{\sffamily \footnotesize 
	\href{http://benung.nfshost.com}{http://benung.nfshost.com} · 
	\href{https://github.com/carbeck/benung-pdfs}{github:carbeck/benung-pdfs} · 
	\href{https://creativecommons.org/licenses/by-sa/4.0/}{CC~BY-SA~4.0}
    }
    
    \renewcommand{\headrulewidth}{0.5pt}
    \setlength\footskip{0.5in}
}

\usepackage{ifthen}
\ifthenelse{\value{page}=1}{\thispagestyle{firstpage}{\pagestyle{fancy}}}

% Line spacing
\usepackage{setspace}
\onehalfspacing
%\linespread{1.5}

% Rotate big tables
\usepackage{rotating}

% Line numbering in verse environment
\usepackage{lineno}

% Formatting of glosses
\usepackage{gb4e}
\let\eachwordone=\itshape
%\def\cgdepthstrut{\vrule height 0pt depth 2ex width 0pt}
%\def\eachwordone{\cgdepthstrut\itshape}
%\pretocmd{\glt}{\vspace{0.75ex}}{}{}
%\AtBeginEnvironment{exe}{\smaller}

% Nicer footnotes
\usepackage[bottom]{footmisc}
\deffootnote{0em}{1.5em}{\textsuperscript\thefootnotemark\enskip}
\renewcommand{\footnoterule}{\rule{0pt}{0pt}{\vspace*{-0pt}}}
\setlength{\footnotesep}{1.5em}

% Smaller font in block quotes
\usepackage{relsize,etoolbox}
\AtBeginEnvironment{quote}{\noindent\smaller}
\AtBeginEnvironment{quotation}{\smaller}

% Macros
\newcommand{\fw}[1]{\textit{#1}} % Foreign Word
\newcommand{\tit}[1]{\textit{#1}} % Title of a work
\newcommand{\q}[1]{\enquote{#1}} % Context-aware quotation
\newcommand{\qq}[1]{\enquote*{#1}} % Explicit sublevel quotation
\newcommand{\tsup}[1]{\textsuperscript{#1}} % Superscript
\newcommand{\markyellow}[1]{\colorbox{yellow}{#1}} % Yellow highlighter

\newcommand{\divider}{\vspace{0.5\baselineskip} \centerline{* * *} \vspace{0.5\baselineskip}}

\newcommand{\mor}[1]{\textsc{\lowercase{#1}}}
\newcommand{\ayr}[1]{{\Tagati #1}}
\newenvironment{ayeri}{\begin{multicols}{2} \Tagati}{\end{multicols} \par}
\newenvironment{mytitle}{\begin{center} \Large \sffamily\bfseries ~\\}{\end{center}}

% blah
\usepackage{lipsum}

%% META INFORMATION IN PDF FILE %%%%%%%%%%%%%%%%%%%%%%%%%%%%%%%%%%%%%%%%%%%%%%%%

\hypersetup{%
    pdfinfo={%
	Title={Translation Challenge: The Beginning of Tolstoy's "Anna Karenina"},
	Author={Carsten Becker}
    }
}

%% BIBLIOGRAPHY DATABASE %%%%%%%%%%%%%%%%%%%%%%%%%%%%%%%%%%%%%%%%%%%%%%%%%%%%%%%

\begin{filecontents*}{\jobname.bib}
@article{becker,
    author = {Carsten Becker},
    title = {Flicking Switches: {Ayeri} and the {Austronesian} Alignment},
    journaltitle = {Benung. The Ayeri Language Resource},
    date = {2012-06-27},
    urldate = {2014-10-26},
    url = {http://benung.nfshost.com/archives/2444},
}

@incollection{schachter,
    author = {Paul Schachter},
    title = {The Subject in {Philippine} Languages: Topic, Actor, Actor-Topic, or None of the Above?},
    booktitle = {{Subject and Topic}},
    editor = {Charles N. Li},
    location = {New York},
    publisher = {Academic P},
    date = {1976},
    pages = {493--518},
}

@online{tolstoy,
    author = {Leo Tolstoy},
    title = {Anna Karenina},
    translator = {Constance Garnett},
    editor = {David Brannan and David Widger and Andrew Sly},
    organization = {Project Gutenberg},
    date = {2013-02-22},
    origdate = {1878},
    urldate = {2014-10-26},
    url = {http://www.gutenberg.org/ebooks/1399},
}

@online{universalsarchive,
    title = {The Universals Archive},
    editor = {Frans Plank and Thomas Mayer and Tatsiana Mayorava and Elena Filimonova},
    organization = {U Konstanz},
    date = {2009},
    urldate = {2014-10-26},
    url = {http://typo.uni-konstanz.de/archive/intro},
}
\end{filecontents*}
\addbibresource{\jobname.bib}

%% END OF PREAMBLE %%%%%%%%%%%%%%%%%%%%%%%%%%%%%%%%%%%%%%%%%%%%%%%%%%%%%%%%%%%%%

\begin{document}

%% MAIN PART %%%%%%%%%%%%%%%%%%%%%%%%%%%%%%%%%%%%%%%%%%%%%%%%%%%%%%%%%%%%%%%%%%%

\begin{mytitle}
    Translation Challenge: The Beginning of Tolstoy's \q{\citetitle{tolstoy}}
\end{mytitle}

\section{Text in English}

The text to be translated in this Translation Challenge is the initial passage
of \citeauthor{tolstoy}'s 1878 novel \citetitle{tolstoy}.\footnote{Hat tip to 
Steven Lytle for suggesting it.} The Ayeri translation here follows the English 
one by Constance Garnett (1901), which can be found on Project Gutenberg.

\blockcquote{tolstoy}{
\noindent Happy families are all alike; every unhappy family is unhappy in its 
own way.

Everything was in confusion in the Oblonskys’ house. The wife had discovered 
that the husband was carrying on an intrigue with a French girl, who had been a 
governess in their family, and she had announced to her husband that she could 
not go on living in the same house with him. This position of affairs had now 
lasted three days, and not only the husband and wife themselves, but all the 
members of their family and household, were painfully conscious of it. Every 
person in the house felt that there was no sense in their living together, and 
that the stray people brought together by chance in any inn had more in common 
with one another than they, the members of the family and household of the 
Oblonskys. The wife did not leave her own room, the husband had not been at 
home for three days. The children ran wild all over the house; the English 
governess quarreled with the housekeeper, and wrote to a friend asking her to 
look out for a new situation for her; the man-cook had walked off the day 
before just at dinner time; the kitchen-maid, and the coachman had given warning.
}

\section{Text in Ayeri}
\begin{ayeri}
kmyonF pMdhye\_aNF/henF mino – minrY mirneri sitNF/tonF pMdhaaNF/henF minrY.

EnYrENF Atauy kaarYo nNy pMdhn ObFlonFsFki. - - - ANF srojye EnFvnF sNlsF yen,
ANF yomojy AyonF rNY tonF lug bhisY kj.

...
\end{ayeri}

\noindent Kamayon pandahajang-hen mino; minarya miraneri sitang-ton 
pandahāng-hen minarya.

Enyareng atauya kāryo nangaya pandahana Oblonski. . . . Ang saroyye envan 
sangalas yena, ang yomoyya ayon rangya ton luga bahisya kay.

...

\section{Interlinear Breakdown}

\begin{exe}
    \ex \label{ex:1}
    \gll Kama -yon pandaha -ye -ang =hen mino; mino -arya miran -eri sitang- ton 
	pandaha -ang =hen mino -arya. \\
    be.like \mor{-3PL.N} family \mor{-PL} \mor{-A} =all happy; happy \mor{-NEG} 
	way \mor{-INS} self- \mor{3PL.N.GEN} family \mor{-A} =every happy
	\mor{-NEG}. \\
    \glt \enquote{All happy families are alike; every unhappy family is unhappy 
	in its own way.}
\end{exe}

I was looking for a way to express \qq{alike} here, first trying to go with
\ayr{kmo} \fw{kamo} \qq{equal, same} and then realizing that it might in fact
be better to use the related verb \ayr{km/} \fw{kama-} \qq{be as … as …}
intransitively to mean \qq{be alike}, i.e. expressing similarity rather than 
equality.

I am saying about \mor{N ADJ} order in an earlier blog article that 
\textcquote{becker}{the predication in equative sentences seems to be 
interpreted in the way of a transitive sentence, although it is lacking an 
overt predicate}: in Ayeri, adjectives in predicative sentences come after
the subject NP, as though they were objects. This, in fact, goes counter to 
Universal 145 in the \citetitle{universalsarchive} \autocite{universalsarchive},
since \citeauthor{universalsarchive} predict predicative adjectives in VSO 
languages to be treated like (or at least similar to, I suppose) verbs. I 
decided to break my rule and follow the universal in this case for stylistic 
reasons, because I thought that \q{pandahāng-hen minarya minarya miraneri 
sitang-ton} does not flow as well as with the constituents reversed.

\divider

\begin{exe}
    \ex
    \gll Enya -reng atau -ya kāryo nanga -ya pandaha -na Oblonski. \\
    everything \mor{-A.INAN} chaos \mor{-LOC} big house \mor{-LOC} family 
	\mor{-GEN} Oblonski. \\
    \glt \enquote{Everything was in a big chaos in the Oblonski family's house.}
\end{exe}

This sentence shows Ayeri's normal constituent order again, as compared to 
(\ref{ex:1}), the predication being the NP \q{atauya kāryo}. Tagalog,
which I have been returning to as an inspiration for Ayeri, seems to place 
not only predicative adjectives, but also predicative NPs before actors, however:

\begin{exe}
    \exi{(i)} Tagalog \autocite[499]{schachter}:
    \begin{xlist}
	\ex
	\gll Abogado ang lalaki. \\
	lawyer \mor{T-} man \\
	\glt \enquote{The man is a lawyer.}
	
	\ex
	\gll Matalino ang lalaki. \\
	intelligent \mor{T-} man \\
	\glt \enquote{The man is intelligent.}
    \end{xlist}
\end{exe}

\divider

The wife had discovered that the husband was carrying on an intrigue with a French girl, who had been a governess in their family, and she had announced to her husband that she could not go on living in the same house with him.

\begin{exe}
    \ex ...
\end{exe}

\divider

This position of affairs had now lasted three days, and not only the husband and wife themselves, but all the members of their family and household, were painfully conscious of it.

\begin{exe}
    \ex ...
\end{exe}

\divider

Every person in the house felt that there was no sense in their living together, and that the stray people brought together by chance in any inn had more in common with one another than they, the members of the family and household of the Oblonskys.

\begin{exe}
    \ex ...
\end{exe}

\divider

\begin{exe}
    \ex
    \gll Ang sara -oy -ye envan {} sangal -as yena, ang yoma -oy -ya ayon {} 
	rang -ya ton luga bahis -ya kay. \\
    \mor{AT} leave \mor{-NEG} \mor{-3SG.F} wife \mor{(.T)} room \mor{-P} 
	\mor{3SG.F.GEN}, \mor{AT} exist \mor{-NEG} \mor{-3SG.M} husband 
	\mor{(.T)} home \mor{-LOC} \mor{3PL.N.GEN} for day \mor{-LOC} three. \\
    \glt \enquote{The wife did not leave her room, the husband had not been
	at home for three days.}
\end{exe}

\divider

The children ran wild all over the house; the English governess quarreled with the housekeeper, and wrote to a friend asking her to look out for a new situation for her; the man-cook had walked off the day before just at dinner time; the kitchen-maid, and the coachman had given warning.

\begin{exe}
    \ex ...
\end{exe}

%% BIBLIOGRAPHY %%%%%%%%%%%%%%%%%%%%%%%%%%%%%%%%%%%%%%%%%%%%%%%%%%%%%%%%%%%%%%%%

\printbibliography

\end{document}
