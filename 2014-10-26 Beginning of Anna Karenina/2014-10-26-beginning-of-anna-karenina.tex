\documentclass[12pt,paper=letter]{scrartcl}

% Handle language and quotation marks
\usepackage[american]{babel}
\usepackage{csquotes} % Put quotations in \enquote{}!
\SetBlockEnvironment{quotation}
\renewcommand*{\mkccitation}[1]{ (#1)}

% Set all margins to 2.54 cm
\usepackage[margin=1in]{geometry}
\widowpenalty10000 % Avoid widows like the plague!
\clubpenalty10000 % Avoid orphans like the plage, too!

% Make multiple columns available in single-column document
\usepackage{multicol}

% Make text colors and color names available
\usepackage[xetex]{xcolor}

% Load font stuff for XeTeX, including Unicode support
\usepackage{fontspec}
\usepackage{xunicode}
\usepackage{xltxtra}

% Set main fonts
\newfontfamily{\Tagati}[
    Scale=1.1,
    BoldFont={* Italic},
    HyphenChar=/,
]{Tagati Book G}

\setmainfont[
    Ligatures=TeX,
    Numbers=Lowercase,
]{Junicode}

\setsansfont[
    Ligatures=TeX,
    Numbers=Lowercase,
    Scale=MatchUppercase,
    BoldFont={Open Sans Condensed Bold},
]{Open Sans Condensed Light}

% Load BibLaTeX (using Biber), configure citation styles
% Uses <https://github.com/semprag/biblatex-sp-unified>
\usepackage[
    bibstyle=biblatex-sp-unified,
    citestyle=authoryear-comp,
    maxnames=2,
    backend=biber,
    safeinputenc,
]{biblatex}

% To make \textcite look like "Doe (2014: 213)"
\renewcommand*{\postnotedelim}{\addcolon\addspace}
\DeclareFieldFormat{postnote}{#1}
\DeclareFieldFormat{multipostnote}{#1}

% Enable generating files from this .tex file (for the bibliography) 
\usepackage{filecontents}

% Date etc.
\usepackage{datetime}
\renewcommand{\dateseparator}{/}

% Clickable links in footnotes, TOC, etc.
\usepackage[
    xetex,
    bookmarks=true,
    colorlinks=false,
    linktoc=section,
    hidelinks,
]{hyperref}

\usepackage{url}
\urlstyle{rm}

% Ability to include graphics and dealing with footnotes in descriptions
\usepackage{graphicx}
\usepackage[font={small,sf},labelfont={small,sf},format=plain]{caption}
\usepackage{subcaption}
\usepackage{wrapfig}
\setlength{\columnsep}{2\baselineskip}

% General headers and footers
\usepackage{fancyhdr}
\pagestyle{fancy}

\fancyhead[L]{} % empty
\fancyhead[C]{} % empty
\fancyhead[R]{\thepage}

\fancyfoot[L]{} % empty
\fancyfoot[C]{} % empty
\fancyfoot[R]{} % empty

\renewcommand{\headrulewidth}{0pt}
\renewcommand{\footrulewidth}{0pt}

% First page headers and footers are different
\fancypagestyle{firstpage}{
    \fancyhead[L]{\sffamily \footnotesize \textbf{Benung. The Ayeri Language Resource}}
    \fancyhead[C]{} % empty
    \fancyhead[R]{\sffamily \footnotesize Carsten Becker · \yyyymmdddate\today}
    
    \fancyfoot[L]{} % empty
    \fancyfoot[C]{} % empty
    \fancyfoot[R]{\sffamily \footnotesize 
	\href{http://benung.nfshost.com}{http://benung.nfshost.com} · 
	\href{https://github.com/carbeck/benung-pdfs}{github:carbeck/benung-pdfs} · 
	\href{https://creativecommons.org/licenses/by-sa/4.0/}{CC~BY-SA~4.0}
    }
    
    \renewcommand{\headrulewidth}{0.5pt}
    \setlength\footskip{0.5in}
}

\usepackage{ifthen}
\ifthenelse{\value{page}=1}{\thispagestyle{firstpage}{\pagestyle{fancy}}}

% Line spacing
\usepackage{setspace}
\onehalfspacing
%\linespread{1.5}

% Rotate big tables
\usepackage{rotating}

% Line numbering in verse environment
\usepackage{lineno}

% Formatting of glosses
\usepackage{gb4e}
\let\eachwordone=\itshape
%\def\cgdepthstrut{\vrule height 0pt depth 2ex width 0pt}
%\def\eachwordone{\cgdepthstrut\itshape}
%\pretocmd{\glt}{\vspace{0.75ex}}{}{}
%\AtBeginEnvironment{exe}{\smaller}

% Nicer footnotes
\usepackage[bottom]{footmisc}
\deffootnote{0em}{1.5em}{\textsuperscript\thefootnotemark\enskip}
\renewcommand{\footnoterule}{\rule{0pt}{0pt}{\vspace*{-0pt}}}
\setlength{\footnotesep}{1.5em}

% Smaller font in block quotes
\usepackage{relsize,etoolbox}
\AtBeginEnvironment{quote}{\noindent\smaller}
\AtBeginEnvironment{quotation}{\smaller}

% Macros
\newcommand{\fw}[1]{\textit{#1}} % Foreign Word
\newcommand{\tit}[1]{\textit{#1}} % Title of a work
\newcommand{\q}[1]{\enquote{#1}} % Context-aware quotation
\newcommand{\qq}[1]{\enquote*{#1}} % Explicit sublevel quotation
\newcommand{\tsup}[1]{\textsuperscript{#1}} % Superscript
\newcommand{\markyellow}[1]{\colorbox{yellow}{#1}} % Yellow highlighter

\newcommand{\divider}{
    \vspace{\baselineskip}
}

\newcommand{\mor}[1]{\textsc{\lowercase{#1}}}
\newcommand{\ayr}[1]{{\Tagati #1}}

\newenvironment{ayeri}{
    %\hyphenpenalty=10000
    %\hbadness=10000
    \doublespacing
    \begin{multicols}{2}
    \Tagati
}{
    \end{multicols} \par
}

\newenvironment{mytitle}{
    \hfill
    \begin{minipage}{0.667\textwidth}
	\vspace{\baselineskip}
	\begin{center}
	    \Large
	    \sffamily\bfseries
}{
	\end{center}
	\vspace{1em}
    \end{minipage}
    \hfill
}

% blah
\usepackage{lipsum}

%% META INFORMATION IN PDF FILE %%%%%%%%%%%%%%%%%%%%%%%%%%%%%%%%%%%%%%%%%%%%%%%%

\hypersetup{%
    pdfinfo={%
	Title={Translation Challenge: The Beginning of Tolstoy's "Anna Karenina"},
	Author={Carsten Becker}
    }
}

%% BIBLIOGRAPHY DATABASE %%%%%%%%%%%%%%%%%%%%%%%%%%%%%%%%%%%%%%%%%%%%%%%%%%%%%%%

\begin{filecontents*}{\jobname.bib}
@article{becker,
    author = {Carsten Becker},
    title = {Flicking Switches: {Ayeri} and the {Austronesian} Alignment},
    journaltitle = {Benung. The Ayeri Language Resource},
    date = {2012-06-27},
    urldate = {2014-10-26},
    url = {http://benung.nfshost.com/archives/2444},
}

@incollection{schachter,
    author = {Paul Schachter},
    title = {The Subject in {Philippine} Languages: Topic, Actor, Actor-Topic, or None of the Above?},
    booktitle = {{Subject and Topic}},
    editor = {Charles N. Li},
    location = {New York},
    publisher = {Academic P},
    date = {1976},
    pages = {493--518},
}

@online{tolstoy,
    author = {Leo Tolstoy},
    title = {Anna Karenina},
    translator = {Constance Garnett},
    editor = {David Brannan and David Widger and Andrew Sly},
    organization = {Project Gutenberg},
    date = {2013-02-22},
    origdate = {1878},
    urldate = {2014-10-26},
    url = {http://www.gutenberg.org/ebooks/1399},
}

@online{universalsarchive,
    title = {The Universals Archive},
    editor = {Frans Plank and Thomas Mayer and Tatsiana Mayorava and Elena Filimonova},
    organization = {U Konstanz},
    date = {2009},
    urldate = {2014-10-26},
    url = {http://typo.uni-konstanz.de/archive/intro},
}
\end{filecontents*}
\addbibresource{\jobname.bib}

%% END OF PREAMBLE %%%%%%%%%%%%%%%%%%%%%%%%%%%%%%%%%%%%%%%%%%%%%%%%%%%%%%%%%%%%%

\begin{document}

%% MAIN PART %%%%%%%%%%%%%%%%%%%%%%%%%%%%%%%%%%%%%%%%%%%%%%%%%%%%%%%%%%%%%%%%%%%

\begin{mytitle}
    Translation Challenge: The Beginning of Tolstoy's \q{\citetitle{tolstoy}}
\end{mytitle}

\section{Text in English}

The text to be translated in this Translation Challenge is the initial passage
of \citeauthor{tolstoy}'s 1878 novel \citetitle{tolstoy}.\footnote{Hat tip to 
Steven Lytle for suggesting it.} The Ayeri translation here follows the English 
one by Constance Garnett (1901), which can be found on \tit{Project Gutenberg}.

\blockcquote{tolstoy}{
\noindent Happy families are all alike; every unhappy family is unhappy in its 
own way.

Everything was in confusion in the Oblonskys’ house. The wife had discovered 
that the husband was carrying on an intrigue with a French girl, who had been a 
governess in their family, and she had announced to her husband that she could 
not go on living in the same house with him. This position of affairs had now 
lasted three days, and not only the husband and wife themselves, but all the 
members of their family and household, were painfully conscious of it. Every 
person in the house felt that there was no sense in their living together, and 
that the stray people brought together by chance in any inn had more in common 
with one another than they, the members of the family and household of the 
Oblonskys. The wife did not leave her own room, the husband had not been at 
home for three days. The children ran wild all over the house; the English 
governess quarreled with the housekeeper, and wrote to a friend asking her to 
look out for a new situation for her; the man-cook had walked off the day 
before just at dinner time; the kitchen-maid, and the coachman had given warning.
}

\section{Text in Ayeri}
\begin{ayeri}
\noindent
kmyonF pMdhye\_aNF/henF mino – minrY mirneri sitNF/tonF pMdhaaNF/henF minrY.

EnYrENF Atauy kaarYo nNy pMdhn ObFlonFsFki.
silF\-visye sris EnFvnNF, ANF mN miry AyonF yen tYaan/tYaansF lyeri khni, seri 
    gnFvyaasF pMdhy tonF, nj ANF nrisye AyonF\-ymF yen, ANF miNF sjliNojye 
    mitnYmF nNy kmo kjvo yaaj. ENF mN yomaarnF Ed/mineye lug bhisY kj, nj toNF 
    vksF tenF puleNeri, sitNF/toNF/nmoj AyonNF nj EnFvnNF, naarY 
    nsimyye\_aNF/henF pMdhn nj nNaanen tonF njnj. 
ANF myyo nYaanF/henF nNy, miNF tenubisojrej, mitMtoNF kdnY.
ANF ENYonF vihYmF miromaanYe\_asF kejnmF si s lMtYonF kdnY Apineri koMdNy, 
    nsimyye\_asF pMdhn nj nNaanen ObFlonFsFki.
ANF srojye EnFvnF sNlsF yen, ANF mN yomojy AyonF rNY tonF lug bhisY kj.
s senYonF gnYe nNy/henF – ANF rnYe gnF\-vy ANFli kjvo lomaayy vismF nj ANF 
    thnYe ledoymF, ymF mY blNYeNF pinYnF ynolej gumo hiro ye – ANF sry ErFsy
    bhisY sris pidimY trik sirutjyaanen – ANF nristonF lomaay risNF nj lMty
    vptnsF tonF.
\end{ayeri}

\noindent
\tsup{1}~Kamayon pandahajang-hen mino; minarya miraneri sitang-ton 
    pandahāng-hen minarya.

\tsup{2}~Enyareng atauya kāryo nangaya pandahana Oblonski.
\tsup{3}~Silvisaye sarisa envanang, ang manga miraya ayon yena cān-cānas layeri 
    Kahani, seri ganvayās pandahaya ton, nay ang narisaye ayonyam yena, ang 
    ming saylingoyye mitanyam nangaya kamo kayvo yāy.
\tsup{4}~Eng manga yomāran eda-mineye luga bahisya kay, nay tong vakas ten 
    pulengeri, sitang-tong-namoy ayonang nay envanang, nārya nasimayajang-hen 
    pandahana nay nangānena ton naynay.
\tsup{5a}~Ang mayayo nyān-hen nangaya, ming tenubisoyrey, mitantong kadanya.
\tsup{5b}~Ang engyon vihyam miromānjas keynam si sa lancon kadanya apineri 
    konda\-ngaya, nasimayajas pandahana nay nangānena Oblonski.
\tsup{6}~Ang saroyye envan sangalas yena, ang manga yomoyya ayon rangya ton 
    luga bahisya kay.
\tsup{7}~Sa senyon ganye nangaya-hen; ang ranye ganvaya Angli kayvo lomāyaya 
    visam nay ang tahanye ledoyam, yam mya balangyeng pinyan yanoley gumo hiro 
    ye; ang saraya ersaya bahisya sarisa pidimya tarika sirutayyānena; ang 
    narisaton lomāya risang nay lantaya vapatanas ton.


\section{Interlinear Breakdown}

\begin{exe} % SENTENCE 1
    \ex \label{ex:1}
    \gll Kama -yon pandaha -ye -ang =hen mino; mino -arya miran -eri sitang- ton 
	pandaha -ang =hen mino -arya. \\
    be.like \mor{-3PL.N} family \mor{-PL} \mor{-A} =all happy; happy \mor{-NEG} 
	way \mor{-INS} self- \mor{3PL.N.GEN} family \mor{-A} =every happy
	\mor{-NEG}. \\
    \glt \q{All happy families are alike; every unhappy family is unhappy 
	in its own way.}
\end{exe}

I was looking for a way to express \qq{alike} here, first trying to go with
\ayr{kmo} \fw{kamo} \qq{equal, same} and then realizing that it might in fact
be better to use the related verb \ayr{km/} \fw{kama-} \qq{be as … as …}
intransitively to mean \qq{be alike}, i.e. expressing likeness rather than 
equality.

In an earlier article on the blog, I wrote about constituent order in Ayeri that
\textcquote{becker}{the predication in equative sentences seems to be 
interpreted in the way of a transitive sentence, although it is lacking an 
overt predicate}: in Ayeri, adjectives and nominals in predicative sentences 
come after the subject NP, as though they were objects. This, in fact, 
goes counter to Universal 145 in \citetitle{universalsarchive} \autocite{universalsarchive},
since \citeauthor{universalsarchive} predict predicative adjectives in VSO 
languages to be treated like (or at least similar to, I suppose) verbs. I 
decided to break my rule and follow the universal in this case for stylistic 
reasons, since I thought that \fw{pandahāng-hen minarya minarya miraneri 
sitang-ton} does not flow as well as the sentence with the constituents reversed.

\divider

\begin{exe} % SENTENCE 2
    \ex
    \gll Enya -reng atau -ya kāryo nanga -ya pandaha -na Oblonski. \\
    everything \mor{-A.INAN} chaos \mor{-LOC} big house \mor{-LOC} family 
	\mor{-GEN} Oblonski. \\
    \glt \q{Everything was in a big chaos in the Oblonski family's house.}
\end{exe}

This sentence shows Ayeri's normal constituent order again, as compared to 
(\ref{ex:1}), the predication being the NP \q{atauya kāryo.} Tagalog, as a VSO
language which I have been returning to as an inspiration for Ayeri, however, 
seems to place not only predicative adjectives before actors like the universal 
cited above predicts, but also predicative NPs:

\begin{exe}
    \exi{(i)} Tagalog \autocite[499]{schachter}:
    \begin{xlist}
	\ex
	\gll Abogado ang lalaki. \\
	lawyer \mor{T-} man \\
	\glt \q{The man is a lawyer.}
	
	\ex
	\gll Matalino ang lalaki. \\
	intelligent \mor{T-} man \\
	\glt \q{The man is intelligent.}
    \end{xlist}
\end{exe}

Since I could not find out anything about the etymology of the name Oblonsky,
I left it as it is and respelled it with the graphemes available in Ayeri 
transcription.

\divider

\begin{exe} % SENTENCE 3
    \ex \label{ex:3}
    \begin{xlist}
	\ex \gll Silvisa -ye sarisa envan -ang, ang manga mira -ya ayon {} yena 
	    cān \textasciitilde{}cān -as lay -eri Kahani, si -eri ganvaya -as 
	    pandaha -ya ton,~… \\
	discover \mor{-3SG.F} previously wife \mor{-A}, \mor{AT} \mor{PROG} do
	    \mor{-3SG.M} husband \mor{(-T)} \mor{3SG.F.GEN} love 
	    \mor{\textasciitilde{}DIM} \mor{-P} girl \mor{INS} French, 
	    \mor{REL} \mor{-INS} governess \mor{-P} family \mor{-LOC} 
	    \mor{3PL.N.GEN},~… \\
	\glt \q{The wife had previously discovered that her husband
	    had been having an affair with a French girl, who had been a
	    governess in their family,~…}
	
	\ex \gll …, nay ang narisa -ye {} ayon -yam yena, ang ming sayling -oy 
	    -ye {} mitan -yam nanga -ya kamo kayvo yāy. \\
	…, and \mor{AT} announce \mor{-3SG.F} \mor{(-T)} husband \mor{-DAT}
	    \mor{3SG.F.GEN}, \mor{AT} can continue \mor{-NEG} \mor{-3SG.F} 
	    \mor{(-T)} live \mor{-PTCP} house \mor{-LOC} same with 
	    \mor{3SG.M.LOC}. \\
	\glt \q{…, and had announced to her husband that she could not 
	    continue living in the same house with him.}
    \end{xlist}
\end{exe}

This is the first section that necessitated coming up with new vocabulary:
\ayr{tYaanF/tYaanF} \fw{cān-cān} \qq{(love) affair} as a diminutive of 
\ayr{tYaanF} \fw{cān} \qq{love} (a little like German \fw{Liebelei} and 
\fw{Techtelmechtel}); \ayr{gnFvy} \fw{ganvaya} \qq{governess} from a 
combination of \ayr{gnF} \fw{gan} \qq{child} and the feminine occupational 
derivative suffix \ayr{/vy} \fw{-vaya}; and \ayr{nris/} \fw{narisa-} 
\qq{announce} as a causative derivation of \ayr{nr/} \fw{nara-} \qq{say, 
speak}: \qq{make sth. said}.

This sentence is also a good illustration for Ayeri's tendency to underspecify
tense. For one, narratives are not marked by the epic past tense, and secondly,
the aspect of actions happening before the time of narration or leading up to it
is only signified by the time adverb \ayr{sris} \fw{sarisa} \qq{previous(ly), 
earlier; former(ly)}.

\divider

\begin{exe}  % SENTENCE 4
    \ex \gll Eng manga yoma -aran eda- mine -ye {} luga bahis -ya kay, nay tong 
	vakas ten puleng -eri, sitang- tong =nama =oy ayon -ang nay envan -ang, 
	nārya nasimaya -ye -ang =hen pandaha -na nay nangān -ena ton naynay. \\
    \mor{AT.INAN} \mor{PROG} exist \mor{-3PL.INAN} this- issue \mor{-PL} 
	\mor{(-T)} for day \mor{-LOC} three, and \mor{3PL.N.A} aware 
	\mor{3PL.INAN.GEN} pain \mor{-INS}, self- \mor{3PL.N.A} =just \mor{=NEG}
	husband \mor{-A} and wife \mor{-A}, but member \mor{-PL} \mor{-A} 
	=all family \mor{-GEN} and household \mor{-GEN} \mor{3PL.N.GEN} as.well \\
    \glt \q{These issues had continued to exist for three days, and they had 
	all been aware of them with pain, not only the husband the wife 
	themselves, but all members of their family and household as well.}
\end{exe}

Words new or modified in this sentence are \ayr{vksF} \fw{vakas} \qq{conscious,
awake}, whose definition should also include \qq{aware}, and \ayr{nNaanF}
\fw{nangān} \qq{household}, which is a renominalization of \ayr{nN} \fw{nanga}
\qq{house}. The ongoing aspect of the conflict is indicated by the progressive
marker \ayr{mN} \fw{manga} in addition to the time reference \fw{luga bahisya
kay} \qq{for three days}.

\divider

\begin{exe}  % SENTENCE 5
    \ex
    \begin{xlist}
	\ex \label{ex:5a}
	\gll Ang maya -yo nyān {} =hen nanga -ya, ming tenubis -oy -rey, 
	    mitan -tong kadanya. \\
	\mor{AT} feel \mor{-3SG} person \mor{(-T)} =every house \mor{-LOC}, can 
	    make.sense.of \mor{-NEG} \mor{-3SG.INAN}, live \mor{-3PL.N.A} 
	    together.\\
	\glt \q{Every person in the house felt that it couldn't be made sense 
	    of them living together.}
	
	\ex \label{ex:5b}
	\gll Ang eng -yon vih -yam miromān -ye -as keynam {} si sa lant -yon {} 
	    kadanya apin -eri kondanga -ya, nasimaya -ye -as pandaha -na nay 
	    nangān -ena Oblonski. \\
	\mor{AT} be.more \mor{3PL.N} share \mor{-PTCP} trait \mor{-PL} \mor{-P} 
	    people \mor{(-T)} \mor{REL} \mor{PT} lead \mor{-3PL.N} \mor{(-T)} 
	    together luck \mor{-INS} inn \mor{-LOC}, member 
	    \mor{-PL} \mor{-P} family \mor{-GEN} and household \mor{-GEN} 
	    Oblonski. \\
	\glt \q{People who were led together in an inn by luck shared more 
	    traits than the members of the family and the household Oblonski.}
    \end{xlist}
\end{exe}

Sentence (\ref{ex:5a}) posed a difficulty in translating \q{there was no sense 
in their living together} due to the nominalized VP, \q{living together} 
\parencite{tolstoy}. I decided to leave the phrase unnominalized in my 
translation, although in principle, \ayr{mitMkdnYaanF} \fw{mitankadanyān} 
\qq{living-together} would have been possible. The problem then is, however, 
how to express the genitive part of the original phrase. Fortunately, Ayeri 
allows complement phrases, so \fw{mitantong kadanya} is here literally 
\qq{that they live together}, the conjunction being inferred from context.

Furthermore, I did not want to literally translate \q{there was no sense in …} 
after I discovered that I had earlier coined a verb \ayr{tenubisF/} \fw{tenubis-} 
\qq{make sense of}. Since I wanted to avoid a non-specific \qq{you} here, I 
rendered the subject of this clause as a patient, thus turning the verb into 
a passive form.

Sentence (\ref{ex:5b}) contained the difficulty of a comparative statement that
was beyond a simple \qq{\mor{CMP} is more \mor{QTY} than \mor{STD}}. To evoke a 
more literary style, I used the comparative verb \ayr{ENF/} \fw{eng-} \qq{be 
more \mor{QTY} than \mor{STD}} instead of a simpler construction with the 
corresponding suffix \ayr{/ENF} \fw{-eng}. I am actually not completely sure if 
the verb construction makes sense syntactically, but I assumed that \ayr{ENF/} 
\fw{eng-} would be the head in this case and the phrase headed by \ayr{vihF/} 
\fw{vih-} \qq{share} would be subordinate to it, essentially serving as the 
quality \mor{QTY} of the standard \mor{STD} that the comparee \mor{CMP} is 
compared to:

\begin{exe}
    \exi{(ii)}
    \begin{xlist}
	\ex \mor{CMP} be.more [\fakesubscript{\mor{QTY}} big] than \mor{STD}.
	\ex \mor{CMP} be.more [\fakesubscript{\mor{QTY}} sharing traits] than \mor{STD}.
    \end{xlist}
\end{exe}

The rather complex comparative structure is, however, further complicated by the 
relative clause attributed to the agent \ayr{kejnmF(ANF)} \fw{keynam[ang]} 
\qq{people}. Due to information flow and syntactic weight, this constituent 
moves to the right edge of the clause.

\divider

\begin{exe}  % SENTENCE 6
    \ex \gll Ang sara -oy -ye envan {} sangal -as yena, ang manga yoma -oy -ya 
	ayon {} rang -ya ton luga bahis -ya kay. \\
    \mor{AT} leave \mor{-NEG} \mor{-3SG.F} wife \mor{(-T)} room \mor{-P} 
    \mor{3SG.F.GEN}, \mor{AT} \mor{PROG} exist \mor{-NEG} \mor{-3SG.M} 
	husband \mor{(-T)} home \mor{-LOC} \mor{3PL.N.GEN} for day \mor{-LOC} 
	three. \\
    \glt \q{The wife did not leave her room, the husband had not been at home 
	for three days.}
\end{exe}

I am a little tempted to insert a conjunction between the two main clauses here,
but I do not know whether it should be \ayr{nj} \fw{nay} \qq{and} or \ayr{ynoymF}
\fw{yanoyam} \qq{because}, since either reading is possible.

\divider

\begin{exe}  % SENTENCE 7
    \ex
    \begin{xlist}
	\ex \gll Sa sen -yon gan -ye {} nanga -ya =hen;~… \\
	\mor{PT} neglect \mor{-3PL.N} child \mor{-PL} \mor{(-T)} house 
	    \mor{-LOC} =all;~… \\
	\glt \q{Children were neglected all over the house;~…}
	
	\ex \gll …; ang ran -ye ganvaya {} Angli kayvo lomāya -ya visam nay ang 
	    tahan -ye {} ledo -yam, yam mya balang -yeng pinyan yano -ley gumo 
	    hiro  ye {} ;~… \\
	…; \mor{AT} argue \mor{-3SG.F} governess \mor{(-T)} English with servant 
	    \mor{-LOC} main and \mor{AT} write \mor{-3SG.F} \mor{(-T)} friend 
	    \mor{-DAT}, \mor{DATT} shall find \mor{-3SG.F.A} please place 
	    \mor{-P.INAN} work new \mor{3SG.F} \mor{(-T)} ;~… \\
	\glt \q{…; the English governess argued with the head servant and wrote
	    to a friend that she may please find her a new work place;~…}
	
	\ex \gll …; ang sara -ya ersaya {} bahis -ya sarisa pidim -ya tarika 
	    sirutayyān -ena;~… \\
	…; \mor{AT} leave \mor{-3SG.M} cook \mor{(-T)} day \mor{-LOC} previous 
	    hour \mor{-LOC} exact dinner \mor{-GEN};~… \\
	\glt \q{…; the cook left the previous day at the exact hour of dinner;~…}
	    
	\ex \gll …; ang narisa -ton lomāya {} risang nay lantaya {} vapatan -as 
	    ton. \\
	…; \mor{AT} express \mor{-3PL.N} servant \mor{(-T)} kitchen and driver
	    \mor{(-T)} warning \mor{-P} \mor{3PL.N.GEN}. \\
	\glt \q{…; the kitchen servant and the driver expressed their warning.}
    \end{xlist}
\end{exe}

For this passage as well, new words had to be coined or extended: \ayr{senF/} 
\fw{sen-} \qq{neglect}, and \ayr{nris} \fw{narisa-} \qq{announce}, coined in 
(\ref{ex:3}), gains the additional meanings \qq{express, pronounce}. \ayr{lMty} 
\fw{lantaya} \qq{driver} and \ayr{vptnF} \fw{vapatan} \qq{warning} are not in 
the dictionary, but they are nouns regularly derived from verbs, in so far, they 
are transparent and not in desperate need of a new dictionary entry.

\section{Conclusive Thoughts}

This translation exercise was originally proposed in a discussion thread 
as a stand-in for the popular \q{Babel Text} from the biblical book of 
Genesis (11:1--9) for those who do not wish to use a religious mythical story to
compare fictional languages and showcase their own. And while the passage from
\citetitle{tolstoy} contains many simple sentences, it still poses some 
syntactic difficulties that may be hard to tackle in a rather newly developed 
conlang. At least, this is the context in which I have most often seen the 
\q{Tower of Babel} story used in.

However, this passage from Tolstoy provides vocabulary that seems more 
immediately useful to get a conlang started than the \q{Tower of Babel} story.
How often, after all, does one talk about \qq{mortar} and \qq{bitumen}, or
translate texts into one's conlang that do? At least in my own experience,
vocabulary of the domestic and everyday life spheres appears much more 
frequently.

%% BIBLIOGRAPHY %%%%%%%%%%%%%%%%%%%%%%%%%%%%%%%%%%%%%%%%%%%%%%%%%%%%%%%%%%%%%%%%

\printbibliography

\end{document}
