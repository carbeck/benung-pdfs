\documentclass[12pt,paper=letter]{scrartcl}

% Language be German, with quotation marks accordingly
\usepackage[ngerman,american]{babel}
\usepackage{csquotes} % Put quotations in \enquote{}!
\SetBlockEnvironment{quotation}
\renewcommand*{\mkccitation}[1]{ (#1)}

% Set all margins to 2.54 cm
\usepackage[margin=1in]{geometry}
\widowpenalty10000 % Avoid widows like the plague!
\clubpenalty10000 % Avoid orphans like the plage, too!

% Make text colors and color names available
\usepackage[xetex]{xcolor}

% Load font stuff for XeTeX, including Unicode support
\usepackage{fontspec}
\usepackage{xunicode}
\usepackage{xltxtra}

% Set main fonts
\newfontfamily{\Tagati}[
    Scale=1.1,
    BoldFont={* Italic},
]{Tagati Book G}

\setmainfont[
    Ligatures=TeX,
    Numbers=Lowercase,
]{Junicode}

\setsansfont[
    Ligatures=TeX,
    Numbers=Lowercase,
    Scale=MatchUppercase,
    BoldFont={Open Sans Condensed Bold},
]{Open Sans Condensed Light}

% Load BibLaTeX (using Biber), configure citation styles
% Uses <https://github.com/semprag/biblatex-sp-unified>
\usepackage[
    style=biblatex-sp-unified,
    citestyle=sp-authoryear-comp,
    maxnames=2,
    backend=biber,
    safeinputenc,
]{biblatex}

% To make \textcite look like "Doe (2014: 213)"
\renewcommand*{\postnotedelim}{\addcolon\addspace}
\DeclareFieldFormat{postnote}{#1}
\DeclareFieldFormat{multipostnote}{#1}

% Enable generating files from this .tex file (for the bibliography) 
\usepackage{filecontents}

% Date etc.
\usepackage{datetime}
\renewcommand{\dateseparator}{/}

% Clickable links in footnotes, TOC, etc.
\usepackage[
    xetex,
    bookmarks=true,
    colorlinks=false,
    linktoc=section,
    hidelinks,
]{hyperref}

\usepackage{url}
\urlstyle{rm}

% Ability to include graphics and dealing with footnotes in descriptions
\usepackage{graphicx}
\usepackage[font={small,sf},labelfont={small,sf},format=plain]{caption}
\usepackage{subcaption}
\usepackage{wrapfig}
\setlength{\columnsep}{2\baselineskip}

% General headers and footers
\usepackage{fancyhdr}
\pagestyle{fancy}

\fancyhead[L]{} % empty
\fancyhead[C]{} % empty
\fancyhead[R]{\thepage}

\fancyfoot[L]{} % empty
\fancyfoot[C]{} % empty
\fancyfoot[R]{} % empty

\renewcommand{\headrulewidth}{0pt}
\renewcommand{\footrulewidth}{0pt}

% First page headers and footers are different
\fancypagestyle{firstpage}{
    \fancyhead[L]{\sffamily \footnotesize \textbf{Benung. The Ayeri Language Resource}}
    \fancyhead[C]{} % empty
    \fancyhead[R]{\sffamily \footnotesize Carsten Becker · \yyyymmdddate\today}
    
    \fancyfoot[L]{} % empty
    \fancyfoot[C]{} % empty
    \fancyfoot[R]{\sffamily \footnotesize 
	\href{http://benung.nfshost.com}{http://benung.nfshost.com} · 
	\href{https://github.com/carbeck/benung-pdfs}{github:carbeck/benung-pdfs} · 
	\href{https://creativecommons.org/licenses/by-sa/4.0/}{CC~BY-SA~4.0}
    }
    
    \renewcommand{\headrulewidth}{0.5pt}
    \setlength\footskip{0.5in}
}

\usepackage{ifthen}
\ifthenelse{\value{page}=1}{\thispagestyle{firstpage}{\pagestyle{fancy}}}

% Line spacing
\usepackage{setspace}
\onehalfspacing
%\linespread{1.5}

% Rotate big tables
\usepackage{rotating}

% Line numbering in verse environment
\usepackage{lineno}

% Formatting of glosses
\usepackage{gb4e}
\let\eachwordone=\itshape
%\def\cgdepthstrut{\vrule height 0pt depth 2ex width 0pt}
%\def\eachwordone{\cgdepthstrut\itshape}
%\pretocmd{\glt}{\vspace{0.75ex}}{}{}
%\AtBeginEnvironment{exe}{\smaller}

% Nicer footnotes
\usepackage[bottom]{footmisc}
\deffootnote{0em}{1.5em}{\textsuperscript\thefootnotemark\enskip}
\renewcommand{\footnoterule}{\rule{0pt}{0pt}{\vspace*{-0pt}}}
\setlength{\footnotesep}{1.5em}

% Smaller font in block quotes
\usepackage{relsize,etoolbox}
\AtBeginEnvironment{quote}{\noindent\smaller}
\AtBeginEnvironment{quotation}{\smaller}

% Macros
\newcommand{\fw}[1]{\textit{#1}} % Foreign Word
\newcommand{\tit}[1]{\textit{#1}} % Title of a work
\newcommand{\q}[1]{\enquote{#1}} % Context-aware quotation
\newcommand{\qq}[1]{\enquote*{#1}} % Explicit sublevel quotation
\newcommand{\tsup}[1]{\textsuperscript{#1}} % Superscript
\newcommand{\markyellow}[1]{\colorbox{yellow}{#1}} % Yellow highlighter

\newcommand{\mor}[1]{\textsc{\lowercase{#1}}}
\newcommand{\ayr}[1]{{\Tagati #1}}
\newenvironment{ayeri}{\Tagati}{\par}
\newenvironment{mytitle}{\begin{center} \Large \sffamily\bfseries ~\\}{\end{center}}

% blah
\usepackage{lipsum}

%% META INFORMATION IN PDF FILE %%%%%%%%%%%%%%%%%%%%%%%%%%%%%%%%%%%%%%%%%%%%%%%%

\hypersetup{%
    pdfinfo={%
	Title={Translation Challenge: The Beginning of Tolstoy's "Anna Karenina"},
	Author={Carsten Becker}
    }
}

%% BIBLIOGRAPHY DATABASE %%%%%%%%%%%%%%%%%%%%%%%%%%%%%%%%%%%%%%%%%%%%%%%%%%%%%%%

\begin{filecontents*}{\jobname.bib}
@online{tolstoy,
author = {Leo Tolstoy},
title	= {{Anna Karenina}},
translator = {Constance Garnett},
editor = {David Brannan and David Widger and Andrew Sly},
organization = {Project Gutenberg},
date = {2013-02-22},
urldate = {2014-10-26},
url = {http://www.gutenberg.org/ebooks/1399},
}
\end{filecontents*}
\addbibresource{\jobname.bib}

%% END OF PREAMBLE %%%%%%%%%%%%%%%%%%%%%%%%%%%%%%%%%%%%%%%%%%%%%%%%%%%%%%%%%%%%%

\begin{document}

%% MAIN PART %%%%%%%%%%%%%%%%%%%%%%%%%%%%%%%%%%%%%%%%%%%%%%%%%%%%%%%%%%%%%%%%%%%

\begin{mytitle}
    Translation Challenge: The Beginning of Tolstoy's \q{\citetitle{tolstoy}}
\end{mytitle}

\section{Text in English}

The text to be translated in this Translation Challenge is the initial passage
of \citeauthor{tolstoy}'s 1878 novel \citetitle{tolstoy}. The Ayeri translation 
here follows the English one by Constance Garnett (1901), which can be
found on Project Gutenberg.

\blockcquote{tolstoy}{
\noindent Happy families are all alike; every unhappy family is unhappy in its 
own way.

Everything was in confusion in the Oblonskys’ house. The wife had discovered 
that the husband was carrying on an intrigue with a French girl, who had been a 
governess in their family, and she had announced to her husband that she could 
not go on living in the same house with him. This position of affairs had now 
lasted three days, and not only the husband and wife themselves, but all the 
members of their family and household, were painfully conscious of it. Every 
person in the house felt that there was no sense in their living together, and 
that the stray people brought together by chance in any inn had more in common 
with one another than they, the members of the family and household of the 
Oblonskys. The wife did not leave her own room, the husband had not been at 
home for three days. The children ran wild all over the house; the English 
governess quarreled with the housekeeper, and wrote to a friend asking her to 
look out for a new situation for her; the man-cook had walked off the day 
before just at dinner time; the kitchen-maid, and the coachman had given warning.
}

\section{Text in Ayeri}
\ayr{kmyonF pMdhye\_aNF/henF mino – minrY mirneri sitNF/tonF pMdhaaNF/henF minrY.}

...

\noindent Kamayon pandahajang-hen mino; minarya miraneri sitang-ton 
pandahāng-hen minarya.

...

\section{Interlinear Breakdown}

\begin{exe}
    \ex
    \gll Kama -yon pandaha -j -ang =hen mino; mino -arya miran -eri sitang =ton 
	pandaha -ang =hen mino -arya \\
    be.like -\mor{3PL.N} family -\mor{PL} -\mor{A} =all happy; happy -\mor{NEG} 
	way -\mor{INS} self =\mor{3PL.N.GEN} family -\mor{A} =every happy
	-\mor{NEG} \\
    \glt \enquote{All happy families are alike; every unhappy family is unhappy 
	in its own way.}

    \ex ...
\end{exe}

%% BIBLIOGRAPHY %%%%%%%%%%%%%%%%%%%%%%%%%%%%%%%%%%%%%%%%%%%%%%%%%%%%%%%%%%%%%%%%

%\newpage
\nocite{*} % gibt alle Einträge in der Bibliografiedatenbank aus
\printbibliography

\end{document}
